\documentclass{article}
% Packages
\usepackage{amsmath}
\usepackage{amssymb}
\usepackage{amsthm}
\usepackage{mathtools}

% Dummy Text NOTE: Remove before Distribution
\usepackage{lipsum}

\usepackage{hyperref} % Links
\usepackage{cleveref} % Referring to Statements
\crefdefaultlabelformat{#2\bfseries\upshape(#1)#3}          
\creflabelformat{equation}{#2\bfseries\upshape(#1)#3}
\hypersetup{colorlinks=true,citecolor=red, linkcolor=blue}
\usepackage{tikz-cd} % Diagrams
\usepackage{parskip} % No Paragraph Indentation
\usepackage{xparse} % Multiple Optional Arguments
\usepackage{ifthen} % If-Else Statements
\usepackage{enumerate} % Enumerate with romans

% leftbar
\usepackage{framed}
\newlength{\leftbarwidth}
\setlength{\leftbarwidth}{1.5pt}
\newlength{\leftbarsep}
\setlength{\leftbarsep}{7.5pt}
\newcommand*{\leftbarcolorcmd}{\color{leftbarcolor}}% as a command to be more flexible
\colorlet{leftbarcolor}{black}
\renewenvironment{leftbar}{%
    \def\FrameCommand{{\leftbarcolorcmd{\vrule width \leftbarwidth\relax\hspace {\leftbarsep}}}}%
    \MakeFramed {\advance \hsize -\width \FrameRestore }%
}{%
    \endMakeFramed
}

% Underline
\usepackage{soul}

% Images
\usepackage{graphicx}
\graphicspath{{images/}}

% \setcounter{chapter}{-1}
\usepackage{emptypage}

% \usepackage[
% backend=biber,
% style=alphabetic,
% sorting=ynt
% ]{biblatex}
% \cref
\usepackage{cite}

\setcounter{tocdepth}{1}
% Global Definitions
% scope
\newenvironment{scope}{}{}

% statement
\newtheorem{statement}{}
\numberwithin{statement}{chapter}
% statement :: Definition
\theoremstyle{definition}
\newtheorem{definition}[statement]{Definition}
% statement :: Theorem
\theoremstyle{plain}
\newtheorem{theorem}[statement]{Theorem}
% statement :: Proposition
\theoremstyle{plain}
\newtheorem{proposition}[statement]{Proposition}
% statement :: Corollary
\theoremstyle{plain}
\newtheorem{corollary}[statement]{Corollary}
% statement :: Lemma
\theoremstyle{plain}
\newtheorem{lemma}[statement]{Lemma}
% statement :: Remark
\theoremstyle{remark}
\newtheorem{remark}[statement]{Remark}
% statement :: Example
\theoremstyle{definition}
\newtheorem{example}[statement]{Example}

% break
\newcommand{\blankbreak}{$ $\\}

% myparagraph
\newenvironment{myparagraph}{%
  \hspace{\parindent}%
} {%
  
}
% mydefinition
\newenvironment{mydefinition}[2][]{%
  \begin{definition}[#1]\label{#2}\blankbreak{}%
} {%
  \end{definition}%
}
% mytheorem
\newenvironment{mytheorem}[2][]{%
  \begin{theorem}[#1]\label{#2}\blankbreak{}%
} {%
  \end{theorem}%
}
% myproposition
\newenvironment{myproposition}[2][]{%
  \begin{proposition}[#1]\label{#2}\blankbreak{}%
} {%
  \end{proposition}%
}
% mycorollary
\newenvironment{mycorollary}[2][]{%
  \begin{corollary}[#1]\label{#2}\blankbreak{}%
} {%
  \end{corollary}%
}
% mylemma
\newenvironment{mylemma}[2][]{%
  \begin{lemma}[#1]\label{#2}\blankbreak{}%
} {%
  \end{lemma}%
}
% myremark
\newenvironment{myremark}[2][]{%
  \begin{remark}[#1]\label{#2}\blankbreak{}%
} {%
  \end{remark}%
}
% myexample
\newenvironment{myexample}[2][]{%
  \begin{example}[#1]\label{#2}\blankbreak{}%
} {%
  \end{example}%
}
% myproof
\newenvironment{myproof}[1][]{%
  \ifthenelse{\equal{#1}{}} {%
    \begin{proof}\blankbreak{}%
  } {%
    \begin{proof}[Proof #1.]\blankbreak{}%
  }
} {%
  \end{proof}%
}
% number
\newcommand{\N} {%
    \mathbb{N}%
}
\newcommand{\Z} {%
    \mathbb{Z}%
}
\newcommand{\Q} {%
    \mathbb{Q}%
}
\newcommand{\R} {%
    \mathbb{R}%
}
\newcommand{\C} {%
    \mathbb{C}%
}

% arrow
\newcommand{\xto}[1] {%
    \xrightarrow{#1}%
}
\newcommand{\isomto} {%
    \xto{\sim}%
}
\newcommand{\incl} {%
    \hookrightarrow{}%
}
\newcommand{\xincl}[1] {%
    \overset{#1}{\incl}%
}
\newcommand{\subm} {%
    \twoheadrightarrow{}%
}
\newcommand{\xsubm}[1] {%
    \overset{#1}{\subm}%
}

% set
\newcommand{\set}[2] {%
    \{#1:#2\}%
}
\newcommand{\coll}[1] {%
    \{#1\}%
}
% set :: operation
\newcommand{\cross}{\times}
\newcommand{\cmpl}[1]{#1^c}
% set :: subset
\newcommand{\sub} {%
    \subseteq{}%
}
\newcommand{\bus} {%
    \supseteq{}%
}

% Restrict
\newcommand{\restr}[1] {%
    \vert_{#1}%
}

% calculus
% calculus :: absolute
\newcommand{\abs}[1] {%
    \vert{} #1 \vert{}%
}
% calculus :: ball
\NewDocumentCommand{\ball}{O{1}O{0}}{%
    B_{#1} (0)%
}
\NewDocumentCommand{\clball}{O{1}O{0}}{%
    \overline{\ball[#1][#2]}%
}

% text
\newcommand{\tand} {%
    \text{ and }%
}
\newcommand{\tor} {%
    \text{ or }%
}
\newcommand{\twith} {%
    \text{ with }%
}

% function
\newcommand{\inv}[1] {%
    #1^{-1}%
}
\newcommand{\frst}[1] {%
    #1^{(1)}%
}
\newcommand{\snd}[1] {%
    #1^{(2)}%
}

% interval
\newcommand{\I} {%
    [0, 1]%
}
% interval :: cylinder
\newcommand{\cyl}[1] {%
    #1 \cross{} \I%
}
\newcommand{\cylup}[1] {%
    #1 \cross{} \{1\}%
}
\newcommand{\cyldown}[1] {%
    #1 \cross{} \{0\}%
}

% intern
\newcommand{\supp} {%
    \text{supp}%
}

% fraction
\newcommand{\half} {%
    \frac{1}{2}%
}

% variable
\newcommand{\eps} {%
    \varepsilon{}%
}

% newcommand
\newcommand{\case}[3] {
    \begin{cases} #1 & \text{if } #3 \\ #2 & \text{else } \end{cases}
}
\newcommand{\casenif}[2] {
    \begin{cases} #1 \\ #2 \end{cases}
}

\newcommand{\bundledef}[5] {
#1: #2 \xto{#4} #3 \xto{#5} #2{}
}
\newcommand{\bundle}[4] {
#1 \xto{#3} #2 \xto{#4} #1{}
}
\newcommand{\ba} {
\mathfrak{a}
}
\NewDocumentCommand{\bb}{o}{
\IfValueTF{#1}{\mathfrak{b}\restr{#1}}{\mathfrak{b}}
}
\newcommand{\bc} {
\mathfrak{c}
}
\newcommand{\bd} {
\mathfrak{d}
}
\newcommand{\be}[2][n]{
\mathfrak{e}^{#1}_{#2}
}
\newcommand{\bt} {
\mathfrak{t}
}
\newcommand{\bn} {
\mathfrak{n}
}

\newcommand{\ind}[1] {
#1^*
}

\newcommand{\germdef}[3] {
#1: #2 \Rightarrow{} #3
}
\newcommand{\germ}[2] {
#1 \Rightarrow{} #2
}

\newcommand{\whitney}[2] {
#1 \oplus{} #2
}

\newcommand{\bab} {
\ba\restr{a_0}
}
\newcommand{\bbb} {
\bb[b_0]
}
\newcommand{\bcb} {
\bc\restr{c_0}
}
\newcommand{\defterm}[1] {\emph{#1}}
\newcommand{\myintref}[1] {\Cref{#1}}
\newcommand{\extref}[1] {[#1]}

% Document
\begin{document}
% Document :: Titlepage
\begin{titlepage}
    \begin{center}
        \vspace{1.5cm}
        
        \begin{Large}Bachelorarbeit\end{Large}
        
        \vspace*{1cm}

        \begin{LARGE}\textbf{Microbundles on Toplogical Manifolds}\end{LARGE}
             
        \vspace{1cm}
 
        \begin{large}\textbf{Florian Burger}\end{large}
 
        \vfill
             
        \begin{large}
            Betreut durch Prof. Dr. Markus Banagl                
            
            \vspace{0.5cm}

            Fakultät für Mathematik und Informatik\\
            Ruprecht-Karls-Universität Heidelberg\\
            Heidelberg, den 1. Juli 2024
        \end{large}
    \end{center}
 \end{titlepage}
% Document :: MAKETITLE
% \maketitle
% \clearpage
% Document :: Abstract
\begin{abstract}
    This paper presents the concept of microbundles
    as introduced in 1964 by John Milnor.
    After showing some basic properties and constructions,
    including the induced microbundle and the Whitney sum,
    we discuss tangent- and normal microbundles
    over topological manifolds.
    We prove that for every microbundle over a
    simplicial complex, there exists an inverse in respect to the Whitney sum.
    Furthermore, we show that homotopic maps yield isomorphic induced microbundles.
    These results permit the proof
    that every topological submanifold $N \sub M$ has a normal microbundle
    in a stabilization $M \cross \R^q$ of the surrounding manifold.
\end{abstract}
% Document :: Table of Contents
\tableofcontents
\clearpage
% Document :: Introduction
\section{Introduction}\label{chapter::introduction}
% Document :: Introduction :: Motivation
\subsection*{Motivation}\label{section::motivation}{\blankbreak}
\chapter{Introduction}\label{chapter::introduction}

\begin{myparagraph}
    In differential geometry,
    the tangent- and normal bundle of smooth manifolds
    play an essential role in understanding their underlying geometry.

    One can define the tangent space
    in a point $p \in M$ using derivations
    \[ T_p M = \set{\nu: C^\infty(M) \to \R^n \text{ linear}}{\nu(fg) = f(p)\nu(g) + \nu(f)g(p)} \]    
    or using tangent curves
    \[ T_p M = \set{\gamma \in C^\infty((-1, 1), M)}{\gamma(0) = p } / \sim \]
    where $\gamma \sim \gamma' \iff \frac{d}{dx}(\psi \circ \gamma)(0) = \frac{d}{dx}(\psi \circ \gamma')(0)$
    and $\psi$ is a chart for $p$.
    
    The tangent space in a point allows for the definition of the tangent bundle
    \[ TM := \bigsqcup_{p \in M} T_p M\]
    together with the section
    \[ TM \xto{\pi} M \twith \pi(p, \nu) = p. \]

    Lets say one wants to define a tangent bundle over a topological manifold
    without given any smooth structure.
    We cannot use the same constructions as in the smooth case, because
    as we have just seen the definition of
    the tangent space always requires the notion of differentiability.

    What we can do is to generalize the concept of tangent bundles
    so that it can be applied to topological manifolds,
    in the hope that many results transfer to this generalization.

    In the paper `Microbundles, Part I', J. Milnor introduces a concept
    of tangent bundles over topological manifolds.
    This tangent bundle is not a vector bundle as in the smooth case,
    instead the tangent bundle is a `microbundle'
    which is a weakening of the definition of a vector bundle.

    One can transfer many constructions and results for
    vector bundles over to microbundles,
    for example the `whitney sum' or the `induced bundle'. 

    Furthermore, one can also define a microbundle analogue
    to the normal bundle for smooth manifolds.

    In the smooth case, given an embedded submanifold $P \sub M$,
    the normal space in a point $p \in P$
    is defined to be the quotient
    \[ N_p P = T_p M / T_p N. \]
    Similar to the tangent bundle, the normal bundle is defined as
    \[ NP  = \bigsqcup_{p \in P} N_p P \]
    together with the section
    \[ NP \xto{\pi} N \twith \pi(p, \nu) = p. \]
    
    So for smooth embedded manifolds there always exists a normal bundle
    in its surrounding space.

    This result doesn't transfer over to microbundles.
    One can find examples for embedded topological manifolds
    such that there doesn't exist a normal microbundle.

    However, Milnor shows that one can always find a normal microbundle
    of the submanifold $P$ in a tubular space $M \cross \R^q$
    for a sufficiently large $q \in \N$.

    Proving this statement while presenting the concept of
    microbundles along the way is the content of this thesis.

    It is based on Milnors paper `Microbundles, Part I',
    adopting much of its structure and proofs.
    Mostly there will be provided more details, proving every statement explicitly, than in Milnors original work.
\end{myparagraph}
% Document :: Introduction :: Microbundle
\subsection*{Introduction to Microbundles}\label{section::microbundle}{\blankbreak}
\begin{myparagraph}
    This section introduces the concept of microbundles
    together with some basic properties.
    We clarify what a microbundle is, what it means for a microbundle to be trivial and 
    cover some fundamental examples for microbundles.
\end{myparagraph}
% definition
\definition{(microbundle)} \\
A \defterm{microbundle} $\bb$ is \emph{hallo} a tuple $\bb := (B, E, i, j)$ satisfying the following properties:
\begin{itemize}
    \item $B$ is a topological space called the \defterm{base space}
    \item $E$ is a topological space called the \defterm{total space}
    \item $i: B \to E$ (\defterm{injection}) and $j: E \to B$ (\defterm{projection}) are continuous maps with $id_B = j \circ i$
    \item Every $b \in B$ is \defterm{locally trivializable}, i.e there exist open neighborhoods $U \sub B$ of $b$ and $V \sub E$ of $i(U)$ such that the following diagram commutes:
    \[\begin{tikzcd}[column sep=tiny]
        & V \ar[dr, "j"] \ar[dd, "\psi"] & \\
        U \ar[ur, "i"] \ar[dr, "{(id, 0)}"'] & & U \\
        & U \times\R^n \ar[ur, "\pi_1"'] &
    \end{tikzcd}\]
\end{itemize}
We call $n$ the \defterm{fibre dimension} of $\bb$.
% foreword
\begin{myparagraph}
    Before we look at some examples of microbundles,
    we first define what it means for two microbundles to be isomorphic.
\end{myparagraph}

% definition
\begin{mystatement}{definition}[isomorphy]{microbundle::isomorphy}[56]
    Two microbundles $\bundledef{\bb_1}{B}{E_1}{i_1}{j_1}$ and $\bundledef{\bb_2}{B}{E_2}{i_2}{j_2}$
    are \defterm{isomorphic} if there exist neighborhoods $V_1$ of $i_1(B)$ and $V_2$ of $i_2(B)$
    together with a homeomorphism $\psi: V_1 \isomto V_2$
    such that the following diagram commutes: 
    \begin{center}
        \begin{tikzcd}
            & V_1 \ar[dr, "j_1\vert_{V_1}"] \ar[dd, "\psi"] & \\
            B \ar[ur, "i_1"] \ar[dr, "i_2"'] & & B \\
            & V_2 \ar[ur, "j_2\vert_{V_2}"'] &
        \end{tikzcd}
    \end{center}
\end{mystatement}
% foreword
The following proposition shows that, when studying microbundles,
we are not interested in the entire total space
but only in an arbitrarily small neighborhood of the base space.
This is actually one of the biggest differences behind
the concept of microbundles and classical vector-bundles.

% prop: restricting the total space
\begin{myproposition}
    % statement
    For a microbundle $\bundledef{\bb}{B}{E}{i}{j}$ over $B$, we can be restrict the total space $E$ to an arbitray neighborhood $E' \sub E$ of $i(B)$
    where the resulting microbundle is isomorphic to $\bb$.
    % proof
    \begin{myproof}
        For an arbitray $b \in B$, choose a local trivialization $(U, V, \phi)$.

        The intersection $V \cap E'$ is a neighborhood of $i(b)$ because $V$ and $E'$ both are.
        It follows that $\phi(V \cap E')$ is a neighborhood of $(b, 0)$. Hence there exist $U' \sub B$ open and $\ball[\varepsilon] \sub \R^n$ such that $U' \times \ball[\varepsilon] \sub \phi(V \cap E')$. 
        Now we construct our local trivialization by choosing $V' := \phi^{-1}(U' \times \ball[\varepsilon])$ and the fact that $\ball[\varepsilon] \cong \R^n$:
        \[ U' \times \R^n \cong U' \times \ball[\varepsilon] \cong V' \]

        We easily see that the resulting microbundle is isomorphic to $\bb$ via the identity.
    \end{myproof}
\end{myproposition}
% foreword
The most obvious example for a microbundle is the standard microbundle.

% trivial microbundle
\begin{myexample}[trivial microbundle]
    % def: trivial microbundle
    For a topological space $B$, the \defterm{standard microbundle} $\be^n_B$ over $B$ is a diagram
    \[ \bundle{B}{B \times \R^n}{\iota}{\pi} \]
    where $\iota(b) := (b, 0)$ and $\pi(b, x) := b$.
    
    Additionally, a microbundle $\bb$ over $B$ is \defterm{trivial} if it is isomorphic to $\be^n_B$.
    % lemma: criteria for triviality
    \begin{mylemma}[criteria for triviality]
        A microbundle $\bb$ over $B$ is trivial if and only if there exists an open neighborhood $U$ of $i(B)$ such that $U \cong B \times \R^n$.
    \end{mylemma}
    % proof
    \begin{myproof}
        TODO
    \end{myproof}
\end{myexample}
% foreword
\begin{myparagraph}
    We are primarily interested in microbundles over topological manifolds.
    In this case, as topological manifolds are paracompact and hausdorff,
    triviality has stronger implications for the total space.
\end{myparagraph}

% lemma
\begin{mystatement}{lemma}{microbundle::paracompact}[57]
    A microbundle $\bb$ over a paracompact hausdorff space $B$ is trivial
    if and only if there exists an open neighborhood $V$ of $i(B)$ such that $V \cong B \cross \R^n$
    with injection and projection maps being compatible with this homeomorphism.
\end{mystatement}

% afterword
\begin{myparagraph}
    This means that there exists an open subset of $E(\bb)$ being homeomorphic to the entire $B \cross \R^n$,
    instead of only a neighborhood of $\cyldown{B}$ given by the definition of triviality.
\end{myparagraph}

% proof
\begin{myproof}
    We show both implications.

    `$\implies$'

    By restricting $E(\bb)$ to an open neighborhood
    and applying \myintref{microbundle::total} if necessary,
    we may assume that the entire $E(\bb)$ is an open subset of $B \cross \R^n$.

    Hence,
    there exist $B_i \sub B$ open and $0 < \eps_i < 1$ with $\bigcup_{i \in I} B_i = B$ such that
    \[ \bigcup_{i \in I } B_i \cross \ball[\eps_i] \sub E(\bb). \]
    Without loss of generality,
    we may assume that the collection $\{B_i\}$ is locally finite by refining this collection if necessary,
    utilizing the fact that $B$ is paracompact.

    Furthermore, $B$ being paracompact hausdorff yields a partition of unity
    \[ f_i: B \to [0, 1] \twith \supp{f_i} \sub B_i\]
    such that $\sum_{i \in J}f_i = 1$.
    
    We define a map $\lambda: B \to (0, \infty)$ with $\lambda = \sum_{i \in J} \eps_i f_i$,
    which has the property that $\abs{x} < \lambda(b) \implies (b, x) \in E(\bb)$ because
    \[ \abs{x} < \lambda(b) \]
    \[ \iff  \abs{x} < \eps_{i_1} f_{i_1}(b) + \cdots + \eps_{i_n} f_{i_n}(b) \]
    \[ \iff 0 < (\eps_{i_1} - \abs{x}) f_{i_1}(b) + \cdots + (\eps_{i_n} - \abs{x}) f_{i_n}(b) \]
    \[ \implies \exists i \in J: 0 < (\eps_{i} - \abs{x}) f_{i}(b) \]
    \[ \implies (b, x) \in B_i \cross \ball[\eps_i]  \implies (b, x) \in E(\bb). \]

    Finally, we have a homeomorphism between the open subset
    $\set{(b, x) \in B \cross \R^n}{\abs{x} < \lambda(b)} \sub E(\bb)$ and $B \cross \R^n$ via
    \[ (b, x) \mapsto (b, \frac{x}{\lambda(b) - \abs{x}}). \]
    Since $(b, 0)$ is mapped to $(b, 0)$, it follows that
    the homeomorphism commutes with the injection and projection maps.
    % TODO: explain commutativity

    `$\impliedby$'

    This is simply a weakening of the definition of triviality.
\end{myproof}
% foreword
\begin{myparagraph}
    The following definition is the microbundle analog
    to the tangent vector bundle.
\end{myparagraph}

% definition
\begin{mystatement}{definition}[tangent microbundle]{microbundle::tangent}[55]
    The \defterm{tangent microbundle} $\bt_M$ over a topological $d$-manifold $M$ is a diagram
    \[ \bundle{M}{M \cross M}{\Delta}{\pi_1} \]
    where $\Delta(m) := (m, m)$ denotes the diagonal map.
\end{mystatement}

% proof
\begin{myproof}[that $\bt_M$ is a microbundle]
    % continuity and compatibility
    The maps $\Delta$ and $\pi_1$ are continuous and clearly $\pi_1 \circ \Delta = id_M$.

    % local triviality
    % local triviality :: homeomorphism
    For an arbitrary $p \in M$,
    choose a chart $(U, \psi)$ over $p$.
    We have a local trivialization $(U, U \cross U, \phi)$ of $p$ in $\bt_M$ given by
    \[ \phi: U \cross U \isomto U \cross \R^n \twith \phi(u, u') = (u, \psi(u) - \psi(u')). \]
    Homeomorphy of $\phi$ follows from homeomorphy of $\psi$.
    
    % local triviality :: commutativity
    Commutativity between the injection and $id \times 0$ is given by
    \[ \phi(\Delta(m)) = \phi(m, m) = (m, \psi(m) - \psi(m)) = (m, 0) = (id \cross 0)(m)\]
    and between the projection and $\pi_1$ by
    \[ \pi_1(u, u') = u = \pi_1(u, \snd{\phi}(u, u')) = \pi_1(\phi(u, u')). \]
    Note that $\snd{\phi}$ denotes the map on
    the second component of $\phi$, i.e. $\pi_2 \circ \phi$.
\end{myproof}

% remark
\begin{mystatement}{remark}{microbundle::tangentdim}
    The tangent microbundle $\bt_M$ has fiber dimension $d$.
\end{mystatement}
% Docuement :: Constructions
\section{Standard Constructions}\label{chapter::constructions}
\begin{myparagraph}
    This section introduces two standard constructions for microbundles,
    the `induced microbundle' and the `Whitney sum'.
    Both constructions have their vector bundle analogue
    and many results carry over immediately to microbundles.
\end{myparagraph}
% Document :: Constructions :: Induced
\subsection*{Induced Microbundles}\label{section::induced}{\blankbreak}
\begin{myparagraph} Given a microbundle $\bb$ over $B$ and a map $f: A \to B$, one can define a microbundle $\ind{f}\bb$ over $A$. This is achieved by `pulling back' the base space $B$ to $A$ with the use of the map $f$.

After showing the existence of such a microbundle, we prove some basic properties such as triviality criteria and compatibility with map composition. Afterwards, we study induced microbundles over cones and simplicial complexes. \end{myparagraph} % definition
\begin{mystatement}{definition}[induced microbundle]{induced::definition}[58]
    Let $\bb$ be a microbundle over $B$ and let $f: A \to B$ be a map.
    The \defterm{induced microbundle} $\ind{f}\bb$ is a microbundle $\bundle{A}{E_f}{i_f}{j_f}$
    defined as follows:
    \begin{itemize}
        \item $E_f = \{ (a, e) \in A \cross E(\bb) \mid f(a) = j(e) \}$
        \item $i_f(a) = (a, (i \circ f)(a))$
        \item $j_f(a, e) = a$
    \end{itemize}
\end{mystatement}

% fiber bundle
\begin{myparagraph}
    The construction is identical to the one over vector bundles,
    more precisely over fiber bundles (compare to \cite[ch.2,sec.14]{brendon}).
\end{myparagraph}

% proof
\begin{myproof}[that $\ind{f}\bb$ is a microbundle]
    % continuity and compatibility
    Both $i_f$ and $j_f$ are continuous
    since they are composed by continuous functions.
    Additionally, $j_f(i_f(a)) = j_f(a, i(f(a))) = a$ and hence $j_f \circ i_f = id_A$.

    % local triviality
    It remains to be shown that $\ind{f}\bb$ is locally trivial.

    % local triviality :: homeomorphism
    For an arbitrary $a \in A$,
    choose a local trivialization $(U, V, \phi)$ of $i(a)$ in $\bb$.
    We construct a local trivialization of $a$ in $\ind{f}\bb$ as follows:
    \begin{itemize}
        \item $U_f = \inv{f}(U) \sub A$,
        which is an open neighborhood of $a$ since $f$ is continuous
        and $U$ is an open neighborhood of $i(a)$.
        \item $V_f = (U_f \cross V) \cap E_f \sub E_f$,
        which is an open neighborhood of $i_f(a)$
        since both $U' \cross V$ and $E_f$ are open neighborhoods of $i_f(a)$.
        \item $\phi_f: V_f \isomto U_f \cross \R^n$ with $\phi_f(a', e) = (a', \snd{\phi}(e))$
    \end{itemize}
    The existence of an inverse $\inv{\phi_f}(a', v) = (a', \inv{\phi}(f(a'), v))$
    and component-wise continuity for both $\phi_f$ and $\inv{\phi_f}$ show that $\phi_f$ is a homeomorphism.
    
    % local triviality :: commutativity
    Commutativity with $i_f$ and $id \times 0$ is given by
    \[ \phi_f(i_f(a')) = \phi(a', i(f(a'))) = (a', \snd{\phi}(i(f(a')))) = (a', 0) = (id \cross 0)(a') \]
    and with $j_f$ and $\pi_1$ by
    \[ j_f(a', e) = a'  = \pi_1(a', \phi(e)) = \pi_1(\phi_f(a', e)), \]
    which completes the proof.
\end{myproof} % definition
\begin{mystatement}{example}[restricted microbundle]{induced::restricted}[54]   
    Let $\bundledef{\bb}{B}{E}{i}{j}$ be a microbundle and $A \sub B$ be a subspace.
    The \defterm{restricted microbundle} $\bb[A]$ is the induced microbundle $\ind{\iota}\bb$
    where $\iota: A \incl B$ denotes the inclusion map.
\end{mystatement}

% remark
\begin{mystatement}{remark}{induced::total}
    From now on, we consider $E(\bb[A])$ to be a subset of $E(\bb)$.
    This is justified because there exists an embedding
    \begin{center}
        $\iota: E(\bb[A]) \to E(\bb)$ with $(a, e) \mapsto e$
    \end{center}
    together with the inverse $e \mapsto (j(e), e)$.
    Note that the same reasoning can be applied to
    induced microbundles over any injective map.
\end{mystatement} \begin{myparagraph} Next, we provide two criteria to show that an induced microbundle is trivial. \end{myparagraph} % lemma: induced trival microbundle
\begin{mylemma}\label{induced::trivial}
    Let $\bb$ be a microbundle over $B$ and $f: A \to B$ be a map.
    The induced microbundle $f^*\bb$ is trivial if $\bb$ is already trivial.
\end{mylemma}
% proof
\begin{myproof}
Let $(V, \phi)$ be a global trivialization, that is $\phi: V \isomto B \times \R^n$.
\begin{itemize}
    \item $V' := (A \times V) \cap E(f^*\bb)$ a neighborhood of $i'(A)$.
    \item $\phi': V' \isomto B \times \R^n, \phi'(a, e) := (a, \pi_2(\phi(e)))$.
\end{itemize}
The existence of an inverse $\phi'^{(-1)}(a, x) = (a, \phi^{-1}(f(a), x))$ and component-wise continuity show that $\phi'$ is a homeomorphism.
This proves that $(V', \phi')$ is a global trivialization for $f^*\bb$. 
\end{myproof} % lemma
\begin{mystatement}{lemma}{induced::const}
    Let $\bb$ be a microbundle over $B$.
    The induced microbundle $\ind{c_{A, b_0}}\bb$ over the constant map
    $c_{A, b_0}: A \to B$ with $c_{A, b_0}(a) = b_0$
    is trivial.
\end{mystatement}

% proof
\begin{myproof}
    % homeomorphism
    The total space $E(\ind{c_{A, b_0}}\bb)$ is defined as
    \[ \set{(a, e) \in A \cross E(\bb)}{f(a) = b_0 = j(e)} = A \cross \inv{j}(b_0). \]

    Let $(U, V, \phi)$ be a local trivialization for $b_0$ in $\bb$.
    Restricting $\phi$ to the fiber $\inv{j}(b_0)$ yields a homeomorphism
    \[ \phi\restr{\inv{j}(b_0)}: V \cap \inv{j}(b_0) \isomto b_0 \times \R^n. \]
    It follows that $\psi: A \cross (V \cap \inv{j}(b_0)) \isomto A \cross \R^n$ with
    $\psi(a, e) = (a, \snd{\phi}(e))$
    is a homeomorphism as well.

    % neighborhoods
    The product $A \cross (V \cap \inv{j}(b_0))$ is open in $E(\ind{c_{A, b_0}}\bb)$,
    since  $V \cap \inv{j}(b_0)$ is open in $\inv{j}(b_0)$ with the subspace topology.
    Furthermore, from
    \[ i_{c}(a) = (a, i(c_{A, b_0}(a))) = (a, i(b_0)) \tand \snd{\phi}(i(b_0)) = 0 \]
    it follows that $\psi(A \cross (V \cap \inv{j}(b_0)))$ is a neighborhood of $\cyldown{A}$.
    Hence, $\psi$ maps a neighborhood of $i_c(A)$
    to a neighborhood of $\cyldown{A}$.
    
    % commutativity
    Commutativity with the injection maps is given by
    \[ \psi(i_c(a)) = \psi(a, i(b_0)) = (a, \snd{\phi}(i(b_0))) = (a, 0) = (id \cross 0)(a) \]
    and with the projection maps by
    \[ j_c(a, e) = a = \pi_1(a, \snd{\phi}(e)) = \pi_1(\psi(a, e)). \]

    We conclude that $\ind{c_{A, b_0}}\bb$ is trivial.
\end{myproof} % foreword
\begin{myparagraph}
    The following lemma shows that induced microbundles are compatible
    with map composition.
\end{myparagraph}

% lemma
\begin{mystatement}{lemma}{induced::composition}
    Let $\bundledef{\bc}{C}{E}{i}{j}$ be a microbundle and let $A \xto{f} B \xto{g} C$ be a map diagram.
    Then the two microbundles
    \[ \bundledef{\ind{(g \circ f)}\bc}{A}{E_1}{i_1}{j_1} \tand \bundledef{\ind{f}(\ind{g}\bc)}{A}{E_2}{i_2}{j_2} \]
    are isomorphic.
\end{mystatement}

% proof
\begin{myproof}
    % homoemorphism
    First, compare the two total spaces:
    \begin{itemize}
        \item $E(\ind{(g \circ f)}) = \{ (a, e) \in A \cross E \mid g(f(a)) = j(e)\}$ 
        \item $E(\ind{f}(\ind{g}\bc)) = \{ (a, b, e) \in A \cross (B \cross E) \mid f(a) = b$ and $ g(b) = j(e) \}$
    \end{itemize}
    We define a homeomorphism $\psi: E(\ind{(g \circ f)}) \isomto E(\ind{f}(\ind{g}\bc))$ with
    \[ \psi(a, e) = (a, f(a), e) \tand \inv{\psi}(a, b, e) = (a, e) \]
    which is a homeomorphism since both $\psi$ and $\inv{\psi}$ are component-wise continuous.
    % commutativity
    Commutativity with the injection maps is given by
    \[ \psi(i_1(a)) = \psi(a, i(g(f(a)))) = (a, f(a), i(g(f(a)))) = i_2(a) \]
    and with the projection maps by
    \[ j_1(a, e) = a = j_2(a, f(a), e) = j_2(\psi(a, e)), \]
    which concludes the proof.
\end{myproof} % NOTE: Check proof

% foreword
For a topological space $X$, we define the \defterm{cone} of $X$ to be 
\[ CX := X \times [0, 1] / X \times \{1\} \]
and for a map $f: A \to B$ the \defterm{mapping cone} of $f$ to be
\[ B \sqcup_f CA := B \sqcup CA / \sim \]
where $(a, 0) \sim b :\iff f(a) = b$.

Similarly, we define the \defterm{cylinder} of $X$ to be
\[ MX := X \times [0, 1] \]
and for a map $f: A \to B$ the \defterm{mapping cylinder} of $f$ to be
\[ B \sqcup_f MA := B \sqcup CA / \sim \]
where $(a, 0) \sim b :\iff f(a) = b$.

% lemma: extending over a mapping cone
\begin{mylemma}
A microbundle $\bb$ over $B$ can be extended to a microbundle over the mapping cone $B \sqcup_f CA$ if and only if $f^*\bb$ is trivial.
\end{mylemma}
% proof
\begin{myproof}
We show both implications.

% =>
``$\implies$''

Let $\bb'$ be an extension of $\bb$ over $B \sqcup_f CA$.
Considering $A \xto{f} B \incl B \sqcup_f CA$, the composition $\iota \circ f$ is null-homotopic with homotopy
\[ H_t(a) := [(a, t)] \]
Note that $H_0(a) = [(a, 0)] = [f(a)] = (\iota \circ f)(a)$ and $H_1(a) = [(a, 1)] = [(\tilde{a}, 1)] = H_1(\tilde{a})$.
From the Homotopy Theorem~\ref{homotopy::theorem} follows that $(\iota \circ f)^*\bb'$ is trivial. 

Since $(\iota \circ f)^*\bb' = f^*(\iota^*\bb') = f^*\bb$, it follows that $f^*\bb$ is trivial.

% <=
``$\impliedby$''

Let $f^*\bb$ be trivial.

In contrast to the mapping cone, there exists a natural retraction from the mapping cylinder to the attached space
\[ \pi: B \sqcup_f MA \to B, \pi([(a, t)]) := f(a) \]
The diagram
\[ A \times \{1\} \incl B \sqcup_f MA \xto{\pi} B \]
equals $f$ if we consider $A = A \times \{1\}$.
It follows that
\[ \pi^*\bb\restr{A \times \{1\}} = (\pi \circ \iota)^*\bb \cong f^*\bb = \mathfrak{e}^n_A\]
is trivial.
From \cref{induced::trivial} and $(a, t) \mapsto (a, 1)$ it follows that $\pi^*\bb\restr{A \times [\frac{1}{2}, 1]}$ is trivial, so
\[ \exists \phi: E(\bb\restr{A \times [\frac{1}{2}, 1]}) \isomto A \times [\frac{1}{2}, 1] \times \R^n \]
Now we explicitly construct our desired extended microbundle $\bundledef{\bb'}{B \sqcup_f CA}{E'}{i'}{j'}$
\begin{itemize}
    \item $E' := E(\bb\restr{A \times [\frac{1}{2}, 1]}) / \phi^{-1}(A \times [\frac{1}{2}, 1] \times \{x\})$ (for every $x \in \R^n$)
    \item $i' := \pi \circ i$ the projection map.
    \item $j'([e]) := [j(e)]$ is well-defined since $[e] = [\tilde{e}] \implies [j(e)] = [j(e')]$
\end{itemize}
We derive the microbundle conditions from 
Now that we have constructed $\bb'$, the prove is complete.
\end{myproof} % cor: extending over a d-simplex
\begin{mycorollary}
Let $B$ be a $(d + 1)$-simplicial complex, $B'$ it's $d$-skeleton and $\sigma \sub B$ a $(d + 1)$-simplex.
A microbundle $\bb$ over $B'$ can be extended to a microbundle over $B' \cup \sigma$ if and only if its restriction to the boundary $\bb \restr{\partial \sigma}$ is trivial.
\end{mycorollary}
% proof
\begin{myproof}
    By choosing $f: \partial \sigma \incl B'$ and applying the previous lemma,
    we see that there exists a microbundle $\bb'$ over $B' \cup_f C\sigma$ extending $\bb$.

    Now, consider the homeomorphism $\phi: C\partial \sigma \isomto \sigma$ with
    \[ \phi((t_1, \dots, t_{d + 1}), \lambda) := (1 - \lambda) (t_1, \dots, t_{d + 1}) + \frac{\lambda}{d + 1} (1, \dots, 1) \]
    In particular, $\phi(\partial \sigma \times \{0\}) = \partial \sigma$.

    It follows that $B' \cup_f C\sigma \cong B' \cup \sigma$ which concludes the proof.
\end{myproof}
% Document :: Constructions :: Whitney
\subsection*{The Whitney Sum}\label{section::whitney}
\section{Whitney sums}
\begin{mystatement}{definition}{whitney::definition}[59] Let $\bb_1$ and $\bb_1$ be two microbundles over $B$ with fiber dimensions $n_1$ and $n_2$. The \defterm{Whitney sum} $\whitney{\bb_1}{\bb_2}$ is a microbundle $\bundle{B}{E}{i}{j}$ where \begin{itemize} \item $E = \set{ (e_1, e_2) \in E(\bb_1) \cross E(\bb_2)}{j_1(e_1) = j_2(e_2)}$ \item $i(b) = (i_1(b), i_2(b))$ \item $j(e_1, e_2) = j_1(e_1) = j_2(e_2)$ \end{itemize} with fiber dimension $n_1 + n_2$. \end{mystatement}

\begin{myproof}[that $\whitney{\bb_1}{\bb_2}$ is a microbundle] Both $i$ and $j$ are continuous since they are composed by continuous functions. Additionally, $j(i(b)) = j(i_1(b), i_2(b)) = j_1(i_1(b)) = b$ and hence $j \circ i = id_B$.

It remains to be shown that $\whitney{\bb_1}{\bb_2}$ is locally trivial:

For an arbitrary $b \in B$, choose local trivializations $(U_1, V_1, \phi_1)$ and $(U_2, V_2, \phi_2)$ of $b$ in $\bb_1$ and $\bb_2$.

We construct a local trivialization $(U, V, \phi)$ of $b$ in $\whitney{\bb_1}{\bb_2}$ as follows: \begin{itemize} \item $U = U_1 \cap U_2$, which is an open neighborhood of $b$ since both $U_1$ and $U_2$ are open neighborhoods of $b$. \item $V = (V_1 \cross V_2) \cap E$, which is an open neighborhood of $i(U)$ since $V_1$ and $V_2$ are open and $i(U) \sub (i_1(U) \times i_2(U)) \cap E \sub (V_1 \cross V_2) \cap E$. \item $\phi: V \isomto U \cross \R^{n_1 + n_2}$ with $\phi(e_1, e_2) = (j_1(e_1), (\snd{\phi_1}(e_1), \snd{\phi_2}(e_2)))$, which is a homeomorphism with the inverse \[ \inv{\phi}(b, (x_1, x_2)) = (\inv{\phi_1}(b, x_1), \inv{\phi_2}(b, x_2)) \] since both $\phi$ and $\inv{\phi}$ are component-wise continuous. \end{itemize}

Commutativity with $i$ and $id \times 0$ is given by \[ \phi(i(b)) = \phi(i_1(b), i_2(b)) = (b, (\snd{\phi_1}(i_1(b)), \snd{\phi_2}(i_2(b)))) = (b, (0, 0)) = (id \cross 0)(b) \] and with $j$ and $\pi_1$ by \[ j(e_1, e_2) = j_1(e_1) = \pi_1(j_1(e_1), \snd{\phi}(e_1, e_2)) = \pi_1(\phi(e_1, e_2)), \] which completes the proof. \end{myproof}

\begin{mystatement}{remark}{whitney::remark} The Whitney sum is associative and commutative. \end{mystatement}

\begin{myparagraph} Alternatively, one could define the Whitney sum between $\bb_1$ and $\bb_2$ to be the induced microbundle $\ind{\Delta}(\bb_1 \cross \bb_2)$ where $\Delta$ denotes the diagonal map and $\bb_1 \cross \bb_2$ denotes the intuitive cross-product between the two microbundles. \end{myparagraph}
\begin{scope} \newcommand{\wleft}{\ind{f} (\whitney{\bb_1}{\bb_2})} \newcommand{\wright}{\whitney{\ind{f}\bb_1}{\ind{f}\bb_2}}

\begin{mystatement}{lemma}{whitney::compatibility} Let $\bb_1$ and $\bb_1$ be two microbundles over $B$ and let $f: A \to B$ be a map. The induced microbundle and the Whitney sum are compatible, that is, \begin{center} $\wleft \cong \wright$. \end{center} \end{mystatement}

\begin{myproof} The total space $E(\wleft)$ is defined as \[ \{(a, (e_1, e_2)) \in A \cross (E(\bb_1) \cross E(\bb_2)) \mid j_1(e_1) = j_2(e_2) = f(a) \} \] and $E(\wright)$ as \[ \set{((a_1, e_1), (a_2, e_2)) \in E(\ind{f}\bb_1) \cross E(\ind{f}\bb_2)}{j_1(a_1, e_1) = j(a_2, e_2)}. \]

We have a homeomorphsim $\psi: E(\wleft) \isomto E(\wright)$ given by \[ \psi(a, (e_1, e_2)) = ((a, e_1), (a, e_2)) \] with the inverse $\inv{\psi}((a, e_1), (a, e_2)) = (a, (e_1, e_2))$. Since both $\psi$ and $\inv{\psi}$ are component-wise continuous, $\psi$ is a homeomorphism.

Commutativity with the injection maps is given by \[ \psi(i_f(a)) = \psi(a, (i_1(f(a)), i_2(f(a)))) = ((a, i_1(f(a))), (a, i_2(f(a)))) = i_\oplus(a) \] and with the projection maps by \[ j_f(a, (e_1, e_2)) = a = j_\oplus((a, e_1), (a, e_2)) = j_\oplus(\psi(a, (e_1, e_2))). \] Here, $i_f$ and $j_f$ denote the injection and projection for $\wleft$ and $i_\oplus$ and $j_\oplus$ the injection and projection for $\wright$. \end{myproof} \end{scope}

\begin{myparagraph} In the remainder of this section, we show the above-mentioned theorem about the existence of an `inverse' in respect to the Whitney sum. This statement is essential for the proof of Milnors \Cref{normal::milnor}.

In order to show this, we require the following lemma, whose proof will be deferred until \myintref{chapter::suspension}.

\end{myparagraph}

\begin{mystatement}{lemma}[Bouquet Lemma]{whitney::bouquet}[59] Let $\bb$ be a microbundle over a `bouquet' of spheres $B$, meeting in a single point. Then there exists a map $r: B \to B$ such that $\whitney{\bb}{\ind{r}\bb}$ is trivial. \end{mystatement}

\begin{mystatement}{theorem}{whitney::theorem}[59] Let $\bb$ be a microbundle over a $d$-simplicial complex $B$. Then there exists a microbundle $\bn$ over $B$ such that the Whitney sum $\whitney{\bb}{\bn}$ is trivial. \end{mystatement}

\begin{myproof} We prove the theorem by induction over $d$.

(Start of induction)

A $1$-simplicial complex is just a bouquet of circles. Hence, the start of induction follows directly from \myintref{whitney::bouquet}.

(Inductive Step)

Let $B'$ be the $(d - 1)$-skeleton of $B$ and let $\bn'$ be its corresponding microbundle such that $\whitney{\bb[B']}{\bn'}$ is trivial.

\begin{steps} \item $\whitney{\bn'}{\be{B'}}$ can be extended over any $d$-simplex $\sigma$

Consider the equation \[ (\whitney{\bn'}{\be{B'}})\restr{\partial\sigma} = \whitney{\bn'\restr{\partial\sigma}}{\be{B'}\restr{\partial\sigma}} = \whitney{\bn'\restr{\partial\sigma}}{\bb[\partial\sigma]} = (\whitney{\bn'}{\bb[B']})\restr{\partial\sigma} \] in which we used \myintref{induced::simplex} for $\be{B'}\restr{\partial\sigma} = \bb[\partial\sigma]$. Since $(\whitney{\bn'}{\bb[B']})\restr{\partial\sigma}$ is trivial, the claim follows from \myintref{induced::simplex}.

\item $\whitney{\bn'}{\be{B'}}$ can be extended over $B$

One difficulty is that the individual $d$-simplices are not well-separated. To deal with this, we consider $B''$ which is defined to be $B$ with small open $d$-cells removed from every $d$-simplex. Since $B'$ is a retract of $B''$, we can extend $\whitney{\bn'}{\be{B'}}$ to a microbundle $\nu$ over $B''$.

Now we extend $\nu$ over $B$ by taking all extensions of $\nu$ over every simplex using (Step 1), and identifying its total spaces together along $E(\nu)$. Similarly, injection and projection are obtained by piecing the injection and projection maps of the individual extensions together.

We denote the resulting microbundle by $\eta$.

\item \blankbreak{} Consider the mapping cone $B \sqcup_\iota CB'$ over the inclusion $B' \incl B$. The following equation shows that $(\whitney{\bb}{\eta})\restr{B'}$ is trivial: \[ (\whitney{\bb}{\eta})\restr{B'} \cong \whitney{\bb[B']}{\eta\restr{B'}} \cong \whitney{\bb[B']}{(\whitney{\bn'}{\be{B'}})} \cong \whitney{(\whitney{\bb[B']}{\bn'})}{\be{B'}} \cong \whitney{\be{B'}}{\be{B'}} \] \myintref{induced::cone} then yields a microbundle $\xi$ over $B \sqcup_\iota CB'$ extending $\whitney{\bb}{\eta}$.

The mapping cone $B \sqcup_\iota CB'$ has the homotopy type of a bouquet of spheres, which can be seen as follows: \begin{itemize} \item A $d$-simplex is homotopic to a $d$-disc. \item A $d$-disc whose boundary is collapsed to a single point is a $d$-sphere. \item One can define a homotopy between $\iota(B')$ and the tip of the cone by traveling along $CB'$. \end{itemize}

Using \myintref{homotopy::theorem} and \myintref{whitney::bouquet}, we conclude that there exists a microbundle $\bn$ such that $(\whitney{\xi}{\bn})\restr{B}$ is trivial. The equation \[ \be{B} = (\whitney{\xi}{\bn})\restr{B} = \whitney{\xi\restr{B}}{\bn\restr{B}} = \whitney{(\whitney{\bb}{\eta})}{\bn\restr{B}} = \whitney{\bb}{(\whitney{\eta}{\bn\restr{B}})} \] completes the proof. \end{steps} \end{myproof}
% Document :: Homotopy
\section{The Homotopy Theorem}\label{chapter::homotopy}
\begin{myparagraph}
    In this section, we will prove the Homotopy Theorem,
    which is a fundamental result over microbundles.
    It states the following.
\end{myparagraph}
\begin{mystatement}{theorem}[Homotopy Theorem]{homotopy::theorem}[58] Let $\bb$ be a microbundle over $B$ and let $f, g: A \to B$ be two maps where $A$ is paracompact hausdorff. If $f$ and $g$ are homotopic, then $\ind{f}\bb$ and $\ind{g}\bb$ are isomorphic. \end{mystatement}
\begin{myparagraph}
    In order to prove this theorem,
    we introduce the concept of map-germs
    over microbundles which provide another way
    to think about isomorphy besides \myintref{microbundle::isomorphy}.
\end{myparagraph}
\subsection*{Map-Germs}\label{section::germs}{\blankbreak}
% germ
\begin{mystatement}{definition}[map-germ]{homotopy::germ}[65]
    A \defterm{map-germ} $\germdef{F}{(X, A)}{(Y, B)}$
    between topological pairs $(X, A)$ and $(Y, B)$
    is an equivalence class of maps $(X, A) \to (Y, B)$
    where $f \sim g \iff f\restr{U} = g\restr{U}$
    for an arbitrary neighborhood $U \sub X$ of $A$. 
\end{mystatement}

% composition
\begin{myparagraph}
    We can form the composition of two map-germs
    $\germdef{F}{(X, A)}{(Y, B)}$ and $\germdef{G}{(Y, B)}{(Z, C)}$
    by choosing representatives $f: U_f \to Y$ and $g: U_g \to Z$
    and defining $(f \circ g) \restr{\inv{f}(U_g)}$ to be
    a representative for $G \circ F$.
\end{myparagraph}

% homeomorphism
\begin{mystatement}{definition}[homeomorphism-germ]{homotopy::homeomorphism}[65]
    A \defterm{homeomorphism-germ} $\germdef{F}{(X, A)}{(Y, B)}$ is a map-germ
    such that there exists a representative $f: U_f \to Y$
    that maps $U_f$ homeomorphically to a neighborhood of $B$.  
\end{mystatement}

\begin{myparagraph} Let $\bb$ and $\bb'$ be two isomorphic microbundles over $B$. There exists a homeomorphism $\psi: V \isomto V'$ where $V \sub E(\bb)$ is a neighborhood of $i(B)$ and $V' \sub E(\bb')$ is a neighborhood of $i'(B)$. We can view $\psi$ as a representative for a homeomorphism-germ \[ \germdef{[\psi]}{(E, i(B))}{(E', i'(B))}. \]

Studying isomorphy between $\bb$ and $\bb'$ using map-germs is useful because we do not care what $\psi$ does on its initial domain, but only what it does on arbitrary small neighborhoods of $i(B)$. Hence, every representative of $[\psi]$ describes the `same' isomorphy between $\bb$ and $\bb'$. Now, naturally, the question arises whether the existence of a homeomorphism-germ \[ \germdef{F}{(E, i(B))}{(E', i'(B))} \] already implies that $\bb$ and $\bb'$ are isomorphic. The answer is generally no, because isomorphy between microbundles additionally requires the homeomorphism to commute with the injection and projection maps. Hence, we need an extra condition (`fiber-preservation') for this implication to be true. This justifies the following definition.

Let $\germdef{J}{(E(\bb), i(B))}{(B, B)}$ and $\germdef{J'}{(E(\bb'), i(B))}{(B, B)}$ denote the map-germs represented by the projections of $\bb$ and $\bb'$. \end{myparagraph}

\begin{mystatement}{definition}[isomorphism-germ]{homotopy::isomorphism}[65] An \defterm{isomorphism-germ} between $\bb$ and $\bb'$ is a homeomorphism-germ \[ \germdef{F}{(E(\bb), B)}{(E(\bb'), B)} \] which is \defterm{fiber-preserving}, that is, $J' \circ F = J$. \end{mystatement}

\begin{mystatement}{remark}{homotopy::isomorphismremark}[65] There exists an isomorphism-germ between $\bb$ and $\bb'$ if and only if $\bb$ and $\bb'$ are isomorphic. \end{mystatement}
% generalize
\begin{myparagraph}
    We can take this even further by dropping the assumption
    that the base spaces of the two microbundles equal.
    Note that in this case no comparison to isomorphy can be drawn,
    because we haven't defined isomorphy between microbundles over different base spaces.
\end{myparagraph}

% definition
\begin{mystatement}{definition}[bundle-germ]{homotopy::bundle}[66]
    Let $\bb$ and $\bb'$ be two microbundles over $B$ and $B'$
    with the same fiber dimension.
    A \defterm{bundle-germ} $\germdef{F}{\bb}{\bb'}$ is a map-germ
    \[ \germdef{F}{(E(\bb), B)}{(E(\bb'), B')} \]
    such that there exists a representative $f: U_f \to E(\bb')$
    that maps each fiber $\inv{j}(b)$ injectively to a fiber $\inv{j'}(b')$.
\end{mystatement}

% diagram
\begin{myparagraph}
    For a bundle-germ $\germdef{F}{\bb}{\bb'}$, the following diagram commutes:
    \begin{center}
        \begin{tikzcd}
            (E(\mathfrak{b}), B) \arrow[r, Rightarrow, "F"] \ar[d, "i"] & (E(\mathfrak{b}'), B') \ar[d, "i'"] \\
            B \ar[r, "F\vert_B"] & B' \\
        \end{tikzcd}
    \end{center}
    We say $F$ \defterm{is covered by} $F\restr{B}$.
    % intentionally left blank
    
\end{myparagraph}
% foreword williamson
\begin{myparagraph}
    The bundle-germ is indeed a generalization for the isomorphism-germ,
    as the following proposition shows.
\end{myparagraph}

% williamson
\begin{mystatement}{proposition}[Williamson]{homotopy::williamson}[66]
    Let $\bb$ and $\bb'$ be two microbundles over $B$.
    A bundle-germ $\germdef{F}{\bb}{\bb'}$ covering
    the identity map is an isomorphism-germ.
\end{mystatement}

% foreword
\begin{myparagraph}
    The following lemma will be necessary for the proof of the proposition.    
\end{myparagraph}

% lemma
\begin{mystatement}{lemma}{homotopy::ball}
    If a homeomorphism $f: \clball[2] \isomto V \sub \R^n$ satisfies
    \[ \abs{f(x) - x} < 1, \forall x \in \clball[2], \]
    then $\clball[1] \sub V$.
\end{mystatement}

% proof lemma
\begin{myproof}[of the lemma]
    We provide a proof by contradiction.

    Suppose $v \in \clball[1] - V$.
    Let $[f(0), v]$ denote the line-segment
    \[ \set{\lambda f(0) + (1 - \lambda) v}{\lambda \in \I} \sub \R^n. \]
    The intersection $[f(0), v] \cap V$ is compact
    since both $[f(0), v]$ and $V$ are compact.
    Hence, the intersection has a maximum $v'$ when ordered via $\lambda$.
    Note that the intersection is non-empty since $f(0) \in V$
    and that $v' \neq v$ since $v \notin V$ by definition.

    The maximum $v'$ is contained in $\partial V$:
    \begin{enumerate}
        \item $v' \in [f(0), v] \cap V \sub V = \overline{V}$
        \item $v' \notin \inner{V}$, because otherwise
        $\ball[\eps][v'] \sub \inner{V}$ for some $\eps > 0$
        which contradicts with $v'$ being the maximum in $[f(0), v] \cap V$.
    \end{enumerate}

    From $\abs{f(0) - 0} < 1$ and $v \in \clball[1]$ it follows that $\abs{v'} < 1$. 

    This leads to a contradiction, because
    \[ \inv{f}(v') \in \partial\clball[2] \implies \abs{\inv{f}(v') - v'} = 2 - \abs{v'} > 2 - 1 = 1. \]
\end{myproof}
% proof
\begin{myproof}[of the proposition]
    Let $f: U_f \to E(\bb')$ be a representative for $F$.
    First we assume a special case. Then we use this result show that $f$
    is open in some neighborhood of every $v \in U_f$, hence $f$ being open.
    \begin{steps}
        % special case
        \item Let $f$ map from $B \cross \R^n$ to $B \cross \R^n$
        
        Since $F$ covers the identity, $f$ is of the form
        \[ f(b, x) = (b, g_b(x)) \]
        where $g_b: \R^n \to \R^n$ are individual maps.
        Since the $g_b$ are continuous and injective,
        it follows from the Invariance of Domain Theorem (see \cite[cor.19.9]{brendon}) that the $g_b$ are open maps.
        
        Let $(b_0, x_0) \in B \cross \R^n$ and let $\eps > 0$.
        Since $g_{b_0}$ is an open map, there exists a $\delta > 0$ such that
        $\clball[2\delta][x_1] \sub g_{b_0}(\clball[\eps][x_0])$ where $x_1 = g_{b_0}(x_0)$.

        % estimation
        We claim that there exists a neighborhood $V \sub B$ of $b_0$ such that
        \[ \abs{g_b(x) - g_{b_0}(x)} < \delta \]
        for every $b \in V$ and $x \in \clball[\eps][x_0]$.
        
        To show that, consider $\phi(b, x) = g_b(x) - g_{b_0}(x)$.
        The open set $\inv{\phi}(\ball[\delta])$ is a neighborhood of $\{b_0\} \cross \R^n$ since $\phi(b_0, x) = 0$.
        Hence, there exist open subsets $V_x \sub B$ and $W_x \sub \R^n$ such that
        \[ \bigcup_{x \in \clball[\eps][x_0]} V_x \cross W_x \sub \inv{\phi}(\clball[\delta]) \]
        and $x \in W_x$.
        Since $\clball[\eps][x_0]$ is compact, there exist $x_1, \dots, x_n \in \clball[\eps][x_0]$ with $\clball[\eps][x_0] \sub \bigcup_{i = 1}^{n} V_{x_i}$.
        The claim follows with $V = V_{x_1} \cap \dots \cap V_{x_n}$
        which is open by forming the intersection over finitely many open sets.

        % apply lemma
        Now we want to apply the previous lemma.

        Consider the homeomorphism
        $(g_b \circ \inv{g_{b_0}})\restr{\clball[2\delta][x_1]}$
        for an arbitrary $b \in V$.
        Together with
        \[ \clball[2\delta][x_1] \sub g_{b_0}(\clball[\eps][x_0]) \implies \inv{g_{b_0}}(\clball[2\delta][x_1]) \sub \clball[\eps][x_0], \]
        we conclude from the above that
        \[ \abs{(g_b \circ \inv{g_{b_0}})(x) - x} < \delta, \forall x \in \clball[2\delta][x_1]. \]
        It follows that, by translation and scaling, $g_b \circ \inv{g_{b_0}}\restr{\clball[2\delta][x_1]}$ satisfies the conditions of \myintref{homotopy::ball}.
        Therefore, $\clball[\delta][x_1] \sub (g_b \circ \inv{g_{b_0}})(\clball[2\delta][x_0])$ and hence $\clball[\delta][x_1] \sub g_b(\clball[\eps][x_0])$.
        % conclusion
        From
        \[ V \cross \clball[\delta][x_1] \sub g(V \cross \clball[\eps][x_0]) \]
        it follows that $f$ is an open map.

        % general case
        \item Gluing together $f: U_f \to E(\bb')$ along local trivializations

        For an arbitrary $b \in B$, choose local trivializations $(U, V, \phi)$ and $(U', V', \phi')$ of $b$ in $\bb$ and $\bb'$.
        Without loss of generality,
        we may assume that $U = U'$ by choosing $V = \inv{\phi}(U \cap U')$ and $V' = \inv{\phi'}(U \cap U')$
        and restricting $\phi$ and $\phi'$ accordingly if necessary. 
        
        First, we restrict $f$ to $V \cap \inv{f}(V')$.
        Since $V \cap \inv{f}(V')$ is an open neighborhood of $i(b)$, we can choose
        an open neighborhood $U_b \sub U$ of $i(b)$ and $\eps_b > 0$ such that $\inv{\phi}(U_b \cross \ball[\eps_b]) \sub V \cap \inv{f}(V')$.
        
        We define a map $U_b \cross \R^n \to U_b \cross \R^n$ given by
        \[ U_b \cross \R^n \cong  U_b \cross \ball[\eps_b] \xto{\phi \circ f \circ \inv{\phi}} U_b \cross \R^n \]
        that is injective and fiber-preserving, and hence an open map (Step 1).
        It follows that $f: \inv{\phi}(U_b \cross \ball[\eps_b]) \to V'$
        must be an open map, as the other composing maps are homeomorphisms.

        We conclude from 
        \[ f = \bigcup_{b \in B}f\restr{\inv{\phi}(U_b \cross \ball[\eps_b])} \]
        that $f$ is an open map.
    \end{steps}
    This completes the proof.
\end{myproof}

% foreword corollary
\begin{myparagraph}
    We can easily generalize this result to bundle-germs between microbundles over different base spaces.
\end{myparagraph}

% corollary
\begin{mystatement}{corollary}{homotopy::corollary}[67]
    If a map $g: B \to B'$ is covered by a bundle-germ $\germdef{F}{\bb}{\bb'}$, then $\bb$ is isomorphic to the induced microbundle $\ind{g}\bb'$.
\end{mystatement}

% proof corollary
\begin{myproof}
    Let $f: U_f \to E'$ be a representative map for $F$.
    We define $\germdef{F'}{\bb}{\ind{g}\bb'}$ by the representative
    \[ f': U_f \to E(\ind{g}\bb') \twith f'(e) = (j(e), f(e)). \]
    Every $f'(e)$ lies in $E(\ind{g}\bb')$ because
    $g(j(e)) = j'(f(e))$
    as we can see from the commutative diagram for bundle-germs.

    The germ $F'$ is a bundle-germ covering the identity because
    \[ j(e) = j_g'(j(e), f(e)) = j_g'(f'(e)) \]
    and because $f'$ is injective ($f$ is injective).
    Applying the previous proposition on $F'$ proves the claim.
\end{myproof}
\section{Homotopy and Microbundles}
% foreword
% TODO

% definition
\definition{\parttitle{map-germ}} \\
Let $(X, A)$ and $(Y, B)$ be two topological pairs.
A \defterm{map-germ} $F: (X, A) \Rightarrow (Y, B)$ is an equivalence class of maps $f: (X, A) \to (Y, B)$, where
$f \sim g :\iff f\restr{U} = g\restr{U}$ for some neighborhood $U \sub X$ of $A$. 

% afterword
ad

% remark
\begin{remark}
The composition of two map-germs $(X, A) \overset{F}{\Rightarrow} (Y, B) \overset{G}{\Rightarrow} (Z, C)$ is well defined.
\end{remark}
% definition
\definition{\parttitle{isomorphism-germ}} \\
% TODO
% definition
\definition{\parttitle{bundle map-germ}} \\
% TODO
% foreword williamson
\begin{myparagraph}
    The bundle-germ is indeed a generalization of the isomorphism germ, as the following proposition illustrates.
\end{myparagraph}

% williamson
\begin{myproposition}[Williamson]{homotopy::williamson}
    Let $\bb$ and $\bb'$ be two microbundles over $B$.
    A bundle-germ $\bgerm{F}{\bb}{\bb'}$ covering the identity map is an isomorphism-germ.
\end{myproposition}

% foreword lemma
\begin{myparagraph}
    First, however, we show a lemma that helps us to prove the proposition.
\end{myparagraph}

% lemma
\begin{mylemma}{homotopy::ball}
    If a homeomorphism $\phi: \clball[2] \isomto \phi(\R^n) \sub \R^n$ satisfies
    \[ \abs{\phi(x) - x} < 1, \forall x \in \clball[2] \]
    then $\clball[1] \sub \phi(\clball[2])$.
\end{mylemma}

% proof of lemma
\begin{myproof}[of the lemma]
    % TODO: Check proof
    Consider $\phi(2S^n)$ where $2S^n$ denotes the $n$-sphere of radius $2$.
    The condition for $\phi$ yields $1 < \abs{\phi(s)}, \forall s \in 2S^n$.
    Since $\clball[2]$ has trivial homology groups which are preserved under homeomorphisms,
    $\phi(\clball[2])$ must have trivial homology groups as well.

    From this we can conclude that $\clball[1]$ must be contained in $\phi(\clball[2])$,
    because otherwise ``holes'' would form which would result in non-trivial homology groups of $\phi(\clball[2])$.
    % Consider the subspace $\R^n - 2S^n$ where $2S^n$ denotes the $n$-sphere of radius $2$.
    % This subspace consists of the two path-components $\ball[2]$ and $\R^n - \clball[2]$.
    % Since homeomorphisms preserve path-components, we know that $\phi(\R^n - 2S^n)$ consists of the two path-components $\phi(\ball[2])$ and $\phi(\R^n - \clball[2])$.
    % Now, applying the requirement yields $\abs{\phi(0)} < 1$ and $1 < \abs{\phi(s)}$ for every $s \in 2S^n$.
    % Hence, there exists a path between $\phi(0)$ and any $x \in \clball[1]$ in $\phi(\R^n - 2S^n)$, e.g. a straight line.
    % Therefore, every $x \in \clball[1]$ is contained in the same path-component of $\phi(\R^n - 2S^n)$.
    % The statement follows from $\phi(0) \in \phi(\clball[2])$.
\end{myproof}

% proof of williamson
\begin{myproof}[of the proposition]
    Let $f$ be a representative for $F$.
    First we assume a special and then generalize the result to show the proposition.
    \begin{enumerate}
        % special case
        \item Let $f$ map from $B \times \R^n$ to $B \times \R^n$:
        
        Since $F$ covers the identity, $f$ is of the form
        \[ f(b, x) = (b, g_b(x)) \]
        where $g_b: \R^n \to \R^n$ are individual maps.
        Since the $g_b$ are continuous and injective, it follows from the \extref{domain invariance theorem} that the $g_b$ are open maps.
        
        Let $(b_0, x_0) \in B \times \R^n$ and let $\varepsilon > 0$.
        Since $g_{b_0}$ is an open map, there exists a $\delta > 0$ such that
        $\clball[2\delta][x_1] \sub g_{b_0}(\clball[\varepsilon][x_0])$ where $x_1 := g_{b_0}(x_0)$.

        % estimation
        We claim that there exists a neighborhood $V \sub B$ of $b_0$ such that
        \[ \abs{g_b(x) - g_{b_0}(x)} < \delta \]
        for every $b \in V$ and $x \in \clball[\varepsilon][x_0]$.
        
        To show that, consider $\phi(b, x) := g_b(x) - g_{b_0}(x)$.
        The open set $\phi^{-1}(\ball[\delta])$ is a neighborhood of $\{b_0\} \times \R^n$ since $\phi(b_0, x) = 0$.
        Hence, there exist open subsets $V_x \sub B$ and $W_x \sub \R^n$ such that
        \[ \bigcup_{x \in \clball[\varepsilon][x_0]} V_x \times W_x \sub \inv{\phi}(\clball[\delta]) \]
        and $x \in W_x$.
        Since $\clball[\varepsilon][x_0]$ is compact, there exist $x_1, \dots, x_n \in \clball[\varepsilon][x_0]$ with $\clball[\varepsilon][x_0] \sub \bigcup_{i = 1}^{n} V_{x_i}$.
        The claim follows with $V := V_{x_1} \cap \dots \cap V_{x_n}$ which is open by forming the intersection over finitely many open sets.

        % apply lemma
        Now we want to apply the previous lemma:

        Consider the homeomorphism
        \[ \clball[2\delta][x_1] \isomto g_b \circ \inv{g_{b_0}}(\clball[2\delta][x_1]) \]
        for an arbitrary $b \in V$.
        Since
        \[ \clball[2\delta][x_1] \sub g_{b_0}(\clball[\varepsilon][x_0]) \implies g_{b_0}^{-1}(\clball[2\delta][x_1]) \sub \clball[\varepsilon][x_0] \]
        we conclude from the above that
        \[ \abs{(g_b \circ g_{b_0}^{-1})(x) - x} < \delta, \forall x \in \clball[2\delta][x_1] \]
        It follows that, by translation and scaling, $g_b \circ g_{b_0}^{-1}\restr{\clball[2\delta][x_1]}$ satisfies the conditions of \intref{homotopy::ball}.
        Therefore, $\clball[\delta][x_1] \sub (g_b \circ g_{b_0}^{-1})(\clball[2\delta][x_0])$ and so $\clball[\delta][x_1] \sub g_b(\clball[\varepsilon][x_0])$.

        % conclusion
        From
        \[ V \times \clball[\delta][x_1] \sub g(V \times \clball[\varepsilon][x_0]) \]
        it follows that $f$ is an open map.

        % general case
        \item Glue together $f: U_f \to E(\bb')$ along local trivializations:

        For an arbitrary $b \in B$, choose local trivializations $(U, V, \phi)$ and $(U', V', \phi')$ of $b$ in $\bb$ and $\bb'$.
        Without loss of generality we may assume that $U = U'$ because otherwise we can choose $V = \inv{\phi}(U \cap U')$ and $V' = \inv{\phi'}(U \cap U')$
        and restrict $\phi$ and $\phi'$ accordingly.   
        
        First, we restrict $f$ to $V \cap \inv{f}(V')$. Since $V \cap \inv{f}(V')$ is an open neighborhood of $i(b)$ and contained in $V$, we can choose
        an open neighborhood $\tilde{U} \sub U$ of $i(b)$ and $\varepsilon > 0$ such that $\inv{\phi}(\tilde{U} \times \ball[\varepsilon]) \sub V \cap \inv{f}(V')$.
        
        This yields a map $U' \times \R^n \to U' \times \R^n$ with
        \[ \tilde{U} \times \R^n \cong  \tilde{U} \times \ball[\varepsilon] \isomto \inv{\phi}(\tilde{U} \times \ball[\varepsilon]) \xto{f} U' \times \R^n \sub U \times \R^n \]
        that is injective and fibre-preserving and therefore an open map (apply 1.).
        It follows that $f: \inv{\phi}(\tilde{U} \times \ball[\varepsilon]) \to V'$ must be an open map as well since the other composing maps are homeomorphisms.

        By glueing the $\inv{\phi}(\tilde{U} \times \ball[\varepsilon])$ together over all $b \in B$, we see that $f$ is an open map.
    \end{enumerate}
\end{myproof}

% foreword corollary
\begin{myparagraph}
    We can easily generalize this to bundle-germs between microbundles over different base spaces:
\end{myparagraph}

% corollary
\begin{mycorollary}{homotopy::corollary}
    If a map $g: B \to B'$ is covered by a bundle-germ $F: \bb \double \bb'$, then $\bb$ is isomorphic to the induced microbundle $g^*\bb'$.
\end{mycorollary}
% proof
\begin{myproof}
    Let $f: U_f \to E'$ be a representative map for $F$.
    We define $F': \bb \double g^*\bb'$ by a representative
    \[ f': U_f \to E(g^*\bb') \twith f'(e) := (j(e), f(e)). \]
    Every $f'(e)$ actually lies in $E(g^*\bb')$ because
    \[ g(j(e)) = j'(f(e)) \]
    as we can see from the commutative diagram for bundle-germs.
    The germ $F'$ is an isomorphism-germ because $F$ is an isomorphism-germ.
    Applying the previous proposition on $F'$ proves the claim.
    % TODO: why is F' an isomorphims-germ?
\end{myproof}
% foreword
\begin{myparagraph}
    The following lemma will allow us to glue
    together bundle-germs over locally finite, closed domains if they agree on their intersection.
\end{myparagraph}

% lemma
\begin{mystatement}{lemma}{homotopy::glueing}[67]
    % conditions
    Let $\bb$ be a microbundle over $B$ and $\coll{B_\alpha}$
    a locally finite collection of closed sets covering $B$.
    Additionally, we are given a collection
    of bundle-germs $\germdef{F_\alpha}{\bb[B_\alpha]}{\bb'}$
    such that $F_\alpha = F_\beta$ on $\bb[B_\alpha \cap B_\beta]$.
    % conclusion
    Then there exists a bundle-germ $\germdef{F}{\bb}{\bb'}$ extending $F_\alpha$,
    that is $F$ and $F_\alpha$ agree on a sufficiently small neighborhood of $i(B_\alpha)$.
\end{mystatement}

% proof
\begin{myproof}
    Choose representative maps $f_\alpha: U_\alpha \to E(\bb')$ for $F_\alpha$ such that the $U_\alpha$ are open.
    For every $\alpha$ and $\beta$, choose an open neighborhood $U_{\alpha\beta}$ of $i(B_\alpha \cap B_\beta)$ on which the representative maps $f_\alpha$ and $f_\beta$ agree.
    Now consider
    \[ U := \set{ e \in E}{j(e) \in B_\alpha \cap B_\beta \implies e \in U_{\alpha\beta}} \]
    which is an open neighborhood of $i(B)$ because
    \begin{enumerate}
        % open
        \item $U$ is open:

        Let $e \in U$ be arbitrary.

        Since $\coll{B_\alpha}$ is locally finite,
        there exists an open neighborhood $V$ of $j(e)$
        that intersects with only finitely many $B_{\alpha_1}, \dots, B_{\alpha_n}$.
        Note that with $e \in U$ it follows that $e \in U_{\alpha_i\alpha_j}, \forall 1 \le i, j \le n$.

        Now we have an open neighborhood of $e$
        \[ \bigcap_{1 \le i, j \le n} U_{\alpha_i\alpha_j} \cap \inv{j}(V) \]
        which is contained in $U$ by construction.

        % subset
        \item $i(B) \sub U$:
        
        This follows from
        \[ j(i(b)) = b \in B_\alpha \cap B_\beta \implies i(b) \in i(B_\alpha \cap B_\beta) \sub U_{\alpha\beta}. \]
    \end{enumerate}
    Now we can define $f: U \to E(\bb')$ in the obvious way
    \[ f(u \in U_{\alpha\beta}) := f_\alpha(u) = f_\beta(u) \]
    which is continuous according to the \extref{glueing lemma}.
    In particular, $f$ agrees with $f_\alpha$ on $U_{\alpha\alpha}$
    and hence $f$ is a representative for a bundle-germ $\germdef{F}{\bb}{\bb'}$ extending $\coll{F_\alpha}$.
\end{myproof}
\begin{file}
% local
\newcommand{\A}[1][] {
    A_\alpha#1
}

% local
\newcommand{\bbleft} {
    \bb\restr{B \times{} [0, \half]}
}
\newcommand{\bbright} {
    \bb\restr{B \times{} [\half, 1]}
}
\newcommand{\bbhalf} {
    \bb\restr{B \times{} \{ \half \}}
}

% statement
\lemma{\parttitle{piecewise triviality}} \\
Let $\bb$ be a microbundle over $B \times [0, 1]$ such that both $\bbleft$ and $\bbright$ are trivial.
Then $\bb$ itself is already trivial.
% proof
\begin{proof}
Since $\bbright$ is trivial, we can extend the identity bundle map-germ on $\bbhalf$ to $\bbright \double \bbhalf$.
Using the previous lemma, we can piece this together with the identity bundle map-germ on $\bbleft$ to
\[ \bb \double \bbleft \]

From the corollary it follows that $\bb \cong \bbleft$.
\end{proof}

% statement
\lemma{\parttitle{}} \\
Let $\bb$ be a microbundle over $B \times [0, 1]$.
Every $b \in B$ has a neighborhood $V$ where $\bb\restr{V \times [0, 1]}$ is trivial.
%proof
\begin{proof}
Let $b \in B$.
For every $t \in [0, 1]$, choose a neighborhood $U_t := V_t \times (t - \varepsilon_t, t + \varepsilon_t)$ of $(b, t)$ such that $\bb\restr{U_t}$ is trivial.
Since $\{b\} \times [0, 1]$ is compact, we can choose a finite covering of the $U_t$ and define $V$ to be the intersection of the corresponding $V_t$.
Then there exists a subdivision $0 = t_0 < \cdots < t_k = 1$ where the $\bb\restr{V \times [t_{i}, t_{i + 1}]}$ are trivial.
Iteratively applying the previous lemma, it follows that $\bb\restr{V \times [0, 1]}$ is trivial.
\end{proof}

% lemma
\lemma{\parttitle{}} \\
Let $\bb$ be a microbundle over $B \times [0, 1]$ where $B$ is paracompact.
Then there exists a bundle map-germ $F: \bb \to \bb\restr{B \times \{1\}}$
covering the standard retraction $r: B \times [0, 1] \to B \times \{1\}$ .
% proof
\begin{proof}
First, we assume a locally finite covering $\{V_\alpha\}$ of closed sets where $\bb\restr{V_\alpha \times [0, 1]}$ is trivial.
The existence of such a covering is justified by the previous lemmas.
Since $B$ is paracompact, we can choose bump functions 
\[ \lambda_\alpha: B \to [0, 1] \]
with $\text{supp}(\lambda_\alpha) \sub V_\alpha$.
Additionally, we assume that 
\[ \max_{\alpha}(\lambda_\alpha(b)) = 1, \forall b \in B \]
Now we define a retraction $r_\alpha: B \times [0, 1] \to B \times [0, 1]$ with
\[ r_\alpha(b, t) := (b, \max(t, \lambda_\alpha(b))) \]

We construct bundle map-germs $R_\alpha: \bb \to \bb$ covering $r_\alpha$.
We can divide $B \times [0, 1]$ into $A_\alpha := \text{supp}(\lambda_\alpha) \times [0, 1]$ and $A'_\alpha := \{(b, t) \mid t \ge \lambda_\alpha(b)\}$.
Since $A_\alpha \sub V_\alpha \times [0, 1]$, $\bb\restr{A_\alpha}$ is trivial.
It follow that the identity bundle germ on $\bb\restr{\A \cap \A[']}$ can be extended to a bundle germ $\bb\restr{\A} \double \bb\restr{\A \cap \A[']}$.
Piecing this together with the identity bundle germ $\bb\restr{\A[']}$, we obtain the desired bundle germ $R_\alpha$.

Applying the well-ordering theorem, which is equivalent to the axiom of choice, we can assume an ordering of $\{ V_\alpha \}$.
Let $\{B_\beta\}$ be a locally finite covering of $B$ with closed sets where $B_\beta$ intersects only $V_{\alpha_1} < \cdots < V_{\alpha_k}$ a finite collection.
Now the composition $R_{\alpha_1} \circ \ldots \circ R_{\alpha_k}$ restricts to a bundle germ $R(\beta): \bb\restr{B_\beta} \times [0, 1] \double \bb\restr{B_\beta} \times [1]$.
Pieced together with the previous lemma, we obtain $R: \bb \times [0, 1]\to \bb \times [1]$ which concludes the proof.
\end{proof}

% theorem
\theorem{\parttitle{Homotopy Theorem}} \\
Let $\bb$ be a microbundle of $B$ and $f, g: A \to B$ be two maps.
\[ f \simeq g \implies f^*\bb \cong g^*\bb \]
% proof
\begin{proof}
Let $H: A \times [0, 1] \to B$ be a homtopy between $f$ and $g$.
By the previous lemma, there exists a bundle germ $R: H^*\bb \double H^*\bb\restr{B \times [1]}$ covering the standard retraction $B \times [0, 1] \to B \times [1]$.
From the composition
\[ f^*\bb \sub H^*\bb \double_R H^*\bb\restr{B \times [1]} = g^*\bb \]
we obtain an isomorphism germ $f^*\bb \double g^*\bb$.
It follow that $f^*\bb \cong g^*\bb$.
\end{proof}

\end{file}
% Docuemnt :: Suspension
\section{Rooted Microbundles and the Bouquet Lemma}\label{chapter::suspension}
\begin{myparagraph}
    In this section,
    we provide a proof for the Bouquet Lemma stated in \myintref{section::whitney}.
    To this end, we introduce the concept of `rooted microbundles',
    which allows us to define the wedge sum of two microbundles in a precise manner.
    Additionally, we show a version of the Homotopy Theorem
    that is compatible with rooted-microbundles.

    Throughout this section,
    we assume that every topological space is equipped with an arbitrary base point
    which we will denote with subscript $0$.
\end{myparagraph}
\subsection*{Rooted Microbundles}\label{section::rooted}
% definition
\begin{mystatement}{definition}{suspension::rooted}[69]
    % rooted
    A \defterm{rooted microbundle} $\bb$ over $B$ is a microbundle
    over $B$ together with an isomorphism-germ
    \[ \germdef{R}{\bbb}{\be{b_0}}. \]
    % isomorphy
    Two rooted microbundles $\bb$ and $\bb'$ over $B$ are \defterm{rooted isomorphic}
    if there exists an isomorphism-germ $\germ{\bb}{\bb'}$ extending
    \[ \germdef{\inv{R'} \circ R}{\bbb}{\bb'\restr{b_0}}. \]
\end{mystatement}

% remark
\begin{mystatement}{remark}{suspension::rooting}
    One can always define a rooting for a given microbundle 
    by choosing a local trivialization in the base point and restricting
    it to the fiber of $b_0$.
\end{mystatement}
\begin{mystatement}{definition}{suspension::induced}[57] Let $\bb$ be a rooted microbundle over $B$ and let $f: A \to B$ be a based map. We equip the induced microbundle $\ind{f}\bb$ with the rooting \[ \germdef{R_f}{E(\ind{f}\bb[a_0]) = a_0 \cross E(\bb[b_0])}{e^n_{a_0}} \] that coincides with $R$ if we consider $a_0 \cross E(\bb[b_0]) = E(\bb[b_0])$ and $e^n_{a_0} = e^n_{b_0}$. \end{mystatement}

\begin{myparagraph} The total space $E(\ind{f}\bb[a_0])$ is the same as $a_0 \cross E(\bbb)$ because \[ E(\ind{f}\bb[a_0]) = \set{(a, e) \in A \cross E(\bb)}{a = a_0 \tand f(a) = b_0 = j(e)} \] \[ = a_0 \cross \set{e \in E(\bb)}{j(e) = b_0} = a_0 \cross E(\bbb). \]

\end{myparagraph}
% theorem
\begin{mytheorem}[Rooted Homotopy Theorem]{suspension::homotopy}
    Let $\bb$ be a rooted microbundle over $B$ and $f, g: A \to B$ be two based maps.
    If there exists a homotopy $H: \cyl{A} \to B$ between $f$ and $g$ that leaves the base point fixed,
    then the two rooted microbundles $\ind{f}\bb$ and $\ind{g}\bb$ are isomorphic.
\end{mytheorem}

% foreword lemma
\begin{myparagraph}
    We need to show a rooted version of \intref{homotopy::lemma2}.
    Before we prove the lemma, note that 
    \[ E(\ind{H}\bb\restr{\cyl{a_0}}) \]
    is just
    \[ \set{e \in E(\ind{H}\bb)}{j(e) \in \cyl{a_0}} \]
    \[ = \set{ (a, t, e) \in \cyl{A} \times E(\bb)}{a = a_0 \land H(a, t) = j(e)} \]
    \[ = \cyl{a_0} \times E(\bbb). \]
    Based on this, we can define an isomorphism-germ
    \[ \germdef{\bar{R}}{\ind{H}\bb\restr{\cyl{a_0}}}{\be_{\cyl{a_0}}} \]
    via a representative
    \[ \bar{r}: \cyl{a_0} \times V \to \cyl{a_0} \times \R^n \]
    with
    \[ \bar{r}(a_0, t, v) = (a_0, t, \snd{r}(v) )\]
    where $r: V \to b_0 \times \R^n$ is a representative for $R$.
    The representative $\bar{r}$ is a homoemorphism on its image
    because it is a product of the identity and $r$, which are both homoemorphisms on their image. 
\end{myparagraph}

% lemma
\begin{mylemma}{suspension::sharper}
    Let $\bb$ be a rooted microbundle over $B$ and let $H: \cyl{A} \to B$ be a map that leaves the base point fixed.
    There exists a neighborhood $V$ of $a_0$ with an isomorphism-germ
    \[ \germ{\ind{H}\bb\restr{\cyl{V}}}{\be_{\cyl{V}}} \]
    extending $\bar{R}$ (as defined above).
\end{mylemma}

% proof lemma
\begin{myproof}
    By applying \intref{homotopy::lemma2}, it follows that there exists an isomorphism-germ
    \[ \germdef{Q}{\ind{H}\bb\restr{\cyl{V}}}{\be_{\cyl{V}}} \]
    for a sufficiently small neighborhood $V$ of $a_0$.

    Now consider
    \[ \germdef{Q \circ \inv{\bar{R}}}{\be_{\cyl{a_0}}}{\be_{\cyl{a_0}}}. \]
    Similarly to the construction of $\bar{R}$ we can construct an isomorphism-germ
    \[ \germdef{P}{\be_{\cyl{V}}}{\be_{\cyl{V}}} \]
    extending $Q \circ \inv{\bar{R}}$ represented by
    \[ p(v, t, x) = (v, q(a_0, t, x)) \]
    where $q$ is a representative for $Q \circ \inv{\bar{R}}$.

    Restricted to $\cyl{a_0}$, $P$ agrees with $Q \times \inv{\bar{R}}$ and thus
    \[ \inv{P} \circ Q = (\bar{R} \circ \inv{Q}) \circ Q = \bar{R} \]
    Since $P$ and $Q$ are both isomorphism-germs, $\inv{P} \circ Q$ is an isomorphism-germ as well.
    Therefore, $\inv{P} \circ Q$ suffices our requirements which concludes the proof.
\end{myproof}

% proof theorem
\begin{myproof}[of the Rooted Homotopy Theorem]
    Follow the steps for proving the initial Homotopy Theorem,
    however using \intref{suspension::sharper} instead of \intref{homotopy::lemma2}.
    % TODO
\end{myproof}
% foreword
\begin{myparagraph}
    Now that we introduced rooted-microbundles,
    we are able to define the wedge sum.
    As we will see in the subsequent proof,
    the definition of the wedge sum
    it is necessary to have a fixed rooting given because otherwise
    one would have to choose a rooting which the resuling microbundle depends on,
    hence not being well-defined.

    Given a quotient space $A \sqcup B / \sim$ and maps $f: A \to C$ and $g: B \to C$, we define
    \[ f \cup g: (A \sqcup B / \sim) \to C \twith \]
    \[ x \mapsto \begin{cases} f(x) & \text{if } x \in A \\ g(x) &\text{if } x \in B \end{cases}.\]
    Clearly, this map is only well-defined if $a \sim b \implies f(a) = g(b)$.
\end{myparagraph}

% definition
\begin{mystatement}{definition}{suspension::wedge}[70] % ? remove citation
    Let $\ba$ and $\bb$ be two rooted microbundles over $A$ and $B$.
    % microbundle
    The \defterm{wedge sum} $\ba \vee \bb$ of $\ba$ and $\bb$ is a microbundle
    \[ A \vee B \xto{i_a \cup i_b} E(\ba \vee \bb) \xto{j_a \cup j_b} A \vee B \]
    with the total space defined as
    \[ (E(\ba) \sqcup E(\bb)) / f(e_a) \sim e_a \]
    where $f: W_a \isomto W_b$ is a representative for $\inv{R_b} \circ R_a$.
    
    % rooting
    We equip $\ba \vee \bb$ with a rooting
    \[ \germdef{R}{E((\ba \vee \bb)\restr{a_0})}{\be{a_0}} \]
    represented by any representative for $R_a$ (or $R_b$).
\end{mystatement}

% proof
\begin{myproof}[that $\ba \vee \bb$ is a rooted microbundle]
    Let $f: W_a \isomto W_b$ be a representative for $\inv{R_b} \circ R_a$.

    % rooted microbundle
    \begin{sectionize}
        \item $\ba \vee \bb$ is a rooted microbundle
        \begin{itemize}
            % injection
            \item The injection map $i_a \cup i_b$ is well-defined because 
            \[ [i(a_0)] = [i_a(a_0)] = [f(i_a(a_0))] = [i_b(b_0)] = [i(b_0)] \]
            and continuous since both $i_a$ and $i_b$ are continuous.
            % projection
            \item The projection map $j_a \cup j_b$ is well-defined because
            \[ \forall e \in W_a: [j(e)] = [j_a(e)] = [a_0] = [b_0] = [j_b(f(e))] = [j(f(e))] \]
            and continuous since both $j_a$ and $j_b$ are continuous.
            % compatibility
            \item The composition $j \circ i$ is the identity because
            \[ \forall a \in A: j(i(a)) = j(i_a(a)) = j_a(i_a(a)) = a \]
            since $j_a \circ i_a = id_A$ (symmetrical for $B$).
        \end{itemize}
        % local triviality
        It remains to be shown that $\ba \vee \bb$ is locally trivial.

        Let $x \in A \vee B$.
        For reasons of symmetry, we can assume that $x \in A$.
        \begin{caselist}
            % trivial case
            \item $x \neq a_0$
            
            % construct
            Choose a local trivialization $(U, V, \phi)$ for $x$ in $\ba$.
            Without loss of generality,
            we can assume that $U \cap B = \emptyset$ by subtracting
            $\{a_0\}$ from $U$ if necessary.
            Note that $\{a_0\}$ is closed since $A$ is hausdorff.
            
            % show
            Now we can simply use this trivialization for $\ba \vee \bb$, because
            $U \sub A$ is open in $A \vee B$ and
            $V \sub E(\ba)$ is open in $E(\ba \vee \bb)$.
            Furthermore, since $i$ and $j$ reduce to $i_a$ and $j_a$,
            it follows that $\phi$ commutes with $i$ and $id \cross 0$
            as well as with $j$ and $\pi_1$.

            % special case
            \item $x = a_0$
            
            Let $(U_a, V_a, \phi_a)$ and $(U_b, V_b, \phi_b)$ be local trivializations
            for $a_0 = b_0$ in $\ba$ and $\bb$.

            % open
            Since $W_a \sub E(\bab)$ is open,
            there exists an open subset
            $W_a' \sub E(\ba)$ such that $W_a = W_a' \cap E(\bab)$.

            Let $U_a' \sub A$ be an open neighborhood of $a_0$ and $\eps > 0$ such that
            \[ V_a' := U_a' \cross \ball[\eps] \sub \phi_a(W_a'). \]
            This allows us to define the map
            \[ \phi_a': V_a' \isomto \phi_a'(V_a') \sub A \cross \R^n \twith \]
            \[ \phi_a'(e) = (j_a(e), (\snd{\phi_b} \circ f \circ \inv{\phi_a})(a_0, \snd{\phi_a}(e))). \]

            Now we can show local triviality in $a_0$ using the homeomorphism
            \[ \phi_a' \cup \phi_b: V_a' \cup V_b \isomto \phi_a'(V_a' \cup V_b) \sub (A \vee B) \cross \R^n \]

            This map is well-defined, because
            \[ \phi_a'(e) = (a_0, (\snd{\phi_b} \circ f \circ \inv{\phi_a})(a_0, \snd{\phi_a}(e))) \]
            \[ = (b_0, \snd{\phi_b}(f(e))) = (j_b(f(e)), \snd{\phi_b}(f(e))) = \phi_b(f(e)). \]

            Homeomorphy follows from the fact
            that both $\phi_a'$ and $\phi_b$ are homeomorphisms,
            and that $\phi_a'(e_a) = \phi(e_b) \implies f(e_a) = e_b$. 

            Commutativity between $i_a \cup i_b$ and $id \cross 0$
            as well as between $j_a \cup j_b$ and $\pi_1$
            is inherited from $\phi_a$ and $\phi_b$.
            Note that $\phi_a(i_a(a)) = (a, 0) = \phi_a'(i_a(a))$.

            Applying \myintref{microbundle::local} yields that $\ba \vee \bb$ is locally trivial.
        \end{caselist}
        
        % well-defined
        \item $\ba \vee \bb$ is well-defined

        Let $f'$ be another representative for $\inv{R_b} \circ R_a$ and $(\ba \vee \bb)'$ the resulting wedge sum.
        We need to find an isomorphism-germ that extends $\inv{R'} \circ R$.
        
        In order to do this,
        choose an open neighborhood $V \sub E(\ba\restr{a_0})$
        of $i_a(a)$ where $f$ and $f'$ agree.
        
        By subtracting the closed set $\inv{j_a}(a_0) - V$
        from $E(\ba \vee \bb)$ and $E(\ba \vee \bb)'$,
        the microbundles remain unchanged due to \myintref{microbundle::total}.
        
        But now the total spaces $E(\ba \vee \bb)$ and $E((\ba \vee \bb)')$ are the same.
        That is because $E(\ba \vee \bb)$ and $E((\ba \vee \bb)')$
        could only differ in $\inv{j_a}(a_0) - V$.

        
        Furthermore, since injection and projection are defined exactly the same,
        it follows that the identity $\germ{(\ba \vee \bb)}{(\ba \vee \bb)'}$
        is an isomorphism-germ.
        Together with
        \[ \inv{R'} \circ R = \inv{R} \circ R = I, \]
        which completes the proof.
    \end{sectionize}
\end{myproof}
\begin{myparagraph}
    In the following,
    we provide a proof for the Bouquet Lemma presented in \myintref{section::whitney}.
    To this end, we introduce the concept of `rooted microbundles',
    which allows us to define the wedge sum of two microbundles in a precise manner.
    Additionally, we show a version of the Homotopy Theorem
    that is compatible with rooted-microbundles.

    In this section,
    we assume that every topological space is equipped with an arbitrary base point
    which we will denote with subscript $0$.
\end{myparagraph}
\subsection*{Rooted Microbundles}\label{section::rooted}{\blankbreak}
% definition
\begin{mystatement}{definition}{suspension::rooted}
    % rooted
    A \defterm{rooted microbundle} $\bb$ over $B$ is a microbundle
    over $B$ together with an isomorphism-germ
    \[ \germdef{R}{\bbb}{\be{b_0}}. \]
    % isomorphy
    Two rooted microbundles $\bb$ and $\bb'$ are \defterm{rooted isomorphic}
    if there exists an isomorphism-germ $\germ{\bb}{\bb'}$ extending
    \[ \germdef{\inv{R'} \circ R}{\bbb}{\bb'\restr{b_0}}. \]
\end{mystatement}

% remark
\begin{mystatement}{remark}{suspension::rooting}
    One can always define a rooting for a given microbundle 
    by choosing a local trivialization in the base point and restricting
    it to the fiber of $b_0$.
\end{mystatement}
% definition
\begin{mystatement}{definition}{suspension::induced}
    Let $\bb$ be a rooted microbundle over $B$ and $f: A \to B$ a based map.
    The induced microbundle $\ind{f}\bb$ is equipped with the rooting
    \[
        \germdef{\ind{f}R}{E(\ind{f}\bb[a_0])
        = a_0 \cross E(\bb[b_0])}{e^n_{a_0}}
    \]
    that coincides with $R$ if we consider
    $a_0 \cross E(\bb[b_0]) = E(\bb[b_0])$ and $e^n_{a_0} = e^n_{b_0}$.
\end{mystatement}

% afterword
\begin{myparagraph}
    The total space $E(\ind{f}\bb[a_0])$ equals $a_0 \cross E(\bbb)$, because
    \[
        E(\ind{f}\bb[a_0])
        = \set{(a, e) \in A \cross E(\bb)}{a = a_0 \tand f(a) = b_0 = j(e)}
    \]
    \[
        = a_0 \cross \set{e \in E(\bb)}{j(e) = b_0}
        = a_0 \cross E(\bbb).
    \]
\end{myparagraph}
% foreword theorem
\begin{myparagraph}
    Given a rooted microbundle $\bb$ and homotopic based maps $f, g: A \to B$,
    the Homotopy Theorem yields that $\ind{f}\bb$ and $\ind{g}\bb$
    are isomorphic (not rooted-isomorphic).
    
    With the preliminary work in \myintref{section::homotopy_prove}, we can derive
    a version of the Homotopy Theorem that also accounts for rooted isomorphy.
\end{myparagraph}

% theorem
\begin{mystatement}{theorem}[Rooted Homotopy Theorem]{suspension::homotopy}
    Let $\bb$ be a rooted microbundle over $B$ and $f, g: A \to B$ be two based maps
    where $A$ is paracompact hausdorff.
    If there exists a homotopy $H: \cyl{A} \to B$ between $f$ and $g$ that leaves the base point fixed,
    then the two rooted microbundles $\ind{f}\bb$ and $\ind{g}\bb$ are rooted isomorphic.
\end{mystatement}

% foreword lemma
\begin{myparagraph}
    In order to proof this,
    we need to show a `rooted version' of \myintref{homotopy::lemma2}.
    
    First, note that 
    \[
        E(\ind{H}\bb[\cyl{a_0}]) = E(\ind{\iota}(\ind{H}(\bb)))
        \cong E(\ind{(H \circ \iota)}\bb) = E(\ind{c_{\cyl{a_0}, b_0}}\bb),
    \]
    whose total space is of the form $(a_0 \cross [0, 1]) \cross E(\bb)$.
    Based on this, we can define an isomorphism-germ
    \[ \germdef{\overline{R}}{\ind{H}\bb[\cyl{a_0}]}{\be{\cyl{a_0}}} \]
    represented by
    \[ \overline{r}(a_0, t, v) = (a_0, t, \snd{r}(v)), \]
    where $r: V \to b_0 \cross \R^n$ is a representative for $R$.
    The representative $\overline{r}$ is a homeomorphism on its image,
    as its components are homeomorphisms on their image. 
\end{myparagraph}

% lemma
\begin{mystatement}{lemma}{suspension::sharper}
    Let $\bb$ be a rooted microbundle over $B$ and
    let $H: \cyl{A} \to B$ be a map that leaves the base point fixed.
    Then there exists a neighborhood $V$ of $a_0$ together with an isomorphism-germ
    \[ \germ{\ind{H}\bb[\cyl{V}]}{\be{\cyl{V}}} \]
    extending $\overline{R}$ (as defined above).
\end{mystatement}

% proof lemma
\begin{myproof}
    By applying \myintref{homotopy::lemma2},
    it follows that there exists an isomorphism-germ
    \[ \germdef{Q}{\ind{H}\bb[\cyl{V}]}{\be{\cyl{V}}} \]
    for a sufficiently small neighborhood $V$ of $a_0$.
    However, $Q$ does not extend $\overline{R}$ in general.

    In order to fix this, consider
    \[ \germdef{Q \circ \inv{\overline{R}}}{\be{\cyl{a_0}}}{\be{\cyl{a_0}}} \]
    together with a representative $f: U_f \to (\cyl{a_0}) \cross \R^n$.

    Similar to the construction of $\overline{R}$, we can construct an isomorphism-germ
    \[ \germdef{P}{\be{\cyl{V}}}{\be{\cyl{V}}} \]
    extending $Q \circ \inv{\overline{R}}$ represented by
    \[ p(a, t, x) = (a, f(a_0, t, x)) \]
    considering $f(a_0, t, x) \in \I \cross \R^n$.

    Restricted to $\be{\cyl{a_0}}$, $P$ agrees with $Q \circ \inv{\overline{R}}$ and thus
    \[
        \inv{Q} \circ P\restr{\be{\cyl{a_0}}}
        = (\inv{Q} \circ (Q \circ \inv{\overline{R}}))
        = ((\inv{Q} \circ Q) \circ \inv{\overline{R}})
        = \inv{\overline{R}}.
    \]
    % \[ \implies (\inv{P} \circ Q)\restr{\ind{H}\bb[\cyl{a_0}]} = \overline{R}. \]
    Since $P$ and $Q$ are both isomorphism-germs,
    \[ \germdef{\inv{P} \circ Q}{\ind{H}\bb[\cyl{V}]}{\be{\cyl{V}}} \]
    is an isomorphism-germ extending $\overline{R}$.
\end{myproof}

% foreword proof
\begin{myparagraph}
    We are now able to show the Rooted Homotopy Theorem.

    To understand the proof,
    it is useful to have the constructions of \myintref{homotopy::lemma3} in mind,
    because we will modify them slightly in order to preserve the rootings.
\end{myparagraph}

% proof theorem
\begin{myproof}[of the Rooted Homotopy Theorem]
    We need to show that $\ind{f}\bb$ and $\ind{g}\bb$ are rooted isomorphic,
    that is there exists an isomorphism-germ $\germ{\ind{f}\bb}{\ind{g}\bb}$
    extending $\inv{R_g} \circ R_f = id_\ba$.

    For the initial Homotopy Theorem,
    we constructed a bundle-germ
    \[ \germdef{F}{\ind{H}\bb}{\ind{H}\bb[\cylup{A}]} \]
    covering $(a, t) \mapsto (a, 1)$
    and restricted it to $\ind{H}\bb[\cyldown{A}]$.
    The required isomorphism-germ was then obtained by
    identifying $\ind{f}\bb$ with $\ind{H}\bb[\cyldown{A}]$ and
    $\ind{g}\bb$ with $\ind{H}\bb[\cyldown{A}]$.

    We must make slight modifications
    to the construction of $F$ such that it extends
    $\germ{\ind{f}\bbb \cong \ind{H}\bb[\cyldown{a_0}]}{\ind{H}\bb[\cylup{a_0}] \cong \ind{g}\bbb}$
    represented by
    \[ (a_0, e) = ((a_0, 0), e) \mapsto ((a_0, 1), e) = (a_0, e). \]

    This can be achieved by choosing a locally finite open cover $\coll{V_\alpha}$
    of $A$ (as in \myintref{homotopy::lemma3}), removing the base point $a_0$ from every set
    and adding $V$ obtained from \myintref{suspension::sharper}.
    Since $a_0 \in V$, the resulting collection is still a
    locally finite open cover of $A$.
    
    In the following, we will denote constructions over $V$
    with subscript $V$ and constructions over the other sets
    from the cover with subscript $\alpha$.

    We continue with the proof of \myintref{homotopy::lemma3}.
    Note that $\lambda_V(a_0) = 1$.
    That is because we removed $a_0$ from every other set and hence $\lambda_\alpha(a_0) = 0$.

    Lastly, we construct the extension $R_V$ for $r_V$
    like in \myintref{section::homotopy_prove},
    but instead of choosing an arbitrary trivialization
    $E(\ind{H}\bb\restr{A_V}) \cong A_V \cross \R^n$
    for the construction we use a representative $r$
    for the bundle-germ constructed in \myintref{suspension::sharper}.
    
    This has the advantage that the representative
    \[
        E(\ind{H}\bb[A_V]) \xto{r}
        A_V \cross \R^n \xto{r_V \cross id} (A_V \cap A'_V) \cross \R^n
        \xto{\inv{r}} E(\ind{H}\bb[A_V \cap A'_V])
    \]
    for $R_V$ maps elements $((a_0, 0), e)$ to $((a_0, 1), e)$.
    Additionally,
    every other $R_\alpha$ leaves $\ind{H}\bb[\cyldown{a_0}]$ unaffected
    because $r_\alpha(a_0, t) = (a_0, \max(\underbrace{\lambda_\alpha(t)}_{= 0}, t)) = (a_0, t)$.

    It follows that, by piecing together the $R_\alpha$ and $R_V$ like in \myintref{homotopy::lemma3},
    we obtain a bundle germ $\germdef{F}{\ind{H}\bb}{\ind{H}\bb[\cylup{A}]}$
    that extends $\inv{R_g} \circ R_f$.
    This completes the proof.
\end{myproof}
% definition wedge sum
\begin{mydefinition}
    Let $\ba_*$ and $\bb_*$ be two rooted microbundles over $A$ and $B$.
    The \defterm{wedge sum} $\ba_* \vee \bb_*$ is a rooted microbundle over $A \vee B$ defined as follows:

    For the rootings $R_\ba: \ba\restr{a_0} \double \be^n_{a_0}$ and $R_\bb: \bb\restr{b_0} \double \be^n_{b_0}$,
    choose representatives $r_\ba: V_\ba \isomto a_0 \times \R^n$ and $r_\bb: V_\bb \isomto b_0 \times \R^n$.
    Denote $r := r_\bb^{-1} \circ r_\ba: V_\ba \isomto V_\bb$.
    \begin{itemize}
        \item $E(\ba \vee \bb) := (E(\ba_*) \sqcup E(\bb_*)) / (\forall e \in V_\ba: e \sim r(e))$
        \item $i := i_\ba \vee i_\bb$ is well-defined, because $i(a_0) = i_\ba(a_0) = r(i_1(a_0)) = i_\bb(b_0) = i(b_0)$
        \item $j := j_\ba \vee j_\bb$ is well-defined, because $\forall e \in V_\ba: j(e) = j_\ba(e) = a_0 = b_0 = j_\bb(r(e)) = j(r(e))$
    \end{itemize}
\end{mydefinition}
\begin{proof}[that $\ba_* \vee \bb_*$ is a microbundle]
    Continuity of $i$ and $j$ follows directly from the quotient topology and from the continuity of $i_a, i_b$ and $j_a, j_b$.
    Additionally, $\forall a \in A: j(i(a)) = j(i_\ba(a)) = j_\ba(i_\ba(a)) = a$ (symmetric for $B$) and therefore $j \circ i = id$.
    It remains to be shown that the diagram is locally trivializable:

    Let $(U_a, V_a, \phi_a)$ and $(U_b, V_b, \phi_b)$ be local trivializations for $\ba$ in $a_0$ and $\bb$ in $b_0$.
    We can construct a local trivialization for $\ba \vee \bb$ in $a_0 = b_0$ as follows:
    \begin{itemize}
        \item $U := U_a \vee U_b$
        \item $V := V_a \vee V_b$
        \item $\phi\restr{V_a}(e) := (j_a(e), r_a(\phi_a^{-1}(a_0, \phi_a^{(2)}(e))))$ and symmetrically over $V_b$
    \end{itemize}
    Openness of $U$ and $V$ follows from the topology of quotients.
    The map $\phi$ is well-defined, because $\forall e_a = e_b \in V_\ba$:
    \[ \phi(e_a) = (j_a(e_a), r_a(\phi_a^{-1}(a_0, \phi_a^{(2)}(e_a)))) \]
    \[ = (a_0, r_a(e_a)) = (b_0, e_b) \]
    \[ = (j_b(e_b), r_b(\phi_b^{-1}(b_0, \phi_b^{(2)}(e_b)))) = \phi(e_b) \]
    Homeomorphy of $\phi$ follows from the homeomorphy of the components it's composed of.

    One can see that regardless of the choices we made for this definition, the resulting microbundle remains the same up to isomorphy.
    TODO: PROOF
\end{proof}

% foreword suspension
\begin{myparagraph}
    In the following, let $B$ be a \defterm{reduced suspension}
    \[ SX = (X \times [0, 1]) / (X \times \{ 0, 1 \} \cup x_0 \times [0, 1])\]
    over $X$.

    Let $\phi: B \to B \vee B$ denote the map that sends
    $X \times [0, \half]$ to the first $B$ via
    \[ \phi(x, t) = [(x, 2t)] \]
    and $X \times [\half, 1]$ to the second $B$ via
    \[ \phi(x, t) = [(x, 2t - 1)]. \]

    Let $c_1: B \vee B \to B$ denote the map that is the identity on the first summand
    and the constant mapt to $b_0$ on the second summand.
\end{myparagraph}

% lemma commutatitivity
\begin{mylemma}\label{suspension::lemma1}
    \[ \phi^*(\bb \oplus \be^n_B) \cong \bb \cong \phi^*(\be^n_B \oplus \bb) \]
\end{mylemma}
\subsection*{Microbundles over a Suspension}\label{section::suspension}{\blankbreak}
% foreword suspension
\begin{myparagraph}
    In the following, let $B$ be a \defterm{reduced suspension}
    \[ SX = (\cyl{X}) / (X \cross \coll{0, 1} \cup \cyl{x_0})\]
    over a topological space $X$.

    Let $\phi: B \to B \vee B$ denote the `duplicate' map given by
    \[ \phi([x, t]) = \case{([x, 2t], 1)}{([x, 2t - 1], 2)}{t \le \half}. \]

    Additionally, let $c_1: B \vee B \to B$ denote the map
    that is the identity on the first summand
    and the constant map $c_{B, b_0}$ on the second summand, i.e.
    \[ c_1(b, i) = \case{b}{b_0}{i = 1}. \]
    We define $c_2$ analogously.
\end{myparagraph}
\begin{mystatement}{lemma}{suspension::triviality}[70] The following (non-rooted) isomorphy holds: \[ \ind{\phi}(\bb \vee \be{B}) \cong \bb \cong \ind{\phi}(\be{B} \vee \bb) \] \end{mystatement}

\begin{myproof} We prove the lemma in two steps. \begin{steps} \item $\ind{c_1}\bb \cong \bb \vee \be{B}$

Let $E(\bb \vee \be{B})$ be constructed via the representative $f: V \to b_0 \cross \R^n$ for $R$.

Without loss of generality, we may assume that $V = E(\bbb)$ by removing the closed set $E(\bbb) - V$ from $E(\bb)$ if necessary.

Consider $\psi: E(\ind{c_1}\bb) \isomto E(\bb \vee \be{B})$ given by \[ \psi((b, i), e) = \case{e}{(b, \snd{f}(e))}{i = 1}. \] Note that $\psi$ is well-defined, because \[ \psi((b_0, 1), e) = e = f(e) = (b_0, \snd{f}(e)) = \psi((b_0, 2), e). \] Furthermore, $\psi$ is a homeomorphism as both of its summands are homeomorphisms.

It remains to be shown that $\psi$ commutes with the injection and projection maps of $\ind{c_1}\bb$ and $\bb \vee \be{B}$. This can be seen with the following equations: \begin{align} &\psi(i_{c_1}(b, i)) = \casenif{\psi((b, 1), i(b)) = i(b) = i_\vee(b, 1)}{\psi((b, 2), i(b_0)) = f(i(b)) = (b, 0) = (id \cross 0)(b) = i_\vee(b, 2)} \\ &j_{c_1}((b, i), e) = \casenif{j(e) = j_\vee(e) = j_\vee(\psi((b, 1), e))}{(b, 2) = \pi_1(\psi((b, 2), e)) = j_\vee(\psi((b, 2), e))} \end{align}

\item $\ind{\phi}(\bb \vee \be{B}) \cong \bb$

Using the fact that $c_1 \circ \phi = id_{B \vee B}$, we conclude that \[ \ind{\phi}(\bb \vee \be{B}) \cong \ind{\phi}(\ind{c_1}\bb) \cong \ind{(c_1 \circ \phi)}\bb \cong \bb. \] For reasons of symmetry, it follows that $\bb \cong \ind{\phi}(\be{B} \vee \bb)$. \end{steps} This completes the proof. \end{myproof}
\begin{mystatement}{lemma}{suspension::distributivity} Let $\ba$ and $\bb$ be two rooted microbundles over $A$ and $B$. Given two based maps $f: A' \to A$ and $g: B' \to B$, the following equality holds: \[ \ind{(f \cup g)}(\ba \vee \bb) = \ind{f}\ba \vee \ind{g}\bb \] \end{mystatement}

\begin{myproof} Consider the following equation: \begin{gather*} E(\ind{(f \cup g)}(\ba \vee \bb)) \\ = \set{(x, e) \in (A' \vee B') \cross E(\ba \vee \bb)}{(f \cup g)(x) = j(e)} \\ = \set{(x, e) \in ((A' \cross E(\ba)) \sqcup (B' \cross E(\bb)) / \sim)}{(f \cup g)(x) = j(e)} \\ = (\set{(x, e) \in A' \cross E(\ba)}{f(x) = j_a(e)} \sqcup \set{(x, e) \in B' \cross E(\bb)}{g(x) = j_b(e)}) / \sim\\ = (E(\ind{f}\ba) \sqcup E(\ind{g}\bb) / \sim) = E(\ind{f}\ba \vee \ind{g}\bb) \end{gather*} Here, $(a, e_a) \sim (b, e_b) \iff a = a_0 = b_0 = b \tand [e_a] = [e_b] \in E(\ba \vee \bb)$.

Furthermore, the injection and projection maps agree. This can be seen with the following equations: \begin{gather} i_{\cup}(a) = i_f(a) = i_{\vee}(a) \tand i_{\cup}(b) = i_g(b) = i_{\vee}(b)\\ j_{\cup}(a, e) = a = j_f(a, e) = i_{\vee}(a, e) \tand j_{\cup}(b, e) = b = j_g(b, e) = i_{\vee}(b, e) \end{gather}

It follows that the two microbundles are equal. \end{myproof}
\begin{myparagraph} Let $r: B \isomto B$ denote the `reflection' given by \[ r([x, t]) = [x, 1 - t] \] and let $c: B \vee B \to B$ denote the identity on the first summand and $r$ on the second summand, i.e, \[ c(b, i) = \case{b}{r(b)}{i = 1}. \] \end{myparagraph}

\begin{mystatement}{lemma}{suspension::reflection}[70] The induced microbundle $\ind{\phi}(\bb \vee \ind{r}\bb)$ is trivial. \end{mystatement}

\begin{myproof} The composition $c \circ \phi$ is null-homotopic via the homotopy $H: B \cross \I \to B$ given by \[ H([x, t], s) = f(\phi(x, t \cdot s)). \] Applying \myintref{homotopy::theorem} yields $\ind{\phi}(\ind{c}\bb) \cong \ind{(c \circ \phi)}\bb \cong \ind{c_{B, b_0}}\bb \cong \be{B}$ as rooted-isomorphy. Note that the induced microbundle preserves rootings (see \myintref{suspension::induced}). By applying the previous lemma, it follows that \[ \ind{\phi}(\bb \vee \ind{r}\bb) = \ind{\phi}(\ind{(id \cup r)}(\bb \vee \bb)) = \ind{\phi}(\ind{c}\bb) \] and hence $\ind{\phi}(\bb \vee \ind{r}\bb) \cong \be{B}$. \end{myproof}
% definition
\begin{mystatement}{definition}{suspension::whitney}[70]
    Given two rooted microbundles $\bb$ and $\bb'$ over $B$,
    we equip the Whitney sum $\whitney{\bb}{\bb'}$ with the rooting
    \[ \germdef{\whitney{R}{R'}}{(\whitney{\bb}{\bb'})\restr{b_0}}{\whitney{\be[n_1]{b_0}}{\be[n_2]{b_0}} = \be[n_1 + n_1]{b_0}} \]
    represented by the direct sum of two representatives for $R$ and $R'$.
\end{mystatement}

% lemma
\begin{mystatement}{lemma}{suspension::compatability}[70]
    Let $\ba$ and $\ba'$ be two rooted microbundles over $A$,
    and let $\bb$ and $\bb'$ be two rooted microbundle over $B$.
    Then the following (non-rooted) isomorphy holds:
    \[ \whitney{(\ba \vee \bb)}{(\ba' \vee \bb')} \cong (\whitney{\ba}{\ba'}) \vee (\whitney{\bb}{\bb'}) \]
\end{mystatement}

% proof
\begin{myproof}
    % homeomorphy
    Consider $\psi: E(\whitney{(\ba \vee \bb)}{(\ba' \vee \bb')}) \isomto E((\whitney{\ba}{\ba'}) \vee (\whitney{\bb}{\bb'}))$
    given by the identity map.
    Note that $\psi$ is well-defined because
    \[ (e, e') \in E(\whitney{(\ba \vee \bb)}{(\ba' \vee \bb')}) \]
    \[ \implies j(e) = j'(e') \implies j(e), j'(e') \in A \tor j(e), j'(e') \in B\]
    \[ \implies (e, e') \in E(\whitney{\ba}{\ba'}) \tor (e, e') \in E(\whitney{\bb}{\bb'}) \]
    \[ \implies (e, e') \in E((\whitney{\ba}{\ba'}) \vee (\whitney{\bb}{\bb'})). \]

    % injection and projection
    Furthermore, the injection and projection maps agree.
    This can be seen with the following equations (symmetrical for $B$):
    \begin{gather}
        i_\oplus(a) = (i_a(a), i_a'(a)) = i_\vee(a) \\
        j_\oplus(e_a, e_a') = j(e_a) = j_a(e_a) =  j_\vee(e_a, e_a')
    \end{gather}

    It follows that the two microbundles are isomorphic.
\end{myproof}
% lemma
\begin{mystatement}{lemma}{suspension::paracompact}
    Let $\bb$ be a rooted microbundle over a paracompact hausdorff space $B$.
    Then there exists a closed neighborhood $W$ of $b_0$ and an isomorphism-germ
    \[ \germ{\bb[W]}{\be{W}} \]
    extending $R$ (rooting of $\bb$) together with a map $\lambda: B \to [0, 1]$ with
    \[ \supp{\lambda} \sub W \tand \lambda(b_0) = 1. \]
\end{mystatement}

% proof
\begin{myproof}
    Let $r: V_r \to b_0 \cross \R^n$ be a representative for $R$.
    
    Choose a local trivialization $(U, V, \phi)$ for $b_0$ such that $V \cap E(\bbb) \sub V_r$.
    Such a trivialization can be obtained by subtracting the closed set
    $E(\bbb) - V_r$ from $E(\bb)$ if necessary
    and choosing a local trivialization for $b_0$ in the resulting microbundle instead.
    
    Consider the (locally) finite open cover of $B$ given by $U$ and $B - \{b_0\}$.
    Since $B$ is paracompact, we can apply the theory
    of partitions of unity which yields a map
    $\lambda: B \to [0, 1] \twith \supp{\lambda} \sub U$ and $\lambda(b_0) = 1$.

    We choose $W = \supp{\lambda} \sub U$,
    which is closed by the definition of the support.
    We are now able to define an isomorphism-germ $\germ{\bb[W]}{\be{W}}$ represented by
    \[ f: V \isomto f(V) \sub U \cross \R^n \twith f(e) = (j(e), \snd{r}(\inv{\phi}(b_0, \snd{\phi}(e)))), \]
    which extends $r$.

    Together with $\lambda$, this completes the proof.
\end{myproof}
% lemma
\begin{mystatement}{lemma}{suspension::commutativity}
    The rooted microbundles $\whitney{\bb}{\be{B}}$ and $\whitney{\be{B}}{\bb}$ are rooted isomorphic. 
\end{mystatement}

% proof
\begin{myproof}
    We need to find an isomorphism-germ
    $\germ{\whitney{\bb}{\be{B}}}{\whitney{\be{B}}{\bb}}$ that extends
    \[ (\whitney{I}{R}) \circ \inv{(\whitney{R}{I})} = \whitney{R}{\inv{R}} \]
    where $I$ denotes the identity germ.

    Ignoring the rooting, we have an
    isomorphism-germ $f: E(\bb) \cross \R^n \isomto \R^n \cross E(\bb)$ with $f(e, x) = (-x, e)$.
    The idea is to change the $f$ near $b_0$ so that it extends the rooting.

    Using the previous lemma, choose a sufficiently small closed neighborhood $U$ of $b_0$
    such that there exists an extension
    $\germdef{Q}{(\whitney{\bb}{\be{B}})\restr{U}}{(\whitney{\be{B}}{\bb})\restr{U}}$ for the rooting.

    The previous lemma also equips us with a map
    \[ \lambda: B \to [0, \frac{\pi}{2}] \]
    such that $\supp \lambda \sub U$ and $\lambda(b_0) = \frac{\pi}{2}$.
    
    Now, we can define a homeomorphism
    \[ \psi: U \cross \R^n \cross \R^n \isomto U \cross \R^n \cross \R^n \twith \]
    \[ \psi(b, x, y) = (b, x \sin(\lambda(b)) - y \cos(\lambda(b)), x \cos(\lambda(b)) - y \sin(\lambda(b))). \]

    Consider the composition
    \[ \germ{(\whitney{\bb}{\be{B}})\restr{U}}{(\whitney{\bb}{\be{B}})\restr{U}} \xto{g} \germ{(\whitney{\bb}{\be{B}})\restr{U}}{(\whitney{\be{B}}{\bb})\restr{U}} \]
    which coincides with $\whitney{R}{\inv{R}}$ over $b_0$
    since $\psi(b_0, x, y) = (b_0, x, y)$ and with $F$ over $U \cap \inv{\lambda}(0)$.
    
    Pieced together with $F\restr{\inv{\lambda}(b)}$ using \myintref{homotopy::glueing},
    we obtain an isomorphism-germ 
    \[ \germ{\whitney{\bb}{\be{B}}}{\whitney{\be{B}}{\bb}} \]
    extending the rooting.
\end{myproof}
% theorem
\begin{mystatement}{theorem}{suspension::theorem}[71]
    If $\ba$ and $\bb$ are rooted microbundles
    over a paracompact hausdorff space $B$, then
    \[ \whitney{\ind{\phi}(\ba \vee \bb)}{\be{B}} = \whitney{\ba}{\bb}. \]
\end{mystatement}

% proof theorem
\begin{myproof}
    The previous lemma yields rooted isomorphy $\whitney{\bb}{\be{B}} \cong \whitney{\be{B}}{\bb}$.
    Hence
    \[
        \ind{\phi}((\whitney{\ba}{\be{B}}) \vee (\whitney{\bb}{\be{B}}))
        \cong \ind{\phi}((\whitney{\ba}{\be{B}}) \vee (\whitney{\be{B}}{\bb})).
    \]
    Furthermore, we have
    \[
        \ind{\phi}((\whitney{\ba}{\be{B}}) \vee (\whitney{\bb}{\be{B}})) \cong
        \whitney{\ind{\phi}((\ba \vee \bb))}{(\be{B} \vee \be{B})}
        \cong \whitney{\ind{\phi}(\ba \vee \bb)}{\be{B}}
    \]
    for the left side and
    \[
        \ind{\phi}((\whitney{\ba}{\be{B}}) \vee (\whitney{\be{B}}{\bb})) \cong
        \ind{\phi}(\whitney{(\ba \vee \be{B})}{(\be{B} \vee \bb)})
        \cong \whitney{\ba}{\bb}
    \]
    for the right side of the isomorphy, which completes the proof.
\end{myproof}

% corollary
\begin{mystatement}{corollary}{suspension::corollary}[72]
    The wedge sum $\whitney{\bb}{\ind{r}\bb}$ is trivial.
\end{mystatement}

% proof corollary
\begin{myproof}
    This follows directly from the previous theorem and
    the fact that $\ind{\phi}(\whitney{\bb}{\ind{r}\bb})$ is trivial.
\end{myproof}

% bouquet
\begin{myparagraph}
    Now the Bouquet Lemma is just \myintref{suspension::corollary} applied to
    a microbundle over a bouquet of spheres.
    Note that a bouquet of $d$-spheres can be regarded
    as a reduced suspension over a bouquet of $(d-1)$-spheres.
\end{myparagraph}
% Document :: Normal
\section{Normal Microbundles and Milnors Theorem}\label{chapter::normal}
\subsection*{The Normal Microbundle}\label{section::normal}{\blankbreak}
\begin{mystatement}{definition}[normal microbundle]{normal::definition}[61]Let $M$ be a topological manifold together with a submanifold $N \sub M$. A \defterm{normal microbundle} $\bn$ of $N$ in $M$ is a microbundle \[ \bundle{N}{U}{\iota}{r} \] where $U \sub M$ is a neighborhood of $N$ and $\iota$ denotes the inclusion $N \incl U$.
\end{mystatement}

\begin{mystatement}{definition}[composition microbundle]{normal::composition}[63] Let $\ba$ be a $n$-dimensional microbundle \[ \bundledef{\ba}{A}{E(\ba)}{i_a}{j_a} \] and let $\bb$ be a $n'$-dimensional microbundle \[ \bundledef{\bb}{E(\ba)}{E(\bb)}{i_b}{j_b}. \] The \defterm{composition microbundle} $\ba \circ \bb$ is a $(n + n')$-dimensional microbundle \[ \bundle{A}{E(\bb)}{i}{j} \] where $i = i_b \circ i_a$ and $j = j_a \circ j_b$. \end{mystatement}

\begin{myproof}[that $\ba \circ \bb$ is a microbundle] Both injection and projection maps are continuous as being composed by continuous maps. Additionally, $j \circ i = j_a \circ (j_b \circ i_b) \circ i_a = j_a \circ i_a = id_A$.

It remains to be shown that $\ba \circ \bb$ is locally trivial.

For an arbitrary $a \in A$, choose local trivializations $(U_a, V_a, \phi_a)$ of $a$ in $\ba$ and $(U_b, V_b, \phi_b)$ of $i_a(a)$ in $\bb$. Note that both $U_b$ and $V_a$ are open neighborhoods of $i_a(a)$.

Without loss of generality, we may assume that $V_a = U_b$:

`$\sub$': Modify $U_a$ such that \[ U_a \cross \ball[\eps] \sub \phi_a(V_a \cap U_b) \] for a sufficiently small $\eps > 0$ and let \[ V_a = \inv{\phi_a}(U_a \cross \ball[\eps]) \sub V_a \cap U_b. \] Composing $\phi_a$ with $\mu_\eps: \ball[\eps] \isomto \R^n$ yields a local trivialization of $a$ in $\ba$ such that $V_a \sub U_a$.

`$\bus$': Restrict $U_b$ to $V_a \cap U_b$ and $V_b$ to $\inv{\phi_b}((V_a \cap U_b) \cross \R^{n'})$.

We have local-trivialization $(U_a, V_b, \phi)$ of $a$ in $\ba \circ \bb$ given by \[ \phi: V_b \xto{\phi_b} U_b \cross \R^{n'} = V_a \cross \R^{n'} \xto{\phi_a \cross id} (U_a \cross \R^{n}) \cross \R^{n'} = U_a \cross \R^{n + n'}, \] which is a homeomorphism since it's composed by homeomorphisms.

Furthermore, $\phi$ commutes with the injection and projection maps, as the following equations show: \begin{gather} \begin{split} \phi(i(a)) = \phi(i_b(i_a(a))) = (\phi_a(i_a(a)), \snd{\phi_b}(i_b(i_a(a)))) \\ = (\snd{\phi_a}(i_a(a)), 0) = (a, (0, 0)) = (id_{U_a} \cross 0)(a)  \end{split} \\ j(e) = j_a(j_b(e)) = \pi_1(j_a(j_b(e)), \snd{\phi}(e)) = \pi_1(\phi(e)) \end{gather} This completes the proof. \end{myproof}
% foreword
\begin{myparagraph}
    Unlike the normal vector bundle for smooth manifolds,
    the normal microbundle is not defined in a constructive manner.
    Therefore, the question arises in which sense the normal microbundle
    of a submanifold $N \sub M$ is unique.
    In fact, it is unknown whether a such normal microbundle is unique up to isomorphy.
    Instead, we have the following statement about uniqueness.
\end{myparagraph}

% statement
\begin{mystatement}{proposition}{normal::uniqueness}[63]
    Let $N \sub M$ be an embedded submanifold.
    Suppose there exists a normal microbundle
    $\bundledef{\bn}{N}{U}{\iota}{r}$ in $M$.
    Then $\bt_N \oplus \bn \cong \bt_M\restr{N}$.
\end{mystatement}

% proof
\begin{myproof}
    We prove this proposition in multiple steps.
    \begin{steps}
        % step 1
        \item $\bt_N \circ \ind{\pi_2}\bn \cong \bt_M\restr{N}$
        
        % total space
        Consider the two total spaces
        \[ E(\bt_N \circ \ind{\pi_2}\bn) = E(\ind{\pi_2}\bn) = \set{(n_1, n_2, u) \in (N \cross N) \cross U}{n_2 = r(u)} \]
        and
        \[ E(\bt_M\restr{N}) = \set{(n, m_1, m_2) \in N \cross (M \cross M)}{n = m_1}. \]

        % homeomorphy
        We can easily define a homeomorphism $\psi: E(\bt_M \circ \ind{\pi_2}\bn) \isomto E(\bt_N\restr{M})$ given by
        \[ \psi(m_1, m_2, u) = (m_1, m_1, u) \tand \inv{\psi}(m, n_1, n_2) = (m, r(n_2), n_2). \]
        Note that $\psi$ is a homeomorphism since $\psi$ and $\inv{\psi}$ are component-wise continuous.

        % commutativity
        It remains to be shown that $\psi$
        commutes with the injection and projection maps of $\bt_M \circ \ind{\pi_2}\bn$ and $\bt_N\restr{M}$.
        This can be seen with the following equations:
        \begin{align}
            \psi(i_{\pi_2}(\Delta(m))) = \psi(m, m, \iota(m)) = (m, m, m) = (m, \Delta(\iota(m))) \\
            \pi_1(j_{\pi_2}(m_1, m_2, u)) = m_1 = j_\iota(m_1, m_1, u) = j_\iota(\psi(m_1, m_2, u)) 
        \end{align}
        
        % step 2
        \item $\bt_M \circ \ind{\pi_1}\bn \cong \whitney{\bt_M}{\bn}$
        
        % total space
        In this case, the two total spaces
        \[ E(\bt_M \circ \ind{\pi_1}\bn) = E(\ind{\pi_1}\bn) = \set{(m_1, m_2, u) \in (M \cross M) \cross U}{m_1 = r(u)} \]
        and
        \[ E(\whitney{\bt_M}{\bn}) = \set{(m_1, m_2, u) \in (M \cross M) \cross U}{m_1 = r(u)} \]
        are equal.
        % homeomorphy + commutativity
        Additionally,
        the injection and projection maps agree, as the following equations show:
        \begin{align}
            & i_{\pi_1}(\Delta(m)) = i_{\pi_1}(m, m) = (m, m, \iota(m)) = (\Delta(m), \iota(m)) \\
            & \pi_1(j_{\pi_1}(m_1, m_2, u)) = \pi_1(m_1, m_2) = m_1 = r(u) = j_{\oplus}(m_1, m_2, u)
        \end{align}

        % step 3
        \item $\bt_M \circ \ind{\pi_1}\bn \cong \bt_M \circ \ind{\pi_2}\bn$
        
        We show that there exists a neighborhood $D \sub N \cross N$ of $\Delta(M)$ such that
        $\pi_1\restr{D}$ is homotopic to $\pi_2\restr{D}$:

        Firstly, we assume that $M$ is embedded in Euclidean space (see\cite[p.60]{dimension}).
        Let $V \sub N$ be a neighborhood retract of $M$.
        We define $D$ as follows:
        \[ D = \set{(m, m') \in M \cross M}{ tm + (1 - t)m' \in V, \forall t \in \I} \]
        We are given a homotopy $H: \cyl{D} \to N$ between $\pi_1$ and $\pi_2$ by
        \[ H((m, m'), t) = tm + (1 - t)m'. \]

        Applying the Homotopy Theorem yields $\ind{\pi_1}\bn\restr{D} \cong \ind{\pi_2}\bn\restr{D}$, and
        by restricting the total spaces accordingly, we get
        $\bt_M \circ \ind{\pi_1}\bn \cong \bt_M \circ \ind{\pi_2}\bn$.
    \end{steps}
    The claim follows by Step 1, 2 and 3.
\end{myproof}

% afterword
\begin{myparagraph}
    This proposition also shows that the normal microbundle
    underlies the same intuition as the normal vector bundle. 
    The sum of the tangent- and the normal microbundle of the submanifold
    `span' the tangent microbundle of its surrounding space.
\end{myparagraph}
\subsection*{Milnors Theorem}\label{section::milnor}{\blankbreak}

\begin{mystatement}{lemma}{normal::transitivity}[62]
Let $P \sub N \sub M$ be a chain of topological submanifolds.
There exists a normal microbundle
\[ \bundledef{\bn}{P}{U}{\iota}{r} \]
of $P$ in $M$ if there exist normal microbundles
\begin{center}
$\bundledef{\bn_p}{P}{U_N}{\iota_P}{j_P}$ in $N$ and $\bundledef{\bn_n}{N}{U_M}{\iota_N}{j_N}$ in $M$.
\end{center}
\end{mystatement}

\begin{myproof}
We are given a normal microbundle $\bn$ of $P$ in $M$ by
\[ \bn_p \circ \bn_n\restr{U_N}. \]
Note that $\iota_N \circ \iota_P$ is just the inclusion $P \incl U_M$.
\end{myproof}
% foreword
\begin{myparagraph}
    Every topological manifold is an absolute neighborhood retract (ANR).
    
    It follows that by restricting $M$, if necessary, to an open neighborhood of $N$,
    there exists a retraction $M \subm N$.

    From now on, let
    \[ r: M \subm N \]
    denote such a retraction and let
    \[ \iota: N \incl M \]
    denote the inclusion $N \sub M$.
\end{myparagraph}
% lemma
\begin{mystatement}{lemma}{normal::homeomorphy}
    Let $\bt_N$ and $\bt_M$ be two tangent microbundles of $N$ and $M$.
    The total spaces $E(\ind{\iota}\bt_M)$ and $E(\ind{r}\bt_N)$ are homeomorphic.
\end{mystatement}

% proof
\begin{myproof}
    The total space
    \[ E(\ind{\iota}\bt_M) = \{ (n, m_1, m_2) \in N \cross (M \cross M) \mid \iota(n) = m_1\} \]
    is homeomorphic to $N \cross M$ via
    \[ (n, m_1, m_2) \mapsto (n, m_2) \]
    with inverse $(n, m) \mapsto (n, \iota(n), m)$.
    Similarly, the total space
    \[ E(\ind{r}\bt_N) = \{ (m, n_1, n_2) \in M \cross (N \cross N) \mid r(m) = n_1\} \]
    is homeomorphic to $M \cross N$ via
    \[ (m, n_1, n_2) \mapsto (m, n) \]
    with inverse $(m, n) \mapsto (m, r(m), n)$.
    
    Composed with the canonic homeomorphism $N \cross M \cong M \cross N$, this yields a homeomorphism
    \[ \psi: E(\ind{\iota}\bt_M) \isomto E(\ind{r}\bt_N) \twith \psi(n, m_1, m_2) := (m_2, r(m_2), n). \]
\end{myproof}

% remark
\begin{mystatement}{remark}{normal::commute}
    The following diagram commutes:
    \[
        \begin{tikzcd}
            N \ar[r, "i_\iota"] \ar[hookrightarrow, "\iota"']{d} & E(\ind{\iota}\bt_M) \ar[d, "\psi"] \\
            M \ar[r, "i_r"] & E(\ind{r}\bt_N)
        \end{tikzcd}
    \]
\end{mystatement}
% foreword
\begin{myparagraph}
    The total space $E(\ind{r}\bt_N)$ is a topological manifold.
    This can be seen with
    \[ E(\ind{r}\bt_N) = \set{(v, n_1, n_2) \in V \cross (N \cross N)}{r(v) = n_1} \cong V \cross N. \]

    Together with $N \incl M \xto{i_{\bt}} E(\ind{r}\bt_N)$, we can assume that
    $N$ is an embedded submanifold of $E(\ind{r}\bt_N)$.
    Note that $i_{\bt}$ is an embedding due to the construction of the induced microbundle.
\end{myparagraph}

% lemma
\begin{mystatement}{lemma}{normal::total}[62]
    Let $N \sub M$ be an embedded submanifold and let $r: V \to N$ be a retraction.
    Then there exists a normal microbundle $\bn$ of $N$ in $E(\ind{r}\bt_N)$. % such that $\bn \cong \ind{\iota}\bt_V$.
\end{mystatement}

% proof
\begin{myproof}
    We are given a normal microbundle of $N$ in $E(\ind{r}\bt_N)$ by $\ind{r}\bt_N\restr{N}$.
    
    Since $\ind{r}\bt_N\restr{N} \cong \ind{(r \circ \iota)}\bt_N = \ind{id}\bt_N \cong \bt_N$,
    it suffices to show that $\bt_N \cong \ind{\iota}\bt_V$.
    % homoemorphism
    We define a homeomorphism $\psi: E(\bt_N) \isomto E(\ind{\iota}\bt_V)$ with
    \[ \psi(n_1, n_2) = (n_1, n_1, n_2) \tand \inv{\psi}(n, v_1, v_2) = (n, v_2) \]
    for which homeomorphy follows from component-wise
    continuity of both $\psi$ and $\inv{\psi}$.
    
    % commutativity
    Commutativity with the injection maps is given by
    \[ \psi(\Delta(n)) = \psi(n, n) = (n, n, n) = (n, \Delta(n)) = i_\iota(n) \]
    and with the projection maps by
    \[ \pi_1(n_1, n_2) = n_1 = j_\iota(n_1, n_1, n_2) = j_\iota(\psi(n_1, n_2)), \]
    which concludes the proof.
    
    % Furthermore, $\ind{r}\bt_N\restr{N}$ is isomorphic to $\ind{\iota}\bt_V$ is given by
    % \[ \psi: E(\ind{\iota}\bt_V) \isomto E(\ind{r}\bt_N). \]
    % where commutativity with the injection and projection maps
    % follow from the diagram in \myintref{normal::commute} and from the following equation:
    % \[ j_\iota(n, v_1, v_2) = n = j_r(v_2) \]
\end{myproof}
\begin{scope}
    % defines
    \newcommand{\rwhitney} {
        \whitney{\ind{r}\bt_N}{\ind{r}\eta}
    }
    \newcommand{\rtn} {
        \ind{r}\bt_N
    }

    % foreword
    \begin{myparagraph}
        Finally, we gathered all the tools to prove Milnor's theorem.
    \end{myparagraph}

    % theorem
    \begin{mystatement}{theorem}[Milnors Theorem]{normal::milnor}
        For a sufficently large $q \in \N$, $N = \cyldown{N}$ has a normal microbundle in $M \cross \R^q$.
    \end{mystatement}

    % proof
    \begin{myproof}
        We assume that $M$ is embedded in euclidean space $\R^{2m + 1}$ \cite[p.60]{dimension}.
        Additionally, let $V$ be an open neighborhood of $N$ in $M$ together with a retraction $r: V \to N$.

        We show the theorem in multiple steps.
        \begin{steps}
            % step 1
            \item $N$ has a normal microbundle $\eta$ in $M$ such that $\whitney{\bt_N}{\eta} \cong \be[q]{N}$
            
            Consider the extension $\ind{r}\bt_N$.
            Since $V$ is an open set, it's a simplicial complex.
            Hence, we can apply \myintref{whitney::theorem} to the extended microbundle $\ind{r}\bt_N$
            to obtain a microbundle $\eta'$ such that $\whitney{\ind{r}\bt_N}{\eta} \cong \be[q]{V}$.

            We conclude that $\whitney{\bt_N}{\eta'\restr{N}} = \whitney{\ind{r}\bt_N\restr{N}}{\eta'\restr{N}} = (\whitney{\ind{r}\bt_N}{\eta'})\restr{N} = \be[q]{N}$.

            % step 2
            \item $E(\rtn)$ has a normal microbundle in $E(\rwhitney)$

            We denote the injection and projection of $\rtn$ with $i_r$ and $j_r$,
            and the injection and projection of $\rwhitney$ with $i_\oplus$ and $j_\oplus$.

            Note that $\rwhitney \cong \ind{r}(\whitney{\bt_N}{\eta})$ is trivial, so
            $E(\rwhitney)$ is an open subset of $\R^qn$ and hence a manifold.

            We consider $E(\rtn)$ to be a subset of $E(\rwhitney)$ via
            \[ (v, e) \mapsto ((v, e), (v, i_{\eta}(v))). \]

            We are given a normal microbundle of $E(\rtn)$ in $E(\rwhitney)$ by $\ind{j_r}(\ind{r}\eta)$.

            Since the total space
            \[ E(\rwhitney) = \set{(e, e') \in E(\rtn) \cross E(\ind{r}\eta)}{j(e) = j'(e)} \]
            we can consider $E(\rtn) \sub E(\rwhitney)$ embedded via
            \[ \iota: e \mapsto (e, i'(j(e))) \]
            with the inverse $\pi_1: (e, e') \mapsto e$.

            Because $\rwhitney \cong \ind{r}(\whitney{\bt_N}{\eta})$ is trivial,
            it follows that $E(\rwhitney) \sub M \cross \R^k$ open and hence being a manifold.

            We have a normal microbundle of $E(\rtn)$ in $E(\rwhitney)$ via
            \[ \bundledef{\bn}{E(\rtn)}{E(\rwhitney)}{\iota}{\pi_1}. \]

            To show local triviality,
            let $(U, V, \phi)$ be a local trivialization of $i'(j(e))$ in $\ind{r}\eta$
            for an arbitrary $e \in E(\rtn)$.

            By choosing
            \begin{itemize}
                \item $U' := \inv{j}(U)$
                \item $V' := (U' \cross V) \cap E(\rwhitney)$
                \item $\phi': V' \isomto U' \cross \R^{n_\eta}$ with $\phi'(e, e') = (e, \snd{\phi}(e'))$
            \end{itemize}
            we have a local trivialization of $e$ in $\bn$.

            That is because both $U' \sub E(\rtn)$ and $V' \sub E(\rwhitney)$ are open sets
            and $\phi'$ is a homeomorphism with its inverse
            $\inv{\phi'}(e, x) = (e, \inv{\phi}(j(e), x))$.
            
            Also, $\phi'$ commutes with injection
            \[ \phi'(\iota(e)) = \phi'(e, i'(j(e))) = (e, \snd{\phi}(i'(j(e)))) = (e, 0) = (id \cross 0)(e) \]
            and projection maps
            \[ \pi_1(e, e') = \pi_1(e, \snd{\phi'}(e, e')) = \pi_1(\phi'(e, e')). \]
            
            % step 3
            \item $N$ has a normal microbundle in $M \times \R^q$
            
            Since $N \sub M \sub E(\rtn)$ has a normal microbundle (using \myintref{normal::total}),
            it follows from \myintref{normal::transitivity} that
            $N \sub E(\whitney{\ind{r}\bt_N}{\ind{r}\bt'})$ has a normal microbundle.
    
            By restricting $E(\rwhitney)$ to an open subset if necessary, we may assume that
             \[ E(\rwhitney) = M \cross \R^q \]
            for some $q \in \N$ using \myintref{induced::trivial}.
        \end{steps}
        Applying \myintref{normal::transitivity} and $E(\rwhitney) = M \cross \R^q$ completes the proof.
    \end{myproof}

    % afterword
    \begin{myparagraph}
        
    \end{myparagraph}
\end{scope}
% TODO: outlook, example that tubular space is necessary
% Document :: References
\renewcommand{\bibname}{References}
\bibliographystyle{alpha}
\bibliography{refs}
\addcontentsline{toc}{chapter}{References}
\addcontentsline{toc}{section}{References}
\end{document}
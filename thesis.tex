\documentclass{article}
\usepackage{amsmath}
\usepackage{amssymb}
\usepackage{amsthm}
\usepackage{mathtools}

% Dummy Text NOTE: Remove before Distribution
\usepackage{lipsum}

\usepackage{hyperref} % Links
\usepackage{cleveref} % Referring to Statements
\crefdefaultlabelformat{#2\bfseries\upshape(#1)#3}          
\creflabelformat{equation}{#2\bfseries\upshape(#1)#3}
\hypersetup{colorlinks=true,citecolor=red, linkcolor=blue}
\usepackage{tikz-cd} % Diagrams
\usepackage{parskip} % No Paragraph Indentation
\usepackage{xparse} % Multiple Optional Arguments
\usepackage{ifthen} % If-Else Statements
\usepackage{enumerate} % Enumerate with romans

% leftbar
\usepackage{framed}
\newlength{\leftbarwidth}
\setlength{\leftbarwidth}{1.5pt}
\newlength{\leftbarsep}
\setlength{\leftbarsep}{7.5pt}
\newcommand*{\leftbarcolorcmd}{\color{leftbarcolor}}% as a command to be more flexible
\colorlet{leftbarcolor}{black}
\renewenvironment{leftbar}{%
    \def\FrameCommand{{\leftbarcolorcmd{\vrule width \leftbarwidth\relax\hspace {\leftbarsep}}}}%
    \MakeFramed {\advance \hsize -\width \FrameRestore }%
}{%
    \endMakeFramed
}

% Underline
\usepackage{soul}

% Images
\usepackage{graphicx}
\graphicspath{{images/}}

% \setcounter{chapter}{-1}
\usepackage{emptypage}

% \usepackage[
% backend=biber,
% style=alphabetic,
% sorting=ynt
% ]{biblatex}
% \cref
\usepackage{cite}

\setcounter{tocdepth}{1}
% break
\newcommand{\blankbreak}{$ $\\}

% scope
\newenvironment{scope}{}{}

% statement
\newtheorem{statement}{}
\numberwithin{statement}{chapter}
% statement :: Definition
\theoremstyle{definition}
\newtheorem{definition}[statement]{Definition}
% statement :: Theorem
\theoremstyle{plain}
\newtheorem{theorem}[statement]{Theorem}
% statement :: Proposition
\theoremstyle{plain}
\newtheorem{proposition}[statement]{Proposition}
% statement :: Corollary
\theoremstyle{plain}
\newtheorem{corollary}[statement]{Corollary}
% statement :: Lemma
\theoremstyle{plain}
\newtheorem{lemma}[statement]{Lemma}
% statement :: Remark
\theoremstyle{remark}
\newtheorem{remark}[statement]{Remark}
% statement :: Example
\theoremstyle{definition}
\newtheorem{example}[statement]{Example}

% myparagraph
\newenvironment{myparagraph}{\hspace{\parindent}}{}

% mystatement
\NewDocumentEnvironment{mystatement}{momo}{%
  % leftbar
  \begin{leftbar}
  % begin
  \ifthenelse{\equal{#1}{definition}}{%
    \IfValueTF{#2}{\begin{definition}[#2]}{\begin{definition}}%
  }{}%
  \ifthenelse{\equal{#1}{theorem}}{%
    \IfValueTF{#2}{\begin{theorem}[#2]}{\begin{theorem}}%
  }{}%
  \ifthenelse{\equal{#1}{proposition}}{%
    \IfValueTF{#2}{\begin{proposition}[#2]}{\begin{proposition}}%
  }{}%
  \ifthenelse{\equal{#1}{corollary}}{%
    \IfValueTF{#2}{\begin{corollary}[#2]}{\begin{corollary}}%
  }{}%
  \ifthenelse{\equal{#1}{lemma}}{%
    \IfValueTF{#2}{\begin{lemma}[#2]}{\begin{lemma}}%
  }{}%
  \ifthenelse{\equal{#1}{remark}}{%
    \IfValueTF{#2}{\begin{remark}[#2]}{\begin{remark}}%
  }{}%
  \ifthenelse{\equal{#1}{example}}{%
    \IfValueTF{#2}{\begin{example}[#2]}{\begin{example}}%
  }{}%
  % label
  \label{#3}%
  % reference
  \IfValueTF{#4} {%
    \extref{#4}%
  }%
  % break
  \blankbreak{}%
} {%
  % end
  \ifthenelse{\equal{#1}{definition}}{\end{definition}}{}%
  \ifthenelse{\equal{#1}{theorem}}{\end{theorem}}{}%
  \ifthenelse{\equal{#1}{proposition}}{\end{proposition}}{}%
  \ifthenelse{\equal{#1}{corollary}}{\end{corollary}}{}%
  \ifthenelse{\equal{#1}{lemma}}{\end{lemma}}{}%
  \ifthenelse{\equal{#1}{remark}}{\end{remark}}{}%
  \ifthenelse{\equal{#1}{example}}{\end{example}}{}%
  % space
  \vspace{2.5pt}
  % leftbar
  \end{leftbar}
}

% myproof
\NewDocumentEnvironment{myproof}{o}{%
  % begin
  \IfValueTF{#1}{%
    \begin{proof}[Proof #1.]%
  } {%
    \begin{proof}%
  }
  % break
  \blankbreak{}%
} {%
  % end
  \end{proof}%
}
% number
\newcommand{\N} {%
    \mathbb{N}%
}
\newcommand{\Z} {%
    \mathbb{Z}%
}
\newcommand{\Q} {%
    \mathbb{Q}%
}
\newcommand{\R} {%
    \mathbb{R}%
}
\newcommand{\C} {%
    \mathbb{C}%
}

% arrow
\newcommand{\xto}[1] {%
    \xrightarrow{#1}%
}
\newcommand{\isomto} {%
    \xto{\sim}%
}
\newcommand{\incl} {%
    \hookrightarrow{}%
}
\newcommand{\xincl}[1] {%
    \overset{#1}{\incl}%
}
\newcommand{\subm} {%
    \twoheadrightarrow{}%
}
\newcommand{\xsubm}[1] {%
    \overset{#1}{\subm}%
}

% set
\newcommand{\set}[2] {%
    \{#1:#2\}%
}
\newcommand{\coll}[1] {%
    \{#1\}%
}
% set :: operation
\newcommand{\cross}{\times}
\newcommand{\cmpl}[1]{#1^c}
% set :: subset
\newcommand{\sub} {%
    \subseteq{}%
}
\newcommand{\bus} {%
    \supseteq{}%
}

% Restrict
\newcommand{\restr}[1] {%
    \vert_{#1}%
}

% calculus
% calculus :: absolute
\newcommand{\abs}[1] {%
    \vert{} #1 \vert{}%
}
% calculus :: ball
\NewDocumentCommand{\ball}{O{1}O{0}}{%
    B_{#1} (0)%
}
\NewDocumentCommand{\clball}{O{1}O{0}}{%
    \overline{\ball[#1][#2]}%
}

% text
\newcommand{\tand} {%
    \text{ and }%
}
\newcommand{\tor} {%
    \text{ or }%
}
\newcommand{\twith} {%
    \text{ with }%
}

% function
\newcommand{\inv}[1] {%
    #1^{-1}%
}
\newcommand{\frst}[1] {%
    #1^{(1)}%
}
\newcommand{\snd}[1] {%
    #1^{(2)}%
}

% interval
\newcommand{\I} {%
    [0, 1]%
}
% interval :: cylinder
\newcommand{\cyl}[1] {%
    #1 \cross{} \I%
}
\newcommand{\cylup}[1] {%
    #1 \cross{} \{1\}%
}
\newcommand{\cyldown}[1] {%
    #1 \cross{} \{0\}%
}

% intern
\newcommand{\supp} {%
    \text{supp}%
}

% fraction
\newcommand{\half} {%
    \frac{1}{2}%
}

% variable
\newcommand{\eps} {%
    \varepsilon{}%
}

% newcommand
\newcommand{\case}[3] {
    \begin{cases} #1 & \text{if } #3 \\ #2 & \text{else } \end{cases}
}
\newcommand{\casenif}[2] {
    \begin{cases} #1 \\ #2 \end{cases}
}
% bundle
\newcommand{\bundledef}[5] {%
    #1: #2 \xto{#4} #3 \xto{#5} #2{}%
}
\newcommand{\bundle}[4] {%
    #1 \xto{#3} #2 \xto{#4} #1{}%
}
% bundle :: generic
\newcommand{\ba} {%
    \mathfrak{a}%
}
\NewDocumentCommand{\bb}{o}{
    \IfValueTF{#1}{\mathfrak{b}\restr{#1}}{\mathfrak{b}}
}
\newcommand{\bc} {%
    \mathfrak{c}%
}
\newcommand{\bd} {%
    \mathfrak{d}%
}
% bundle :: trivial
\newcommand{\be}[2][n]{
    \mathfrak{e}^{#1}_{#2}
}
% bundle :: tangent
\newcommand{\bt} {%
    \mathfrak{t}%
}
% bundle :: normal
\newcommand{\bn} {%
    \mathfrak{n}%
}

% induced
\newcommand{\ind}[1] {%
    #1^*%
}

% germ
\newcommand{\germdef}[3] {%
    #1: #2 \Rightarrow{} #3%
}
\newcommand{\germ}[2] {%
    #1 \Rightarrow{} #2%
}

% whitney
\newcommand{\whitney}[2] {%
    #1 \oplus{} #2%
}

% rooted
\newcommand{\bab} {%
    \ba\restr{a_0}%
}
\newcommand{\bbb} {%
    \bb[b_0]%
}
\newcommand{\bcb} {%
    \bc\restr{c_0}%
}

\newcommand{\defterm}[1] {
\emph{#1}
}
\newcommand{\myintref}[1] {
\Cref{#1}
}
\newcommand{\extref}[1] {
[#1]
}

\begin{document}
\begin{titlepage}
    \begin{center}
        \vspace{1.5cm}
        
        \begin{Large}Bachelorarbeit Mathematik\end{Large}
        
        \vspace*{1cm}

        \begin{LARGE}\textbf{Microbundles on Toplogical Manifolds}\end{LARGE}
             
        \vspace{1cm}
 
        \begin{large}\textbf{Florian Burger, 4229203}\end{large}

        \vfill
             
        \begin{large}
            Betreut durch Prof. Dr. Markus Banagl
            
            \vspace{0.5cm}

            Fakultät für Mathematik und Informatik\\
            Ruprecht-Karls-Universität Heidelberg\\
            Heidelberg, den 16. Juni 2024
        \end{large}
    \end{center}
 \end{titlepage}
\begin{abstract}
    This paper presents the concept of microbundles
    as introduced in 1964 by John Milnor.
    After showing some fundamental properties and constructions,
    including the induced microbundle and the Whitney sum,
    we discuss the concept of tangent- and normal microbundles
    over topological manifolds.
    We prove that for every microbundle over a
    simplicial complex, there exists an inverse in respect to the Whitney sum.
    Furthermore, we show that homotopic maps yield isomorphic induced microbundles.
    These results permit the proof
    that every submanifold $N \sub M$ has a normal microbundle
    in a stabilized surrounding manifold $M \cross \R^q$.
\end{abstract}
\tableofcontents
\clearpage
\section{Introduction}\label{chapter::introduction}
\subsection*{Motivation}\label{section::motivation}
\begin{myparagraph}
    % smootg
    % smooth :: tangent space
    Given a smooth manifold $M$,
    one can define the tangent space
    in a point $p \in M$ over `derivations' in $\mathcal{L}(C^\infty(M), \R^n)$
    or over `tangent curves' in $C^\infty((-1, 1), M)$.

    % smooth :: tangent bundle
    The tangent space allows for the definition of the tangent bundle
    \[ TM = \bigsqcup_{p \in M} T_p M\]
    together with the projection
    \[ TM \xto{\pi} M, \pi(p, \nu) = p. \]

    % topological
    Suppose we want to define the tangent bundle over a topological manifold $M$.
    % topological :: problem
    As we can see from the two ways of defining the tangent space,
    the classic tangent bundle requires the notion of differentiability.
    One intuitive approach would be to equip the topological manifold
    with a differentiable structure and use the same construction as for smooth manifolds.
    However, M. Kervaire showed in 1960 that there exists a $10$-dimensional manifold
    that does not admit a differentiable structure \cite{kervaire}.
    Furthermore, even if there existed a differentiable structure for our topological manifold,
    it is generally not unique. The $7$-sphere for example admits
    multiple different differentiable structures (see \cite{milnor7sphere}).

    Hence,
    we need to take a different approach than the one for the smooth case.
    
    % topological :: naive
    Another plausible approach would be to define the fibers of the tangent bundle
    to be of the form $\coll{p} \cross U_p$,
    where $U_p$ is a neighborhood of $p$ which is
    homeomorphic to $\R^d$ via a chart.
    However, this construction raises the problem
    of choosing neighborhoods $U_p$
    such that they vary continuously over $M$.
    Furthermore,
    it is questionable whether this construction is even a topological
    invariant if it depends on specific choices of neighborhoods $U_p$.

    % topological :: solution
    In 1964, John Milnor published `Microbundles, Part I',
    introducing a unique way to think
    about tangent bundles over topological manifolds.

    The core idea behind this approach is
    to drop the assumption
    that the tangent bundle is a vector bundle and hence
    every fiber must be homeomorphic to euclidean space.
    Instead, we require that the fibers are `germs' of euclidean space,
    i.e. topological spaces with an open subset homeomorphic to euclidean space.
    In contrast to the previous approach,
    we can now choose the neighborhoods $U_p$ of $p$ regardless of any corresponding charts.
    That is because we do not require anymore that these neighborhoods are euclidean spaces.
    Moreover, each neighborhood of $p$ contains
    the domain of a chart, hence satisfying our `germ' condition.

    If the respective neighborhoods can be chosen freely,
    we may as well always choose the entire space $M$ for the sake of simplicity.

    We conclude that our resulting total space is of the form $M \cross M$, which,
    analoguous to the smooth case, comes equipped with a projection
    \[ M \cross M \xto{\pi} M, \pi(m, m') = m. \]
    
    % topological :: microbundle
    In order to develop this approach, Milnor introduces a new type of bundle
    that generalizes this idea.
    He calls them `microbundles'.
    Many fundamental properties
    and constructions of vector bundles carry over other to microbundles,
    e.g. induced microbundles or the Whitney sum (see \myintref{chapter::constructions}).
    Moreover, Milnor shows that if a manifold can be equipped with a smooth structure,
    then the tangent vector bundle regarded as a microbundle is isomorphic to the
    tangent microbundle.

    % topological :: normal bundle
    The theory of microbundles over topological manifolds reaches even further,
    allowing for the definition of a microbundle analogue
    to the normal bundle.
    Given a smooth submanifold $P \sub M$,
    there always exists a normal bundle $NP$ defined fiber-wise by
    \[ N_p P = T_p M / T_p N. \]
    In contrast to this, the two mathematicians C. Rourke and B. Sanderson could construct a $19$-dimensional manifold
    embedded in $S^{29}$ that does not admit a normal microbundle \cite{rourke}.
    So the existence of a normal microbundle of a submanifold $N \sub M$ is not garantueed.
    Instead, Milnor could show that there always exists a normal microbundle
    of $N$ in a stabilization $M \cross \R^q$ for some $q \in \N$.

    % applications
    Milnors studies on microbundles allowed Robert Williamson
    to transfer results in cobordism theory developed by Réne Thom,
    particulary the concept of transverse regularity, from the smooth category
    over to the piecewise category (see \cite{williamson}).

    % thesis structure
    This thesis presents the concept of microbundles as introduced in 1964 by John Milnor,
    providing a proof for the above theorem.
    It is based on Milnors paper `Microbundles, Part I'.
\end{myparagraph}
\subsection*{Introduction to Microbundles}\label{section::microbundle}
\begin{myparagraph} This subsection introduces the concept of microbundles along with some basic properties. We clarify what a microbundle is, what it means for a microbundle to be trivial and cover some basic examples of microbundles, including the tangent microbundle over topological manifolds.

Throughout this thesis, we require that every manifold is paracompact and second-countable. Unless explicitly stated, manifolds are not equipped with a differentiable structure. \end{myparagraph} % def: microbundle
\begin{mydefinition}[microbundle]
    A \defterm{microbundle} $\bb$ over $B$ (with \defterm{fibre-dimension} $n$) is a diagram $\bundle{B}{E}{i}{j}$ satisfying the following:
    \begin{itemize}
        \item $B$ is a topological space (\defterm{base space})
        \item $E$ is a topological space (\defterm{total space})
        \item $i: B \to E$ (\defterm{injection}) and $j: E \to B$ (\defterm{projection}) are continuous maps such that $id_B = j \circ i$
        \item Every $b \in B$ is \defterm{locally trivializable}, that is there exist open neighborhoods $U \sub B$ of $b$ and $V \sub E$ of $i(U)$ with a homeomorphism $\phi: V \isomto U \times \R^n$ such that the following diagram commutes:
        \[
            \begin{tikzcd}[column sep=tiny]
                & V \ar[dr, "j\vert_V"] \ar[dd, "\psi"] & \\
                U \ar[ur, "i"] \ar[dr, "{(id, 0)}"'] & & U \\
                & U \times\R^n \ar[ur, "\pi_1"'] &
            \end{tikzcd}
        \]
    \end{itemize}
\end{mydefinition}

% rmk: omit fibre dimension
\begin{myremark}
    In the following, unless explicitly stated otherwise we assume the fiber dimension of any given microbundle to be $n$.
\end{myremark} \begin{mystatement}{lemma}{microbundle::local} The diagram $\bundle{B}{E}{i}{j}$ is locally trivial in $b \in B$ if and only if there exists a homeomorphism \[ \phi: V \isomto \phi(V) \sub B \times \R^n \] where $V$ is a neighborhood of $i(b)$ and $\phi(V)$ is neighborhood of $(b, 0)$ such that $\phi$ commutes as in \myintref{microbundle::definition}. \end{mystatement}

\begin{myproof} It suffices to show that we can derive local triviality in $b$ assuming only a homeomorphism $\phi: V \isomto \phi(V)$ with the properties required above.

Since $\phi(V)$ is a neighborhood of $(b, 0)$, there exists an open subset $U \sub B$ and $\eps > 0$ such that $U \times \ball[\eps] \sub \phi(V)$.

Note that there exists a homeomorphism \[ \mu_\eps: \ball[\eps] \isomto \R^n \twith \mu_\eps(0) = 0, \] for example given by $\mu_\eps(x) = \tan(\frac{\abs{x}\cdot\pi}{2\eps}) x$.

We construct a local trivialization $\phi': \inv{\phi}(U \times \ball[\eps]) \isomto U \times \R^n$ given by $\phi' = \mu_\eps \circ \phi$.

Commutativity with $i$ and $id \cross 0$ is given by \[ \phi'(i(b)) = \mu_\eps(\phi(i(b))) = \mu_\eps(b, 0) = (b, 0) = (id \cross 0)(b) \] and with $j$ and $\pi_1$ by \[ j(e) = \pi_1(\phi(e)) = \pi_1(\mu_\eps(\phi(e))) = \pi_1(\phi'(e)), \] which concludes the proof. \end{myproof} % foreword
Before we look at examples of microbundles, we should first clarify
what it means for two microbundles to be isomorphic.

% def: isomorphy
\begin{mydefinition}[isomorphy]
    Two microbundles $\bundledef{\bb_1}{B}{E_1}{i_1}{j_1}$ and $\bundledef{\bb_2}{B}{E_2}{i_2}{j_2}$ are \defterm{isomorphic} if
    there exist neighborhoods $V_1 \sub E_1$ of $i_1(B)$ and $V_2 \sub E_2$ of $i_2(B)$ with
    a homeomorphism $\phi: V_1 \isomto V_2$ such that the following diagram commutes: 
    \[
        \begin{tikzcd}
            & V_1 \ar[dr, "j_1\vert_{V_1}"] \ar[dd, "\phi"] & \\
            B \ar[ur, "i_1"] \ar[dr, "i_2"'] & & B \\
            & V_2 \ar[ur, "j_2\vert_{V_2}"'] &
        \end{tikzcd}
    \]
\end{mydefinition} % foreword
\begin{myparagraph}
    As the definition of isomorphy already indicates, when studying microbundles,
    we are not interested in the entire total space
    but only in an arbitrary small neighborhood of the base space (more precise, the image $i(B)$).
    The following proposition makes this even clearer.
    
    % Throughout the paper we will often take use of homeomorphisms
    % \[ \mu_\eps: \ball[\eps] \isomto \R^n \twith \mu_\eps(0) = 0 \]
    % for example given by
    % \[ x \mapsto \tan(\frac{\abs{x}\cdot\pi}{2\eps}) x \]
    % in order to show properties like local triviality.
\end{myparagraph}

% proposition
\begin{mystatement}{proposition}{microbundle::total}[54]
    Given a microbundle $\bundledef{\bb}{B}{E}{i}{j}$ over $B$,
    restricting the total space $E$ to an
    arbitrary neighborhood $E' \sub E$ of $i(B)$ leaves the microbundle unchanged.
    That is, the microbundle
    \[ \bundledef{\bb'}{B}{E'}{i}{j\restr{E'}} \]
    is isomorphic to $\bb$.
\end{mystatement}

% proof
\begin{myproof}
    We prove the proposition in two steps.
    \begin{steps}
        % microbundle
        \item $\bb'$ is a microbundle
        
        Continuity for $i$ and $j$ as well as $j\restr{E'} \circ i = id_B$ are already given since $\bb$ is a microbundle.

        So we only need to show that $\bb'$ is locally trivial.
        For an arbitrary $b \in B$, choose a local trivialization $(U, V, \phi)$ of $b$ in $\bb$.
        By restricting $\phi$ to $V \cap E'$,
        we obtain a homeomorphism on its image as required in \myintref{microbundle::local},
        hence showing local triviality.
        
        % isomorphy
        \item $\bb'$ is isomorphic to $\bb$

        Since $E'$ is a subset of $E$,
        we can simply use the identity $E' \to E' \sub E$
        as our homeomorphism between neighborhoods of $i(B)$.
        Furthermore, the injection and projection maps of $\bb$ and $\bb'$ are the same,
        so they clearly commute with the identity.
    \end{steps}
    This completes the proof.
\end{myproof} \begin{mystatement}{example}[trivial microbundle]{microbundle::trivial}[55] A microbundle $\bb$ over $B$ is \defterm{trivial} if it is isomorphic to $\be{B}$. \end{mystatement} 
\begin{myparagraph} We are primarily interested in microbundles over topological manifolds. In this case, as topological manifolds are paracompact and hausdorff, triviality has stronger implications for the total space. \end{myparagraph}

\begin{mystatement}{lemma}{microbundle::paracompact}[57] A microbundle $\bb$ over a paracompact hausdorff space $B$ is trivial if and only if there exists an open neighborhood $V$ of $i(B)$ such that $V \cong B \cross \R^n$ with injection and projection maps being compatible with this homeomorphism. \end{mystatement}

\begin{myparagraph} This means that there exists an open subset of $E(\bb)$ being homeomorphic to the entire $B \cross \R^n$, instead of only a neighborhood of $\cyldown{B}$ given by the definition of triviality. \end{myparagraph}

\begin{myproof} We show both implications.

`$\implies$'

By restricting $E(\bb)$ to an open neighborhood and applying \myintref{microbundle::total} if necessary, we may assume that the entire $E(\bb)$ is an open subset of $B \cross \R^n$.

Hence, there exist $B_i \sub B$ open and $0 < \eps_i < 1$ with $\bigcup_{i \in I} B_i = B$ such that \[ \bigcup_{i \in I } B_i \cross \ball[\eps_i] \sub E(\bb). \] Without loss of generality, we may assume that the collection $\{B_i\}$ is locally finite by refining this collection if necessary, utilizing the fact that $B$ is paracompact.

Furthermore, $B$ being paracompact hausdorff yields a partition of unity \[ f_i: B \to [0, 1] \twith \supp{f_i} \sub B_i\] such that $\sum_{i \in J}f_i = 1$.

We define a map $\lambda: B \to (0, \infty)$ with $\lambda = \sum_{i \in J} \eps_i f_i$, which has the property that $\abs{x} < \lambda(b) \implies (b, x) \in E(\bb)$ because \[ \abs{x} < \lambda(b) \] \[ \iff \abs{x} < \eps_{i_1} f_{i_1}(b) + \cdots + \eps_{i_n} f_{i_n}(b) \] \[ \iff 0 < (\eps_{i_1} - \abs{x}) f_{i_1}(b) + \cdots + (\eps_{i_n} - \abs{x}) f_{i_n}(b) \] \[ \implies \exists i \in J: 0 < (\eps_{i} - \abs{x}) f_{i}(b) \] \[ \implies (b, x) \in B_i \cross \ball[\eps_i] \implies (b, x) \in E(\bb). \]

Finally, we have a homeomorphism between the open subset $\set{(b, x) \in B \cross \R^n}{\abs{x} < \lambda(b)} \sub E(\bb)$ and $B \cross \R^n$ via \[ (b, x) \mapsto (b, \frac{x}{\lambda(b) - \abs{x}}). \] Since $(b, 0)$ is mapped to $(b, 0)$, it follows that the homeomorphism commutes with the injection and projection maps.

`$\impliedby$'

This is simply a weakening of the definition of triviality. \end{myproof} \begin{myparagraph} Next, we show that vector bundles can be regarded as microbundles. \end{myparagraph}

\begin{mystatement}{example}[underlying microbundle]{microbundle::underlying}[55] Let $\xi: E \xto{\pi} B$ be a $n$-dimensional vector bundle. The \defterm{underlying microbundle} $\abs{\xi}$ of $\xi$ is a microbundle \[ \bundledef{\abs{\xi}}{B}{E}{i}{\pi} \] where $i: B \to E$ denotes the \defterm{zero-cross section} of $\xi$, that is, the section that maps every $b \in B$ to the neutral element $0_b$ of its fiber $\inv{\pi}(b) \cong \R^n$. \end{mystatement}

\begin{myproof}[that $\abs{\xi}$ is a microbundle] First, $\pi$ is an open map:

Let $V \sub E$ be open. For every $b \in \pi(V)$, there exists a neighborhood $U_b$ together with a homeomorphism $\phi_b: \inv{\pi}(U_b) \isomto U_b \times \R^n$. It follows that $\pi\restr{\inv{\pi}(U_b)} = \pi_1 \circ \phi_b$. Hence, $\pi\restr{\inv{\pi}(U_b)}$ is open due to openness of $\pi_1$ and $\phi_b$.

We conclude from \[ \pi(V) = \bigcup_{b \in B} \pi\restr{\inv{\pi}(U_b)}(V) \] that $\pi$ is open.

Now from $\inv{i}(V) = \pi(V)$ it follows that $i$ is continuous. Additionally, $\pi \circ i = id_B$ since $\pi(i(b)) = \pi(0_b) = b$.

Local triviality is immediately inherited from the local triviality condition for vector bundles. \end{myproof}

\begin{myparagraph} This illustrates why we explicitly require an injection for the definition of the microbundle, in contrast to the vector bundle. For vector bundles, we have a canonical section from the base space to the total space given by the zero cross section. For microbundles, the fibers are not associated with a specific chart, hence the neutral element of some underlying Euclidean space is not well-defined. \end{myparagraph} % foreword
\begin{myparagraph}
    The following definition is the microbundle analog
    to the tangent vector bundle.
\end{myparagraph}

% definition
\begin{mystatement}{definition}[tangent microbundle]{microbundle::tangent}[55]
    The \defterm{tangent microbundle} $\bt_M$ over a topological $d$-manifold $M$ is a diagram
    \[ \bundle{M}{M \cross M}{\Delta}{\pi_1} \]
    where $\Delta(m) := (m, m)$ denotes the diagonal map.
\end{mystatement}

% proof
\begin{myproof}[that $\bt_M$ is a microbundle]
    % continuity and compatibility
    The maps $\Delta$ and $\pi_1$ are continuous and clearly $\pi_1 \circ \Delta = id_M$.

    % local triviality
    % local triviality :: homeomorphism
    For an arbitrary $p \in M$,
    choose a chart $(U, \psi)$ over $p$.
    We have a local trivialization $(U, U \cross U, \phi)$ of $p$ in $\bt_M$ given by
    \[ \phi: U \cross U \isomto U \cross \R^n \twith \phi(u, u') = (u, \psi(u) - \psi(u')). \]
    Homeomorphy of $\phi$ follows from homeomorphy of $\psi$.
    
    % local triviality :: commutativity
    Commutativity between the injection and $id \times 0$ is given by
    \[ \phi(\Delta(m)) = \phi(m, m) = (m, \psi(m) - \psi(m)) = (m, 0) = (id \cross 0)(m)\]
    and between the projection and $\pi_1$ by
    \[ \pi_1(u, u') = u = \pi_1(u, \snd{\phi}(u, u')) = \pi_1(\phi(u, u')). \]
    Note that $\snd{\phi}$ denotes the map on
    the second component of $\phi$, i.e. $\pi_2 \circ \phi$.
\end{myproof}

% remark
\begin{mystatement}{remark}{microbundle::tangentdim}
    The tangent microbundle $\bt_M$ has fiber dimension $d$.
\end{mystatement} \begin{myparagraph} The following statement is fundamental for the theory of microbundles over topological manifolds. It justifies that the tangent microbundle can be regarded as a generalization of the tangent vector bundle. \end{myparagraph}

\begin{mystatement}{theorem}{microbundle::equivalence}[56] Let $M$ be a smooth $d$-manifold. Then the underlying microbundle of $\xi: TM \to M$ and the tangent microbundle $\bt_M$ are isomorphic. \end{mystatement}

\begin{myproof} We equip $M$ with a Riemannian metric, which allows us to define the usual exponential map $\exp: V \to M$ where $V \sub TM$ is a neighborhood of the zero-cross section of $M$.

Consider $id \cross \exp$. Using the Inverse Function Theorem for smooth manifolds (see\cite[thm.4.5]{lee}) for arbitrary $(p, \nu) \in V$, it follows that $id \cross \exp$ is a local diffeomorphism and hence a local homeomorphism. Furthermore, the zero-cross section is mapped homeomorphically to the diagonal. By applying Lemma 4.1 from \cite[lm.4.1]{whitehead} (manifolds are locally compact and separable), it follows that $id \cross \exp$ maps a neighborhood of the zero-cross section to a neighborhood of the diagonal. Commutativity with the injection maps is given by \[ (id \cross \exp)(i_{\abs{\eta}}(p)) = (id \cross \exp)(p, 0) = (p, p) = \Delta(p) \] and with the projection maps by \[ j_{\abs{\eta}}(p, \nu) = p = \pi_1(p, \exp(\nu)) = \pi_1((id \cross \exp)(p, \nu)), \] which concludes the proof. \end{myproof}
\section{Standard Constructions}\label{chapter::constructions}
\begin{myparagraph}
This section introduces two standard constructions for microbundles, the `induced microbundle' and the `Whitney sum'. Both constructions have their vector bundle analogue and many results carry over immediately to microbundles.
\end{myparagraph}
\subsection*{Induced Microbundles}\label{section::induced}
% introduction
\begin{myparagraph}
    % sketch
    Given a microbundle $\bb$ over $B$ and a map $f: A \to B$,
    one can define a microbundle $\ind{f}\bb$ over $A$.
    This is achieved by `pulling back' the base space $B$ to $A$
    with the use of the map $f$.

    % roadmap
    After showing the existence of such a microbundle,
    we prove some basic properties
    such as triviality criteria and compatibility with map composition.
    Afterwards, we study induced microbundles
    over cones and simplicial complexes.
\end{myparagraph}
% definition
\begin{mystatement}{definition}[induced microbundle]{induced::definition}[58]
    Let $\bb$ be a microbundle over $B$ and let $f: A \to B$ be a map.
    The \defterm{induced microbundle} $\ind{f}\bb$ is a microbundle $\bundle{A}{E_f}{i_f}{j_f}$
    defined as follows:
    \begin{itemize}
        \item $E_f = \{ (a, e) \in A \cross E(\bb) \mid f(a) = j(e) \}$
        \item $i_f(a) = (a, (i \circ f)(a))$
        \item $j_f(a, e) = a$
    \end{itemize}
\end{mystatement}

% fiber bundle
\begin{myparagraph}
    The construction is identical to the one over vector bundles,
    more precisely over fiber bundles (compare to \cite[ch.2,sec.14]{brendon}).
\end{myparagraph}

% proof
\begin{myproof}[that $\ind{f}\bb$ is a microbundle]
    % continuity and compatibility
    Both $i_f$ and $j_f$ are continuous
    since they are composed by continuous functions.
    Additionally, $j_f(i_f(a)) = j_f(a, i(f(a))) = a$ and hence $j_f \circ i_f = id_A$.

    % local triviality
    It remains to be shown that $\ind{f}\bb$ is locally trivial.

    % local triviality :: homeomorphism
    For an arbitrary $a \in A$,
    choose a local trivialization $(U, V, \phi)$ of $i(a)$ in $\bb$.
    We construct a local trivialization of $a$ in $\ind{f}\bb$ as follows:
    \begin{itemize}
        \item $U_f = \inv{f}(U) \sub A$,
        which is an open neighborhood of $a$ since $f$ is continuous
        and $U$ is an open neighborhood of $i(a)$.
        \item $V_f = (U_f \cross V) \cap E_f \sub E_f$,
        which is an open neighborhood of $i_f(a)$
        since both $U' \cross V$ and $E_f$ are open neighborhoods of $i_f(a)$.
        \item $\phi_f: V_f \isomto U_f \cross \R^n$ with $\phi_f(a', e) = (a', \snd{\phi}(e))$
    \end{itemize}
    The existence of an inverse $\inv{\phi_f}(a', v) = (a', \inv{\phi}(f(a'), v))$
    and component-wise continuity for both $\phi_f$ and $\inv{\phi_f}$ show that $\phi_f$ is a homeomorphism.
    
    % local triviality :: commutativity
    Commutativity with $i_f$ and $id \times 0$ is given by
    \[ \phi_f(i_f(a')) = \phi(a', i(f(a'))) = (a', \snd{\phi}(i(f(a')))) = (a', 0) = (id \cross 0)(a') \]
    and with $j_f$ and $\pi_1$ by
    \[ j_f(a', e) = a'  = \pi_1(a', \phi(e)) = \pi_1(\phi_f(a', e)), \]
    which completes the proof. 
\end{myproof}
% statement
\example{\parttitle{restricted microbundle}} \\
Let $\bb: \bundle{B}{E}{i}{j}$ be a microbundle and $A \sub B$:
The induced microbundle $\iota^*\bb$ with $\iota: A \incl B$ being the inclusion map is called the \defterm{restricted microbundle} and
we write $\bb \restr{A} := \iota^*\bb$.
% remark
\begin{remark}
In the following, we'll consider $E(\bb\restr{A})$ a subset of $E(\bb)$.
This is justified because
$E(\bb\restr{A}) = \{ (a, e) \in A \times E(\bb) \mid a = j(e) \} \cong \{ e \in E(\bb) \mid j(e) \in A \} \sub E(\bb)$.
\end{remark}
\begin{myparagraph}
    Next, we provide two criteria
    to show that an induced microbundle is trivial.
\end{myparagraph}
% statement
\lemma{\parttitle{induced trivial microbundle}} \\
The induced microbundle $f^*\bb$ is trivial for every map $f: A \to B$, if $\bb$ is already trivial.
% proof
\begin{myproof}
Let $(V, \phi)$ be a global trivialization of $\bb$, i.e $V \cong_\phi B \times \R^n$.
Now define $V' := (A \times V) \cap E'$ and $\phi'(a, e) := (a, \phi^{(2)}(e))$.
Obviously, $V'$ is a neighborhood of $i'(A)$ and also $\phi'$ is a homeomorphism with inverse $\phi'^{(-1)}(a, x) = (a, \phi^{-1}(f(a), x))$.
\end{myproof}
\begin{mystatement}{lemma}{induced::const} Let $\bb$ be a microbundle over $B$. The induced microbundle $\ind{c_{A, b_0}}\bb$ over the constant map $c_{A, b_0}: A \to B$ with $c_{A, b_0}(a) = b_0$ is trivial. \end{mystatement}

\begin{myproof} The total space $E(\ind{c_{A, b_0}}\bb)$ is defined as \[ \set{(a, e) \in A \cross E(\bb)}{f(a) = b_0 = j(e)} = A \cross \inv{j}(b_0). \]

Let $(U, V, \phi)$ be a local trivialization for $b_0$ in $\bb$. Restricting $\phi$ to the fiber $\inv{j}(b_0)$ yields a homeomorphism \[ \phi\restr{\inv{j}(b_0)}: V \cap \inv{j}(b_0) \isomto b_0 \times \R^n. \] It follows that $\psi: A \cross (V \cap \inv{j}(b_0)) \isomto A \cross \R^n$ with $\psi(a, e) = (a, \snd{\phi}(e))$ is a homeomorphism as well.

The product $A \cross (V \cap \inv{j}(b_0))$ is open in $E(\ind{c_{A, b_0}}\bb)$, since $V \cap \inv{j}(b_0)$ is open in $\inv{j}(b_0)$ with the subspace topology. Furthermore, from \[ i_{c}(a) = (a, i(c_{A, b_0}(a))) = (a, i(b_0)) \tand \snd{\phi}(i(b_0)) = 0 \] it follows that $\psi(A \cross (V \cap \inv{j}(b_0)))$ is a neighborhood of $\cyldown{A}$. Hence, $\psi$ maps a neighborhood of $i_c(A)$ to a neighborhood of $\cyldown{A}$.

Commutativity with the injection maps is given by \[ \psi(i_c(a)) = \psi(a, i(b_0)) = (a, \snd{\phi}(i(b_0))) = (a, 0) = (id \cross 0)(a) \] and with the projection maps by \[ j_c(a, e) = a = \pi_1(a, \snd{\phi}(e)) = \pi_1(\psi(a, e)). \]

We conclude that $\ind{c_{A, b_0}}\bb$ is trivial. \end{myproof}
% statement
\proposition{\parttitle{composition}} \\
Let $A \xto{f} B \xto{g} C$ be topological spaces and $\mathfrak{c}: \bundle{C}{E}{i}{j}$ be a microbundle:
\[ (g \circ f)^*\mathfrak{c} \cong f^*(g^*\mathfrak{c}) \]
% proof
\begin{proof}
We'll compare the two total spaces and conclude that they are homeomorphic.
\begin{enumerate}
    \item $E((g \circ f)^*\mathfrak{c}) = \{ (a, e) \in A \times E(\mathfrak{c}) \mid g(f(a)) = j(e)\}$ 
    \item $E(f^*(g^*\mathfrak{c})) = \{ (a, (b, e)) \in A \times (B \times E(\mathfrak{c})) \mid f(a) = b$ and $ g(b) = j(e) \}$.
\end{enumerate}
We have the bijection $\phi: E((g \circ f)^*\mathfrak{c}) \isomto E(f^*(g^*\mathfrak{c}))$ with $\phi(a, e) := (a, (f(a), e))$ and $\phi^{-1}(a, (b, e)) = (a, e)$.
Additionally, $\phi$ is a homeomorphism because $\phi$ and $\phi^{-1}$ are componentwise continuous.
It's easy to see that $\phi$ respects both injection and projection, which concludes the proof.
\end{proof}
% foreword
For a topological space $X$, we define the \defterm{cone} of $X$ as 
\[ CX := X \times [0, 1] / X \times \{1\} \]
and for a map $f: A \to B$ the \defterm{mapping cone} of $f$ as
\[ B \sqcup_f CA := B \sqcup CA / \sim \]
where $(a, 0) \sim b :\iff f(a) = b$.
% statement
\lemma{(extending over a mapping cone)} \\
A microbundle $\bb$ over $B$ can be extended to a microbundle over the mapping cone $B \sqcup_f CA$ if and only if $f^*\bb$ is trivial.
% proof
\begin{proof}
We show both implications. \\
$\implies$: \\
Let $\bb'$ be an extension of $\bb$ over $B \sqcup_f CA$.
Considering $A \xto{f} B \incl B \sqcup_f CA$, the composition $\iota \circ f$ is null-homotopic with homotopy
\[ H_t(a) := [(a, t)] \]
Note that $H_0(a) = [(a, 0)] = [f(a)] = (\iota \circ f)(a)$ and $H_1(a) = [(a, 1)] = [(\tilde{a}, 1)] = H_1(\tilde{a})$.
\\ $\ximplies{Hom. Thm.} (\iota \circ f)^*\bb'$ is trivial \\
Since $(\iota \circ f)^*\bb' = f^*(\iota^*\bb') = f^*\bb$, it follows that $f^*\bb$ is trivial.
$\impliedby$: \\
Let $f^*\bb$ be trivial.
Analogous to the cone, we define the \defterm{cylinder} of $X$ as
\[ MX := X \times [0, 1] \]
and for a map $f: A \to B$ the mapping \defterm{cylinder} of $f$ as
\[ B \sqcup_f MA := B \sqcup MA / \sim \]
where $(a, 0) \sim b :\iff f(a) = b$.
In contrast to the mapping cone, there exists a natural retraction from the mapping cylinder to the attached space
\[ \pi: B \sqcup_f MA \to B; \pi([(a, t)]) := f(a) \]
and therefore the induced microbundle $\pi^*\bb$ over $B \sqcup_f MA$.
Considering $A \times \{1\} \incl B \sqcup_f MA \xto{\pi} B$, we see that $\pi \circ \iota \cong f$ and therefore
\[ \pi^*\bb\restr{A \times \{1\}} = (\pi \circ \iota)^*\bb \cong f^*\bb = \mathfrak{e}^n_A\]
is trivial. From the lemma of induced trivial microbundles and $(a, t) \mapsto (a, 1)$ it follows that $\pi^*\bb\restr{A \times [\frac{1}{2}, 1]}$ is trivial.
\\ $\implies \exists \phi: E(\bb\restr{A \times [\frac{1}{2}, 1]}) \isomto A \times [\frac{1}{2}, 1] \times \R^n$ \\
Now we explicitly construct the desired extended microbundle $\bundledef{\bb'}{B \sqcup_f CA}{E'}{i'}{j'}$
\begin{itemize}
    \item $E' := E(\bb\restr{A \times [\frac{1}{2}, 1]}) / \phi^{-1}(A \times [\frac{1}{2}, 1] \times \{x\})$ (for every $x \in \R^n$)
    \item $i' := \pi \circ i$ the projection $i$ to $E'$
    \item $j'([e]) := [j(e)]$ is well defined, because $[e] = [\tilde{e}] \implies [j(e)] = [j(e')]$
\end{itemize}
Now that we have constructed $\bb'$, this proves the claim.
\end{proof}
% statement
\corollary{(extending over a d-simplex)} \\
Let $B$ be a $(d + 1)$-simplicial complex, $B'$ it's $d$-skeleton and $\Delta^{d + 1} \cong \sigma \sub B$.
A microbundle $\bb$ over $B'$ can be extended to a microbundle over $B' \cup \sigma$ if and only if $\bb \restr{\partial \sigma}$ is trivial.
% proof
\begin{proof}
The statement follows from the last lemma: \\
There exists a $\phi: C\partial \sigma \isomto \sigma$ such that $\phi(\partial \sigma \times \{0\}) = \partial \sigma$.
We explicitly construct $\phi((t_1, \dots, t_{d + 1}), \lambda) := (1 - \lambda) (t_1, \dots, t_{d + 1}) + \frac{\lambda}{d + 1} (1, \dots, 1)$.
It's easy to see that $\phi$ suffices all our requirements.
By choosing $f: \partial \sigma \incl B'$ and applying the last lemma, the statement is proven.
\end{proof}
\subsection*{The Whitney Sum}\label{section::whitney}
\begin{myparagraph} Given two vector bundles $E$ and $F$ over the same base space $X$, one can define the Whitney sum $\whitney{E}{F}$ by forming the direct sum of the individual fibers $E_x$ and $F_x$, hence the notation.

This construction carries over to microbundles, as elaborated in the following. The centerpiece of this section will be \myintref{whitney::theorem}, which states that for microbundles over simplicial complexes, one can find an `inverse' microbundle such that their Whitney sum is trivial. \end{myparagraph} % definition
\definition{\parttitle{whitney sum}} \\
Let $\bb_1$ and $\bb_1$ be two microbundles over a topological space $B$.
We define the \defterm{whitney sum} $\bb_1 \oplus \bb_2$ as follows:
\begin{itemize}
    \item $E := \{ (e_1, e_2) \in E(\bb_1) \times E(\bb_2) \mid j_1(e_1) = j_2(e_2)\}$
    \item $i(b) := (i_1(b), i_2(b))$
    \item $j(e_1, e_2) := j_1(e_1) = j_2(e_2)$
\end{itemize}
% proof
\begin{proof}
Let $b \in B$. \\
Choose $U_1, V_1, \phi_1$ and $U_2, V_2, \phi_2$ accordingly from the local trivialization of $b$ over $\bb_1$ and $\bb_2$:
\begin{itemize}
    \item $U := U_1 \cap U_2$
    \item $V := (V_1 \times V_2) \cap E$
    \item $\phi: V \to U \times \R^{n_1 + n_2}; \phi(e_1, e_2) := (\phi_1^{(1)}(e_1), \phi_1^{(2)}(e_1) \times  \phi_2^{(2)}(e_2))$
\end{itemize}
Note that $\phi_1^{(1)}(e_1) = \phi_2^{(1)}(e_2)$.
Local triviality follows directly from it's components.
\end{proof} % statement
\lemma{(compatibility)} \\
Let $\bb_1$ and $\bb_1$ be two microbundles over $B$ and $f: A \to B$ a map.
Induced microbundle and whitney sum are compatible, i.e. $f^*(\bb_1 \oplus \bb_2) \cong f^*\bb_1 \oplus f^*\bb_2$
% proof
\begin{proof}
From the definition of the induced microbundle and the whitney sum, we can derive the total spaces:
\begin{enumerate}
    \item $E(f^*(\bb_1 \oplus \bb_2)) = \{(a, (e_1, e_2)) \in A \times (E_1 \times E_2) \mid j_1(e_1) = j_2(e_2) = f(a) \}$
    \item $E(f^*\bb_1 \oplus f^*\bb_2) = \{((a_1, e_1), (a_2, e_2)) \in (A \times E_1) \times (A \times E_2) \mid j(a_1, e_1) = j(a_2, e_2)$ and $f(a_i) = j(e_i)\}$
\end{enumerate}
Those two total spaces are homeomorphic via $\phi(a, (e_1, e_2)) := ((a, e_1), (a, e_2))$ and
$\phi^{-1}((a, e_1), (a, e_2)) = (a, (e_1, e_2))$. $\phi$ and $\phi^{-1}$ are continuous because they are componentwise continuous.
Obviously, $\phi \circ i = i$ and $\phi \circ j = j$, which concludes the proof. 
\end{proof} % foreword
\begin{myparagraph}
    Lastly, we show the above mentioned theorem about Whitney sums
    which will be essential in the proof of Milnors Theorem.

    For its prove, we need the following proposition
    which will be shown in \myintref{chapter::suspension}.
\end{myparagraph}

% bouquet proposition
\begin{mystatement}{proposition}[Bouquet Lemma]{whitney::bouquet}[59]
    Let $\bb$ be a microbundle over a `bouquet' of spheres $B$, meeting in a single point.
    Then there exists a map $r: B \to B$ such that $\whitney{\bb}{\ind{r}\bb}$ is trivial.
\end{mystatement}

% theorem trivialize
\begin{mystatement}{theorem}{whitney::theorem}[59]
    Let $\bb$ be a microbundle over a $d$-simplicial complex $B$.
    Then there exists a microbundle $\bn$ over $B$ such that
    the Whitney sum $\whitney{\bb}{\bn}$ is trivial.
\end{mystatement}

% proof trivialize
\begin{myproof}
    We prove the theorem by induction over $d$.

    % start
    (Start of induction)

    A $1$-simplicial complex is just a bouquet of circles.
    Hence, the start of induction follows directly from \myintref{whitney::bouquet}.   

    % step
    (Inductive Step)

    Let $B'$ be the $(d - 1)$-skeleton of $B$ and let $\bn'$ be its corresponding microbundle
    such that $\whitney{\bb[B']}{\bn'}$ is trivial.

    \begin{enumerate}
        % simplex
        \item $\whitney{\bn'}{\be{B'}}$ can be extended over any $d$-simplex $\sigma$:

        Consider the equation
        \[
            (\whitney{\bn'}{\be{B'}})\restr{\partial\sigma}
            = \whitney{\bn'\restr{\partial\sigma}}{\be{B'}\restr{\partial\sigma}}
            = \whitney{\bn'\restr{\partial\sigma}}{\bb[\partial\sigma]}
            = (\whitney{\bn'}{\bb[B']})\restr{\partial\sigma}
        \]
        in which we used the previous lemma and \myintref{induced::simplex}
        for $\be{B'}\restr{\partial\sigma} = \bb[\partial\sigma]$.
        Since $(\whitney{\bn'}{\bb[B']})\restr{\partial\sigma}$ is trivial, the claim follows from \myintref{induced::simplex}.

        % extend
        \item $\whitney{\bn'}{\be{B'}}$ can be extended over $B$:

        The difficulty is that the individual $d$-simplices are not well-seperated.
        Let $B''$ denote $B$ with small open $d$-cells removed from every $d$-simplex.
        Since $B'$ is a retract of $B''$ we can extend $\whitney{\bn'}{\be{B'}}$ to a microbundle $\nu$ over $B''$.

        Now we can extend $\nu$ over $B$ by taking all extensions of $\nu$
        over every simplex
        using (1.), and glueing its total spaces together along $E(\nu)$.
        Similarly, the injection and projection can be obtained
        by glueing the injection and projection maps over every simplex together.

        We denote the resulting microbundle by $\eta$.

        % trivial
        \item
        Consider the mapping cone $B \sqcup_\iota CB'$ over the inclusion $B' \incl B$.
        Since
        \[
            (\whitney{\bb}{\eta})\restr{B'}
            = \whitney{\bb[B']}{\eta\restr{B'}}
            = \whitney{\bb[B']}{(\whitney{\bn'}{\be{B'}})}
            = \whitney{(\whitney{\bb[B']}{\bn'})}{\be{B'}}
            = \whitney{\be{B'}}{\be{B'}}
        \]
        % ! associativity not shown
        is trivial, it follows from \myintref{induced::cone} that
        we can extend $\whitney{\bb}{\eta}$ over $B \sqcup_\iota CB'$,
        which will be denoted by $\xi$.

        The mapping cone $B \sqcup_\iota CB'$ has the homotopy type of a bouquet of spheres
        by carrying $B'$ along $CB'$ collapsing to a single point.
        Since any $d$-simplex is homotopic to a $d$-disc and its boundary is collapsed,
        we obtain the homotopy of a $(d - 1)$-sphere.
        
        Using \myintref{homotopy::theorem} and \myintref{whitney::bouquet},
        we conclude that there exists a microbundle $\bn$ such that $(\whitney{\xi}{\bn})\restr{B}$ is trivial.
        The equation
        \[
            \be{B}
            = (\whitney{\xi}{\bn})\restr{B}
            = \whitney{\xi\restr{B}}{\bn\restr{B}}
            = \whitney{(\whitney{\bb}{\eta})}{\bn\restr{B}}
            = \whitney{\bb}{(\whitney{\eta}{\bn\restr{B}})}
        \]
        completes the proof.
    \end{enumerate}
\end{myproof}
\section{The Homotopy Theorem}\label{chapter::homotopy}
\begin{myparagraph}
In this section, we will prove the Homotopy Theorem, which is a fundamental result over microbundles. It states the following.
\end{myparagraph}
\begin{mystatement}{theorem}[Homotopy Theorem]{homotopy::theorem}[58] Let $\bb$ be a microbundle over $B$ and let $f, g: A \to B$ be two maps where $A$ is paracompact hausdorff. If $f$ and $g$ are homotopic, then $\ind{f}\bb$ and $\ind{g}\bb$ are isomorphic. \end{mystatement}
\begin{myparagraph}
In order to prove this theorem, we introduce the concept of map-germs over microbundles which provides another way to think about isomorphy besides \myintref{microbundle::isomorphy}.
\end{myparagraph}
\subsection*{Map-Germs}\label{section::germs}{\blankbreak}
% germ
\begin{mystatement}{definition}[map-germ]{homotopy::germ}[65]
    A \defterm{map-germ} $\germdef{F}{(X, A)}{(Y, B)}$
    between topological pairs $(X, A)$ and $(Y, B)$
    is an equivalence class of maps $(X, A) \to (Y, B)$
    where $f \sim g \iff f\restr{U} = g\restr{U}$
    for an arbitrary neighborhood $U \sub X$ of $A$.
\end{mystatement}

% composition
\begin{myparagraph}
    We can form the composition of two map-germs
    $\germdef{F}{(X, A)}{(Y, B)}$ and $\germdef{G}{(Y, B)}{(Z, C)}$
    by choosing representatives $f: U_f \to Y$ and $g: U_g \to Z$
    and defining $(f \circ g) \restr{\inv{f}(U_g)}$ to be
    a representative for $G \circ F$.
\end{myparagraph}

% homeomorphism
\begin{mystatement}{definition}[homeomorphism-germ]{homotopy::homeomorphism}[65]
    A \defterm{homeomorphism-germ} $\germdef{F}{(X, A)}{(Y, B)}$ is a map-germ
    such that there exists a representative $f: U_f \to Y$
    that maps $U_f$ homeomorphically to a neighborhood of $B$.
\end{mystatement}
% foreword
\begin{myparagraph}
    Let $\bb$ and $\bb'$ be two isomorphic microbundles over $B$.
    There exists a homeomorphism $\psi: V \isomto V'$ where
    $V \sub E(\bb)$ is a neighborhood of $i(B)$
    and $V' \sub E(\bb')$ is a neighborhood of $i'(B)$.
    We can view $\psi$ as a representative for a homeomorphism-germ
    \[ \germdef{[\psi]}{(E, i(B))}{(E', i'(B))}. \]
    
    Studying isomorphy between $\bb$ and $\bb'$
    using map-germs is useful,
    because we don't care what $\psi$ does on its initial domain,
    but only what it does on arbitray small neighborhoods of $i(B)$.
    Hence, every representative of $[\psi]$
    describes the `same' isomorphy between $\bb$ and $\bb'$.
    Now, naturally, the question arises whether
    the existence of a homeomorphism-germ
    \[ \germdef{F}{(E, i(B))}{(E', i'(B))} \]
    already implies that $\bb$ and $\bb'$ are isomorphic.
    The answer is generally no, because isomorphy between microbundles
    additionally requires 
    the homeomorphism to commute with the injection and projection maps.
    Therefore, we need to require an extra condition (`fiber-preservation')
    for this implication to be true.
    This justifies the following definition.
    
    % precondition
    Let $\germdef{J}{(E(\bb), i(B))}{(B, B)}$ and $\germdef{J'}{(E(\bb'), i(B))}{(B, B)}$
    denote the map-germs represented by the projections of $\bb$ and $\bb'$.
\end{myparagraph}

% definition
\begin{mystatement}{definition}[isomorphism-germ]{homotopy::isomorphism}[65]
    An \defterm{isomorphism-germ} between $\bb$ and $\bb'$ is a homeomorphism-germ 
    \[ \germdef{F}{(E(\bb), B)}{(E(\bb'), B)} \]
    which is \defterm{fiber-preserving}, that is $J' \circ F = J$.
\end{mystatement}

% remark
\begin{mystatement}{remark}{homotopy::isomorphismremark}[65]
    There exists an isomorphism-germ between $\bb$ and $\bb'$
    if and only if $\bb$ and $\bb'$ are isomorphic.
\end{mystatement}
% generalize
\begin{myparagraph}
    We can take this even further by dropping the assumption
    that the base spaces of the two microbundles equal.
    Note that in this case no comparison to isomorphy can be drawn,
    because we haven't defined isomorphy between microbundles over different base spaces.
\end{myparagraph}

% definition
\begin{mystatement}{definition}[bundle-germ]{homotopy::bundle}[66]
    Let $\bb$ and $\bb'$ be two microbundles over $B$ and $B'$
    with the same fiber dimension.
    A \defterm{bundle-germ} $\germdef{F}{\bb}{\bb'}$ is a map-germ
    \[ \germdef{F}{(E(\bb), B)}{(E(\bb'), B')} \]
    such that there exists a representative $f: U_f \to E(\bb')$
    that maps each fiber $\inv{j}(b)$ injectively to a fiber $\inv{j'}(b')$.
\end{mystatement}

% diagram
\begin{myparagraph}
    For a bundle-germ $\germdef{F}{\bb}{\bb'}$, the following diagram commutes:
    \begin{center}
        \begin{tikzcd}
            (E(\mathfrak{b}), B) \arrow[r, Rightarrow, "F"] \ar[d, "i"] & (E(\mathfrak{b}'), B') \ar[d, "i'"] \\
            B \ar[r, "F\vert_B"] & B' \\
        \end{tikzcd}
    \end{center}
    We say $F$ \defterm{is covered by} $F\restr{B}$.
    % intentionally left blank
    
\end{myparagraph}
% foreword williamson
\begin{myparagraph}
    The bundle-germ is indeed a generalization of the isomorphism-germ,
    as the following proposition shows.
\end{myparagraph}

% williamson
\begin{mystatement}{proposition}[Williamson]{homotopy::williamson}[66]
    Let $\bb$ and $\bb'$ be two microbundles over $B$.
    A bundle-germ $\germdef{F}{\bb}{\bb'}$ covering
    the identity map is an isomorphism-germ.
\end{mystatement}

% foreword
\begin{myparagraph}
    The following lemma is necessary for the proof of the proposition.    
\end{myparagraph}

% lemma
\begin{mystatement}{lemma}{homotopy::ball}
    If a homeomorphism $f: \clball[2] \isomto V \sub \R^n$ satisfies
    \[ \abs{f(x) - x} < 1, \forall x \in \clball[2], \]
    then $\clball[1] \sub V$.
\end{mystatement}

% proof lemma
\begin{myproof}[of the lemma]
    We provide a proof by contradiction:

    Suppose $v \in \clball[1] - V$.
    Let $[f(0), v]$ denote the line-segment
    \[ \set{\lambda f(0) + (1 - \lambda) v}{\lambda \in \I} \sub \R^n. \]
    The intersection $[f(0), v] \cap V$ is compact
    as $[f(0), v]$ is compact and $V$ is closed.
    Hence, the intersection has a maximum $v'$ when ordered via $\lambda$.
    Note that the intersection is non-empty since $f(0) \in V$.

    The maximum $v'$ is contained in $\partial V$:
    \begin{enumerate}
        \item $v' \in [f(0), v] \cap V \sub V = \overline{V}$
        \item $v' \notin \inner{V}$, because otherwise
        $\ball[\eps][v'] \sub \inner{V}$ for some $\eps > 0$
        contradicting the maximum condition for $v'$.
    \end{enumerate}

    From $\abs{f(0) - 0} < 1$ and $v \in \clball[1]$ it follows that $\abs{v'} < 1$. 

    This leads to a contradiction, because
    \[ \inv{f}(v') \in \partial\clball[2] \implies \abs{\inv{f}(v') - v'} = 2 - \abs{v'} > 2 - 1 = 1. \]
\end{myproof}
\begin{myproof}[of the proposition] Let $f: U_f \to E(\bb')$ be a representative for $F$. First we assume a special case. Then we use this result show that $f$ is open in some neighborhood of every $v \in U_f$, hence $f$ being open. \begin{steps} \item Let $f$ map from $B \cross \R^n$ to $B \cross \R^n$

Since $F$ covers the identity, $f$ is of the form \[ f(b, x) = (b, g_b(x)) \] where $g_b: \R^n \to \R^n$ are individual maps. Since the $g_b$ are continuous and injective, it follows from the Invariance of Domain Theorem (see \cite[cor.19.9]{brendon}) that the $g_b$ are open maps.

Let $(b_0, x_0) \in B \cross \R^n$ and let $\eps > 0$. Since $g_{b_0}$ is an open map, there exists a $\delta > 0$ such that $\clball[2\delta][x_1] \sub g_{b_0}(\clball[\eps][x_0])$ where $x_1 = g_{b_0}(x_0)$.

We claim that there exists a neighborhood $V \sub B$ of $b_0$ such that \[ \abs{g_b(x) - g_{b_0}(x)} < \delta \] for every $b \in V$ and $x \in \clball[\eps][x_0]$.

To show this, consider $\phi(b, x) = g_b(x) - g_{b_0}(x)$. The open set $\inv{\phi}(\ball[\delta])$ is a neighborhood of $\{b_0\} \cross \R^n$ since $\phi(b_0, x) = 0$. Hence, there exist open subsets $V_x \sub B$ and $W_x \sub \R^n$ such that \[ \bigcup_{x \in \clball[\eps][x_0]} V_x \cross W_x \sub \inv{\phi}(\clball[\delta]) \] and $x \in W_x$. Since $\clball[\eps][x_0]$ is compact, there exist $x_1, \dots, x_n \in \clball[\eps][x_0]$ with $\clball[\eps][x_0] \sub \bigcup_{i = 1}^{n} V_{x_i}$. The claim follows with $V = V_{x_1} \cap \dots \cap V_{x_n}$ which is open by forming the intersection over finitely many open sets.

Now we want to apply the previous lemma.

Consider the homeomorphism $(g_b \circ \inv{g_{b_0}})\restr{\clball[2\delta][x_1]}$ for an arbitrary $b \in V$. Together with \[ \clball[2\delta][x_1] \sub g_{b_0}(\clball[\eps][x_0]) \implies \inv{g_{b_0}}(\clball[2\delta][x_1]) \sub \clball[\eps][x_0], \] we conclude from the above that \[ \abs{(g_b \circ \inv{g_{b_0}})(x) - x} < \delta, \forall x \in \clball[2\delta][x_1]. \] It follows that, by translation and scaling, $g_b \circ \inv{g_{b_0}}\restr{\clball[2\delta][x_1]}$ satisfies the conditions of \myintref{homotopy::ball}. Therefore, $\clball[\delta][x_1] \sub (g_b \circ \inv{g_{b_0}})(\clball[2\delta][x_0])$ and hence $\clball[\delta][x_1] \sub g_b(\clball[\eps][x_0])$. From \[ V \cross \clball[\delta][x_1] \sub g(V \cross \clball[\eps][x_0]) \] it follows that $f$ is an open map.

\item Gluing together $f: U_f \to E(\bb')$ along local trivializations

For an arbitrary $b \in B$, choose local trivializations $(U, V, \phi)$ and $(U', V', \phi')$ of $b$ in $\bb$ and $\bb'$. Without loss of generality, we may assume that $U = U'$ by choosing $V = \inv{\phi}(U \cap U')$ and $V' = \inv{\phi'}(U \cap U')$ and restricting $\phi$ and $\phi'$ accordingly if necessary.

First, we restrict $f$ to $V \cap \inv{f}(V')$. Since $V \cap \inv{f}(V')$ is an open neighborhood of $i(b)$, we can choose an open neighborhood $U_b \sub U$ of $i(b)$ and $\eps_b > 0$ such that $\inv{\phi}(U_b \cross \ball[\eps_b]) \sub V \cap \inv{f}(V')$.

We define a map $U_b \cross \R^n \to U_b \cross \R^n$ given by \[ U_b \cross \R^n \cong U_b \cross \ball[\eps_b] \xto{\phi \circ f \circ \inv{\phi}} U_b \cross \R^n \] that is injective and fiber-preserving, and hence an open map (Step 1). It follows that $f: \inv{\phi}(U_b \cross \ball[\eps_b]) \to V'$ must be an open map, as the other composing maps are homeomorphisms.

We conclude from \[ f = \bigcup_{b \in B}f\restr{\inv{\phi}(U_b \cross \ball[\eps_b])} \] that $f$ is an open map. \end{steps} This completes the proof. \end{myproof}

\begin{myparagraph} We can easily generalize this result to bundle-germs between microbundles over different base spaces. \end{myparagraph}

\begin{mystatement}{corollary}{homotopy::corollary}[67] If a map $g: B \to B'$ is covered by a bundle-germ $\germdef{F}{\bb}{\bb'}$, then $\bb$ is isomorphic to the induced microbundle $\ind{g}\bb'$. \end{mystatement}

\begin{myproof} Let $f: U_f \to E'$ be a representative map for $F$. We define $\germdef{F'}{\bb}{\ind{g}\bb'}$ by the representative \[ f': U_f \to E(\ind{g}\bb') \twith f'(e) = (j(e), f(e)). \] Every $f'(e)$ lies in $E(\ind{g}\bb')$ because $g(j(e)) = j'(f(e))$ as we can see from the commutative diagram for bundle-germs.

The germ $F'$ is a bundle-germ covering the identity because \[ j(e) = j_g'(j(e), f(e)) = j_g'(f'(e)) \] and because $f'$ is injective ($f$ is injective). Applying the previous proposition on $F'$ proves the claim. \end{myproof}
\subsection*{Proving the Homotopy Theorem}\label{section::homotopy_prove}{\blankbreak}
% statement
\lemma{\parttitle{glueing together bundle map-germs}} \\
Let $\bb$ be a microbundle over $B$ and $\{B_\alpha\}$ a locally finite collection of closed sets covering $B$.
Additionally, we are given $\bgerm{F_\alpha}{\bb\restr{B_\alpha}}{\bb'}$, a collection of bundle map-germs with
$F_\alpha = F_\beta$ on $\bb\restr{B_\alpha \cap B_\beta}$
Then there exists a bundle map-germ $\bgerm{F}{\bb}{\bb'}$ extending $F_\alpha$.
% proof
\begin{proof}
Choose representative maps $f_\alpha: U_\alpha \to E'$ for $F_\alpha$ with $U_\alpha$ open.
Since $F_\alpha = F_\beta$ on $\bb\restr{B_\alpha \cap B_\beta}$, $f_\alpha = f_\beta$ for an open neighborhood $U_{\alpha\beta}$ of $B_\alpha \cap B_\beta$.
We define
\[ U := \{ e \in E \mid j(e) \in B_\alpha \cap B_\beta \implies e \in U_{\alpha\beta} \}\]
\begin{enumerate}
    \item $U$ is open: \\
    Let $e \in U$ and $j(e) \in B_\alpha \cap B_\beta$.
    From local finiteness there exists an open neighborhood $V \sub B$ of $j(e)$ with $V \sub B_{\alpha_1} \cap \ldots \cap B_{\alpha_n}$.
    W.l.o.g. $V \sub B_\alpha \cap B_\beta$ by exluding a finite number of closed sets if necessary.
    Now $V_{\alpha\beta} := j^{-1}(V) \cap U_{\alpha\beta}$ is an open neighborhood of $e$.
    Since $j(e)$ can only be contained in finitely many $B_\alpha$ we can form the intersection of all these $V_{\alpha'\beta'}$
    which, by construction, is contained in $U$ and is open.
    \item $B \sub U$ considering the cases $U_{\alpha\alpha}$.
\end{enumerate}

Now we can define $f: U \to E'$ in the obvious way
\[ f(u \in U_{\alpha\beta}) := f_\alpha(u) = f_\beta(u) \]
which is a representative map for our desired $F$.
\end{proof}
\begin{scope}
    % defines
    \newcommand{\bbleft} {
        \bb[B {\cross} [0, \half]]
    }
    \newcommand{\bbright} {
        \bb[B {\cross} [\half, 1]]
    }
    \newcommand{\bbhalf} {
        \bb[B {\cross} \coll{\half{}}]
    }

    % % foreword
    % \begin{myparagraph}
    %     Using the previous lemma,
    %     we can show a criteria on whether a microbundle
    %     over a cylinder base space is trivial.        
    % \end{myparagraph}

    % lemma
    \begin{mystatement}{lemma}{homotopy::lemma1}[67]
        Let $\bb$ be a microbundle over $\cyl{B}$.
        If $\bbleft$ and $\bbright$ are both trivial,
        then $\bb$ itself is trivial.
    \end{mystatement}

    % proof
    \begin{myproof}
        Consider the identity bundle-germ over $\bbhalf$,
        which is defined as the bundle-germ represented by the identity map on $E(\bbhalf)$.
        
        Since $\bbright$ and $\bbhalf$ are both trivial,
        there exist isomorphism-germs
        \[
            \germdef{R}{\bbright}{\be{B {\cross} [\half, 1]}} \tand
            \germdef{L}{\bbhalf}{\be{B {\cross} \coll{\half}}}.
        \]
        We define a bundle-germ $\germdef{M}{\be{B \cross [\half, 1]}}{\be{B \cross \coll{\half}}}$
        represented by
        \[ (b, t, x) \mapsto (b, \half, x). \]
        The composition $\inv{L} \circ M \circ R$ then yields a bundle-germ $\germ{\bbright}{\bbhalf}$
        that extends the identity on $\bbhalf$.

        Using the previous lemma, we can glue this together with the
        identity over $\bbleft$
        (note that the bundle-germs agree on $\bbhalf$)
        resulting in a bundle-germ $\germ{\bb}{\bbleft}$.

        \myintref{homotopy::corollary} infers that $\bb$
        is isomorphic to $\ind{r}\bbleft$
        where
        \[ r: \cyl{B} \to B \cross [0, \half] \twith r(b, t) = (b, \min(t, \half)). \]
        But $\bbleft$ is trivial, hence $\ind{r}\bbleft$ is
        trivial as well (see \myintref{induced::trivial}).
        We conclude that $\bb$ is trivial.
    \end{myproof}
\end{scope}
% lemma
\begin{mystatement}{lemma}{homotopy::lemma2}[67]
    Let $\bb$ be a microbundle over $\cyl{B}$.
    Then every $b \in B$ has a neighborhood $V$ such that $\bb[\cyl{V}]$ is trivial.
\end{mystatement}

% proof
\begin{myproof}
    Let $b \in B$ be arbitrary.

    For every $t \in \I$, assume a neighborhood
    $U_t = V_t \cross (t - \eps_t, t + \eps_t)$ of $(b, t)$
    such that $\bb[U_t]$ is trivial.
    Such a neighborhood can be constructed by taking
    a local trivialization $(U', V', \phi')$ of $(b, t)$ in $\bb$
    and restricting $U'$ accordingly.

    Since $\cyl{\coll{b}}$ is compact, we can choose finitely many
    \[ V_1 \cross (t_1 - \eps_1, t_1 + \eps_1), \dots, V_n \cross (t_n - \eps_n, t_n + \eps_n) \]
    covering $\{b\} \cross \I$ and define $V = V_1 \cap \dots \cap V_n$.

    The restricted microbundles $\bb[V \cross (t_i - \eps_i, t_i + \eps_i)]$ are trivial
    as every $\bb[U_t]$ is trivial and $V \cross (t_i - \eps_i, t_i + \eps_i) \sub U_t$.
    It follows that there exists a subdivision $0 = t_0 < \cdots < t_k = 1$ such that every $\bb[V \cross [t_{i}, t_{i + 1}]]$ is trivial.
    
    By iteratively applying the previous lemma on the $\bb[V \cross [t_i, t_{i + 1}]]$,
    we conclude that $\bb[\cyl{V}]$ is trivial.
\end{myproof}
\begin{mystatement}{lemma}{homotopy::paracompact} Let $B$ be a paracompact space and let $\coll{V_\alpha}$ be a locally finite open cover of $B$. Then there exists a locally finite closed cover $\coll{\overline{B_\beta}}$ of $B$ such that every $\overline{B_\beta}$ intersects with only finitely many $V_{\alpha_1}, \dots V_{\alpha_n}$. \end{mystatement}

\begin{myproof} For every $b \in B$, there exists an open neighborhood $U_b$ of $b$ that intersects with only finitely many $V_{\alpha_1}, \dots V_{\alpha_k}$ due to local finiteness of $\coll{V_\alpha}$. Clearly, the collection $\coll{U_b}$ over all $b \in B$ covers $B$.

Since $B$ is paracompact, there exists a locally finite subcover $\coll{B_\beta}$.

The collection $\coll{\overline{B_\beta}}$ then meets our requirements: \begin{sectionize} \item $\coll{\overline{B_\beta}}$ is locally finite

For an arbitrary $b \in B$, let $W_b$ be an open neighborhood of $b$ that intersects only finitely many $B_{\beta_1}, \dots, B_{\beta_k}$. Now $W_b$ intersects only $\overline{B_{\beta_1}}, \dots, \overline{B_{\beta_1}}$, because \[ W_b \cap B_\beta = \emptyset \] \[ \implies B_\beta \sub B - W_b \] \[ \implies \overline{B_\beta} \sub \overline{B - W_b} = B - W_b \] \[ \implies W_b \cap \overline{B_\beta} = \emptyset. \]

\item Every $\overline{B_\beta}$ intersects only finitely many $V_{\alpha_1}, \dots V_{\alpha_k}$

Since $B_\beta \sub U_b$ for some $b \in B$, $B_\beta$ intersects only finitely many $V_{\alpha_1}, \dots V_{\alpha_k}$. By applying the same reasoning as in (1), it follows that $\overline{B_\beta}$ intersects with the same $V_{\alpha_1}, \dots V_{\alpha_k}$. \end{sectionize} This completes the proof. \end{myproof}
\begin{scope}
    % defines
    \newcommand{\A} {
        A_\alpha{}
    }

    % lemma
    \begin{mystatement}{lemma}{homotopy::lemma3}[67]
        Let $\bb$ be a microbundle over $\cyl{B}$ where $B$ is paracompact hausdorff.
        Then there exists a bundle-germ $\germdef{R}{\bb}{\bb[\cylup{B}]}$
        covering the retraction $r: \cyl{B} \to \cylup{B}$ with $r(b, t) = (b, 1)$.
    \end{mystatement}

    % proof
    \begin{myproof}
        % partition of unity
        First, we assume a locally finite covering $\{V_\alpha\}$ of open sets
        where $\bb[\cyl{V_\alpha}]$ is trivial.
        The existence of such a covering is justified
        by \myintref{homotopy::lemma2} and paracompactness of $B$.

        This cover can be equipped with a partition of unity 
        \[ \lambda_\alpha: B \to \I \twith \supp \lambda_\alpha \sub V_\alpha \]
        since $B$ is paracompact hausdorff, that is rescaled in way that
        \[ \max_\alpha(\lambda_\alpha(b)) = 1, \forall b \in B. \]
        Such a rescaling can be achieved by dividing $\lambda_\alpha$ by $\max_\alpha \lambda_\alpha$
        which is well-defined because $\coll{V_\alpha}$ is locally finite and continuous
        because the $\max$ function is continuous.
        Also, $\max_\alpha \lambda_\alpha(b) > 0$ since the initial
        partition of unity adds up to $1$ in every point.

        % construction
        Now we define a retraction $r_\alpha: \cyl{B} \to \cyl{B}$ with
        \[ r_\alpha(b, t) = (b, \max(t, \lambda_\alpha(b))). \]

        In the following,
        we construct bundle-germs $\germdef{R_\alpha}{\bb}{\bb}$
        covering $r_\alpha$ and `compose' them to obtain the required bundle-germ.
        \begin{enumerate}
            % local bundle-germ
            \item 
            We can divide $B \cross \I$ into two subsets
            \[
                A_\alpha = \cyl{\supp{\lambda_\alpha}} \sub \cyl{V_\alpha} \tand
                A'_\alpha = \set{(b, t)}{t \ge \lambda_\alpha(b)}.
            \]
            Since $\bb[A_\alpha]$ is trivial, we can,
            analogous to the proof of \myintref{homotopy::lemma1},
            extend the identity bundle-germ on $\bb[\A \cap \A']$ to a bundle-germ
            \[ \germ{\bb[\A]}{\bb[\A \cap \A']} \]
            using the bundle-germ
            \[ \germ{\be{\A}}{\be{\A \cap \A'}} \twith (a, x) \mapsto (r_\alpha(a), x). \]
            Pieced together with the identity bundle-germ $\bb[\A']$
            (note that $\A$ and $\A'$ are both closed),
            we obtain a bundle-germ $R_\alpha$ covering $r_\alpha$.
            
            % glue together
            \item
            Lastly, we construct a bundle-germ $R$ using the $R_\alpha$.

            Applying the well-ordering theorem,
            which is equivalent to the axiom of choice (see\cite[p.14]{kuczma}),
            we can assume an ordering of $\coll{V_\alpha}$.

            Let $\coll{B_\beta}$ be a locally finite closed cover of $B$
            such that $B_\beta$ intersects only
            finitely many $V_{\alpha_1} < \cdots < V_{\alpha_k}$
            obtained by \myintref{homotopy::paracompact}.

            Now the composition
            $R_{\alpha_1} \circ \ldots \circ R_{\alpha_k}$ restricts to a bundle-germ 
            \[ \germdef{R(\beta)}{\bb[\cyl{B_\beta}]}{\bb[\cylup{B_\beta}]} \]
            covering the retraction $(b, t) \mapsto (b, 1)$.
            That is because for every $b \in B_\beta$,
            we find an $1 \le i \le k$ with
            $\lambda_{\alpha_i}(b) = 1$ and hence $r_{\alpha_i}(b, t) = (b, 1)$.
            
            Pieced together using \myintref{homotopy::lemma1},
            we obtain a bundle-germ
            \[ R: \bb[\cyl{B}] \to \bb[\cylup{B}] \]
            covering $(b, t) \mapsto (b, 1)$.
        \end{enumerate}    
    \end{myproof}
\end{scope}
% foreword
\begin{myparagraph}
    Finally, we gathered all the tools to proof the Homotopy Theorem.
\end{myparagraph}

% proof
\begin{myproof}[of the Homotopy Theorem]
    The previous lemma yields a bundle-germ
    \[ \germdef{R}{\ind{H}\bb}{\ind{H}\bb[\cylup{A}]} \]
    covering the retraction $(a, t) \mapsto (a, 1)$.

    By restricting $R$ to $\ind{H}\bb[\cyldown{A}]$, we obtain a bundle-germ
    \[ \germ{\ind{H}\bb[\cyldown{A}]}{\ind{H}\bb[\cylup{A}]} \]
    covering $\theta: \cyldown{A} \isomto \cylup{A}$ with $\theta(a, 0) = (a, 1)$.
    \myintref{homotopy::corollary} then infers that
    $\ind{H}\bb[\cyldown{A}] \cong \ind{\theta}(\ind{H}\bb[\cylup{A}])$.
    
    Considering $\cyldown{A} = A$,
    we can identify $\ind{H}\bb[\cyldown{A}]$ with $\ind{f}\bb$ as follows:
    \[
        \ind{H}\bb[A \cross \{0\}] =
        \ind{\iota}(\ind{H}\bb) \cong
        \ind{(H \circ \iota)}\bb =
        \ind{f}\bb
    \]
    Analogously, we can identify
    $\ind{\theta}(\ind{H}\bb[\cylup{A}])$ with $\ind{g}\bb$.

    Together with
    $\ind{H}\bb[\cyldown{A}] \cong \ind{\theta}(\ind{H}\bb[\cylup{A}])$,
    it follows that $\ind{f}\bb \cong \ind{g}\bb$.
\end{myproof}
\section{Rooted Microbundles and the Bouquet Lemma}\label{chapter::suspension}
\begin{myparagraph}
In this section, we provide a proof for the Bouquet Lemma stated in \myintref{section::whitney}. To this end, we introduce the concept of `rooted microbundles', which allows us to define the wedge sum of two microbundles in a precise manner. Additionally, we show a version of the Homotopy Theorem that is compatible with rooted-microbundles.

Throughout this section, we assume that every topological space is equipped with an arbitrary base point which we will denote with subscript $0$.
\end{myparagraph}
\subsection*{Rooted Microbundles}\label{section::rooted}
% definition
\begin{mystatement}{definition}{suspension::rooted}[69]
    % rooted
    A \defterm{rooted microbundle} $\bb$ over $B$ is a microbundle
    over $B$ together with an isomorphism-germ
    \[ \germdef{R}{\bbb}{\be{b_0}}. \]
    % isomorphy
    Two rooted microbundles $\bb$ and $\bb'$ are \defterm{rooted isomorphic}
    if there exists an isomorphism-germ $\germ{\bb}{\bb'}$ extending
    \[ \germdef{\inv{R'} \circ R}{\bbb}{\bb'\restr{b_0}}. \]
\end{mystatement}

% remark
\begin{mystatement}{remark}{suspension::rooting}
    One can always define a rooting for a given microbundle 
    by choosing a local trivialization in the base point and restricting
    it to the fiber of $b_0$.
\end{mystatement}
\begin{mystatement}{definition}{suspension::induced}[57] Let $\bb$ be a rooted microbundle over $B$ and let $f: A \to B$ be a based map. We equip the induced microbundle $\ind{f}\bb$ with the rooting \[ \germdef{R_f}{E(\ind{f}\bb[a_0]) = a_0 \cross E(\bb[b_0])}{e^n_{a_0}} \] that coincides with $R$ if we consider $a_0 \cross E(\bb[b_0]) = E(\bb[b_0])$ and $e^n_{a_0} = e^n_{b_0}$. \end{mystatement}

\begin{myparagraph} The total space $E(\ind{f}\bb[a_0])$ is the same as $a_0 \cross E(\bbb)$ because \[ E(\ind{f}\bb[a_0]) = \set{(a, e) \in A \cross E(\bb)}{a = a_0 \tand f(a) = b_0 = j(e)} \] \[ = a_0 \cross \set{e \in E(\bb)}{j(e) = b_0} = a_0 \cross E(\bbb). \]

\end{myparagraph}
% foreword theorem
\begin{myparagraph}
    Given a rooted microbundle $\bb$ and homotopic based maps $f, g: A \to B$,
    the Homotopy Theorem yields that $\ind{f}\bb$ and $\ind{g}\bb$
    are isomorphic (not rooted-isomorphic).
    
    With the preliminary work in \myintref{section::homotopy_prove}, we can derive
    a version of the Homotopy Theorem that also accounts for rooted isomorphy.
\end{myparagraph}

% theorem
\begin{mystatement}{theorem}[Rooted Homotopy Theorem]{suspension::homotopy}[69]
    Let $\bb$ be a rooted microbundle over $B$ and $f, g: A \to B$ be two based maps
    where $A$ is paracompact hausdorff.
    If there exists a homotopy $H: \cyl{A} \to B$ between $f$ and $g$ that leaves the base point fixed,
    then the two rooted microbundles $\ind{f}\bb$ and $\ind{g}\bb$ are rooted isomorphic.
\end{mystatement}

% foreword lemma
\begin{myparagraph}
    In order to proof this,
    we need to show a `rooted version' of \myintref{homotopy::lemma2}.
    
    First, note that 
    \[
        E(\ind{H}\bb[\cyl{a_0}]) = E(\ind{\iota}(\ind{H}(\bb)))
        \cong E(\ind{(H \circ \iota)}\bb) = E(\ind{c_{\cyl{a_0}, b_0}}\bb),
    \]
    whose total space is of the form $(a_0 \cross [0, 1]) \cross E(\bb)$.
    Based on this, we can define an isomorphism-germ
    \[ \germdef{\overline{R}}{\ind{H}\bb[\cyl{a_0}]}{\be{\cyl{a_0}}} \]
    represented by
    \[ \overline{r}(a_0, t, v) = (a_0, t, \snd{r}(v)), \]
    where $r: V \to b_0 \cross \R^n$ is a representative for $R$.
    The representative $\overline{r}$ is a homeomorphism on its image,
    as its components are homeomorphisms on their image. 
\end{myparagraph}

% lemma
\begin{mystatement}{lemma}{suspension::sharper}[69]
    Let $\bb$ be a rooted microbundle over $B$ and
    let $H: \cyl{A} \to B$ be a map that leaves the base point fixed.
    Then there exists a neighborhood $V$ of $a_0$ together with an isomorphism-germ
    \[ \germ{\ind{H}\bb[\cyl{V}]}{\be{\cyl{V}}} \]
    extending $\overline{R}$ (as defined above).
\end{mystatement}

% proof lemma
\begin{myproof}
    By applying \myintref{homotopy::lemma2},
    it follows that there exists an isomorphism-germ
    \[ \germdef{Q}{\ind{H}\bb[\cyl{V}]}{\be{\cyl{V}}} \]
    for a sufficiently small neighborhood $V$ of $a_0$.
    However, $Q$ does not extend $\overline{R}$ in general.

    In order to fix this, consider
    \[ \germdef{Q \circ \inv{\overline{R}}}{\be{\cyl{a_0}}}{\be{\cyl{a_0}}} \]
    together with a representative $f: U_f \to (\cyl{a_0}) \cross \R^n$.

    Similar to the construction of $\overline{R}$, we can construct an isomorphism-germ
    \[ \germdef{P}{\be{\cyl{V}}}{\be{\cyl{V}}} \]
    extending $Q \circ \inv{\overline{R}}$ represented by
    \[ p(a, t, x) = (a, f(a_0, t, x)) \]
    considering $f(a_0, t, x) \in \I \cross \R^n$.

    Restricted to $\be{\cyl{a_0}}$, $P$ agrees with $Q \circ \inv{\overline{R}}$ and thus
    \[
        \inv{Q} \circ P\restr{\be{\cyl{a_0}}}
        = (\inv{Q} \circ (Q \circ \inv{\overline{R}}))
        = ((\inv{Q} \circ Q) \circ \inv{\overline{R}})
        = \inv{\overline{R}}.
    \]
    % \[ \implies (\inv{P} \circ Q)\restr{\ind{H}\bb[\cyl{a_0}]} = \overline{R}. \]
    Since $P$ and $Q$ are both isomorphism-germs,
    \[ \germdef{\inv{P} \circ Q}{\ind{H}\bb[\cyl{V}]}{\be{\cyl{V}}} \]
    is an isomorphism-germ extending $\overline{R}$.
\end{myproof}

% foreword proof
\begin{myparagraph}
    We are now able to show the Rooted Homotopy Theorem.

    To understand the proof,
    it is useful to have the constructions of \myintref{homotopy::lemma3} in mind,
    because we will modify them slightly in order to preserve the rootings.
\end{myparagraph}

% proof theorem
\begin{myproof}[of the Rooted Homotopy Theorem]
    We need to show that $\ind{f}\bb$ and $\ind{g}\bb$ are rooted isomorphic,
    that is there exists an isomorphism-germ $\germ{\ind{f}\bb}{\ind{g}\bb}$
    extending $\inv{R_g} \circ R_f = I$
    where $I$ denotes the identity germ.

    For the initial Homotopy Theorem,
    we constructed a bundle-germ
    \[ \germdef{F}{\ind{H}\bb}{\ind{H}\bb[\cylup{A}]} \]
    covering $(a, t) \mapsto (a, 1)$
    and restricted it to $\ind{H}\bb[\cyldown{A}]$.
    The required isomorphism-germ was then obtained by
    identifying $\ind{f}\bb$ with $\ind{H}\bb[\cyldown{A}]$ and
    $\ind{g}\bb$ with $\ind{H}\bb[\cyldown{A}]$.

    We must make slight modifications
    to the construction of $F$ such that it extends
    $\germ{\ind{f}\bbb \cong \ind{H}\bb[\cyldown{a_0}]}{\ind{H}\bb[\cylup{a_0}] \cong \ind{g}\bbb}$
    represented by
    \[ (a_0, e) = ((a_0, 0), e) \mapsto ((a_0, 1), e) = (a_0, e). \]

    This can be achieved by choosing a locally finite open cover $\coll{V_\alpha}$
    of $A$ (as in \myintref{homotopy::lemma3}), removing the base point $a_0$ from every set
    and adding $V$ obtained from \myintref{suspension::sharper}.
    Since $a_0 \in V$, the resulting collection is still a
    locally finite open cover of $A$.
    
    In the following, we will denote constructions over $V$
    with subscript $V$ and constructions over the other sets
    from the cover with subscript $\alpha$.

    We continue with the proof of \myintref{homotopy::lemma3}.
    Note that $\lambda_V(a_0) = 1$.
    That is because we removed $a_0$ from every other set and hence $\lambda_\alpha(a_0) = 0$.

    Lastly, we construct the extension $R_V$ for $r_V$
    like in \myintref{section::homotopy_prove},
    but instead of choosing an arbitrary trivialization
    $E(\ind{H}\bb\restr{A_V}) \cong A_V \cross \R^n$
    for the construction we use a representative $r$
    for the bundle-germ constructed in \myintref{suspension::sharper}.
    
    This has the advantage that the representative
    \[
        E(\ind{H}\bb[A_V]) \xto{r}
        A_V \cross \R^n \xto{r_V \cross id} (A_V \cap A'_V) \cross \R^n
        \xto{\inv{r}} E(\ind{H}\bb[A_V \cap A'_V])
    \]
    for $R_V$ maps elements $((a_0, 0), e)$ to $((a_0, 1), e)$.
    Additionally,
    every other $R_\alpha$ leaves $\ind{H}\bb[\cyldown{a_0}]$ unaffected
    because $r_\alpha(a_0, t) = (a_0, \max(\underbrace{\lambda_\alpha(t)}_{= 0}, t)) = (a_0, t)$.

    It follows that, by piecing together the $R_\alpha$ and $R_V$ like in \myintref{homotopy::lemma3},
    we obtain a bundle germ $\germdef{F}{\ind{H}\bb}{\ind{H}\bb[\cylup{A}]}$
    that extends $\inv{R_g} \circ R_f$.
    This completes the proof.
\end{myproof}
% foreword
\begin{myparagraph}
    Now that we introduced rooted-microbundles,
    we are able to define the wedge sum.
    As we will see in the subsequent proof,
    we require rootings for its definition.
    Particularly, the wedge sum depends on the specific choices of these rootings,
    justifying the requirement for rooted-microbundles.

    Given a quotient space $A \sqcup B / \sim$ and maps $f: A \to C$ and $g: B \to C$, we define
    $f \cup g: (A \sqcup B / \sim) \to C$ by
    \[ x \mapsto \begin{cases} f(x) & \text{if } x \in A \\ g(x) &\text{if } x \in B \end{cases}.\]
    Clearly, this map is only well-defined if $a \sim b \implies f(a) = g(b)$.
\end{myparagraph}

% definition
\begin{mystatement}{definition}{suspension::wedge}[70] % ? remove citation
    Let $\ba$ and $\bb$ be two rooted microbundles over $A$ and $B$.
    % microbundle
    The \defterm{wedge sum} $\ba \vee \bb$ of $\ba$ and $\bb$ is a microbundle
    \[ A \vee B \xto{i_a \cup i_b} E(\ba \vee \bb) \xto{j_a \cup j_b} A \vee B \]
    with the total space defined as
    \[ (E(\ba) \sqcup E(\bb)) / f(e_a) \sim e_a \]
    where $f: W_a \isomto W_b$ is a representative for $\inv{R_b} \circ R_a$.
    
    % rooting
    We equip $\ba \vee \bb$ with a rooting
    \[ \germdef{R}{E((\ba \vee \bb)\restr{a_0})}{\be{a_0}} \]
    represented by any representative for $R_a$ (or $R_b$).
\end{mystatement}

% proof
\begin{myproof}[that $\ba \vee \bb$ is a (rooted) microbundle]
    Let $f: W_a \isomto W_b$ be a representative for $\inv{R_b} \circ R_a$.

    % rooted microbundle
    \begin{sectionize}
        \item $\ba \vee \bb$ is a rooted microbundle
        \begin{itemize}
            % injection
            \item The injection map $i_a \cup i_b$ is well-defined because 
            \[ [i(a_0)] = [i_a(a_0)] = [f(i_a(a_0))] = [i_b(b_0)] = [i(b_0)] \]
            and continuous, since both $i_a$ and $i_b$ are continuous.
            % projection
            \item The projection map $j_a \cup j_b$ is well-defined because
            \[ \forall e \in W_a: [j(e)] = [j_a(e)] = [a_0] = [b_0] = [j_b(f(e))] = [j(f(e))] \]
            and continuous, since both $j_a$ and $j_b$ are continuous.
            % compatibility
            \item The composition $j \circ i$ is the identity because
            \[ \forall a \in A: j(i(a)) = j(i_a(a)) = j_a(i_a(a)) = a \]
            since $j_a \circ i_a = id_A$ (symmetrical for $B$).
        \end{itemize}
        % local triviality
        It remains to be shown that $\ba \vee \bb$ is locally trivial.

        Let $x \in A \vee B$.
        For reasons of symmetry, we may assume that $x \in A$.
        \begin{caselist}
            % trivial case
            \item $x \neq a_0$
            
            % construct
            Choose a local trivialization $(U, V, \phi)$ for $x$ in $\ba$.
            Without loss of generality,
            we may assume that $U \cap B = \emptyset$ by subtracting
            $\{a_0\}$ from $U$ if necessary.
            Note that $\{a_0\}$ is closed, since $A$ is hausdorff.
            
            % show
            Now we can simply use this trivialization for $\ba \vee \bb$ because
            $U \sub A$ is open in $A \vee B$ and
            $V \sub E(\ba)$ is open in $E(\ba \vee \bb)$.
            Furthermore, since $i$ and $j$ reduce to $i_a$ and $j_a$,
            it follows that $\phi$ commutes with $i$ and $id \cross 0$
            as well as with $j$ and $\pi_1$.

            % special case
            \item $x = a_0$
            
            Let $(U_a, V_a, \phi_a)$ and $(U_b, V_b, \phi_b)$ be local trivializations
            for $a_0 = b_0$ in $\ba$ and $\bb$.

            % open
            Since $W_a \sub E(\bab)$ is open,
            there exists an open subset
            $W_a' \sub E(\ba)$ such that $W_a = W_a' \cap E(\bab)$.

            Let $U_a' \sub A$ be an open neighborhood of $a_0$ and $\eps > 0$ such that
            \[ U_a' \cross \ball[\eps] \sub \phi_a(W_a' \cap V_a) \]
            and define $V_a' = \inv{\phi_a}(U_a' \cross \ball[\eps])$.
            This allows us to construct a homeomorphism
            $\phi_a': V_a' \isomto \phi_a'(V_a') \sub A \cross \R^n$ given by
            \[ \phi_a'(e) = (j_a(e), (\snd{\phi_b} \circ f \circ \inv{\phi_a})(a_0, \snd{\phi_a}(e))). \]
            We can now show local triviality in $a_0$ using the homeomorphism
            \[ \phi_a' \cup \phi_b: V_a' \cup V_b \isomto \phi_a'(V_a' \cup V_b) \sub (A \vee B) \cross \R^n. \]
            This map is well-defined because
            \[ \phi_a'(e) = (a_0, (\snd{\phi_b} \circ f \circ \inv{\phi_a})(a_0, \snd{\phi_a}(e))) \]
            \[ = (b_0, \snd{\phi_b}(f(e))) = (j_b(f(e)), \snd{\phi_b}(f(e))) = \phi_b(f(e)). \]

            Commutativity with $i_a \cup i_b$ and $id \cross 0$
            as well as between $j_a \cup j_b$ and $\pi_1$
            are inherited from $\phi_a$ and $\phi_b$.
            Note that $\phi_a(i_a(a)) = (a, 0) = \phi_a'(i_a(a))$.

            Applying \myintref{microbundle::local} yields that $\ba \vee \bb$ is locally trivial.
        \end{caselist}
        
        % well-defined
        \item $\ba \vee \bb$ is well-defined

        Let $f'$ be another representative for $\inv{R_b} \circ R_a$ and $(\ba \vee \bb)'$ the resulting wedge sum.
        We need to find an isomorphism-germ that extends $\inv{R'} \circ R$.
        
        In order to do this,
        choose an open neighborhood $V \sub E(\ba\restr{a_0})$
        of $i_a(a)$ where $f$ and $f'$ agree.
        
        By subtracting the closed set $\inv{j_a}(a_0) - V$
        from $E(\ba \vee \bb)$ and $E(\ba \vee \bb)'$,
        the microbundles remain unchanged due to \myintref{microbundle::total}.
        
        But now the total spaces $E(\ba \vee \bb)$ and $E((\ba \vee \bb)')$ are the same.
        That is because $E(\ba \vee \bb)$ and $E((\ba \vee \bb)')$
        could only possibly differ in $\inv{j_a}(a_0) - V$.
        
        Furthermore, since injection and projection are defined the same,
        it follows that the identity $\germ{(\ba \vee \bb)}{(\ba \vee \bb)'}$
        is an isomorphism-germ.
        Together with
        \[ \inv{R'} \circ R = \inv{R} \circ R = I, \]
        this completes the proof.
    \end{sectionize}
\end{myproof}
\begin{myparagraph}
    In this chapter, we provide a proof for the Bouquet Lemma in \ref{whitney::bouquet}.
    For this purpose we introduce so called `rooted microbundles'.
    They allow us to properly define the wedge sum of two microbundles.
    We will also show a version of the Homotopy Theorem
    that is compatible with rooted-microbundles.

    Throughout this chapter,
    we assume that every topological space comes with an arbitrary base point
    which we will denote with subscript $0$.
\end{myparagraph}
\section{Rooted Microbundles}\label{section::rooted}
% definition
\begin{mystatement}{definition}{suspension::rooted}
    % rooted
    A \defterm{rooted microbundle} $\bb$ over $B$ is a microbundle
    over $B$ together with an isomorphism-germ
    \[ \germdef{R}{\bbb}{\be{b_0}}. \]
    % isomorphy
    Two rooted microbundles $\bb$ and $\bb'$ are \defterm{rooted isomorphic}
    if there exists an isomorphism-germ $\germ{\bb}{\bb'}$ extending
    \[ \germdef{\inv{R'} \circ R}{\bbb}{\bb'\restr{b_0}}. \]
\end{mystatement}

% remark
\begin{mystatement}{remark}{suspension::rooting}
    One can always define a rooting for a given microbundle 
    by choosing a local trivialization in the base point and restricting
    it to the fiber of $b_0$.
\end{mystatement}
% definition
\begin{mystatement}{definition}{suspension::induced}
    Let $\bb$ be a rooted microbundle over $B$ and $f: A \to B$ a based map.
    The \defterm{induced microbundle} of $f$ over $\bb$
    is the initial induced microbundle $\ind{f}\bb$ together with the rooting
    \[
        \germdef{\ind{f}R}{E(\ind{f}\bb[a_0])
        = a_0 \cross E(\bb[b_0])}{e^n_{a_0}}
    \]
    that coincides with $R$ if we consider
    $a_0 \cross E(\bb[b_0]) = E(\bb[b_0])$ and $e^n_{a_0} = e^n_{b_0}$.
\end{mystatement}

% afterword
\begin{myparagraph}
    Note that the total space $E(\ind{f}\bb[a_0])$ equals $a_0 \cross E(\bbb)$ because
    \[
        E(\ind{f}\bb[a_0])
        = \set{(a, e) \in A \cross E(\bb)}{a = a_0 \land f(a) = b_0 = j(e)}
    \]
    \[
        = a_0 \cross \set{e \in E(\bb)}{j(e) = b_0}
        = a_0 \cross E(\bbb).
    \]
\end{myparagraph}
% foreword theorem
\begin{myparagraph}
    Given a rooted microbundle $\bb$ and homotopic based maps $f, g: A \to B$,
    the Homotopy Theorem yields that $\ind{f}\bb$ and $\ind{g}\bb$
    are isomorphic (not rooted-isomorphic).
    
    With the preliminary work in \myintref{section::homotopy_prove}, we can derive
    a version of the Homotopy Theorem that also accounts for rooted isomorphy.
\end{myparagraph}

% theorem
\begin{mystatement}{theorem}[Rooted Homotopy Theorem]{suspension::homotopy}
    Let $\bb$ be a rooted microbundle over $B$ and $f, g: A \to B$ be two based maps
    where $A$ is paracompact hausdorff.
    If there exists a homotopy $H: \cyl{A} \to B$ between $f$ and $g$ that leaves the base point fixed,
    then the two rooted microbundles $\ind{f}\bb$ and $\ind{g}\bb$ are rooted isomorphic.
\end{mystatement}

% foreword lemma
\begin{myparagraph}
    In order to proof this,
    we need to show a `rooted version' of \myintref{homotopy::lemma2}.
    
    First, note that 
    \[
        E(\ind{H}\bb[\cyl{a_0}]) = E(\ind{\iota}(\ind{H}(\bb)))
        \cong E(\ind{(H \circ \iota)}\bb) = E(\ind{c_{\cyl{a_0}, b_0}}\bb),
    \]
    whose total space is of the form $(a_0 \cross [0, 1]) \cross E(\bb)$.
    Based on this, we can define an isomorphism-germ
    \[ \germdef{\overline{R}}{\ind{H}\bb[\cyl{a_0}]}{\be{\cyl{a_0}}} \]
    represented by
    \[ \overline{r}(a_0, t, v) = (a_0, t, \snd{r}(v)), \]
    where $r: V \to b_0 \cross \R^n$ is a representative for $R$.
    The representative $\overline{r}$ is a homeomorphism on its image,
    as its components are homeomorphisms on their image. 
\end{myparagraph}

% lemma
\begin{mystatement}{lemma}{suspension::sharper}
    Let $\bb$ be a rooted microbundle over $B$ and
    let $H: \cyl{A} \to B$ be a map that leaves the base point fixed.
    Then there exists a neighborhood $V$ of $a_0$ together with an isomorphism-germ
    \[ \germ{\ind{H}\bb[\cyl{V}]}{\be{\cyl{V}}} \]
    extending $\overline{R}$ (as defined above).
\end{mystatement}

% proof lemma
\begin{myproof}
    By applying \myintref{homotopy::lemma2},
    it follows that there exists an isomorphism-germ
    \[ \germdef{Q}{\ind{H}\bb[\cyl{V}]}{\be{\cyl{V}}} \]
    for a sufficiently small neighborhood $V$ of $a_0$.
    However, $Q$ does not extend $\overline{R}$ in general.

    In order to fix this, consider
    \[ \germdef{Q \circ \inv{\overline{R}}}{\be{\cyl{a_0}}}{\be{\cyl{a_0}}} \]
    together with a representative $f: U_f \to (\cyl{a_0}) \cross \R^n$.

    Similar to the construction of $\overline{R}$, we can construct an isomorphism-germ
    \[ \germdef{P}{\be{\cyl{V}}}{\be{\cyl{V}}} \]
    extending $Q \circ \inv{\overline{R}}$ represented by
    \[ p(a, t, x) = (a, f(a_0, t, x)) \]
    considering $f(a_0, t, x) \in \I \cross \R^n$.

    Restricted to $\be{\cyl{a_0}}$, $P$ agrees with $Q \circ \inv{\overline{R}}$ and thus
    \[
        \inv{Q} \circ P\restr{\be{\cyl{a_0}}}
        = (\inv{Q} \circ (Q \circ \inv{\overline{R}}))
        = ((\inv{Q} \circ Q) \circ \inv{\overline{R}})
        = \inv{\overline{R}}.
    \]
    % \[ \implies (\inv{P} \circ Q)\restr{\ind{H}\bb[\cyl{a_0}]} = \overline{R}. \]
    Since $P$ and $Q$ are both isomorphism-germs,
    \[ \germdef{\inv{P} \circ Q}{\ind{H}\bb[\cyl{V}]}{\be{\cyl{V}}} \]
    is an isomorphism-germ extending $\overline{R}$.
\end{myproof}

% foreword proof
\begin{myparagraph}
    We are now able to show the Rooted Homotopy Theorem.

    To understand the proof,
    it is useful to have the constructions of \myintref{homotopy::lemma3} in mind,
    because we will modify them slightly in order to preserve the rootings.
\end{myparagraph}

% proof theorem
\begin{myproof}[of the Rooted Homotopy Theorem]
    We need to show that $\ind{f}\bb$ and $\ind{g}\bb$ are rooted isomorphic,
    that is there exists an isomorphism-germ $\germ{\ind{f}\bb}{\ind{g}\bb}$
    extending $\inv{R_g} \circ R_f = I$
    where $I$ denotes the identity germ.

    For the initial Homotopy Theorem,
    we constructed a bundle-germ
    \[ \germdef{F}{\ind{H}\bb}{\ind{H}\bb[\cylup{A}]} \]
    covering $(a, t) \mapsto (a, 1)$
    and restricted it to $\ind{H}\bb[\cyldown{A}]$.
    The required isomorphism-germ was then obtained by
    identifying $\ind{f}\bb$ with $\ind{H}\bb[\cyldown{A}]$ and
    $\ind{g}\bb$ with $\ind{H}\bb[\cyldown{A}]$.

    We must make slight modifications
    to the construction of $F$ such that it extends
    $\germ{\ind{f}\bbb \cong \ind{H}\bb[\cyldown{a_0}]}{\ind{H}\bb[\cylup{a_0}] \cong \ind{g}\bbb}$
    represented by
    \[ (a_0, e) = ((a_0, 0), e) \mapsto ((a_0, 1), e) = (a_0, e). \]

    This can be achieved by choosing a locally finite open cover $\coll{V_\alpha}$
    of $A$ (as in \myintref{homotopy::lemma3}), removing the base point $a_0$ from every set
    and adding $V$ obtained from \myintref{suspension::sharper}.
    Since $a_0 \in V$, the resulting collection is still a
    locally finite open cover of $A$.
    
    In the following, we will denote constructions over $V$
    with subscript $V$ and constructions over the other sets
    from the cover with subscript $\alpha$.

    We continue with the proof of \myintref{homotopy::lemma3}.
    Note that $\lambda_V(a_0) = 1$.
    That is because we removed $a_0$ from every other set and hence $\lambda_\alpha(a_0) = 0$.

    Lastly, we construct the extension $R_V$ for $r_V$
    like in \myintref{section::homotopy_prove},
    but instead of choosing an arbitrary trivialization
    $E(\ind{H}\bb\restr{A_V}) \cong A_V \cross \R^n$
    for the construction we use a representative $r$
    for the bundle-germ constructed in \myintref{suspension::sharper}.
    
    This has the advantage that the representative
    \[
        E(\ind{H}\bb[A_V]) \xto{r}
        A_V \cross \R^n \xto{r_V \cross id} (A_V \cap A'_V) \cross \R^n
        \xto{\inv{r}} E(\ind{H}\bb[A_V \cap A'_V])
    \]
    for $R_V$ maps elements $((a_0, 0), e)$ to $((a_0, 1), e)$.
    Additionally,
    every other $R_\alpha$ leaves $\ind{H}\bb[\cyldown{a_0}]$ unaffected
    because $r_\alpha(a_0, t) = (a_0, \max(\underbrace{\lambda_\alpha(t)}_{= 0}, t)) = (a_0, t)$.

    It follows that, by piecing together the $R_\alpha$ and $R_V$ like in \myintref{homotopy::lemma3},
    we obtain a bundle germ $\germdef{F}{\ind{H}\bb}{\ind{H}\bb[\cylup{A}]}$
    that extends $\inv{R_g} \circ R_f$.
    This completes the proof.
\end{myproof}
% foreword
\begin{myparagraph}
    Now that we introduced rooted-microbundles,
    we are able to define the wedge sum.
    As we will see in the subsequent proof,
    the definition of the wedge sum
    it is necessary to have a fixed rooting given because otherwise
    one would have to choose a rooting which the resuling microbundle depends on,
    hence not being well-defined.

    Given a quotient space $A \sqcup B / \sim$ and maps $f: A \to C$ and $g: B \to C$, we define
    \[ f \cup g: (A \sqcup B / \sim) \to C \twith \]
    \[ x \mapsto \begin{cases} f(x) & \text{if } x \in A \\ g(x) &\text{if } x \in B \end{cases}.\]
    Clearly, this map is only well-defined if $a \sim b \implies f(a) = g(b)$.
\end{myparagraph}

% definition
\begin{mystatement}{definition}{suspension::wedge}
    Let $\ba$ and $\bb$ be two rooted microbundles over $A$ and $B$.
    % microbundle
    The \defterm{wedge sum} $\ba \vee \bb$ of $\ba$ and $\bb$ is a microbundle
    \[ A \vee B \xto{i_a \cup i_b} E(\ba \vee \bb) \xto{j_a \cup j_b} A \vee B \]
    with the total space defined as
    \[ (E(\ba) \sqcup E(\bb)) / f(e_a) \sim e_a \]
    where $f: W_a \isomto W_b$ is a representative for $\inv{R_b} \circ R_a$.
    
    % rooting
    We equip $\ba \vee \bb$ with a rooting
    \[ \germdef{R}{E((\ba \vee \bb)\restr{a_0})}{\be{a_0}} \]
    represented by any representative for $R_a$ (or $R_b$).
\end{mystatement}

% proof
\begin{myproof}[that $\ba \vee \bb$ is a rooted microbundle]
    Let $f: W_a \isomto W_b$ be a representative for $\inv{R_b} \circ R_a$.

    % rooted microbundle
    \begin{sectionize}
        \item $\ba \vee \bb$ is a rooted microbundle
        \begin{itemize}
            % injection
            \item The injection map $i_a \cup i_b$ is well-defined because 
            \[ [i(a_0)] = [i_a(a_0)] = [f(i_a(a_0))] = [i_b(b_0)] = [i(b_0)] \]
            and continuous since both $i_a$ and $i_b$ are continuous.
            % projection
            \item The projection map $j_a \cup j_b$ is well-defined because
            \[ \forall e \in W_a: [j(e)] = [j_a(e)] = [a_0] = [b_0] = [j_b(f(e))] = [j(f(e))] \]
            and continuous since both $j_a$ and $j_b$ are continuous.
            % compatibility
            \item The composition $j \circ i$ is the identity because
            \[ \forall a \in A: j(i(a)) = j(i_a(a)) = j_a(i_a(a)) = a \]
            since $j_a \circ i_a = id_A$ (symmetrical for $B$).
        \end{itemize}
        % local triviality
        It remains to be shown that $\ba \vee \bb$ is locally trivial.

        Let $x \in A \vee B$.
        For reasons of symmetry, we can assume that $x \in A$.
        \begin{caselist}
            % trivial case
            \item $x \neq a_0$
            
            % construct
            Choose a local trivialization $(U, V, \phi)$ for $x$ in $\ba$.
            Without loss of generality,
            we can assume that $U \cap B = \emptyset$ by subtracting
            $\{a_0\}$ from $U$ if necessary.
            Note that $\{a_0\}$ is closed since $A$ is hausdorff.
            
            % show
            Now we can simply use this trivialization for $\ba \vee \bb$, because
            $U \sub A$ is open in $A \vee B$ and
            $V \sub E(\ba)$ is open in $E(\ba \vee \bb)$.
            Furthermore, since $i$ and $j$ reduce to $i_a$ and $j_a$,
            it follows that $\phi$ commutes with $i$ and $id \cross 0$
            as well as with $j$ and $\pi_1$.

            % special case
            \item $x = a_0$
            
            Let $(U_a, V_a, \phi_a)$ and $(U_b, V_b, \phi_b)$ be local trivializations
            for $a_0 = b_0$ in $\ba$ and $\bb$.

            % open
            Since $W_a \sub E(\bab)$ is open,
            there exists an open subset
            $W_a' \sub E(\ba)$ such that $W_a = W_a' \cap E(\bab)$.

            Let $U_a' \sub A$ be an open neighborhood of $a_0$ and $\eps > 0$ such that
            \[ V_a' := U_a' \cross \ball[\eps] \sub \phi_a(W_a'). \]
            This allows us to define the map
            \[ \phi_a': V_a' \isomto \phi_a'(V_a') \sub A \cross \R^n \twith \]
            \[ \phi_a'(e) = (j_a(e), (\snd{\phi_b} \circ f \circ \inv{\phi_a})(a_0, \snd{\phi_a}(e))). \]

            Now we can show local triviality in $a_0$ using the homeomorphism
            \[ \phi_a' \cup \phi_b: V_a' \cup V_b \isomto \phi_a'(V_a' \cup V_b) \sub (A \vee B) \cross \R^n \]

            This map is well-defined, because
            \[ \phi_a'(e) = (a_0, (\snd{\phi_b} \circ f \circ \inv{\phi_a})(a_0, \snd{\phi_a}(e))) \]
            \[ = (b_0, \snd{\phi_b}(f(e))) = (j_b(f(e)), \snd{\phi_b}(f(e))) = \phi_b(f(e)). \]

            Homeomorphy follows from the fact
            that both $\phi_a'$ and $\phi_b$ are homeomorphisms,
            and that $\phi_a'(e_a) = \phi(e_b) \implies f(e_a) = e_b$. 

            Commutativity between $i_a \cup i_b$ and $id \cross 0$
            as well as between $j_a \cup j_b$ and $\pi_1$
            is inherited from $\phi_a$ and $\phi_b$.
            Note that $\phi_a(i_a(a)) = (a, 0) = \phi_a'(i_a(a))$.

            Applying \myintref{microbundle::local} yields that $\ba \vee \bb$ is locally trivial.
        \end{caselist}
        
        % well-defined
        \item $\ba \vee \bb$ is well-defined

        Let $f'$ be another representative for $\inv{R_b} \circ R_a$ and $(\ba \vee \bb)'$ the resulting wedge sum.
        We need to find an isomorphism-germ that extends $\inv{R'} \circ R$.
        
        In order to do this,
        choose an open neighborhood $V \sub E(\ba\restr{a_0})$
        of $i_a(a)$ where $f$ and $f'$ agree.
        
        By subtracting the closed set $\inv{j_a}(a_0) - V$
        from $E(\ba \vee \bb)$ and $E(\ba \vee \bb)'$,
        the microbundles remain unchanged due to \myintref{microbundle::total}.
        
        But now the total spaces $E(\ba \vee \bb)$ and $E((\ba \vee \bb)')$ are the same.
        That is because $E(\ba \vee \bb)$ and $E((\ba \vee \bb)')$
        could only differ in $\inv{j_a}(a_0) - V$.

        
        Furthermore, since injection and projection are defined exactly the same,
        it follows that the identity $\germ{(\ba \vee \bb)}{(\ba \vee \bb)'}$
        is an isomorphism-germ.
        Together with
        \[ \inv{R'} \circ R = \inv{R} \circ R = I, \]
        which completes the proof.
    \end{sectionize}
\end{myproof}
\section{Microbundles over a Suspension}\label{section::suspension}
\begin{myparagraph} In the following, let $B$ be a \defterm{reduced suspension} \[ SX = (\cyl{X}) / (X \cross \coll{0, 1} \cup \cyl{x_0})\] over a topological space $X$.

Let $\phi: B \to B \vee B$ denote the `duplicate' map given by \[ \phi([x, t]) = \case{([x, 2t], 1)}{([x, 2t - 1], 2)}{t \le \half}. \]

Additionally, let $c_1: B \vee B \to B$ denote the map that is the identity on the first summand and the constant map $c_{B, b_0}$ on the second summand, i.e, \[ c_1(b, i) = \case{b}{b_0}{i = 1}. \] We define $c_2$ analogously. \end{myparagraph}

\begin{mystatement}{definition}{suspension::whitney}[70]
Given two rooted microbundles $\bb$ and $\bb'$ over $B$,
we equip the Whitney sum $\whitney{\bb}{\bb'}$ with the rooting
\[ \germdef{\whitney{R}{R'}}{(\whitney{\bb}{\bb'})\restr{b_0}}{\whitney{\be[n_1]{b_0}}{\be[n_2]{b_0}} = \be[n_1 + n_1]{b_0}} \]
represented by the direct sum of two representatives for $R$ and $R'$.
\end{mystatement}

\begin{mystatement}{lemma}{suspension::compatability}[70]
Let $\ba$ and $\ba'$ be two rooted microbundles over $A$,
and let $\bb$ and $\bb'$ be two rooted microbundle over $B$.
Then the following (non-rooted) isomorphy holds:
\[ \whitney{(\ba \vee \bb)}{(\ba' \vee \bb')} \cong (\whitney{\ba}{\ba'}) \vee (\whitney{\bb}{\bb'}) \]
\end{mystatement}

\begin{myproof}
Consider $\psi: E(\whitney{(\ba \vee \bb)}{(\ba' \vee \bb')}) \isomto E((\whitney{\ba}{\ba'}) \vee (\whitney{\bb}{\bb'}))$
given by the identity map.
Note that $\psi$ is well-defined because
\[ (e, e') \in E(\whitney{(\ba \vee \bb)}{(\ba' \vee \bb')}) \]
\[ \implies j(e) = j'(e') \implies j(e), j'(e') \in A \tor j(e), j'(e') \in B\]
\[ \implies (e, e') \in E(\whitney{\ba}{\ba'}) \tor (e, e') \in E(\whitney{\bb}{\bb'}) \]
\[ \implies (e, e') \in E((\whitney{\ba}{\ba'}) \vee (\whitney{\bb}{\bb'})). \]

Furthermore, the injection and projection maps agree.
This can be seen with the following equations (symmetrical for $B$):
\begin{gather}
i_\oplus(a) = (i_a(a), i_a'(a)) = i_\vee(a) \\
j_\oplus(e_a, e_a') = j(e_a) = j_a(e_a) = j_\vee(e_a, e_a')
\end{gather}

It follows that the two microbundles are isomorphic.
\end{myproof}
\begin{mystatement}{lemma}{suspension::paracompact} Let $\bb$ be a rooted microbundle over a paracompact hausdorff space $B$. Then there exists a closed neighborhood $W$ of $b_0$ and an isomorphism-germ \[ \germ{\bb[W]}{\be{W}} \] extending $R$ (rooting of $\bb$) together with a map $\lambda: B \to [0, 1]$ with \[ \supp{\lambda} \sub W \tand \lambda(b_0) = 1. \] \end{mystatement}

\begin{myproof} Let $r: V_r \to b_0 \cross \R^n$ be a representative for $R$.

Choose a local trivialization $(U, V, \phi)$ for $b_0$ such that $V \cap E(\bbb) \sub V_r$. Such a trivialization can be obtained by subtracting the closed set $E(\bbb) - V_r$ from $E(\bb)$ if necessary and choosing a local trivialization for $b_0$ in the resulting microbundle instead.

Consider the (locally) finite open cover of $B$ given by $U$ and $B - \{b_0\}$. Since $B$ is paracompact, we can apply the theory of Partitions of Unity, which yields a map $\lambda: B \to [0, 1] \twith \supp{\lambda} \sub U$ and $\lambda(b_0) = 1$.

We choose $W = \supp{\lambda} \sub U$, which is closed by the definition of the support. We are now able to define an isomorphism-germ $\germ{\bb[W]}{\be{W}}$ represented by \[ f: V \isomto f(V) \sub U \cross \R^n \twith f(e) = (j(e), \snd{r}(\inv{\phi}(b_0, \snd{\phi}(e)))), \] which extends $r$.

Together with $\lambda$, this completes the proof. \end{myproof}
% lemma
\begin{mystatement}{lemma}{suspension::commutativity}
    The rooted microbundles $\whitney{\bb}{\be{B}}$ and $\whitney{\be{B}}{\bb}$ are rooted-isomorphic. 
\end{mystatement}

% proof
\begin{myproof}
    We need to find an isomorphism-germ
    $\germ{\whitney{\bb}{\be{B}}}{\whitney{\be{B}}{\bb}}$ that extends
    \[ (\whitney{I}{R}) \circ \inv{(\whitney{R}{I})} = \whitney{R}{\inv{R}} \]
    where $I$ denotes the identity germ.

    Ignoring the rooting, we have an
    isomorphism-germ $f: E(\bb) \cross \R^n \isomto \R^n \cross E(\bb)$ with $f(e, x) = (-x, e)$.
    The idea is to change the $f$ near $b_0$ so that it extends the rooting.

    Using the previous lemma, choose a sufficiently small closed neighborhood $U$ of $b_0$
    such that there exists an extension
    $\germdef{Q}{(\whitney{\bb}{\be{B}})\restr{U}}{(\whitney{\be{B}}{\bb})\restr{U}}$ for the rooting.

    The previous lemma also equips us with a map
    \[ \lambda: B \to [0, \frac{\pi}{2}] \]
    such that $\supp \lambda \sub U$ and $\lambda(b_0) = \frac{\pi}{2}$.
    
    Now, we can define a homeomorphism
    \[ \psi: U \cross \R^n \cross \R^n \isomto U \cross \R^n \cross \R^n \twith \]
    \[ \psi(b, x, y) = (b, x \sin(\lambda(b)) - y \cos(\lambda(b)), x \cos(\lambda(b)) - y \sin(\lambda(b))). \]

    Consider the composition
    \[ \germ{(\whitney{\bb}{\be{B}})\restr{U}}{(\whitney{\bb}{\be{B}})\restr{U}} \xto{g} \germ{(\whitney{\bb}{\be{B}})\restr{U}}{(\whitney{\be{B}}{\bb})\restr{U}} \]
    which coincides with $\whitney{R}{\inv{R}}$ over $b_0$
    since $\psi(b_0, x, y) = (b_0, x, y)$ and with $F$ over $U \cap \inv{\lambda}(0)$.
    
    Pieced together with $F\restr{\inv{\lambda}(b)}$ using \myintref{homotopy::glueing},
    we obtain an isomorphism-germ 
    \[ \germ{\whitney{\bb}{\be{B}}}{\whitney{\be{B}}{\bb}} \]
    extending the rooting.
\end{myproof}
\begin{mystatement}{theorem}{suspension::theorem}[71] If $\ba$ and $\bb$ are rooted microbundles over a paracompact hausdorff space $B$, then \[ \whitney{\ind{\phi}(\ba \vee \bb)}{\be{B}} \cong \whitney{\ba}{\bb}. \] \end{mystatement}

\begin{myproof} The previous lemma yields rooted isomorphy $\whitney{\bb}{\be{B}} \cong \whitney{\be{B}}{\bb}$. Hence, \[ \ind{\phi}((\whitney{\ba}{\be{B}}) \vee (\whitney{\bb}{\be{B}})) \cong \ind{\phi}((\whitney{\ba}{\be{B}}) \vee (\whitney{\be{B}}{\bb})). \] Furthermore, we have \[ \ind{\phi}((\whitney{\ba}{\be{B}}) \vee (\whitney{\bb}{\be{B}})) \cong \whitney{\ind{\phi}((\ba \vee \bb))}{(\be{B} \vee \be{B})} \cong \whitney{\ind{\phi}(\ba \vee \bb)}{\be{B}} \] for the left side and \[ \ind{\phi}((\whitney{\ba}{\be{B}}) \vee (\whitney{\be{B}}{\bb})) \cong \ind{\phi}(\whitney{(\ba \vee \be{B})}{(\be{B} \vee \bb)}) \cong \whitney{\ba}{\bb} \] for the right side of the isomorphy, which completes the proof. \end{myproof}

\begin{mystatement}{corollary}{suspension::corollary}[72] The wedge sum $\whitney{\bb}{\ind{r}\bb}$ is trivial. \end{mystatement}

\begin{myproof} This follows directly from the previous theorem and the fact that $\ind{\phi}(\whitney{\bb}{\ind{r}\bb})$ is trivial. \end{myproof}

\begin{myparagraph} Finally, the Bouquet Lemma is just \myintref{suspension::corollary} applied to a microbundle over a bouquet of spheres. Note that a bouquet of $d$-spheres can be regarded as a reduced suspension over a bouquet of $(d-1)$-spheres. \end{myparagraph}
\section{Normal Microbundles and Milnors Theorem}\label{chapter::normal}
\section{Normal Microbundles}
\begin{mystatement}{definition}[normal microbundle]{normal::definition}[61]Let $M$ be a topological manifold together with a submanifold $N \sub M$. A \defterm{normal microbundle} $\bn$ of $N$ in $M$ is a microbundle \[ \bundle{N}{U}{\iota}{r} \] where $U \sub M$ is a neighborhood of $N$ and $\iota$ denotes the inclusion $N \incl U$.
\end{mystatement}

\begin{mydefinition}
Again, let $M$ and $N$ be two topological manifolds with $N \sub M$.
We say that $N$ has a \defterm{product neighborhood} in $M$ if
there exists a trivial normal microbundle of $N$ in $M$.
\end{mystatement}

\begin{mylemma}
A submanifold $N \sub M$ has a product neighborhood if and only if
there exists a neighborhood $U$ of $N$ with $(U, M) \cong (M \cross R^n, \cyldown{M})$.
\end{mystatement}

\begin{myproof}
This follows directly from the definition of normal microbundles and the criteria for trivial microbundles.
\end{myproof}
% definition
\begin{mystatement}{definition}[composition microbundle]{normal::composition}[63]
    Let $\ba$ be a $n$-dimensional microbundle
    \[ \bundledef{\ba}{A}{E(\ba)}{i_a}{j_a} \]
    and let $\bb$ be a $n'$-dimensional microbundle
    \[ \bundledef{\bb}{E(\ba)}{E(\bb)}{i_b}{j_b}. \]
    The \defterm{composition microbundle} $\ba \circ \bb$ is a $(n + n')$-dimensional microbundle
    \[ \bundle{A}{E(\bb)}{i}{j} \]
    where $i = i_b \circ i_a$ and $j = j_a \circ j_b$.
\end{mystatement}

% proof
\begin{myproof}[that $\ba \circ \bb$ is a microbundle]
    % continuity and compatibility
    Both injection and projection maps are continuous as being composed by continuous maps.
    Additionally, $j \circ i = j_a \circ (j_b \circ i_b) \circ i_a = j_a \circ i_a = id_A$.

    It remains to be shown that $\ba \circ \bb$ is locally trivial.

    % without loss
    For an arbitrary $a \in A$, choose local trivializations
    $(U_a, V_a, \phi_a)$ of $a$ in $\ba$ and $(U_b, V_b, \phi_b)$ of $i_a(a)$ in $\bb$.
    Note that both $U_b$ and $V_a$ are open neighborhoods of $i_a(a)$.
    
    Without loss of generality, we may assume that $V_a = U_b$:
    
    `$\sub$':
    Modify $U_a$ such that
    \[ U_a \cross \ball[\eps] \sub \phi_a(V_a \cap U_b) \]
    for a sufficiently small $\eps > 0$ and let
    \[ V_a = \inv{\phi_a}(U_a \cross \ball[\eps]) \sub V_a \cap U_b. \]
    Composing $\phi_a$ with $\mu_\eps: \ball[\eps] \isomto \R^n$
    yields a local trivialization of $a$ in $\ba$ such that $V_a \sub U_a$.

    `$\bus$': Restrict $U_b$ to $V_a \cap U_b$ and $V_b$ to $\inv{\phi_b}((V_a \cap U_b) \cross \R^{n'})$.

    % homeomorphy
    We have local-trivialization $(U_a, V_b, \phi)$ of $a$ in $\ba \circ \bb$ given by
    \[
        \phi: V_b \xto{\phi_b} U_b \cross \R^{n'}
        = V_a \cross \R^{n'}
        \xto{\phi_a \cross id} (U_a \cross \R^{n}) \cross \R^{n'}
        = U_a \cross \R^{n + n'},
    \]
    which is a homeomorphism since it's composed by homeomorphisms.

    % commutativity
    Furthermore, $\phi$ commutes with the injection and projection maps,
    as the following equations show:
    \begin{gather}
        \begin{split}
            \phi(i(a)) = \phi(i_b(i_a(a))) = (\phi_a(i_a(a)), \snd{\phi_b}(i_b(i_a(a)))) \\
            = (\snd{\phi_a}(i_a(a)), 0) = (a, (0, 0)) = (id_{U_a} \cross 0)(a)    
        \end{split} \\
        j(e) = j_a(j_b(e)) = \pi_1(j_a(j_b(e)), \snd{\phi}(e)) = \pi_1(\phi(e))
    \end{gather}
    This completes the proof.
\end{myproof}
\begin{mystatement}{lemma}{normal::transitivity}[62] Let $P \sub N \sub M$ be a chain of topological submanifolds. There exists a normal microbundle \[ \bundledef{\bn}{P}{U}{\iota}{r} \] of $P$ in $M$ if there exist normal microbundles \begin{center} $\bundledef{\bn_p}{P}{U_N}{\iota_P}{j_P}$ in $N$ and $\bundledef{\bn_n}{N}{U_M}{\iota_N}{j_N}$ in $M$. \end{center} \end{mystatement}

\begin{myproof} We are given a normal microbundle $\bn$ of $P$ in $M$ by \[ \bn_p \circ \bn_n\restr{U_N}. \] Note that $\iota_N \circ \iota_P$ is just the inclusion $P \incl U_M$. \end{myproof}
% foreword
\begin{myparagraph}
    Every topological manifold is an absolute neighborhood retract (ANR).
    
    It follows that by restricting $M$, if necessary, to an open neighborhood of $N$,
    there exists a retraction $M \subm N$.

    From now on, let
    \[ r: M \subm N \]
    denote such a retraction and let
    \[ \iota: N \incl M \]
    denote the inclusion $N \sub M$.
\end{myparagraph}
% statement
\lemma{(homeomorphism of total spaces)} \\
Let $\bt_N$ and $\bt_M$ be the tangent microbundles of $N$ and $M$.
The total space $E(\iota^*\bt_M)$ and $E(r^*\bt_N)$ are homeomorphic.
% proof
\begin{proof}
We explicitly construct a homeomorphism:
\begin{enumerate}
    \item $E(\iota^*\bt_M) = \{ (n, (m_1, m_2)) \in N \times (M \times M) \mid \iota(n) = m_1\}$
    \item $E(r^*\bt_N) = \{ (m, (n_1, n_2)) \in M \times (N \times N) \mid r(m) = n_1\}$
\end{enumerate}
Now, we have the homeomorphism $\phi: E(\iota^*\bt_M) \to E(r^*\bt_N)$ with
$\phi(n, (m_1, m_2)) = (m_2, (r(m_2), n))$ and $\phi^{-1}(m, (n_1, n_2)) = (n_2, (n_2, m))$.
We easily see that $\phi$ suffices all requirements of $E(\iota^*\bt_M)$ and $E(r^*\bt_N)$.
\end{proof}

% remark
\begin{remark}
Note that the following diagram commutes
\[\begin{tikzcd}
    N \ar[r] \ar[hookrightarrow]{d} & E(\iota^*\bt_M) \ar[d, "\phi"] \\
    M \ar[r] & E(r^*\bt_N)
\end{tikzcd}\]
\end{remark}

% statement
\lemma{(normal microbundle on total space)}
There exists a normal microbundle $\bn$ of $N$ in $E(r^*\bt_N)$ with $\bn \cong \iota^*\bt_M$.
% proof
\begin{proof}
Obviously, $\bn := r^*\bt_N \restr{N}$ is a normal microbundle of $N$ in $E(r^*\bt_N)$.
Since $E(r^*\bt_N \restr{N}) \sub E(r^*\bt_N)$, isomorphy follows from the previous lemma and remark.
\end{proof}
\begin{scope}
    % defines
    \newcommand{\rwhitney} {
        \whitney{\ind{r}\bt_N}{\ind{r}\eta}
    }
    \newcommand{\rtn} {
        \ind{r}\bt_N
    }

    % foreword
    \begin{myparagraph}
        Finally, we gathered all the tools to prove Milnors theorem.
    \end{myparagraph}

    % theorem
    \begin{mystatement}{theorem}[Milnors Theorem]{normal::milnor}[62]
        For a sufficiently large $q \in \N$, $N = \cyldown{N}$ has a normal microbundle in $M \cross \R^q$.
    \end{mystatement}

    % proof
    \begin{myproof}
        We assume that $M$ is embedded in Euclidean space $\R^{2m + 1}$ \cite[p.60]{dimension}.
        Additionally, let $V$ be an open neighborhood of $N$ in $M$ together with a retraction $r: V \to N$.

        We show the theorem in multiple steps.
        \begin{steps}
            % step 1
            \item $N$ has a normal microbundle $\eta$ in $M$ such that $\whitney{\bt_N}{\eta} \cong \be[q]{N}$
            
            Consider the extension $\ind{r}\bt_N$ for $\bt_N$.
            Since $V$ is an open set, it's a simplicial complex.
            Hence, we can apply \myintref{whitney::theorem} to $\ind{r}\bt_N$
            to obtain a microbundle $\eta'$ such that $\whitney{\ind{r}\bt_N}{\eta'} \cong \be[q]{V}$.

            We conclude that $\whitney{\bt_N}{\eta'\restr{N}} = \whitney{\ind{r}\bt_N\restr{N}}{\eta'\restr{N}} \cong (\whitney{\ind{r}\bt_N}{\eta'})\restr{N} \cong \be[q]{N}$.

            % step 2
            \item $E(\rtn)$ has a normal microbundle in $E(\rwhitney)$

            Note that $\rwhitney \cong \ind{r}(\whitney{\bt_N}{\eta})$ is trivial.
            Hence, by restricting the total space $E(\rwhitney)$ if necessary,
            we may assume that
            \[ E(\rwhitney) = V \cross \R^q \]
            for some $q \in \N$.
            Particularly, we derive that $E(\rwhitney)$ is a manifold.

            We denote the injection and projection of $\rtn$ by $i_{\bt}$ and $j_{\bt}$,
            and the injection and projection of $\ind{r}\eta$ by $i_{\eta}$ and $j_{\eta}$.

            We consider $E(\rtn)$ to be a submanifold of $E(\rwhitney)$ embedded by
            \[ e \mapsto (e, j_{r}(i_{\eta}(e))). \]
            We have a normal microbundle of $E(\rtn)$ in $E(\rwhitney)$ given by $\ind{j_{\bt}}(\ind{r}\eta)$.
            That is because the total space
            \[ E(\ind{j_r}(\ind{r}\eta)) = \set{(e, e') \in E(\rtn) \cross E(\ind{r}\eta)}{j_{\bt}(e) = j_\eta(e')} \]
            equals $E(\rwhitney)$ and because its inclusion matches the above embedding.
            
            % step 3
            \item $N$ has a normal microbundle in $M \times \R^q$
            
            Since $N$ has a normal microbundle in $E(\rtn)$ (\myintref{microbundle::total})
            and $E(\rtn)$ has a normal microbundle in $E(\rwhitney)$,
            we conclude with \myintref{normal::transitivity} that
            $N$ has a normal microbundle in $E(\rwhitney)$.
            
            As elaborated in Step 2, we may assume that $E(\rwhitney) = V \cross \R^n$.
            Since $V$ is open in $M$, $V \cross \R^n$ is open in $M \cross \R^n$.
        \end{steps}
        We conclude that $N$ has a normal microbundle in $M \cross \R^n$.
    \end{myproof}

    % dimension
    \begin{myparagraph}
        With the provided proofs for \myintref{whitney::theorem} and \myintref{normal::milnor},
        we can calculate an upper bound $m(2^{2m + 2} - 2)$ for $q$, where $m$ is the dimension of $M$ \cite[p.63]{milnor}.
        However, when provided with sharper proofs,
        one can substantially reduce the upper bound
        to a quadratic $(m + 1)^2 - 1$ (see \cite[p.232]{hirsch}). 
    \end{myparagraph}
\end{scope}
\section{Outlook}\label{chapter::outlook}
\begin{myparagraph}
    Now that we have seen that Milnors Theorem holds,
    one may ask whether we can do better than this.
    Can we prove the existence of normal microbundles without having to stabilize
    the ambient manifold? As already mentioned in \myintref{section::motivation},
    Rourke and Sanderson found a submanifold that does not admit a normal microbundle
    in its ambient manifold.
    So in the theory of microbundles, one can not do substantially better.

    When we broaden our view,
    one can indeed develop a bundle-theory over non-smooth manifolds
    where the existence of a normal bundle is garantueed.
    However, this is not free.
    Rourke and Sanderson developed a theory of so called `block bundles'.
    They are defined over simplicial complexes:
    
    Given a simplicial complex $K$, then a \defterm{q-block bundle} $\xi^q/K$
    consists of a total space $E(\xi)$ with $\abs{K} \sub E(\xi)$ such that
    there exists a collection of \defterm{blocks} $\coll{\beta_i}$ satisfying the following:
    \begin{itemize}
        \item for each $n$-cell $\sigma \in K$, there exists a block $\beta_i$, such that $(\beta_i, \sigma) \cong (I^{n + q}, I^n)$.
        \item $E(\xi) = \bigcup_{i} \beta_i$
        \item $i \neq j \implies \inner{\beta_i} \cap \inner{\beta_j} = \emptyset$
        \item $\beta_i \cap \beta_j$ is the union of the blocks of the faces for $\sigma_i \cap \sigma_j$.
    \end{itemize}

    The normal block bundle over a PL-submanifold $M \sub N$ can now be defined to be a block bundle $\eta/K$ such that $E(\eta) = N$.
    Theorem 4.3a) in \cite{block} then states that every compact PL-submanifold $M \sub N$ admits a normal block bundle.
    
    So block bundles do not require a stabilization.
    However, this comes to the major cost that block bundles are not equipped with a projection map.
    So if we want the existence of normal bundle, we would have to sacrifice the existence of a projection map.
\end{myparagraph}
\bibliographystyle{alpha}
\bibliography{refs}
\addcontentsline{toc}{section}{References}
\end{document}
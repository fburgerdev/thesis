\section{Outlook}\label{chapter::outlook}
\begin{myparagraph}
    Now that we have seen that Milnors Theorem holds,
    one may ask whether we can do better than this.
    Can we prove the existence of normal microbundles without having to stabilize
    the ambient manifold? As already mentioned in \myintref{section::motivation},
    Rourke and Sanderson found a submanifold that does not admit a normal microbundle
    in its ambient manifold.
    So in the theory of microbundles, one can not do substantially better.

    When we broaden our view,
    one can indeed develop a bundle-theory over non-smooth manifolds
    where the existence of a normal bundle is garantueed.
    However, this is not free.
    Rourke and Sanderson developed a theory of so called `block bundles'.
    They are defined over simplicial complexes:
    
    Given a simplicial complex $K$, then a \defterm{q-block bundle} $\xi^q/K$
    consists of a total space $E(\xi)$ with $\abs{K} \sub E(\xi)$ such that
    there exists a collection of \defterm{blocks} $\coll{\beta_i}$ satisfying the following:
    \begin{itemize}
        \item for each $n$-cell $\sigma \in K$, there exists a block $\beta_i$, such that $(\beta_i, \sigma) \cong (I^{n + q}, I^n)$.
        \item $E(\xi) = \bigcup_{i} \beta_i$
        \item $i \neq j \implies \inner{\beta_i} \cap \inner{\beta_j} = \emptyset$
        \item $\beta_i \cap \beta_j$ is the union of the blocks of the faces for $\sigma_i \cap \sigma_j$.
    \end{itemize}

    The normal block bundle over a PL-submanifold $M \sub N$ can now be defined to be a block bundle $\eta/K$ such that $E(\eta) = N$.
    Theorem 4.3a) in \cite{block} then states that every compact PL-submanifold $M \sub N$ admits a normal block bundle.
    
    So block bundles do not require a stabilization.
    However, this comes to the major cost that block bundles are not equipped with a projection map.
    So if we want the existence of normal bundle, we would have to sacrifice the existence of a projection map.
\end{myparagraph}
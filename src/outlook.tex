\section{Outlook}\label{chapter::outlook}
\subsection*{Block bundles}
\begin{myparagraph}
    Now that we have seen that Milnors Theorem holds, one may ask whether a stabilization of the surrounding manifold is even necessary. As mentioned in \myintref{section::motivation}, Rourke and Sanderson constructed a submanifold that does not admit a normal microbundle in its surrounding manifold. So within the theory of microbundles, we can not significantly improve the results from Milnors Theorem.

    But what if we look beyond the theory of microbundles. Is there some other approach to define a tangent bundle over topological manifolds without the need for a stabilization in order to guarantee the existence of a normal bundle?

    In 1968, Rourke and Sanderson published a paper introducing so called `block bundles', which are defined over simplicial complexes (see \cite{block}).
    
    Let $K$ be a simplicial complex. A \defterm{q-block bundle} $\xi^q/K$ is defined by a total space $E(\xi)$ with $\abs{K} \sub E(\xi)$ such that there exists a collection of \defterm{blocks} $\coll{\beta_i}$ satisfying the following:
    \begin{itemize}
        \item Every block $\beta_i$ corresponds to a $n$-cell $\sigma_i \in K$ in a way that
        \begin{center}
            $(\beta_i, \sigma_i) \cong (\I^{n + q}, \I^n)$.
        \end{center}
        \item The blocks cover the entire total space $E(\xi)$.
        \item The interior of any two different blocks are disjoint.
        \item $\beta_i \cap \beta_j$ is the union of the blocks of the faces for $\sigma_i \cap \sigma_j$.
    \end{itemize}

    A normal block bundle over a PL-submanifold $M \sub N$ is defined to be a block bundle $\eta/K$ such that $E(\eta) = N$. Theorem 4.3a in `block bundles I' \cite{block} then states that every compact PL-submanifold $M \sub N$, with a locally flat and proper embedding, admits a normal block bundle. To prove this, Rourke and Sanderson utilize the regular neighborhood theorem in \cite[p.293]{regular} and the construction of dual complexes.

    So in this theory, the existence of normal block bundles is guaranteed without the need for a stabilization. However, this comes to the major cost that block bundles are not equipped with a projection map. Thus, for block bundles, the existence of a normal bundle is attained at the expense of the projection map.
\end{myparagraph}
\subsection*{Fiber bundles}
\begin{myparagraph}
    In order to construct the tangent microbundle, we gave up on the requirement that the bundle is a vector bundle or even a fiber bundle. Suprisingly, in 1964 J. Kister could show that every microbundle $\bundledef{\bb}{B}{E}{i}{j}$ has a restriction $\bb[E']$ such that $E' \xto{j\restr{E'}} B$ is a fiber bundle (see \cite{kister}). The fibers are Euclidean and the structural group is the set of all based homeomorphisms $(\R^n, 0) \isomto (\R^n, 0)$. This is an extraordinary result. It makes microbundles a very natural analogue to vector bundles in the topological and piecewise category.
\end{myparagraph}
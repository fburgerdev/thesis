% foreword
\begin{myparagraph}
    The total space $E(\ind{r}\bt_N)$ is a topological manifold with
    \[ E(\ind{r}\bt_N) \cong M \times N \]
    as described in the previous lemma.

    The fact that the above diagram commutes,
    allows us to consider $N$ to be a submanifold of $E(\ind{r}\bt_N)$ via
    \[ N \incl M \xto{i_r} E(\ind{r}\bt_N). \]
    The composition $\iota \circ i_r$ is an embedding since $i_r$ is an embedding
    due to the construction of the induced microbundle.
\end{myparagraph}

% lemma
\begin{mylemma}{normal::total}
    Let $M$ be a topological manifold together with a submanifold $N \sub M$.
    Then there exists a normal microbundle $\bn$ of $N$ in $E(\ind{r}\bt_N)$ such that $\bn \cong \ind{\iota}\bt_M$.
\end{mylemma}

% proof
\begin{myproof}
    We are already given a normal microbundle of $N$ in $E(\ind{r}\bt_N)$ with $\ind{r}\bt_N\restr{N}$.
    Isomorphy between $\ind{r}\bt_N\restr{N}$ and $\ind{\iota}\bt_M$ follows from the homeomorphy
    \[ \psi: E(\ind{\iota}\bt_M) \isomto E(\ind{r}\bt_N) \]
    and from the diagram which shows that injection and projection maps commute with $\psi$.
\end{myproof}
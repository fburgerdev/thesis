% foreword
\begin{myparagraph}
    The total space $E(\ind{r}\bt_N)$ carries the structure of a topological manifold with
    \[ E(\ind{r}\bt_N) \cong M \times N \]
    as described in the previous lemma.
    That is since $M \times N$ comes equipped with the product manifold structure.

    The fact that the diagram
    \[
        \begin{tikzcd}
            N \ar[r, "i_1"] \ar[hookrightarrow, "\iota"']{d} & E(\ind{\iota}\bt_M) \ar[d, "\psi"] \\
            M \ar[r, "i_2"] & E(\ind{r}\bt_N)
        \end{tikzcd}
    \]
    commutes ($i_1$ and $i_2$ denote the injections of $\ind{\iota}\bt_M$ and $\ind{r}\bt_N$),
    lets us consider $N$ to be a submanifold of $E(\ind{r}\bt_N)$ via $N \incl M \xto{i_2} E(\ind{r}\bt_N)$.
    The composition $\iota \circ i_2$ is an embedding because $i_2$ is an embedding
    due to the construction of the induced microbundle.
\end{myparagraph}

% lemma
\begin{mylemma}{normal::total}
    There exists a normal microbundle $\bn$ of $N$ in $E(\ind{r}\bt_N)$ such that $\bn \cong \ind{\iota}\bt_M$.
\end{mylemma}

% proof
\begin{myproof}
    We are already given a normal microbundle of $N$ in $E(\ind{r}\bt_N)$ with $\ind{r}\bt_N\restr{N}$.
    Isomorphy between $\ind{r}\bt_N\restr{N}$ and $\ind{\iota}\bt_M$ follows from
    \[ \psi: E(\ind{\iota}\bt_M) \isomto E(\ind{r}\bt_N) \]
    together with the fact that $E(\ind{r}\bt_N\restr{N})$ is a neighborhood of $i_2(B)$ in $E(\ind{r}\bt_N)$
    and from the diagram which shows that injection and projection maps commute with $\psi$.
\end{myproof}
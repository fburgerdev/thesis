% foreword
Finally, we gathered all the tools to prove Milnor's theorem.

\newcommand{\whitney} {
    r^*\bt_N \oplus{} r^*\bt'
}
\newcommand{\rtn} {
    r^*\bt_N
}

% theorem milnor
\begin{mytheorem}[Milnors Theorem]{normal::milnor}
    For a sufficently large $q \in \N$, $N = N \times \{0\}$ has a normal microbundle in $M \times \R^q$.
\end{mytheorem}

% proof milnor
\begin{myproof}
    We show the theorem in multiple steps:
    \begin{enumerate}
        % statement
        \item There exists a microbundle $\eta$ over $N$ such that $\bt_N \oplus \eta \cong \be[q]_N$:
        
        % proof
        From the \extref{Whitney Embedding Thereom} it follows that we can consider $M$ to be embedded in euclidean space $\R^{2m + 1}$.
        Additionally, since we can find a retraction $r: V \to N$ where $V$ is an open neighborhood of $N$ in $M$ we can extend $\bt_N$ over $V$.
        Now we can apply the \intref{whitney::theorem} from the ``Whitney Sum'' Chapter to the extended microbundle
        to receive a $\eta$ such that $\bt_N \oplus \eta \cong \be[q]_N$.

        % statement
        \item $E(\rtn) \sub E(\whitney)$ has a normal microbundle:

        %proof
        Consider the microbundle
        \[ \bundledef{j^*(\whitney)}{E(\rtn)}{E(\whitney)}{i'}{j'} \]
        where $j$ is the projection map for $\rtn$.
        Since $i'$ is injective, we can assume $E(\rtn) \sub E(\whitney)$.
        Now $j^*(\whitney)$ is a normal microbundle if we equip $\whitney$ with a manifolds structure as shown above.
        % TODO: how?
    \end{enumerate}
    Since $N \sub M \sub E(\rtn)$ has a normal microbundle (\intref{normal::total}) it follows from \intref{normal::transitivity} that $N \sub E(r^*\bt_N \oplus r^*\bt')$ has a normal microbundle.
    But $r^*\bt_N \oplus r^*\bt'$ is trivial and therefore w.l.o.g. $E(\whitney) \cong N \times \R^q$
    % TODO: last portion is unclear
\end{myproof}
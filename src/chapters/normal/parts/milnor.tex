\begin{scope}
    % defines
    \newcommand{\rwithney} {
        \whitney{\ind{r}\bt_N}{\ind{r}\bt'}
    }
    \newcommand{\rtn} {
        \ind{r}\bt_N
    }

    % foreword
    \begin{myparagraph}
        Finally, we gathered all the tools to prove Milnor's theorem.
    \end{myparagraph}

    % theorem
    \begin{mytheorem}[Milnors Theorem]{normal::milnor}
        For a sufficently large $q \in \N$, $N = \cyldown{N}$ has a normal microbundle in $M \times \R^q$.
    \end{mytheorem}

    % proof
    \begin{myproof}
        We show the theorem in multiple steps:
        \begin{enumerate}
            % statement
            \item There exists a microbundle $\eta$ over $N$ such that $\whitney{\bt_N}{\eta} \cong \be[q]{N}$:
            
            % proof
            From the \extref{Whitney Embedding Thereom} it follows that we can consider $M$ to be embedded in euclidean space $\R^{2m + 1}$.
            Additionally, since we can find a retraction $r: V \to N$ where $V$ is an open neighborhood of $N$ in $M$ we can extend $\bt_N$ over $V$.
            Now we can apply the \intref{whitney::theorem} from the ``Whitney Sum'' Chapter to the extended microbundle
            to receive a $\eta$ such that $\whitney{\bt_N}{\eta} \cong \be[q]{N}$.

            % statement
            \item $E(\rtn) \sub E(\whitney)$ has a normal microbundle:

            %proof
            Consider the microbundle
            \[ \bundledef{\ind{j}(\rwithney)}{E(\rtn)}{E(\rwithney)}{i'}{j'} \]
            where $j$ is the projection map for $\rtn$.
            Since $i'$ is injective, we can assume $E(\rtn) \sub E(\rwithney)$.
            Now $\ind{j}(\rwithney)$ is a normal microbundle if we equip $\rwithney$ with a manifolds structure as shown above.
            % TODO: how?
        \end{enumerate}
        Since $N \sub M \sub E(\rtn)$ has a normal microbundle (\intref{normal::total}) it follows from \intref{normal::transitivity} that $N \sub E(\whitney{\ind{r}\bt_N}{\ind{r}\bt'})$ has a normal microbundle.
        But $\whitney{\ind{r}\bt_N}{\ind{r}\bt'}$ is trivial and therefore w.l.o.g. $E(\rwithney) \cong N \times \R^q$
        % TODO: last portion is unclear
    \end{myproof}
\end{scope}
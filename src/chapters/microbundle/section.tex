\chapter{Introduction to Microbundles}
\begin{myparagraph}
    In order to construct the tangent bundle on a manifold, we need a differential structure.
    However, this is generally not given for topological manifolds.
    In order to still have a structure ``similar'' to the tangent bundle on topological manifolds,
    we need a different, weaker, concept of the tangent bundle.
    Therefore we introduce so called ``microbundles'' which act as a weaker alternative to vector bundles.
    The concept of microbundles as well as some basic properties and examples are presented in this chapter.
\end{myparagraph}
% definition
\begin{mydefinition}[microbundle]{microbundle::definition}{}
    A \defterm{microbundle} $\bb$ over $B$ (with \defterm{fibre dimension} $n$)
    is a diagram $\bundle{B}{E}{i}{j}$ satisfying the following:
    \begin{enumerate}[(i)]
        \item $B$ is a topological space (\defterm{base space})
        \item $E$ is a topological space (\defterm{total space})
        \item $i: B \to E$ (\defterm{injection}) and $j: E \to B$ (\defterm{projection})
        are maps such that $id_B = j \circ i$
        \item Every $b \in B$ is \defterm{locally trivializable},
        that is there exist open neighborhoods $U \sub B$ of $b$ and $V \sub E$ of $i(U)$
        together with a homeomorphism $\phi: V \isomto U \times \R^n$ such that the following diagram commutes:
        \[
            \begin{tikzcd}[column sep=tiny]
                & V \ar[dr, "j\vert_V"] \ar[dd, "\psi"] & \\
                U \ar[ur, "i"] \ar[dr, "{id \times 0}"'] & & U \\
                & U \times\R^n \ar[ur, "\pi_1"'] &
            \end{tikzcd}
        \]
    \end{enumerate}
\end{mydefinition}

% remark
\begin{myremark}{microbundle::dimension}{}
    In the following, unless explicitly stated otherwise,
    we assume the fibre dimension of any given microbundle to be $n$.
\end{myremark}
% foreword
\begin{myparagraph}
    Before we look at some examples of microbundles,
    we first define what it means for two microbundles to be isomorphic.
\end{myparagraph}

% definition
\begin{mystatement}{definition}[isomorphy]{microbundle::isomorphy}
    Two microbundles $\bundledef{\bb_1}{B}{E_1}{i_1}{j_1}$ and $\bundledef{\bb_2}{B}{E_2}{i_2}{j_2}$
    are \defterm{isomorphic} if there exist neighborhoods $V_1$ of $i_1(B)$ and $V_2$ of $i_2(B)$
    together with a homeomorphism $\psi: V_1 \isomto V_2$
    such that the following diagram commutes: 
    \begin{center}
        \begin{tikzcd}
            & V_1 \ar[dr, "j_1\vert_{V_1}"] \ar[dd, "\psi"] & \\
            B \ar[ur, "i_1"] \ar[dr, "i_2"'] & & B \\
            & V_2 \ar[ur, "j_2\vert_{V_2}"'] &
        \end{tikzcd}
    \end{center}
\end{mystatement}
% foreword
\begin{myparagraph}
    As the definition of isomorphy already indicates, when studying microbundles,
    we are not interested in the entire total space
    but only in an arbitrarily small neighborhood of the base space.
    This is certainly one of the strongest conceptual differences between microbundles and classical vector bundles.
\end{myparagraph}

% proposition
\begin{myproposition}{microbundle::total}
    Given a microbundle $\bundledef{\bb}{B}{E}{i}{j}$ over $B$,
    restricting the total space $E$ to an arbitrary neighborhood $E' \sub E$ of $i(B)$ leaves the microbundle unchanged.
    That is, the microbundle
    \[ \bundledef{\bb'}{B}{E'}{i}{j\restr{E'}} \]
    is isomorphic to $\bb$.
\end{myproposition}

% proof
\begin{myproof}
    We prove this proposition in two steps.
    \begin{enumerate}
        % microbundle
        \item $\bb'$ is a microbundle:
        
        Since we take $i$ and $j$ from $\bb$, we only need to show local triviality.

        For an arbitrary $b \in B$, choose a local trivialization $(U, V, \phi)$ of $b$ in $\bb$.
        
        The image $\phi(V \cap E')$ is a neighborhood of $(b, 0)$.
        That follows from $\phi(i(b)) = (b, 0)$ and $V \cap E'$ being a neighborhood of $i(b)$.

        Hence, there exists a $U' \times \ball[\eps] \sub \phi(V \cap E')$ with $U'$ open and $\eps$ sufficiently small.

        By utilising the fact that $\ball[\eps] \cong \R^n$, we have a local trivialization $(U', V', \phi')$ with
        \[ \phi': V' \xto{\phi} U' \times \ball[\eps] \xto{id \times \mu_\eps} U' \times \R^n \]
        and $V' := \inv{\phi}(U' \times \ball[\eps])$.

        Note that homeomorphism commutes with injection
        \[ \phi'(i(b)) =  (id \times \mu_\eps)(\phi(i(b))) = (id \times \mu_\eps)(b, 0) = (b, 0) = (id \times 0)(b)\]
        and projection maps
        \[ j(e) = \pi_1(\phi(e)) = \pi_1((id \times \mu_\eps)(\phi(e))) = \pi_1(\phi'(e)). \]
        
        % isomorphy
        \item $\bb'$ is isomorphic to $\bb$:

        Since $E' \sub E$, we can simply take the identity $E' \to E' \sub E$
        as our homeomorphism between neighborhoods of $i(B)$.
        Furthermore, the injection and projection maps for $\bb$ and $\bb'$ are the same,
        so they clearly commute with the identity.
    \end{enumerate}
\end{myproof}
\begin{myparagraph}
    Now that we introduced the basic concept of microbundles, we will take a look at some key examples.
\end{myparagraph}
% foreword example
\begin{myparagraph}
    The most obvious example for a microbundle is the standard microbundle.
\end{myparagraph}

% example trivial
\begin{myexample}[trivial microbundle]{microbundle::trivial}
    Given a topological space $B$, the \defterm{standard microbundle} $\be_B$ over $B$ is a diagram
    \[ \bundle{B}{B \times \R^n}{\iota}{\pi} \]
    where $\iota(b) := (b, 0)$ and $\pi(b, x) := b$.
    Furthermore, a microbundle $\bb$ over $B$ is \defterm{trivial} if it is isomorphic to $\be_B$.
\end{myexample}

% foreword lemma
\begin{myparagraph}
    In order to make it easier classifying microbundles as trivial,
    we provide a sharper description of what it means for a microbundle to be trival. 
\end{myparagraph}

% lemma trivial
\begin{mylemma}{microbundle::triviallemma}
    A microbundle $\bb$ over a paracompact hausdorff space $B$ is trivial
    if and only if there exists an open neighborhood $V$ of $i(B)$ such that $V \cong B \times \R^n$
    with injection and projection maps being compatible with this homeomorphism.
\end{mylemma}

% proof trivial
\begin{myproof}
    ``$\implies$''

    By restricting $E(\bb)$ to an open neighborhood and applying \intref{microbundle::total},
    we may assume that $E(\bb)$ is an open subset of $B \times \R^n$.

    Since $E(\bb)$ is a neighborhood of $B \times \{0\}$, there exist $B_i \sub B$ open and $0 < \varepsilon_i < 1$ with
    \[ \bigcup_{i \in I } B_i \times \ball[\varepsilon_i] \sub E(\bb)\]
    such that $\bigcup_{i \in I} B_i = B$.
    Without loss of generality, we may assume that the collection $\{B_i\}$ is locally finite because if not
    we can simply choose a locally finite refinement using the fact that $B$ is paracompact.

    Furthermore, from paracompactness and the hausdorff property we derive a partition of unity over $\{B_i\}$
    \[ f_i: B \to [0, 1] \twith \supp{f_i} \sub B_i\]
    such that $\sum_{i \in I}f_i = 1$.
    
    Now we define a map $\lambda: B \to (0, \infty)$ via
    \[ \lambda := \sum_{i \in I} \varepsilon_i f_i \]
    which has the property that $\abs{x} < \lambda(b) \implies (b, x) \in E(\bb)$ because
    \[ \abs{x} < \lambda(b) \]
    \[ \iff  \abs{x} < \varepsilon_{i_1} f_{i_1}(b) + \cdots + \varepsilon_{i_n} f_{i_n}(b) \]
    \[ \iff 0 < (\varepsilon_{i_1} - \abs{x}) f_{i_1}(b) + \cdots + (\varepsilon_{i_n} - \abs{x}) f_{i_n}(b) \]
    \[ \implies \exists i \in I: 0 < (\varepsilon_{i_1} - \abs{x}) f_{i_1}(b) \]
    \[ \implies (b, x) \in B_i \times \ball[\varepsilon_i]  \implies (b, x) \in E(\bb). \]

    Lastly, we have a homeomorphism between the open subset $\set{(b, x) \in B \times \R^n}{\abs{x} < \lambda(b)} \sub E(\bb)$ and $B \times \R^n$ via
    \[ (b, x) \mapsto (b, \frac{x}{\lambda(b) - \abs{x}}). \]
    Note that $(b, 0) \mapsto (b, 0)$ and hence this homeomorphism is compatible with injection and projection maps.

    ``$\impliedby$''

    This is simply a weakening of the definition of triviality. 
\end{myproof}
% foreword
\begin{myparagraph}
    The following example acts as the microbundle analog to the tangent bundle on a smooth manifold.
\end{myparagraph}

% tangent microbundle
\begin{myexample}[tangent microbundle]
    % def: tangent microbundle
    The \defterm{tangent microbundle} $\bt_M$ over a topological $d$-manifold $M$ is a diagram
    \[ \bundle{M}{M \times M}{\Delta}{\pi_1} \]
    where $\Delta(m) := (m, m)$ is the diagonal map and $\pi_1(m_1, m_2) := m_1$ is the projection map on the first component.
    % proof: microbundle properties
    \begin{myproof}[$\bt_M$ is a microbundle]
        Let $p \in M$ and $(U, \phi)$ be a chart over $p$.
        We explicitly construct a local trivialization
        % diagram
        \[\begin{tikzcd}
            & U \times U \ar[dr, "\pi_1"] \ar[dd, "{\psi}"] & \\
            U \ar[ur, "\Delta"] \ar[dr, "{(0, id)}"'] & & U \\
            & U \times \R^d \ar[ur, "\pi_1"'] &
        \end{tikzcd}\]
        where $\psi(u, \tilde{u}) := (u, \phi(u) - \phi(\tilde{u}))$.
        It's obvious that $(U, U \times U, \psi)$ meets all local triviality conditions.
    \end{myproof}
\end{myexample}
% underlying microbundle
\begin{myexample}[underlying microbundle]{microbundle::underlying}
    % def: underlying microbundle
    Let $\xi: E \xto{\pi} B$ be a $n$-dimensional vector bundle.
    The microbundle $\lvert \xi \rvert: \bundle{B}{E}{i}{\pi}$ where $i(b) := \phi_b(b, 0)$, where
    $\phi_b: U_b \times \R^n \to \pi^{-1}(U_b)$ is the local trivialization over a
    neighborhood $U_b \sub B$ of $b$. We call $\lvert \xi \rvert$ the \defterm{underlying microbundle} of $\xi$
    % proof: microbundle properties
    \begin{myproof}
        TODO
    \end{myproof}
\end{myexample}
% foreword example
\begin{myparagraph}
    The most obvious example for a microbundle is the standard microbundle.
\end{myparagraph}

% example trivial
\begin{myexample}[trivial microbundle]{microbundle::trivial}
    Given a topological space $B$, the \defterm{standard microbundle} $\be_B$ over $B$ is a diagram
    \[ \bundle{B}{B \times \R^n}{\iota}{\pi} \]
    where $\iota(b) := (b, 0)$ and $\pi(b, x) := b$.
    Furthermore, a microbundle $\bb$ over $B$ is \defterm{trivial} if it is isomorphic to $\be_B$.
\end{myexample}

% foreword lemma
\begin{myparagraph}
    In order to make it easier classifying microbundles as trivial,
    we provide a sharper description of what it means for a microbundle to be trival. 
\end{myparagraph}

% lemma trivial
\begin{mylemma}{microbundle::triviallemma}
    A microbundle $\bb$ over a paracompact hausdorff space $B$ is trivial
    if and only if there exists an open neighborhood $V$ of $i(B)$ such that $V \cong B \times \R^n$
    with injection and projection maps being compatible with this homeomorphism.
\end{mylemma}

% proof trivial
\begin{myproof}
    ``$\implies$''

    By restricting $E(\bb)$ to an open neighborhood and applying \intref{microbundle::total},
    we may assume that $E(\bb)$ is an open subset of $B \times \R^n$.

    Since $E(\bb)$ is a neighborhood of $B \times \{0\}$, there exist $B_i \sub B$ open and $0 < \varepsilon_i < 1$ with
    \[ \bigcup_{i \in I } B_i \times \ball[\varepsilon_i] \sub E(\bb)\]
    such that $\bigcup_{i \in I} B_i = B$.
    Without loss of generality, we may assume that the collection $\{B_i\}$ is locally finite because if not
    we can simply choose a locally finite refinement using the fact that $B$ is paracompact.

    Furthermore, from paracompactness and the hausdorff property we derive a partition of unity over $\{B_i\}$
    \[ f_i: B \to [0, 1] \twith \supp{f_i} \sub B_i\]
    such that $\sum_{i \in I}f_i = 1$.
    
    Now we define a map $\lambda: B \to (0, \infty)$ via
    \[ \lambda := \sum_{i \in I} \varepsilon_i f_i \]
    which has the property that $\abs{x} < \lambda(b) \implies (b, x) \in E(\bb)$ because
    \[ \abs{x} < \lambda(b) \]
    \[ \iff  \abs{x} < \varepsilon_{i_1} f_{i_1}(b) + \cdots + \varepsilon_{i_n} f_{i_n}(b) \]
    \[ \iff 0 < (\varepsilon_{i_1} - \abs{x}) f_{i_1}(b) + \cdots + (\varepsilon_{i_n} - \abs{x}) f_{i_n}(b) \]
    \[ \implies \exists i \in I: 0 < (\varepsilon_{i_1} - \abs{x}) f_{i_1}(b) \]
    \[ \implies (b, x) \in B_i \times \ball[\varepsilon_i]  \implies (b, x) \in E(\bb). \]

    Lastly, we have a homeomorphism between the open subset $\set{(b, x) \in B \times \R^n}{\abs{x} < \lambda(b)} \sub E(\bb)$ and $B \times \R^n$ via
    \[ (b, x) \mapsto (b, \frac{x}{\lambda(b) - \abs{x}}). \]
    Note that $(b, 0) \mapsto (b, 0)$ and hence this homeomorphism is compatible with injection and projection maps.

    ``$\impliedby$''

    This is simply a weakening of the definition of triviality. 
\end{myproof}
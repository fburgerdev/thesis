% foreword
\begin{myparagraph}
    As the definition of isomorphy already indicates, when studying microbundles,
    we are not interested in the entire total space
    but only in an arbitrarily small neighborhood of the base space.
    This is certainly one of the strongest conceptual differences between microbundles and classical vector bundles.
\end{myparagraph}

% proposition total
\begin{myproposition}{microbundle::total}
    Given a microbundle $\bundledef{\bb}{B}{E}{i}{j}$ over $B$,
    restricting the total space $E$ to an arbitrary neighborhood $E' \sub E$ of $i(B)$ leaves the microbundle unchanged.
    That is, the microbundle
    \[ \bundledef{\bb'}{B}{E'}{i}{j\restr{E'}} \]
    is isomorphic to $\bb$.
\end{myproposition}

% proof total
\begin{myproof}
    We prove this proposition in two steps.
    \begin{enumerate}
        % microbundle
        \item $\bb'$ is a microbundle:
        
        Since we take $i$ and $j$ from $\bb$, we only need to show local triviality.

        For an arbitrary $b \in B$, choose a local trivialization $(U, V, \phi)$ of $b$ in $\bb$.
        
        The image $\phi(V \cap E')$ is a neighborhood of $(b, 0)$.
        That follows from $\phi(i(b)) = (b, 0)$ and $V \cap E'$ being a neighborhood of $i(b)$.

        Hence, there exists a $U' \times \ball[\eps] \sub \phi(V \cap E')$ with $U'$ open and $\eps$ sufficiently small.

        By utilising the fact that $\ball[\eps] \cong \R^n$, we have a local trivialization $(U', V', \phi')$ with
        \[ \phi': V' \xto{\phi} U' \times \ball[\eps] \xto{id \times \mu_\eps} U' \times \R^n \]
        and $V' := \phi^{-1}(U' \times \ball[\eps])$.

        Note that homeomorphism commutes with injection
        \[ \phi'(i(b)) =  (id \times \mu_\eps)(\phi(i(b))) = (id \times \mu_\eps)(b, 0) = (b, 0) = (id \times 0)(b)\]
        and projection maps
        \[ j(e) = \pi_1(\phi(e)) = \pi_1((id \times \mu_\eps)(\phi(e))) = \pi_1(\phi'(e)). \]
        
        % isomorphy
        \item $\bb'$ is isomorphic to $\bb$:

        Since $E' \sub E$, we can simply take the identity $E' \to E' \sub E$ as our homeomorphism between neighborhoods of $i(B)$.
        Furthermore, the injection and projection maps for $\bb$ and $\bb'$ are the same, so they clearly commute with the identity.
    \end{enumerate}
\end{myproof}
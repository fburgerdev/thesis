% definition extend
\begin{mydefinition}{induced::extend}{}
    Let $\bb$ and $\bb'$ be two microbundles over $B$ and $B'$ where $B' \sub B$.
    We say that $\bb$ is an \defterm{extension} of $\bb'$ over $B$, if
    \[ \bb\restr{B'} \cong \bb'. \]
\end{mydefinition}

% foreword cone / cylinder
\begin{myparagraph}
    % cone
    For a topological space $X$, we define the \defterm{cone} of $X$ to be 
    \[ CX := \cyl{X} / X \times \{1\} \]
    and the \defterm{mapping cone} of a map $f: A \to B$ to be
    \[ B \sqcup_f CA := B \sqcup CA / \sim \]
    where $(a, 0) \sim b \iff f(a) = b$.
    
    % cylinder
    Similarly, we define the \defterm{cylinder} of $X$ to be
    \[ MX := \cyl{X} \]
    and the \defterm{mapping cylinder} of a map $f: A \to B$ to be
    \[ B \sqcup_f MA := B \sqcup MA / \sim \]
    where $(a, 0) \sim b \iff f(a) = b$.
\end{myparagraph}

% lemma cone
\begin{mylemma}{induced::cone}{}
    Let $A$ be a paracompact hausdorff space.
    A microbundle $\bb$ over $B$ can be extended to a microbundle
    over the mapping cone $B \sqcup_f CA$ if and only if $\ind{f}\bb$ is trivial.
\end{mylemma}

% proof cone
\begin{myproof}
    We show both implications.

    % =>
    `$\implies$'

    Let $\bb'$ be an extension of $\bb$ over $B \sqcup_f CA$.

    The composition $A \xto{f} B \incl B \sqcup_f CA$ is
    null-homotopic via the homotopy
    \[ H(a, t) := [a, t] \]
    because $H(a, 0) = [a, 0] = [f(a)] = (\iota \circ f)(a)$
     and $H(a, 1) = [a, 1] = [\tilde{a}, 1] = H(\tilde{a}, 1)$.
    From the Homotopy \intref{homotopy::theorem},
    which we will prove in \intref{chapter::homotopy},
    it follows that $\ind{(\iota \circ f)}\bb'$ is
    isomorphic to $\ind{const}\bb'$ and hence trivial.

    Since $\ind{(\iota \circ f)}\bb' = \ind{f}(\ind{\iota}\bb') = \ind{f}\bb$,
    it follows that $\ind{f}\bb$ is trivial.

    % <=
    `$\impliedby$'

    Let $\ind{f}\bb$ be trivial.

    In contrast to the mapping cone,
    there exists a natural retraction from the mapping cylinder to the attached space
    \[ r: B \sqcup_f MA \to B \twith r([a, t]) = f(a) \]
    The diagram
    \[ A \times \{1\} \incl B \sqcup_f MA \xto{r} B \]
    equals $f$ if we consider $A = \cylup{A}$.
    It follows that
    \[ \ind{r}\bb\restr{A \times \{1\}} = \ind{(r \circ \iota)}\bb = \ind{f}\bb \]
    is trivial.

    From \intref{induced::trivial} and the retraction $(a, t) \mapsto (a, 1)$
    it follows that $\ind{r}\bb\restr{A \times [\half, 1]}$ is trivial.
    Since $A$ is paracompact hausdorff and by \intref{microbundle::paracompact},
    there exists a homoemorphism
    \[ \psi: V \isomto A \times [\half, 1] \times \R^n \]
    where $V$ is a neighborhood of $i_r(B)$ in $E(\ind{r}\bb\restr{A \times [\half, 1]})$.
    Without loss of generality,
    we may assume that $V = E(\ind{r}\bb\restr{A \times [\half, 1]})$
    by removing a closed subset of $E(\ind{r}\bb\restr{A \times [\half, 1]})$
    if necessary and applying \intref{microbundle::total}.

    Now we construct an extension
    \[ \bundledef{\bb'}{B \sqcup_f CA}{E'}{i'}{j'} \]
    with
    \begin{itemize}
        \item $E' := E(\ind{r}\bb) / \inv{\psi}(A \times \{1\} \times \{x\})$ (for every $x \in \R^n$)
        \item $i'([a, t]) := [i_r([a, t])]$
        \item $j'([[a, t], e]) := [j_r([a, t], e)] = [a, t]$
    \end{itemize}
    The injection $i'$ is well-defined because $i_r$ maps every
    representative $[a, 1]$ to the same equivalence class of $E'$.
    Similarly, the projection $j'$ is well-defined because
    \[ [[a, t], e] = [[a', t'], e'] \implies  [a, t] = [a, t']. \]

    Both $i'$ and $j'$ are continuous
    by the construction of the quotient space topology.
    Also, $j' \circ i' = id_{B \sqcup_f CA}$ because
    \[ j'(i'([a, t])) = j'([i_r(a, t)]) = [j_r(i_r(a, t))] = [a, t]. \]
    
    It remains to be shown that $\bb'$ is locally trivial:

    For points $[a, t] \in B \sqcup_f CA $ far away from the critical point $[a, 1]$,
    we can simply take a local trivialization of $[a, t]$ in $\ind{r}\bb$.

    When a point is `close' to the critical point, say $t > \half$,
    we can take $\psi$ to be the homeomorphism for our local trivialization.
    By construction, $\psi$ respects the quotient $\pi: E(\ind{r}\bb) \subm E'$.
    It follows that $(\frst{\psi}(\pi(V)), \pi(V), \psi)$
    yields a local trivialization in $\bb'$.
\end{myproof}
% foreword
\begin{myparagraph}
    Last, we show a theorem about whitney sums that will be essential in the proof of Milnor's theorem.
    For its prove, we need to use the following proposition that will be deferred until \autoref{chapter::suspension}.
\end{myparagraph}

% bouqet lemma
\begin{myproposition}{whitney::bouqet}
    Let $\bb$ be a microbundle over a ``bouqet'' of spheres $B$, meeting at a single point.
    There exists a map $r: B \to B$ such that $\whitney{\bb}{\ind{r}\bb}$ is trivial.
\end{myproposition}

% theorem trivialize
\begin{mytheorem}{whitney::theorem}
    Let $\bb$ be a microbundle over a $d$-dimensional simplicial complex $B$.
    Then there exists a microbundle $\bn$ over $B$ so that the whitney sum $\whitney{\bb}{\bn}$ is trivial.
\end{mytheorem}

% proof trivialize
\begin{myproof}
    We prove this theorem by induction over $d$.

    % start
    (Start of induction)

    A $1$-dimensional simplicial complex is just a bouquet of circles.
    Therefore, the start of induction follows directly from \intref{whitney::bouqet}.   

    % step
    (Inductive Step)

    Let $B'$ be the $(d - 1)$-skeleton of $B$ and $\bn'$ be it's corresponding microbundle
    such that $\whitney{\bb\restr{B'}}{\bn'}$ is trivial.

    \begin{enumerate}
        % simplex
        \item $\whitney{\bn'}{\be{B'}}$ can be extended over any $d$-simplex $\sigma$:

        Consider the equation
        \[
            (\whitney{\bn'}{\be{B'}})\restr{\partial\sigma}
            = \whitney{\bn'\restr{\partial\sigma}}{\be{B'}\restr{\partial\sigma}}
            = \whitney{\bn'\restr{\partial\sigma}}{\bb\restr{\partial\sigma}}
            = (\whitney{\bn'}{\bb\restr{B'}})\restr{\partial\sigma}
        \]
        in which we used the previous lemma and \intref{induced::simplex}
        for $\be{B'}\restr{\partial\sigma} = \bb\restr{\partial\sigma}$.
        Since $(\whitney{\bn'}{\bb\restr{B'}})\restr{\partial\sigma}$ is trivial, the claim follows from \intref{induced::simplex}.

        % extend
        \item $\whitney{\bn'}{\be{B'}}$ can be extended over $B$:

        The difficulty is that the individual $d$-simplices are not well-seperated.
        Let $B''$ denote $B$ with small open $d$-cells removed from every $d$-simplex.
        Since $B'$ is a retract of $B''$ we can extend $\whitney{\bn'}{\be{B'}}$ to a microbundle $\nu$ over $B''$.

        Now we can extend $\nu$ over every $d$-simplex by taking the extensions
        over every simplex individually (using 1.) and glueing its total spaces together along $E(\nu)$.
        The injection and projection maps can be constructed
        by glueing the injetion and projection maps of the individual extensions together.

        We denote the resulting microbundle by $\eta$.

        % trivial
        \item
        Consider the mapping cone $B \sqcup CB'$ over the inclusion $B' \incl B$.
        Since
        \[
            (\whitney{\bb}{\eta})\restr{B'}
            = \whitney{\bb\restr{B'}}{\eta\restr{B'}}
            = \whitney{\bb\restr{B'}}{(\whitney{\bn'}{\be{B'}})}
            = \whitney{(\whitney{\bb\restr{B'}}{\bn'})}{\be{B'}}
            = \whitney{\be{B'}}{\be{B'}}
        \]
        is trivial, it follows from \intref{induced::cone} that we can extend $\whitney{\bb}{\eta}$ over $B \sqcup CB'$
        which will be denoted by $\xi$.

        The mapping cone $B \sqcup CB'$ has the homotopy type of a bouquet of spheres
        by transfering $B'$ along $CB'$ collapsing to a single point.
        Since any $d$-simplex is homotopic to a $d$-disc and it's border is collapsed, we get the homotopy of a $(d - 1)$-sphere.
        
        With \intref{homotopy::theorem} and \intref{whitney::bouqet},
        we conclude that there exists a microbundle $\bn$ such that $(\whitney{\xi}{\bn})\restr{B}$ is trivial.
        The equation
        \[
            \be{B}
            = (\whitney{\xi}{\bn})\restr{B}
            = \whitney{\xi\restr{B}}{\bn\restr{B}}
            = \whitney{(\whitney{\bb}{\eta})}{\bn\restr{B}}
            = \whitney{\bb}{(\whitney{\eta}{\bn\restr{B}})}
        \]
        completes the proof.
    \end{enumerate}
\end{myproof}
% foreword
\begin{myparagraph}
    Last, we show a theorem about whitney sums that will be essential in the proof of Milnor's theorem.
    For its prove, we need to use the following proposition that will be deferred until \autoref{chapter::suspension}.
\end{myparagraph}

% bouqet lemma
\begin{myproposition}{whitney::bouqet}
    Let $\bb$ be a microbundle over a ``bouqet'' of spheres $B$, meeting at a single point.
    There exists a map $r: B \to B$ such that $\bb \oplus r^*\bb$ is trivial.
\end{myproposition}

% theorem trivialize
\begin{mytheorem}{whitney::theorem}
    Let $\bb$ be a microbundle over a $d$-dimensional simplicial complex $B$.
    Then there exists a microbundle $\bn$ over $B$ so that the whitney sum $\bb \oplus \bn$ is trivial.
\end{mytheorem}

% proof trivialize
\begin{myproof}
    We prove this theorem by induction over $d$.

    % start
    (Start of induction)

    A $1$-dimensional simplicial complex is just a bouquet of circles.
    Therefore, the start of induction follows directly from \intref{whitney::bouqet}.   

    % step
    (Inductive Step)

    Let $B'$ be the $(d - 1)$-skeleton of $B$ and $\bn'$ be it's corresponding microbundle
    such that $\bb\restr{B'} \oplus \bn'$ is trivial.

    \begin{enumerate}
        % simplex
        \item $\bn' \oplus \be_{B'}$ can be extended over any $d$-simplex $\sigma$:

        Consider the equation
        \[
            (\bn' \oplus \be_{B'})\restr{\partial\sigma}
            = \bn'\restr{\partial\sigma} \oplus \be_{B'}\restr{\partial\sigma}
            = \bn'\restr{\partial\sigma} \oplus \bb\restr{\partial\sigma}
            = (\bn' \oplus \bb\restr{B'})\restr{\partial\sigma}
        \]
        in which we used the previous lemma and \intref{induced::simplex} for $\be_{B'}\restr{\partial\sigma} = \bb\restr{\partial\sigma}$.
        Since $(\bn' \oplus \bb\restr{B'})\restr{\partial\sigma}$ is trivial, the claim follows from \intref{induced::simplex}.

        % extend
        \item $\bn' \oplus \be_{B'}$ can be extended over $B$:

        The difficulty is that the individual $d$-simplices are not well-seperated.
        Let $B''$ denote $B$ with small open $d$-cells removed from every $d$-simplex.
        Since $B'$ is a retract of $B''$ we can extend $\bn' \oplus \be_{B'}$ over $B''$ and now apply the first statement.
        We denote the resulting microbundle by $\eta$.
        % TODO: explain further by providing the total space

        % trivial
        \item
        Consider the mapping cone $B \sqcup CB'$ over the inclusion $B' \incl B$.
        Since
        \[
            (\bb \oplus \eta)\restr{B'}
            = \bb\restr{B'} \oplus \eta\restr{B'}
            = \bb\restr{B'} \oplus (\bn' \oplus \be_{B'})
            = (\bb\restr{B'} \oplus \bn') \oplus \be_{B'}
            = \be_{B'} \oplus \be_{B'}
        \]
        is trivial, it follows from \intref{induced::cone} that we can extend $\bb \oplus \eta$ over $B \sqcup CB'$
        which will be denoted by $\xi$.

        The mapping cone $B \sqcup CB'$ has the homotopy type of a bouquet of spheres
        by transfering $B'$ along $CB'$ collapsing to a single point.
        Since any $d$-simplex is homotopic to a $d$-disc and it's border is collapsed, we get the homotopy of a $(d - 1)$-sphere.
        
        With \intref{homotopy::theorem} and \intref{whitney::bouqet}, we conclude that there exists a microbundle $\bn$ such that $(\xi \oplus \bn)\restr{B}$ is trivial.
        The equation
        \[
            \be_{B}
            = (\xi \oplus \bn)\restr{B}
            = \xi\restr{B} \oplus \bn\restr{B}
            = (\bb \oplus \eta) \oplus \bn\restr{B}
            = \bb \oplus (\eta \oplus \bn\restr{B})
        \]
        completes the proof.
    \end{enumerate}
\end{myproof}
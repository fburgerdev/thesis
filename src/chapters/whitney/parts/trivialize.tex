% foreword
\begin{myparagraph}
    Last, we show a theorem about whitney sums that will be essential in the proof of Milnor's theorem.
    For its prove, we need to use the following proposition that will be deferred until \autoref{chap:suspension}.
\end{myparagraph}

% bouqet lemma
\begin{myproposition}\label{whitney::bouqet}
    Let $\bb$ be a microbundle over a ``bouqet'' of spheres $B$, meeting at a single point.
    There exists a map $r: B \to B$ such that $\bb \oplus r^*\bb$ is trivial.
\end{myproposition}

% statement
\begin{mytheorem}\label{whitney::theorem}
    Let $\bb$ be a microbundle over a $d$-dimensional simplicial complex $B$.
    Then there exists a microbundle $\bn$ over $B$ so that the Whitney sum $\bb \oplus \bn$ is trivial.
\end{mytheorem}
% proof
\begin{myproof}
    We prove this theorem by induction over $d$.

    % Start if Induction
    (Start of induction)

    A $1$-dimensional simplicial complex is just a bouquet of circles, therefore
    the start of induction follows directly from the \intref{whitney::bouqet}.   

    % Start if Induction
    (Inductive Step)

    Let $B'$ be the $(d - 1)$-skeleton of $B$ and $\bn'$ it's corresponding microbundle
    such that $\bb\restr{B'} \oplus \bn'$ is trivial.

    \begin{enumerate}
        % statement
        \item $\bn' \oplus \be_{B'}$ can be extended over any $d$-simplex $\sigma$:

        % proof
        Consider the following:
        \[
            (\bn' \oplus \be_{B'})\restr{\partial\sigma}
            = \bn'\restr{\partial\sigma} \oplus \be_{B'}\restr{\partial\sigma}
            = \bn'\restr{\partial\sigma} \oplus \bb\restr{\partial\sigma}
            = (\bn' \oplus \bb\restr{B'})\restr{\partial\sigma}
        \]
        Since $(\bn' \oplus \bb\restr{B'})\restr{\partial\sigma}$ is trivial, the claim follows from \intref{induced::simplex}.

        % statement
        \item $\bn' \oplus \be_{B'}$ can be extended over $B$:

        % proof
        The difficulty is that the individual $d$-simplices are not well-seperated.
        Let $B''$ denote $B$ with small open $d$-cells removed from every $d$-simplex.
        Since $B'$ is a retract of $B''$ we can extend $\bn' \oplus \be_{B'}$ over $B''$ and now apply the first statement.
        We denote the resulting microbundle by $\eta$.

        % finish
        \item
        Consider the mapping cone $B \sqcup CB'$ over the inclusion $B' \incl B$.
        Since
        \[
            (\bb \oplus \eta)\restr{B'}
            = \bb\restr{B'} \oplus \eta\restr{B'}
            = \bb\restr{B'} \oplus (\bn' \oplus \be_{B'})
            = (\bb\restr{B'} \oplus \bn') \oplus \be_{B'}
            = \be_{B'} \oplus \be_{B'}
        \]
        which is trivial, by \intref{induced::simplex} we can extend $\bb \oplus \eta$ over $B \sqcup CB'$ denoted by $\xi$.
        However, $B \sqcup CB'$ has the homotopy type of a bouquet of spheres and by \intref{homotopy::theorem} and \intref{whitney::bouqet} there exists a microbundle $\bn$ such that $(\xi \oplus \bn)\restr{B}$ is trivial.
        The formula
        \[
            \be_{B}
            = (\xi \oplus \bn)\restr{B}
            = \xi\restr{B} \oplus \bn\restr{B}
            = (\bb \oplus \eta) \oplus \bn\restr{B}
            = \bb \oplus (\eta \oplus \bn\restr{B})
        \]
        utilizing the compatibility between whitney sums and induced microbundles completes the proof.
    \end{enumerate}
\end{myproof}
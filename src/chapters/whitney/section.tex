\chapter{The Whitney Sum}
\begin{myparagraph}
    In the last chapter we saw how we can pull back the base space of a given microbundle using a map.
    In this chapter, another central construction is introduced, the ``Whitney Sum''.
    It allows us to construct a microbundle given two microbundles over the same base space.
    The fiber dimension of the resulting microbundle is just the sum of the fiber dimensions of the initial microbundles.
\end{myparagraph}
% definition
\begin{mydefinition}
    Let $\bb_1$ and $\bb_1$ be two microbundles over $B$ with fibre-dimension $n_1$ and $n_2$.
    The \defterm{whitney sum} $\bb_1 \oplus \bb_2$ is a microbundle $\bundle{B}{E}{i}{j}$ where
    \begin{itemize}
        \item $E := \{ (e_1, e_2) \in E(\bb_1) \times E(\bb_2) \mid j_1(e_1) = j_2(e_2)\}$
        \item $i(b) := (i_1(b), i_2(b))$
        \item $j(e_1, e_2) := j_1(e_1) = j_2(e_2)$
    \end{itemize}
    with fibre-dimension $n_1 + n_2$.
\end{mydefinition}
% proof
\begin{myproof}
Let $b \in B$. \\
Choose $U_1, V_1, \phi_1$ and $U_2, V_2, \phi_2$ accordingly from the local trivialization of $b$ over $\bb_1$ and $\bb_2$:
\begin{itemize}
    \item $U := U_1 \cap U_2$
    \item $V := (V_1 \times V_2) \cap E$
    \item $\phi: V \to U \times \R^{n_1 + n_2}; \phi(e_1, e_2) := (\phi_1^{(1)}(e_1), \phi_1^{(2)}(e_1) \times  \phi_2^{(2)}(e_2))$
\end{itemize}
Note that $\phi_1^{(1)}(e_1) = \phi_2^{(1)}(e_2)$.
Local triviality follows directly from it's components.
\end{myproof}
% lemma
\begin{mylemma}{whitney::compatibility}
    Let $\bb_1$ and $\bb_1$ be two microbundles over $B$ and let $f: A \to B$ be a map.
    The induced microbundle and the whitney sum are compatible, that is
    \[ \ind{f}(\whitney{\bb_1}{\bb_2}) \cong \whitney{\ind{f}\bb_1}{\ind{f}\bb_2}. \]
\end{mylemma}

% proof
\begin{myproof}
    From the definition of the induced microbundle and the whitney sum, we can explicitly write the total spaces
    \[ E(\ind{f}(\whitney{\bb_1}{\bb_2})) \]
    \[ = \{(a, (e_1, e_2)) \in A \times (E(\bb_1) \times E(\bb_2)) \mid j_1(e_1) = j_2(e_2) = f(a) \} \]
    and
    \[ E(\whitney{\ind{f}\bb_1}{\ind{f}\bb_2}) \]
    \[ = \{(e_1, e_2) \in E(\ind{f}\bb_1) \times E(\ind{f}\bb_2) \mid j_1(e_1) = j(e_2)\}\]
    \[ = \{((a_1, e_1), (a_2, e_2)) \in (A \times E(\bb_1)) \times (A \times E(\bb_2)) \mid \]
    \[ j(a_1, e_1) = j(a_2, e_2) \land f(a_i) = j(e_i)\} \]
    The two total spaces are homeomorphic via $\phi(a, (e_1, e_2)) := ((a, e_1), (a, e_2))$ with $\inv{\phi}((a, e_1), (a, e_2)) = (a, (e_1, e_2))$.
    Homeomorphy of $\phi$ follows from the continuity of $\phi$ and $\inv{\phi}$, which is given since both $\phi$ and $\inv{\phi}$ are composed by identity maps.
    
    It remains to be shown that injection and projection maps $i$ and $j$ for $E(\ind{f}(\whitney{\bb_1}{\bb_2}))$
    and $i'$ and $j'$ for $\whitney{\ind{f}\bb_1}{\ind{f}\bb_2}$ agree under $\phi$.

    This follows from
    \[ \phi(i(a)) = \phi(a, i_1(f(a)), i_2(f(a))) \]
    \[ = ((a, i_1(f(a))), (a, i_2(f(a)))) = (i_1'(a), i_2'(a)) = i'(a) \]
    and
    \[ j(a, e_1, e_2) = a = j'((a, e_1), (a, e_2)) = j'(\phi(a, e_1, e_2)). \]
\end{myproof}
% foreword
\begin{myparagraph}
    Last, we show a theorem about whitney sums that will be
    essential in the proof of Milnor's theorem.
    For its prove, we need to use the following proposition
    that will be deferred until \intref{chapter::suspension}.
\end{myparagraph}

% bouqet lemma
\begin{myproposition}{whitney::bouqet}{}
    Let $\bb$ be a microbundle over a `bouqet' of spheres $B$, meeting at a single point.
    Then there exists a map $r: B \to B$ such that $\whitney{\bb}{\ind{r}\bb}$ is trivial.
\end{myproposition}

% theorem trivialize
\begin{mytheorem}{whitney::theorem}{}
    Let $\bb$ be a microbundle over a $d$-dimensional simplicial complex $B$.
    Then there exists a microbundle $\bn$ over $B$ so that the whitney sum $\whitney{\bb}{\bn}$ is trivial.
\end{mytheorem}

% proof trivialize
\begin{myproof}
    We prove this theorem by induction over $d$.

    % start
    (Start of induction)

    A $1$-dimensional simplicial complex is just a bouquet of circles.
    Hence, the start of induction follows directly from \intref{whitney::bouqet}.   

    % step
    (Inductive Step)

    Let $B'$ be the $(d - 1)$-skeleton of $B$ and $\bn'$ be it's corresponding microbundle
    such that $\whitney{\bb\restr{B'}}{\bn'}$ is trivial.

    \begin{enumerate}
        % simplex
        \item $\whitney{\bn'}{\be{B'}}$ can be extended over any $d$-simplex $\sigma$:

        Consider the equation
        \[
            (\whitney{\bn'}{\be{B'}})\restr{\partial\sigma}
            = \whitney{\bn'\restr{\partial\sigma}}{\be{B'}\restr{\partial\sigma}}
            = \whitney{\bn'\restr{\partial\sigma}}{\bb\restr{\partial\sigma}}
            = (\whitney{\bn'}{\bb\restr{B'}})\restr{\partial\sigma}
        \]
        in which we used the previous lemma and \intref{induced::simplex}
        for $\be{B'}\restr{\partial\sigma} = \bb\restr{\partial\sigma}$.
        Since $(\whitney{\bn'}{\bb\restr{B'}})\restr{\partial\sigma}$ is trivial, the claim follows from \intref{induced::simplex}.

        % extend
        \item $\whitney{\bn'}{\be{B'}}$ can be extended over $B$:

        The difficulty is that the individual $d$-simplices are not well-seperated.
        Let $B''$ denote $B$ with small open $d$-cells removed from every $d$-simplex.
        Since $B'$ is a retract of $B''$ we can extend $\whitney{\bn'}{\be{B'}}$ to a microbundle $\nu$ over $B''$.

        Now we can extend $\nu$ over every $d$-simplex by taking the extensions
        over every simplex individually (using 1.) and glueing its total spaces together along $E(\nu)$.
        The injection and projection maps can be constructed
        by glueing the injetion and projection maps of the individual extensions together.

        We denote the resulting microbundle by $\eta$.

        % trivial
        \item
        Consider the mapping cone $B \sqcup CB'$ over the inclusion $B' \incl B$.
        Since
        \[
            (\whitney{\bb}{\eta})\restr{B'}
            = \whitney{\bb\restr{B'}}{\eta\restr{B'}}
            = \whitney{\bb\restr{B'}}{(\whitney{\bn'}{\be{B'}})}
            = \whitney{(\whitney{\bb\restr{B'}}{\bn'})}{\be{B'}}
            = \whitney{\be{B'}}{\be{B'}}
        \]
        is trivial, it follows from \intref{induced::cone} that we can extend $\whitney{\bb}{\eta}$ over $B \sqcup CB'$
        which will be denoted by $\xi$.

        The mapping cone $B \sqcup CB'$ has the homotopy type of a bouquet of spheres
        by transfering $B'$ along $CB'$ collapsing to a single point.
        Since any $d$-simplex is homotopic to a $d$-disc and it's border is collapsed, we get the homotopy of a $(d - 1)$-sphere.
        
        With \intref{homotopy::theorem} and \intref{whitney::bouqet},
        we conclude that there exists a microbundle $\bn$ such that $(\whitney{\xi}{\bn})\restr{B}$ is trivial.
        The equation
        \[
            \be{B}
            = (\whitney{\xi}{\bn})\restr{B}
            = \whitney{\xi\restr{B}}{\bn\restr{B}}
            = \whitney{(\whitney{\bb}{\eta})}{\bn\restr{B}}
            = \whitney{\bb}{(\whitney{\eta}{\bn\restr{B}})}
        \]
        completes the proof.
    \end{enumerate}
\end{myproof}
% foreword williamson
\begin{myparagraph}
    The bundle-germ is indeed a generalization of the isomorphism germ, as the following proposition illustrates.
\end{myparagraph}

% williamson
\begin{myproposition}[Williamson]{homotopy::williamson}
    Let $\bb$ and $\bb'$ be two microbundles over $B$.
    A bundle-germ $\bgerm{F}{\bb}{\bb'}$ covering the identity map is an isomorphism-germ.
\end{myproposition}

% foreword lemma
\begin{myparagraph}
    First, however, we show a lemma that helps us to prove the proposition.
\end{myparagraph}

% lemma
\begin{mylemma}{homotopy::ball}
    If a homeomorphism $\phi: \clball[2] \isomto \phi(\R^n) \sub \R^n$ satisfies
    \[ \abs{\phi(x) - x} < 1, \forall x \in \clball[2] \]
    then $\clball[1] \sub \phi(\clball[2])$.
\end{mylemma}

% proof of lemma
\begin{myproof}[of the lemma]
    % TODO: Check proof
    Consider $\phi(2S^n)$ where $2S^n$ denotes the $n$-sphere of radius $2$.
    The condition for $\phi$ yields $1 < \abs{\phi(s)}, \forall s \in 2S^n$.
    Since $\clball[2]$ has trivial homology groups which are preserved under homeomorphisms,
    $\phi(\clball[2])$ must have trivial homology groups as well.

    From this we can conclude that $\clball[1]$ must be contained in $\phi(\clball[2])$,
    because otherwise ``holes'' would form which would result in non-trivial homology groups of $\phi(\clball[2])$.
    % Consider the subspace $\R^n - 2S^n$ where $2S^n$ denotes the $n$-sphere of radius $2$.
    % This subspace consists of the two path-components $\ball[2]$ and $\R^n - \clball[2]$.
    % Since homeomorphisms preserve path-components, we know that $\phi(\R^n - 2S^n)$ consists of the two path-components $\phi(\ball[2])$ and $\phi(\R^n - \clball[2])$.
    % Now, applying the requirement yields $\abs{\phi(0)} < 1$ and $1 < \abs{\phi(s)}$ for every $s \in 2S^n$.
    % Hence, there exists a path between $\phi(0)$ and any $x \in \clball[1]$ in $\phi(\R^n - 2S^n)$, e.g. a straight line.
    % Therefore, every $x \in \clball[1]$ is contained in the same path-component of $\phi(\R^n - 2S^n)$.
    % The statement follows from $\phi(0) \in \phi(\clball[2])$.
\end{myproof}

% proof of williamson
\begin{myproof}[of the proposition]
    Let $f$ be a representative for $F$.
    First we assume a special and then generalize the result to show the proposition.
    \begin{enumerate}
        % special case
        \item Let $f$ map from $B \times \R^n$ to $B \times \R^n$:
        
        Since $F$ covers the identity, $f$ is of the form
        \[ f(b, x) = (b, g_b(x)) \]
        where $g_b: \R^n \to \R^n$ are individual maps.
        Since the $g_b$ are continuous and injective, it follows from the \extref{domain invariance theorem} that the $g_b$ are open maps.
        
        Let $(b_0, x_0) \in B \times \R^n$ and let $\varepsilon > 0$.
        Since $g_{b_0}$ is an open map, there exists a $\delta > 0$ such that
        $\clball[2\delta][x_1] \sub g_{b_0}(\clball[\varepsilon][x_0])$ where $x_1 := g_{b_0}(x_0)$.

        % estimation
        We claim that there exists a neighborhood $V \sub B$ of $b_0$ such that
        \[ \abs{g_b(x) - g_{b_0}(x)} < \delta \]
        for every $b \in V$ and $x \in \clball[\varepsilon][x_0]$.
        
        To show that, consider $\phi(b, x) := g_b(x) - g_{b_0}(x)$.
        The open set $\phi^{-1}(\ball[\delta])$ is a neighborhood of $\{b_0\} \times \R^n$ since $\phi(b_0, x) = 0$.
        Hence, there exist open subsets $V_x \sub B$ and $W_x \sub \R^n$ such that
        \[ \bigcup_{x \in \clball[\varepsilon][x_0]} V_x \times W_x \sub \inv{\phi}(\clball[\delta]) \]
        and $x \in W_x$.
        Since $\clball[\varepsilon][x_0]$ is compact, there exist $x_1, \dots, x_n \in \clball[\varepsilon][x_0]$ with $\clball[\varepsilon][x_0] \sub \bigcup_{i = 1}^{n} V_{x_i}$.
        The claim follows with $V := V_{x_1} \cap \dots \cap V_{x_n}$ which is open by forming the intersection over finitely many open sets.

        % apply lemma
        Now we want to apply the previous lemma:

        Consider the homeomorphism
        \[ \clball[2\delta][x_1] \isomto g_b \circ \inv{g_{b_0}}(\clball[2\delta][x_1]) \]
        for an arbitrary $b \in V$.
        Since
        \[ \clball[2\delta][x_1] \sub g_{b_0}(\clball[\varepsilon][x_0]) \implies g_{b_0}^{-1}(\clball[2\delta][x_1]) \sub \clball[\varepsilon][x_0] \]
        we conclude from the above that
        \[ \abs{(g_b \circ g_{b_0}^{-1})(x) - x} < \delta, \forall x \in \clball[2\delta][x_1] \]
        It follows that, by translation and scaling, $g_b \circ g_{b_0}^{-1}\restr{\clball[2\delta][x_1]}$ satisfies the conditions of \intref{homotopy::ball}.
        Therefore, $\clball[\delta][x_1] \sub (g_b \circ g_{b_0}^{-1})(\clball[2\delta][x_0])$ and so $\clball[\delta][x_1] \sub g_b(\clball[\varepsilon][x_0])$.

        % conclusion
        From
        \[ V \times \clball[\delta][x_1] \sub g(V \times \clball[\varepsilon][x_0]) \]
        it follows that $f$ is an open map.

        % general case
        \item Glue together $f: U_f \to E(\bb')$ along local trivializations:

        For an arbitrary $b \in B$, choose local trivializations $(U, V, \phi)$ and $(U', V', \phi')$ of $b$ in $\bb$ and $\bb'$.
        Without loss of generality we may assume that $U = U'$ because otherwise we can choose $V = \inv{\phi}(U \cap U')$ and $V' = \inv{\phi'}(U \cap U')$
        and restrict $\phi$ and $\phi'$ accordingly.   
        
        First, we restrict $f$ to $V \cap \inv{f}(V')$. Since $V \cap \inv{f}(V')$ is an open neighborhood of $i(b)$ and contained in $V$, we can choose
        an open neighborhood $\tilde{U} \sub U$ of $i(b)$ and $\varepsilon > 0$ such that $\inv{\phi}(\tilde{U} \times \ball[\varepsilon]) \sub V \cap \inv{f}(V')$.
        
        This yields a map $U' \times \R^n \to U' \times \R^n$ with
        \[ \tilde{U} \times \R^n \cong  \tilde{U} \times \ball[\varepsilon] \isomto \inv{\phi}(\tilde{U} \times \ball[\varepsilon]) \xto{f} U' \times \R^n \sub U \times \R^n \]
        that is injective and fibre-preserving and therefore an open map (apply 1.).
        It follows that $f: \inv{\phi}(\tilde{U} \times \ball[\varepsilon]) \to V'$ must be an open map as well since the other composing maps are homeomorphisms.

        By glueing the $\inv{\phi}(\tilde{U} \times \ball[\varepsilon])$ together over all $b \in B$, we see that $f$ is an open map.
    \end{enumerate}
\end{myproof}

% foreword corollary
\begin{myparagraph}
    We can easily generalize this to bundle-germs between microbundles over different base spaces:
\end{myparagraph}

% corollary
\begin{mycorollary}{homotopy::corollary}
    If a map $g: B \to B'$ is covered by a bundle-germ $F: \bb \double \bb'$, then $\bb$ is isomorphic to the induced microbundle $g^*\bb'$.
\end{mycorollary}
% proof
\begin{myproof}
    Let $f: U_f \to E'$ be a representative map for $F$.
    We define $F': \bb \double g^*\bb'$ by a representative
    \[ f': U_f \to E(g^*\bb') \twith f'(e) := (j(e), f(e)). \]
    Every $f'(e)$ actually lies in $E(g^*\bb')$ because
    \[ g(j(e)) = j'(f(e)) \]
    as we can see from the commutative diagram for bundle-germs.
    The germ $F'$ is an isomorphism-germ because $F$ is an isomorphism-germ.
    Applying the previous proposition on $F'$ proves the claim.
    % TODO: why is F' an isomorphims-germ?
\end{myproof}
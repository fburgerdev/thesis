% foreword williamson
\begin{myparagraph}
    The bundle-germ is indeed a generalization of the isomorphism germ, as the following proposition shows.
\end{myparagraph}

% williamson
\begin{myproposition}[Williamson]
    Let $\bb$ and $\bb'$ be two microbundles over $B$.
    A bundle-germ $\bgerm{F}{\bb}{\bb'}$ covering the identity map is an isomorphism-germ.
\end{myproposition}

% foreword lemma
\begin{myparagraph}
    First we show a lemma, which we will use to prove the proposition.
\end{myparagraph}

% lemma
\begin{mylemma}
    If a map $\phi: \R^n \isomto \phi(\R^n) \sub \R^n$ is a homeomorphism such that 
    \[ \abs{\phi(x) - x} < 1, \forall x \in \clball[2] \]
    then $\clball[1] \sub \phi(\clball[2])$.
\end{mylemma}

% proof of lemma
\begin{myproof}[of the lemma]
    Consider the subspace $\R^n - 2S^n$, which consists of the two path-components $\ball[2]$ and $\R^n - \clball[2]$.
    Since homeomorphism preserve path-components it follows that $\phi(\R^n - 2S^n)$ consists of the two path-components $\phi(\ball[2])$ and $\phi(\R^n - \clball[2])$.

    Now, applying the requirement yields $\abs{\phi(0)} < 1$ and $1 < \abs{\phi(s)}$ for every $s \in 2S^n$.
    Hence, there exists a path between $\phi(0)$ and any $x \in \clball[1]$ in $\phi(\R^n - 2S^n)$, for example a straight line.
    Therefore, every $x \in \clball[1]$ is contained within the same path-component of $\phi(\R^n - 2S^n)$.
    The statement follows from $\phi(0) \in \phi(\clball[2])$.
\end{myproof}

% proof of williamson
\begin{myproof}[of the proposition]
    Let $f$ be a representative map for $F$.
    We show the proposition in two steps.
    \begin{enumerate}
        % special case
        \item
        Assume $\bb$ and $\bb'$ to be trivial.

        This means that $f: B \times \R^n \to B \times \R^n$ is of the form
        \[ f(b, x) = (b, g_b(x)) \]
        where $g_b: \R^n \to \R^n$ are individual mappings.
        From the \extref{domain invariance theorem} it follows that $g_b$ are open mappings.
        Let $(b_0, x_0) \in B \times \R^n$ and $\varepsilon > 0$.
        Since $g_{b_0}$ is an open mapping, there exists a $\delta > 0$ such that
        $\clball[2\delta][x_1] \sub g_{b_0}(\clball[\varepsilon][x_0])$ where $x_1 := g_{b_0}(x_0)$.

        % estimation
        There exists a neighborhood $V \sub B$ of $b_0$ such that
        \[ \abs{g_b(x) - g_{b_0}(x)} < \delta \]
        for every $b \in V$ and $x \in \clball[\varepsilon][x_0]$.
        To show that, consider $\phi(b, x) := g_b(x) - g_{b_0}(x)$.
        The closed set $\phi^{-1}(\clball[\delta])$ is a neighborhood of $\{b_0\} \times \R^n$ since $\phi(b_0, x) = 0$.
        Therefore, for every $x \in \clball[\delta]$ exist $V_x \sub B$ and $U_x \sub \R^n$ open with $x \in U_x$ and $V_x \times U_x \sub \phi^{-1}(\clball[\varepsilon][x_0])$.
        Obviously, $\bigcup_{x \in \clball[\delta][x_1]} U_x$ is an open covering of $\clball[\delta][x_1]$ and since $\clball[\delta][x_1]$ is compact,
        there exist $x_1, \dots, x_n \in \clball[\delta][x_1]$ with $\clball[\delta][x_1] \sub \bigcup_{i = 1}^{n} U_{x_i}$.
        The claim follows via $V := V_{x_1} \cap \dots \cap V_{x_n}$.

        % apply lemma
        Now we want to apply the previous lemma:

        Consider the homeomorphism $g_b \circ g_{b_0}^{-1}$ for an arbitrary $b \in V$.
        Since
        \[ \clball[2\delta][x_1] \sub g_{b_0}(\clball[\varepsilon][x_0]) \implies g_{b_0}^{-1}(\clball[2\delta][x_1]) \sub \clball[\varepsilon][x_0] \]
        we conclude from the above that
        \[ \abs{(g_b \circ g_{b_0}^{-1})(x) - x} < \delta \]
        It follows that, by translation and scaling, $g_b \circ g_{b_0}^{-1}$ satisfies the requirements of the lemma.
        Therefore, $\clball[\delta][x_1] \sub (g_b \circ g_{b_0}^{-1})(\clball[2\delta][x_0])$ and so $\clball[\delta][x_1] \sub g_b(\clball[\varepsilon][x_0])$.

        % conclusion
        From
        \[ V \times \clball[\delta][x_1] \sub g(V \times \clball[\varepsilon][x_0]) \]
        it follows that $f$ is an open mapping.

        % general case
        \item
        Glue together $f$ from its local trivializations

        Choose a local trivialization $(U, V, \phi)$ over $b \in B$.
        First, we restrict $f$ to $f^{-1}(V)$.
        Since $f^{-1}(V)$ is a neighborhood of $i(b)$, we can choose an open neighborhood $V' \sub f^{-1}(V) \cap V$ of $i(b)$ of the form $U' \times \ball[\varepsilon]$.
        Now we have
        \[ U' \times \R^n \cong U' \times \ball[\varepsilon] \xto{f} U' \times \R^n \sub U \times \R^n \]
        a map $U' \times \R^n \to U' \times \R^n$ that is injective and fibre-preserving and therefore an open mapping (apply 1.).
        It follows that $f: V' \to V$ must have been injective and fibre-preserving as well.

        By glueing the $V'$ over all $b \in B$ together, we see that $f$ is an open mapping.
        This concludes the proof.
    \end{enumerate}
\end{myproof}

% corollary
\begin{mycorollary}
    If a map $g: B \to B'$ is covered by a bundle germ $F: \bb \double \bb'$ then $\bb$ is isomorphic to the induced bundle $g^*\bb'$.
\end{mycorollary}
% proof
\begin{myproof}
    Let $f: U_f \to E'$ be a representative map for $F$.
    We define $F': \bb \to g^*\bb'$ with a representative map $f'$ as follows:
    \[ f': U_f \to E(g^*\bb'), f'(e) := (j(e), f(e)) \]
    The element $f'(e)$ actually lies in $E(g^*\bb')$ because
    \[ g(j(e)) = f(g(j(e))) = f(j'(e)) = j'(f(e)) \]
    We use the fact that $f$ and $j'$ commute.
    Applying the previous proposition concludes the proof.
\end{myproof}
% foreword
\begin{myparagraph}
    Now consider two isomorphic microbundles $\bb$ and $\bb'$ over $B$.
    There exists a homeomorphism $\phi: V \isomto V'$ where $V \sub E$ is a neighborhood of $i(B)$ and $V' \sub E'$ is a neighborhood of $i'(B)$.
    The homeomorphism $\phi$ is a representative for a homeomorphism-germ
    \[ \germdef{[\phi]}{(E, i(B))}{(E', i'(B))}. \]
    
    Studying isomorphy between microbundles in this way is useful because we don't care what such a
    homeomorphism does on particular neighborhoods of the base spaces but only what it does on arbitray small ones.
    Hence every representative of $[\phi]$ describes the `same' isomorphy between $\bb$ and $\bb'$.
    Now, naturally, the question arises whether the existence of a homeomorphism-germ
    \[ \germdef{F}{(E, i(B))}{(E', i'(B))} \]
    already implies that $\bb$ and $\bb'$ are isomorphic.
    The answer is generally no, because isomorphy of microbundles requires 
    a homeomorphism that \ul{commutes with injection and projection maps}.
    Therefore, we must assume an extra condition called `fibre-preservation' for this implication to be true.
    This justifies the following definition.
    
    % precondition
    Let $\bb$ and $\bb'$ be two microbundles over $B$ and
    let $\germdef{J}{(E, i(B))}{(B, B)}$ and $\germdef{J'}{(E', i(B))}{(B, B)}$ denote the map-germs representing its projection maps.
\end{myparagraph}

% definition
\begin{mydefinition}[isomorphism-germ]{homotopy::isomorphism}
    An \defterm{isomorphism-germ} between $\bb$ and $\bb'$ is a homeomorphism-germ 
    \[ \germdef{F}{(E, B)}{(E', B)} \]
    which is \defterm{fibre-preserving}, that is $J' \circ F = J$.
\end{mydefinition}

% remark
\begin{myremark}{homotopy::isomorphismremark}
    There exists an isomorphism-germ between $\bb$ and $\bb'$ if and only if $\bb$ is isomorphic to $\bb'$.
\end{myremark}
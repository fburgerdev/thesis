% definition wedge sum
\begin{mydefinition}
    Let $\ba_*$ and $\bb_*$ be two rooted microbundles over $A$ and $B$.
    The \defterm{wedge sum} $\ba_* \vee \bb_*$ is a rooted microbundle over $A \vee B$ defined as follows:

    For the rootings $R_\ba: \ba\restr{a_0} \double \be^n_{a_0}$ and $R_\bb: \bb\restr{b_0} \double \be^n_{b_0}$,
    choose representatives $r_\ba: V_\ba \isomto a_0 \times \R^n$ and $r_\bb: V_\bb \isomto b_0 \times \R^n$.
    Denote $r := r_\bb^{-1} \circ r_\ba: V_\ba \isomto V_\bb$.
    \begin{itemize}
        \item $E(\ba \vee \bb) := (E(\ba_*) \sqcup E(\bb_*)) / (\forall e \in V_\ba: e \sim r(e))$
        \item $i := i_\ba \vee i_\bb$ is well-defined, because $i(a_0) = i_\ba(a_0) = r(i_1(a_0)) = i_\bb(b_0) = i(b_0)$
        \item $j := j_\ba \vee j_\bb$ is well-defined, because $\forall e \in V_\ba: j(e) = j_\ba(e) = a_0 = b_0 = j_\bb(r(e)) = j(r(e))$
    \end{itemize}
\end{mydefinition}
\begin{proof}[that $\ba_* \vee \bb_*$ is a microbundle]
    Continuity of $i$ and $j$ follows directly from the quotient topology and from the continuity of $i_a, i_b$ and $j_a, j_b$.
    Additionally, $\forall a \in A: j(i(a)) = j(i_\ba(a)) = j_\ba(i_\ba(a)) = a$ (symmetric for $B$) and therefore $j \circ i = id$.
    It remains to be shown that the diagram is locally trivializable:

    Let $(U_a, V_a, \phi_a)$ and $(U_b, V_b, \phi_b)$ be local trivializations for $\ba$ in $a_0$ and $\bb$ in $b_0$.
    We can construct a local trivialization for $\ba \vee \bb$ in $a_0 = b_0$ as follows:
    \begin{itemize}
        \item $U := U_a \vee U_b$
        \item $V := V_a \vee V_b$
        \item $\phi\restr{V_a}(e) := (j_a(e), r_a(\phi_a^{-1}(a_0, \phi_a^{(2)}(e))))$ and symmetrically over $V_b$
    \end{itemize}
    Openness of $U$ and $V$ follows from the topology of quotients.
    The map $\phi$ is well-defined, because $\forall e_a = e_b \in V_\ba$:
    \[ \phi(e_a) = (j_a(e_a), r_a(\phi_a^{-1}(a_0, \phi_a^{(2)}(e_a)))) \]
    \[ = (a_0, r_a(e_a)) = (b_0, e_b) \]
    \[ = (j_b(e_b), r_b(\phi_b^{-1}(b_0, \phi_b^{(2)}(e_b)))) = \phi(e_b) \]
    Homeomorphy of $\phi$ follows from the homeomorphy of the components it's composed of.

    One can see that regardless of the choices we made for this definition, the resulting microbundle remains the same up to isomorphy.
    TODO: PROOF
\end{proof}

% foreword suspension
\begin{myparagraph}
    In the following, let $B$ be a \defterm{reduced suspension}
    \[ SX = (X \times [0, 1]) / (X \times \{ 0, 1 \} \cup x_0 \times [0, 1])\]
    over $X$.

    Let $\phi: B \to B \vee B$ denote the map that sends
    $X \times [0, \half]$ to the first $B$ via
    \[ \phi(x, t) = [(x, 2t)] \]
    and $X \times [\half, 1]$ to the second $B$ via
    \[ \phi(x, t) = [(x, 2t - 1)]. \]

    Let $c_1: B \vee B \to B$ denote the map that is the identity on the first summand
    and the constant mapt to $b_0$ on the second summand.
\end{myparagraph}

% lemma commutatitivity
\begin{mylemma}\label{suspension::lemma1}
    \[ \phi^*(\bb \oplus \be^n_B) \cong \bb \cong \phi^*(\be^n_B \oplus \bb) \]
\end{mylemma}
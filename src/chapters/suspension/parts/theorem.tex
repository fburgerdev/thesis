% lemma commutativity
\begin{mylemma}\label{suspension::commutativity}%
    The rooted microbundles $\bb \oplus \be_B$ and $\be_B \oplus \bb$ are isomorphic. 
\end{mylemma}
\begin{myproof}
    We need to find an isomorphism germ $\bb \oplus \be_B \double \be_B \oplus \bb$ that extends
    \[ (I \oplus R) \circ (R \oplus I)^{-1} = R \oplus R^{-1} \]
    where $I$ denotes the identity germ.

    Ignoring the rooting, we have an isomorphism-germ $f: E(\bb) \times \R^n \isomto \R^n \times E(\bb)$ with $f(e, x) = (-x, e)$.
    The idea is to change to $f$ near $b_0$ so that it extends the rooting.

    For this, choose a sufficiently small closed neighborhood $U$ of $b_0$
    such that there exists an extension $Q: (\bb \oplus \be)\restr{U} \double (\be \oplus \bb)\restr{U}$ for the rooting.

    Since $B$ is Tychonoff, there exists a map
    \[ \lambda: B \to [0, \frac{\pi}{2}] \]
    with $supp \lambda \sub U$ and $\lambda(b_0) = \frac{\pi}{2}$.
    With this map, we can define a homeomorphism
    \[ g: U \times \R^n \times \R^n \isomto U \times \R^n \times \R^n \]
    by
    \[ g(b, x, y) = (b, x \sin(\lambda(b)) - y \cos(\lambda(b)), x \cos(\lambda(b)) - y \sin(\lambda(b))). \]

    Now, we can consider
    \[ (\bb \oplus \be)\restr{U} \double (\bb \oplus \be)\restr{U} \xto{g} (\bb \oplus \be)\restr{U} \double (\be \oplus \bb)\restr{U}\]
    which coincides with $R \oplus R^{-1}$ over $b_0$ since $g(b_0, x, y) = (b_0, x, y)$ and with $F$ over $U \cap \lambda^{-1}(0)$.
    Pieced together with $F\restr{\lambda^{-1}(b)}$, we have an isomorphism germ $\bb \oplus \be_B \double \be_B \oplus \bb$ that extends the rooting, which completes the proof.
\end{myproof}

% theorem
\begin{mytheorem}
    If $\ba$ and $\bb$ are rooted microbundles over a completly regular space $B$, then
    \[ \phi^*(\ba \vee \bb) \oplus \be_B = \ba \oplus \bb. \]
\end{mytheorem}
\begin{myproof}
    The previous lemma yields $\bb \oplus \be \cong \be \oplus \bb$.
    Hence
    \[ \phi^*((\ba \oplus \be) \vee (\bb \oplus \be)) \cong \phi^*((\ba \oplus \be) \vee (\be \oplus \bb)). \]
    Additionally we have
    \[ \phi^*((\ba \vee \bb) \oplus (\be \vee \be)) \cong \phi^*(\ba \vee \bb) \oplus \be \]
    for the left side of the isomorphy and
    \[ \phi^*((\ba \vee \be) \oplus (\be \vee \bb)) \cong \ba \oplus \bb\]
    for the right side of the isomorphy which concludes the proof.
\end{myproof}

% corollary
\begin{mycorollary}
    The wedge sum $\bb \oplus r^*\bb$ is trivial.
\end{mycorollary}
\begin{myproof}
    This follows directly from the Theorem and the fact that $\phi^*(\bb \oplus r^*\bb)$ is trivial.
\end{myproof}

% TODO: bouqet of circles
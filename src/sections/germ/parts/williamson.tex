% proof
\begin{myproof}[of the proposition]
    Let $f: U_f \to E(\bb')$ be a representative for $F$.
    First we assume a special case. Then we use this result show that $f$
    is open in some neighborhood of every $v \in U_f$, hence $f$ being open.
    \begin{steps}
        % special case
        \item Let $f$ map from $B \cross \R^n$ to $B \cross \R^n$
        
        Since $F$ covers the identity, $f$ is of the form
        \[ f(b, x) = (b, g_b(x)) \]
        where $g_b: \R^n \to \R^n$ are individual maps.
        Since the $g_b$ are continuous and injective,
        it follows from the Invariance of Domain Theorem (see \cite[cor.19.9]{brendon}) that the $g_b$ are open maps.
        
        Let $(b_0, x_0) \in B \cross \R^n$ and let $\eps > 0$.
        Since $g_{b_0}$ is an open map, there exists a $\delta > 0$ such that
        $\clball[2\delta][x_1] \sub g_{b_0}(\clball[\eps][x_0])$ where $x_1 = g_{b_0}(x_0)$.

        % estimation
        We claim that there exists a neighborhood $V \sub B$ of $b_0$ such that
        \[ \abs{g_b(x) - g_{b_0}(x)} < \delta \]
        for every $b \in V$ and $x \in \clball[\eps][x_0]$.
        
        To show that, consider $\phi(b, x) = g_b(x) - g_{b_0}(x)$.
        The open set $\inv{\phi}(\ball[\delta])$ is a neighborhood of $\{b_0\} \cross \R^n$ since $\phi(b_0, x) = 0$.
        Hence, there exist open subsets $V_x \sub B$ and $W_x \sub \R^n$ such that
        \[ \bigcup_{x \in \clball[\eps][x_0]} V_x \cross W_x \sub \inv{\phi}(\clball[\delta]) \]
        and $x \in W_x$.
        Since $\clball[\eps][x_0]$ is compact, there exist $x_1, \dots, x_n \in \clball[\eps][x_0]$ with $\clball[\eps][x_0] \sub \bigcup_{i = 1}^{n} V_{x_i}$.
        The claim follows with $V = V_{x_1} \cap \dots \cap V_{x_n}$
        which is open by forming the intersection over finitely many open sets.

        % apply lemma
        Now we want to apply the previous lemma.

        Consider the homeomorphism
        $(g_b \circ \inv{g_{b_0}})\restr{\clball[2\delta][x_1]}$
        for an arbitrary $b \in V$.
        Together with
        \[ \clball[2\delta][x_1] \sub g_{b_0}(\clball[\eps][x_0]) \implies \inv{g_{b_0}}(\clball[2\delta][x_1]) \sub \clball[\eps][x_0], \]
        we conclude from the above that
        \[ \abs{(g_b \circ \inv{g_{b_0}})(x) - x} < \delta, \forall x \in \clball[2\delta][x_1]. \]
        It follows that, by translation and scaling, $g_b \circ \inv{g_{b_0}}\restr{\clball[2\delta][x_1]}$ satisfies the conditions of \myintref{homotopy::ball}.
        Therefore, $\clball[\delta][x_1] \sub (g_b \circ \inv{g_{b_0}})(\clball[2\delta][x_0])$ and hence $\clball[\delta][x_1] \sub g_b(\clball[\eps][x_0])$.
        % conclusion
        From
        \[ V \cross \clball[\delta][x_1] \sub g(V \cross \clball[\eps][x_0]) \]
        it follows that $f$ is an open map.

        % general case
        \item Gluing together $f: U_f \to E(\bb')$ along local trivializations

        For an arbitrary $b \in B$, choose local trivializations $(U, V, \phi)$ and $(U', V', \phi')$ of $b$ in $\bb$ and $\bb'$.
        Without loss of generality,
        we may assume that $U = U'$ by choosing $V = \inv{\phi}(U \cap U')$ and $V' = \inv{\phi'}(U \cap U')$
        and restricting $\phi$ and $\phi'$ accordingly if necessary.
        
        First, we restrict $f$ to $V \cap \inv{f}(V')$.
        Since $V \cap \inv{f}(V')$ is an open neighborhood of $i(b)$, we can choose
        an open neighborhood $U_b \sub U$ of $i(b)$ and $\eps_b > 0$ such that $\inv{\phi}(U_b \cross \ball[\eps_b]) \sub V \cap \inv{f}(V')$.
        
        We define a map $U_b \cross \R^n \to U_b \cross \R^n$ given by
        \[ U_b \cross \R^n \cong  U_b \cross \ball[\eps_b] \xto{\phi \circ f \circ \inv{\phi}} U_b \cross \R^n \]
        that is injective and fiber-preserving, and hence an open map (Step 1).
        It follows that $f: \inv{\phi}(U_b \cross \ball[\eps_b]) \to V'$
        must be an open map, as the other composing maps are homeomorphisms.

        We conclude from
        \[ f = \bigcup_{b \in B}f\restr{\inv{\phi}(U_b \cross \ball[\eps_b])} \]
        that $f$ is an open map.
    \end{steps}
    This completes the proof.
\end{myproof}

% foreword corollary
\begin{myparagraph}
    We can easily generalize this result to bundle-germs between microbundles over different base spaces.
\end{myparagraph}

% corollary
\begin{mystatement}{corollary}{homotopy::corollary}[67]
    If a map $g: B \to B'$ is covered by a bundle-germ $\germdef{F}{\bb}{\bb'}$, then $\bb$ is isomorphic to the induced microbundle $\ind{g}\bb'$.
\end{mystatement}

% proof corollary
\begin{myproof}
    Let $f: U_f \to E'$ be a representative map for $F$.
    We define $\germdef{F'}{\bb}{\ind{g}\bb'}$ by the representative
    \[ f': U_f \to E(\ind{g}\bb') \twith f'(e) = (j(e), f(e)). \]
    Every $f'(e)$ lies in $E(\ind{g}\bb')$ because
    $g(j(e)) = j'(f(e))$
    as we can see from the commutative diagram for bundle-germs.

    The germ $F'$ is a bundle-germ covering the identity because
    \[ j(e) = j_g'(j(e), f(e)) = j_g'(f'(e)) \]
    and because $f'$ is injective ($f$ is injective).
    Applying the previous proposition on $F'$ proves the claim.
\end{myproof}
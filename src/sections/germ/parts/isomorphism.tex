% foreword
\begin{myparagraph}
    Let $\bb$ and $\bb'$ be two isomorphic microbundles over $B$.
    There exists a homeomorphism $\psi: V \isomto V'$ where
    $V \sub E(\bb)$ is a neighborhood of $i(B)$
    and $V' \sub E(\bb')$ is a neighborhood of $i'(B)$.
    We can view $\psi$ as a representative for a homeomorphism-germ
    \[ \germdef{[\psi]}{(E, i(B))}{(E', i'(B))}. \]
    
    Studying isomorphy between $\bb$ and $\bb'$
    using map-germs is useful,
    because we don't care what $\psi$ does on its initial domain,
    but only what it does on arbitray small neighborhoods of $i(B)$.
    Hence, every representative of $[\psi]$
    describes the `same' isomorphy between $\bb$ and $\bb'$.
    Now, naturally, the question arises whether
    the existence of a homeomorphism-germ
    \[ \germdef{F}{(E, i(B))}{(E', i'(B))} \]
    already implies that $\bb$ and $\bb'$ are isomorphic.
    The answer is generally no, because isomorphy between microbundles
    additionally requires 
    the homeomorphism to commute with the injection and projection maps.
    Therefore, we need to require an extra condition (`fiber-preservation')
    for this implication to be true.
    This justifies the following definition.
    
    % precondition
    Let $\germdef{J}{(E(\bb), i(B))}{(B, B)}$ and $\germdef{J'}{(E(\bb'), i(B))}{(B, B)}$
    denote the map-germs represented by the projections of $\bb$ and $\bb'$.
\end{myparagraph}

% definition
\begin{mystatement}{definition}[isomorphism-germ]{homotopy::isomorphism}[65]
    An \defterm{isomorphism-germ} between $\bb$ and $\bb'$ is a homeomorphism-germ 
    \[ \germdef{F}{(E(\bb), B)}{(E(\bb'), B)} \]
    which is \defterm{fiber-preserving}, that is $J' \circ F = J$.
\end{mystatement}

% remark
\begin{mystatement}{remark}{homotopy::isomorphismremark}[65]
    There exists an isomorphism-germ between $\bb$ and $\bb'$
    if and only if $\bb$ and $\bb'$ are isomorphic.
\end{mystatement}
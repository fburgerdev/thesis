\subsection*{The Normal Microbundle}\label{section::normal} \begin{mystatement}{definition}[normal microbundle]{normal::definition}[61]Let $M$ be a topological manifold together with a submanifold $N \sub M$. A \defterm{normal microbundle} $\bn$ of $N$ in $M$ is a microbundle \[ \bundle{N}{U}{\iota}{r} \] where $U \sub M$ is a neighborhood of $N$ and $\iota$ denotes the inclusion $N \incl U$.
\end{mystatement} % foreword
\begin{myparagraph}
    Unlike the normal vector bundle for smooth manifolds,
    the normal microbundle is not defined in a constructive manner.
    Therefore, the question arises in which sense the normal microbundle
    of a submanifold $N \sub M$ is unique.
    In fact, it is unknown whether a such normal microbundle is unique up to isomorphy.
    Instead, we have the following statement about uniqueness.
\end{myparagraph}

% statement
\begin{mystatement}{proposition}{normal::uniqueness}[63]
    Let $N \sub M$ be an embedded submanifold.
    Suppose there exists a normal microbundle
    $\bundledef{\bn}{N}{U}{\iota}{r}$ in $M$.
    Then $\bt_N \oplus \bn \cong \bt_M\restr{N}$.
\end{mystatement}

% proof
\begin{myproof}
    We prove this proposition in multiple steps.
    \begin{steps}
        % step 1
        \item $\bt_N \circ \ind{\pi_2}\bn \cong \bt_M\restr{N}$
        
        % total space
        Consider the two total spaces
        \[ E(\bt_N \circ \ind{\pi_2}\bn) = E(\ind{\pi_2}\bn) = \set{(n_1, n_2, u) \in (N \cross N) \cross U}{n_2 = r(u)} \]
        and
        \[ E(\bt_M\restr{N}) = \set{(n, m_1, m_2) \in N \cross (M \cross M)}{n = m_1}. \]

        % homeomorphy
        We can easily define a homeomorphism $\psi: E(\bt_M \circ \ind{\pi_2}\bn) \isomto E(\bt_N\restr{M})$ given by
        \[ \psi(m_1, m_2, u) = (m_1, m_1, u) \tand \inv{\psi}(m, n_1, n_2) = (m, r(n_2), n_2). \]
        Note that $\psi$ is a homeomorphism since $\psi$ and $\inv{\psi}$ are component-wise continuous.

        % commutativity
        It remains to be shown that $\psi$
        commutes with the injection and projection maps of $\bt_M \circ \ind{\pi_2}\bn$ and $\bt_N\restr{M}$.
        This can be seen with the following equations:
        \begin{align}
            \psi(i_{\pi_2}(\Delta(m))) = \psi(m, m, \iota(m)) = (m, m, m) = (m, \Delta(\iota(m))) \\
            \pi_1(j_{\pi_2}(m_1, m_2, u)) = m_1 = j_\iota(m_1, m_1, u) = j_\iota(\psi(m_1, m_2, u)) 
        \end{align}
        
        % step 2
        \item $\bt_M \circ \ind{\pi_1}\bn \cong \whitney{\bt_M}{\bn}$
        
        % total space
        In this case, the two total spaces
        \[ E(\bt_M \circ \ind{\pi_1}\bn) = E(\ind{\pi_1}\bn) = \set{(m_1, m_2, u) \in (M \cross M) \cross U}{m_1 = r(u)} \]
        and
        \[ E(\whitney{\bt_M}{\bn}) = \set{(m_1, m_2, u) \in (M \cross M) \cross U}{m_1 = r(u)} \]
        are equal.
        % homeomorphy + commutativity
        Additionally,
        the injection and projection maps agree, as the following equations show:
        \begin{align}
            & i_{\pi_1}(\Delta(m)) = i_{\pi_1}(m, m) = (m, m, \iota(m)) = (\Delta(m), \iota(m)) \\
            & \pi_1(j_{\pi_1}(m_1, m_2, u)) = \pi_1(m_1, m_2) = m_1 = r(u) = j_{\oplus}(m_1, m_2, u)
        \end{align}

        % step 3
        \item $\bt_M \circ \ind{\pi_1}\bn \cong \bt_M \circ \ind{\pi_2}\bn$
        
        We show that there exists a neighborhood $D \sub N \cross N$ of $\Delta(M)$ such that
        $\pi_1\restr{D}$ is homotopic to $\pi_2\restr{D}$:

        Firstly, we assume that $M$ is embedded in Euclidean space (see\cite[p.60]{dimension}).
        Let $V \sub N$ be a neighborhood retract of $M$.
        We define $D$ as follows:
        \[ D = \set{(m, m') \in M \cross M}{ tm + (1 - t)m' \in V, \forall t \in \I} \]
        We are given a homotopy $H: \cyl{D} \to N$ between $\pi_1$ and $\pi_2$ by
        \[ H((m, m'), t) = tm + (1 - t)m'. \]

        Applying the Homotopy Theorem yields $\ind{\pi_1}\bn\restr{D} \cong \ind{\pi_2}\bn\restr{D}$, and
        by restricting the total spaces accordingly, we get
        $\bt_M \circ \ind{\pi_1}\bn \cong \bt_M \circ \ind{\pi_2}\bn$.
    \end{steps}
    The claim follows by Step 1, 2 and 3.
\end{myproof}

% afterword
\begin{myparagraph}
    This proposition also shows that the normal microbundle
    underlies the same intuition as the normal vector bundle. 
    The sum of the tangent- and the normal microbundle of the submanifold
    `span' the tangent microbundle of its surrounding space.
\end{myparagraph} \subsection*{Milnors Theorem}\label{section::milnor} \begin{myparagraph} Lastly, we will use all the substantial results presented in this thesis in order to prove Milnors \myintref{normal::milnor}. \end{myparagraph} 
\begin{mystatement}{lemma}{normal::transitivity}[62]
Let $P \sub N \sub M$ be a chain of topological submanifolds.
There exists a normal microbundle
\[ \bundledef{\bn}{P}{U}{\iota}{r} \]
of $P$ in $M$ if there exist normal microbundles
\begin{center}
$\bundledef{\bn_p}{P}{U_N}{\iota_P}{j_P}$ in $N$ and $\bundledef{\bn_n}{N}{U_M}{\iota_N}{j_N}$ in $M$.
\end{center}
\end{mystatement}

\begin{myproof}
We are given a normal microbundle $\bn$ of $P$ in $M$ by
\[ \bn_p \circ \bn_n\restr{U_N}. \]
Note that $\iota_N \circ \iota_P$ is just the inclusion $P \incl U_M$.
\end{myproof} \begin{myparagraph} For the remainder of this section, we utilize the fact that every manifold is an absolute neighborhood retract (ANR). For a proof, see Theorem 3.3 in \cite{hanner}. It follows that for every submanifold $N \sub M$, there exists an open neighborhood $V \sub M$ of $N$ together with a retraction $r: V \to N$.

\end{myparagraph} % foreword
\begin{myparagraph}
    The total space $E(\ind{r}\bt_N)$ is a topological manifold.
    This can be seen with
    \[ E(\ind{r}\bt_N) = \set{(v, n_1, n_2) \in V \cross (N \cross N)}{r(v) = n_1} \cong V \cross N. \]

    Together with $N \incl M \xto{i_{\bt}} E(\ind{r}\bt_N)$, we can assume that
    $N$ is an embedded submanifold of $E(\ind{r}\bt_N)$.
    Note that $i_{\bt}$ is an embedding due to the construction of the induced microbundle.
\end{myparagraph}

% lemma
\begin{mystatement}{lemma}{normal::total}[62]
    Let $N \sub M$ be an embedded submanifold and let $r: V \to N$ be a retraction.
    Then there exists a normal microbundle $\bn$ of $N$ in $E(\ind{r}\bt_N)$. % such that $\bn \cong \ind{\iota}\bt_V$.
\end{mystatement}

% proof
\begin{myproof}
    We are given a normal microbundle of $N$ in $E(\ind{r}\bt_N)$ by $\ind{r}\bt_N\restr{N}$.
    
    Since $\ind{r}\bt_N\restr{N} \cong \ind{(r \circ \iota)}\bt_N = \ind{id}\bt_N \cong \bt_N$,
    it suffices to show that $\bt_N \cong \ind{\iota}\bt_V$.
    % homoemorphism
    We define a homeomorphism $\psi: E(\bt_N) \isomto E(\ind{\iota}\bt_V)$ with
    \[ \psi(n_1, n_2) = (n_1, n_1, n_2) \tand \inv{\psi}(n, v_1, v_2) = (n, v_2) \]
    for which homeomorphy follows from component-wise
    continuity of both $\psi$ and $\inv{\psi}$.
    
    % commutativity
    Commutativity with the injection maps is given by
    \[ \psi(\Delta(n)) = \psi(n, n) = (n, n, n) = (n, \Delta(n)) = i_\iota(n) \]
    and with the projection maps by
    \[ \pi_1(n_1, n_2) = n_1 = j_\iota(n_1, n_1, n_2) = j_\iota(\psi(n_1, n_2)), \]
    which concludes the proof.
    
    % Furthermore, $\ind{r}\bt_N\restr{N}$ is isomorphic to $\ind{\iota}\bt_V$ is given by
    % \[ \psi: E(\ind{\iota}\bt_V) \isomto E(\ind{r}\bt_N). \]
    % where commutativity with the injection and projection maps
    % follow from the diagram in \myintref{normal::commute} and from the following equation:
    % \[ j_\iota(n, v_1, v_2) = n = j_r(v_2) \]
\end{myproof} \begin{scope}
    % defines
    \newcommand{\rwhitney} {
        \whitney{\ind{r}\bt_N}{\ind{r}\eta}
    }
    \newcommand{\rtn} {
        \ind{r}\bt_N
    }

    % foreword
    \begin{myparagraph}
        Finally, we gathered all the tools to prove Milnor's theorem.
    \end{myparagraph}

    % theorem
    \begin{mystatement}{theorem}[Milnors Theorem]{normal::milnor}
        For a sufficently large $q \in \N$, $N = \cyldown{N}$ has a normal microbundle in $M \cross \R^q$.
    \end{mystatement}

    % proof
    \begin{myproof}
        We assume that $M$ is embedded in euclidean space $\R^{2m + 1}$ \cite[p.60]{dimension}.
        Additionally, let $V$ be an open neighborhood of $N$ in $M$ together with a retraction $r: V \to N$.

        We show the theorem in multiple steps.
        \begin{steps}
            % step 1
            \item $N$ has a normal microbundle $\eta$ in $M$ such that $\whitney{\bt_N}{\eta} \cong \be[q]{N}$
            
            Consider the extension $\ind{r}\bt_N$.
            Since $V$ is an open set, it's a simplicial complex.
            Hence, we can apply \myintref{whitney::theorem} to the extended microbundle $\ind{r}\bt_N$
            to obtain a microbundle $\eta'$ such that $\whitney{\ind{r}\bt_N}{\eta} \cong \be[q]{V}$.

            We conclude that $\whitney{\bt_N}{\eta'\restr{N}} = \whitney{\ind{r}\bt_N\restr{N}}{\eta'\restr{N}} = (\whitney{\ind{r}\bt_N}{\eta'})\restr{N} = \be[q]{N}$.

            % step 2
            \item $E(\rtn)$ has a normal microbundle in $E(\rwhitney)$

            We denote the injection and projection of $\rtn$ with $i_r$ and $j_r$,
            and the injection and projection of $\rwhitney$ with $i_\oplus$ and $j_\oplus$.

            Note that $\rwhitney \cong \ind{r}(\whitney{\bt_N}{\eta})$ is trivial, so
            $E(\rwhitney)$ is an open subset of $\R^qn$ and hence a manifold.

            We consider $E(\rtn)$ to be a subset of $E(\rwhitney)$ via
            \[ (v, e) \mapsto ((v, e), (v, i_{\eta}(v))). \]

            We are given a normal microbundle of $E(\rtn)$ in $E(\rwhitney)$ by $\ind{j_r}(\ind{r}\eta)$.

            Since the total space
            \[ E(\rwhitney) = \set{(e, e') \in E(\rtn) \cross E(\ind{r}\eta)}{j(e) = j'(e)} \]
            we can consider $E(\rtn) \sub E(\rwhitney)$ embedded via
            \[ \iota: e \mapsto (e, i'(j(e))) \]
            with the inverse $\pi_1: (e, e') \mapsto e$.

            Because $\rwhitney \cong \ind{r}(\whitney{\bt_N}{\eta})$ is trivial,
            it follows that $E(\rwhitney) \sub M \cross \R^k$ open and hence being a manifold.

            We have a normal microbundle of $E(\rtn)$ in $E(\rwhitney)$ via
            \[ \bundledef{\bn}{E(\rtn)}{E(\rwhitney)}{\iota}{\pi_1}. \]

            To show local triviality,
            let $(U, V, \phi)$ be a local trivialization of $i'(j(e))$ in $\ind{r}\eta$
            for an arbitrary $e \in E(\rtn)$.

            By choosing
            \begin{itemize}
                \item $U' := \inv{j}(U)$
                \item $V' := (U' \cross V) \cap E(\rwhitney)$
                \item $\phi': V' \isomto U' \cross \R^{n_\eta}$ with $\phi'(e, e') = (e, \snd{\phi}(e'))$
            \end{itemize}
            we have a local trivialization of $e$ in $\bn$.

            That is because both $U' \sub E(\rtn)$ and $V' \sub E(\rwhitney)$ are open sets
            and $\phi'$ is a homeomorphism with its inverse
            $\inv{\phi'}(e, x) = (e, \inv{\phi}(j(e), x))$.
            
            Also, $\phi'$ commutes with injection
            \[ \phi'(\iota(e)) = \phi'(e, i'(j(e))) = (e, \snd{\phi}(i'(j(e)))) = (e, 0) = (id \cross 0)(e) \]
            and projection maps
            \[ \pi_1(e, e') = \pi_1(e, \snd{\phi'}(e, e')) = \pi_1(\phi'(e, e')). \]
            
            % step 3
            \item $N$ has a normal microbundle in $M \times \R^q$
            
            Since $N \sub M \sub E(\rtn)$ has a normal microbundle (using \myintref{normal::total}),
            it follows from \myintref{normal::transitivity} that
            $N \sub E(\whitney{\ind{r}\bt_N}{\ind{r}\bt'})$ has a normal microbundle.
    
            By restricting $E(\rwhitney)$ to an open subset if necessary, we may assume that
             \[ E(\rwhitney) = M \cross \R^q \]
            for some $q \in \N$ using \myintref{induced::trivial}.
        \end{steps}
        Applying \myintref{normal::transitivity} and $E(\rwhitney) = M \cross \R^q$ completes the proof.
    \end{myproof}

    % afterword
    \begin{myparagraph}
        
    \end{myparagraph}
\end{scope}
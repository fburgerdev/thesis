\subsection*{The Normal Microbundle}\label{section::normal} \begin{mystatement}{definition}[normal microbundle]{normal::definition}[61]Let $M$ be a topological manifold together with a submanifold $N \sub M$. A \defterm{normal microbundle} $\bn$ of $N$ in $M$ is a microbundle \[ \bundle{N}{U}{\iota}{r} \] where $U \sub M$ is a neighborhood of $N$ and $\iota$ denotes the inclusion $N \incl U$.
\end{mystatement} % foreword
\begin{myparagraph}
    Unlike the normal vector bundle for smooth manifolds,
    the normal microbundle is not defined in a constructive manner.
    Therefore, the question arises in which sense the normal microbundle
    of a submanifold $N \sub M$ is unique.
    In fact, it is unkown whether a such normal microbundle is unique up to isomorphy.
    Instead, we have the following statement about uniqueness.
\end{myparagraph}

% statement
\begin{mystatement}{proposition}{normal::uniqueness}
    Let $N \sub M$ be an embedded submanifold.
    Suppose there exists a normal microbundle
    $\bundledef{\bn}{N}{U}{\iota}{r}$ in $M$.
    Then $\bt_N \oplus \bn \cong \bt_M\restr{N}$.
\end{mystatement}

% proof
\begin{myproof}
    We prove this proposition in multiple steps.
    \begin{steps}
        % step 1
        \item $\bt_N \circ \ind{\pi_2}\bn \cong \bt_M\restr{N}$
        
        % total space
        Consider the two total spaces
        \[ E(\bt_N \circ \ind{\pi_2}\bn) = E(\ind{\pi_2}\bn) = \set{(n_1, n_2, u) \in (N \cross N) \cross U}{n_2 = r(u)} \]
        and
        \[ E(\bt_M\restr{N}) = \set{(n, m_1, m_2) \in N \cross (M \cross M)}{n = m_1}. \]

        % homeomorphy
        We can easily define a homeomorphism $\psi: E(\bt_M \circ \ind{\pi_2}\bn) \isomto E(\bt_N\restr{M})$ given by
        \[ \psi(m_1, m_2, u) = (m_1, m_1, u) \tand \inv{\psi}(m, n_1, n_2) = (m, r(n_2), n_2). \]
        Note that $\psi$ is a homeomorphism since as $\psi$ and $\inv{\psi}$ are component-wise continuous.

        % commutativity
        It remains to be shown that $\psi$
        commutes with the injection and projection maps of $\bt_M \circ \ind{\pi_2}\bn$ and $\bt_N\restr{M}$.
        To check this, consider the following equations:
        \begin{align}
            & \psi(i_{\pi_2}(\Delta(m))) = \psi(i_{\pi_2}(m, m)) = \psi(m, m, \iota(m)) = (m, m, m) = (m, \Delta(\iota(m))) \\
            & \pi_1(j_{\pi_2}(m_1, m_2, u)) = \pi_1(m_1, m_2) = m_1 = j_\iota(m_1, m_1, u) = j_\iota(\psi(m_1, m_2, u)) 
        \end{align}
        
        % step 2
        \item $\bt_M \circ \ind{\pi_1}\bn \cong \whitney{\bt_M}{\bn}$
        
        % total space
        In this case, the two total spaces
        \[ E(\bt_M \circ \ind{\pi_1}\bn) = E(\ind{\pi_1}\bn) = \set{(m_1, m_2, u) \in (M \cross M) \cross U}{m_1 = r(u)} \]
        and
        \[ E(\whitney{\bt_M}{\bn}) = \set{(m_1, m_2, u) \in (M \cross M) \cross U}{m_1 = r(u)}, \]
        are even equal.
        % homeomorphy + commutativity
        Additionally,
        the injection and projection maps agree as the following equations show:
        \begin{align}
            & i_{\pi_1}(\Delta(m)) = i_{\pi_1}(m, m) = (m, m, \iota(m)) = (\Delta(m), \iota(m)) \\
            & \pi_1(j_{\pi_1}(m_1, m_2, u)) = \pi_1(m_1, m_2) = m_1 = r(u) = j_{\oplus}(m_1, m_2, u)
        \end{align}

        % step 3
        \item $\bt_M \circ \ind{\pi_1}\bn \cong \bt_M \circ \ind{\pi_2}\bn$
        
        We show that there exists a neighborhood $D \sub N \cross N$ of $\Delta(M)$ such that
        $\pi_1\restr{D}$ is homotopic to $\pi_2\restr{D}$:

        Firstly, we assume that $N$ is embedded in euclidean space (see\cite[p.60]{dimension}).
        Let $V \sub N$ be a neighborhood retract of $M$.
        We define $D$ as follows:
        \[ D = \set{(m, m') \in M \cross M}{ tm + (1 - t)m' \in V, \forall t \in \I} \]
        We are given a homotopy $H: \cyl{D} \to N$ between $\pi_1$ and $\pi_2$ with
        \[ H((m, m'), t) = tm + (1 - t)m'. \]

        Applying the Homotopy Theorem yields $\ind{\pi_1}\bn\restr{D} \cong \ind{\pi_2}\bn\restr{D}$, and
        by restricting the total spaces accordingly we get
        $\bt_M \circ \ind{\pi_1}\bn \cong \bt_M \circ \ind{\pi_2}\bn$.
    \end{steps}
    The claim follows by Step 1, 2 and 3.
\end{myproof}

% afterword
\begin{myparagraph}
    This proposition also shows that the normal microbundle
    underlies the same intuition as the normal vector bundle. 
    The sum of the tangent- and normal microbundle of the submanifold
    `span' the tangent microbundle of the surrounding space.
\end{myparagraph} \subsection*{Milnors Theorem}\label{section::milnor} \begin{myparagraph} Lastly, we will use all the substantial results presented in this thesis in order to prove Milnors \myintref{normal::milnor}. \end{myparagraph} \begin{mystatement}{lemma}{normal::transitivity}[62] Let $P \sub N \sub M$ be a chain of topological submanifolds. There exists a normal microbundle \[ \bundledef{\bn}{P}{U}{\iota}{r} \] of $P$ in $M$ if there exist normal microbundles \begin{center} $\bundledef{\bn_p}{P}{U_N}{\iota_P}{j_P}$ in $N$ and $\bundledef{\bn_n}{N}{U_M}{\iota_N}{j_N}$ in $M$. \end{center} \end{mystatement}

\begin{myproof} We are given a normal microbundle $\bn$ of $P$ in $M$ by \[ \bn_p \circ \bn_n\restr{U_N}. \] Note that $\iota_N \circ \iota_P$ is just the inclusion $P \incl U_M$. \end{myproof} \begin{myparagraph} For the remainder of this section, we utilize the fact that every manifold is an absolute neighborhood retract (ANR). For a proof, see Theorem 3.3 in \cite{hanner}. It follows that for every submanifold $N \sub M$, there exists an open neighborhood $V \sub M$ of $N$ together with a retraction $r: V \to N$.

\end{myparagraph} % statement
\lemma{(homeomorphism of total spaces)} \\
Let $\bt_N$ and $\bt_M$ be the tangent microbundles of $N$ and $M$.
The total space $E(\iota^*\bt_M)$ and $E(r^*\bt_N)$ are homeomorphic.
% proof
\begin{proof}
We explicitly construct a homeomorphism:
\begin{enumerate}
    \item $E(\iota^*\bt_M) = \{ (n, (m_1, m_2)) \in N \times (M \times M) \mid \iota(n) = m_1\}$
    \item $E(r^*\bt_N) = \{ (m, (n_1, n_2)) \in M \times (N \times N) \mid r(m) = n_1\}$
\end{enumerate}
Now, we have the homeomorphism $\phi: E(\iota^*\bt_M) \to E(r^*\bt_N)$ with
$\phi(n, (m_1, m_2)) = (m_2, (r(m_2), n))$ and $\phi^{-1}(m, (n_1, n_2)) = (n_2, (n_2, m))$.
We easily see that $\phi$ suffices all requirements of $E(\iota^*\bt_M)$ and $E(r^*\bt_N)$.
\end{proof}

% remark
\begin{remark}
Note that the following diagram commutes
\[\begin{tikzcd}
    N \ar[r] \ar[hookrightarrow]{d} & E(\iota^*\bt_M) \ar[d, "\phi"] \\
    M \ar[r] & E(r^*\bt_N)
\end{tikzcd}\]
\end{remark}

% statement
\lemma{(normal microbundle on total space)}
There exists a normal microbundle $\bn$ of $N$ in $E(r^*\bt_N)$ with $\bn \cong \iota^*\bt_M$.
% proof
\begin{proof}
Obviously, $\bn := r^*\bt_N \restr{N}$ is a normal microbundle of $N$ in $E(r^*\bt_N)$.
Since $E(r^*\bt_N \restr{N}) \sub E(r^*\bt_N)$, isomorphy follows from the previous lemma and remark.
\end{proof} \begin{scope}
    % defines
    \newcommand{\rwhitney} {
        \whitney{\ind{r}\bt_N}{\ind{r}\eta}
    }
    \newcommand{\rtn} {
        \ind{r}\bt_N
    }

    % foreword
    \begin{myparagraph}
        Finally, we gathered all the tools to prove Milnors theorem.
    \end{myparagraph}

    % theorem
    \begin{mystatement}{theorem}[Milnors Theorem]{normal::milnor}[62]
        For a sufficiently large $q \in \N$, $N = \cyldown{N}$ has a normal microbundle in $M \cross \R^q$.
    \end{mystatement}

    % proof
    \begin{myproof}
        We assume that $M$ is embedded in Euclidean space $\R^{2m + 1}$ \cite[p.60]{dimension}.
        Additionally, let $V$ be an open neighborhood of $N$ in $M$ together with a retraction $r: V \to N$.

        We show the theorem in multiple steps.
        \begin{steps}
            % step 1
            \item $N$ has a normal microbundle $\eta$ in $M$ such that $\whitney{\bt_N}{\eta} \cong \be[q]{N}$
            
            Consider the extension $\ind{r}\bt_N$ for $\bt_N$.
            Since $V$ is an open set, it's a simplicial complex.
            Hence, we can apply \myintref{whitney::theorem} to $\ind{r}\bt_N$
            to obtain a microbundle $\eta'$ such that $\whitney{\ind{r}\bt_N}{\eta'} \cong \be[q]{V}$.

            We conclude that $\whitney{\bt_N}{\eta'\restr{N}} = \whitney{\ind{r}\bt_N\restr{N}}{\eta'\restr{N}} \cong (\whitney{\ind{r}\bt_N}{\eta'})\restr{N} \cong \be[q]{N}$.

            % step 2
            \item $E(\rtn)$ has a normal microbundle in $E(\rwhitney)$

            Note that $\rwhitney \cong \ind{r}(\whitney{\bt_N}{\eta})$ is trivial.
            Hence, by restricting the total space $E(\rwhitney)$ if necessary,
            we may assume that
            \[ E(\rwhitney) = V \cross \R^q \]
            for some $q \in \N$.
            Particularly, we derive that $E(\rwhitney)$ is a manifold.

            We denote the injection and projection of $\rtn$ by $i_{\bt}$ and $j_{\bt}$,
            and the injection and projection of $\ind{r}\eta$ by $i_{\eta}$ and $j_{\eta}$.

            We consider $E(\rtn)$ to be a submanifold of $E(\rwhitney)$ embedded by
            \[ e \mapsto (e, j_{r}(i_{\eta}(e))). \]
            We have a normal microbundle of $E(\rtn)$ in $E(\rwhitney)$ given by $\ind{j_{\bt}}(\ind{r}\eta)$.
            That is because the total space
            \[ E(\ind{j_r}(\ind{r}\eta)) = \set{(e, e') \in E(\rtn) \cross E(\ind{r}\eta)}{j_{\bt}(e) = j_\eta(e')} \]
            equals $E(\rwhitney)$ and because its inclusion matches the above embedding.
            
            % step 3
            \item $N$ has a normal microbundle in $M \times \R^q$
            
            Since $N$ has a normal microbundle in $E(\rtn)$ (\myintref{microbundle::total})
            and $E(\rtn)$ has a normal microbundle in $E(\rwhitney)$,
            we conclude with \myintref{normal::transitivity} that
            $N$ has a normal microbundle in $E(\rwhitney)$.
            
            As elaborated in Step 2, we may assume that $E(\rwhitney) = V \cross \R^n$.
            Since $V$ is open in $M$, $V \cross \R^n$ is open in $M \cross \R^n$.
        \end{steps}
        We conclude that $N$ has a normal microbundle in $M \cross \R^n$.
    \end{myproof}

    % dimension
    \begin{myparagraph}
        With the provided proofs for \myintref{whitney::theorem} and \myintref{normal::milnor},
        we can calculate an upper bound $m(2^{2m + 2} - 2)$ for $q$, where $m$ is the dimension of $M$ \cite[p.63]{milnor}.
        However, when provided with sharper proofs,
        one can substantially reduce the upper bound
        to a quadratic $(m + 1)^2 - 1$ (see \cite[p.232]{hirsch}). 
    \end{myparagraph}
\end{scope}
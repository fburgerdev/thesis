% foreword
\begin{myparagraph}
    Unlike the normal vector bundle for smooth manifolds,
    the normal microbundle is not defined in a constructive manner.
    Therefore, the question arises in which sense the normal microbundle
    of a submanifold $N \sub M$ is unique.
    In fact, it is unknown whether a such normal microbundle is unique up to isomorphy.
    Instead, we have the following statement about uniqueness.
\end{myparagraph}

% statement
\begin{mystatement}{proposition}{normal::uniqueness}[63]
    Let $N \sub M$ be an embedded submanifold.
    Suppose there exists a normal microbundle
    $\bundledef{\bn}{N}{U}{\iota}{r}$ in $M$.
    Then $\bt_N \oplus \bn \cong \bt_M\restr{N}$.
\end{mystatement}

\begin{myparagraph}
    In order to prove this, we require another construction over microbundles, the composition.
\end{myparagraph}
% definition
\begin{mystatement}{definition}[composition microbundle]{normal::composition}[63]
    Let $\ba$ be a $n$-dimensional microbundle
    \[ \bundledef{\ba}{A}{E(\ba)}{i_a}{j_a} \]
    and let $\bb$ be a $n'$-dimensional microbundle
    \[ \bundledef{\bb}{E(\ba)}{E(\bb)}{i_b}{j_b}. \]
    The \defterm{composition microbundle} $\ba \circ \bb$ is a $(n + n')$-dimensional microbundle
    \[ \bundle{A}{E(\bb)}{i}{j} \]
    where $i = i_b \circ i_a$ and $j = j_a \circ j_b$.
\end{mystatement}

% proof
\begin{myproof}[that $\ba \circ \bb$ is a microbundle]
    % continuity and compatibility
    Both injection and projection maps are continuous as being composed by continuous maps.
    Additionally, $j \circ i = j_a \circ (j_b \circ i_b) \circ i_a = j_a \circ i_a = id_A$.

    It remains to be shown that $\ba \circ \bb$ is locally trivial.

    % without loss
    For an arbitrary $a \in A$, choose local trivializations
    $(U_a, V_a, \phi_a)$ of $a$ in $\ba$ and $(U_b, V_b, \phi_b)$ of $i_a(a)$ in $\bb$.
    Note that both $U_b$ and $V_a$ are open neighborhoods of $i_a(a)$.
    
    Without loss of generality, we may assume that $V_a = U_b$:
    
    `$\sub$':
    Modify $U_a$ such that
    \[ U_a \cross \ball[\eps] \sub \phi_a(V_a \cap U_b) \]
    for a sufficiently small $\eps > 0$ and let
    \[ V_a = \inv{\phi_a}(U_a \cross \ball[\eps]) \sub V_a \cap U_b. \]
    Composing $\phi_a$ with $\mu_\eps: \ball[\eps] \isomto \R^n$
    yields a local trivialization of $a$ in $\ba$ such that $V_a \sub U_a$.

    `$\bus$': Restrict $U_b$ to $V_a \cap U_b$ and $V_b$ to $\inv{\phi_b}((V_a \cap U_b) \cross \R^{n'})$.

    % homeomorphy
    We have local-trivialization $(U_a, V_b, \phi)$ of $a$ in $\ba \circ \bb$ given by
    \[
        \phi: V_b \xto{\phi_b} U_b \cross \R^{n'}
        = V_a \cross \R^{n'}
        \xto{\phi_a \cross id} (U_a \cross \R^{n}) \cross \R^{n'}
        = U_a \cross \R^{n + n'},
    \]
    which is a homeomorphism since it's composed by homeomorphisms.

    % commutativity
    Furthermore, $\phi$ commutes with the injection and projection maps,
    as the following equations show:
    \begin{gather}
        \begin{split}
            \phi(i(a)) = \phi(i_b(i_a(a))) = (\phi_a(i_a(a)), \snd{\phi_b}(i_b(i_a(a)))) \\
            = (\snd{\phi_a}(i_a(a)), 0) = (a, (0, 0)) = (id_{U_a} \cross 0)(a)    
        \end{split} \\
        j(e) = j_a(j_b(e)) = \pi_1(j_a(j_b(e)), \snd{\phi}(e)) = \pi_1(\phi(e))
    \end{gather}
    This completes the proof.
\end{myproof}

% proof
\begin{myproof}[of the proposition]
    We prove this proposition in multiple steps.
    \begin{steps}
        % step 1
        \item $\bt_N \circ \ind{\pi_2}\bn \cong \bt_M\restr{N}$
        
        % total space
        Consider the two total spaces
        \[ E(\bt_N \circ \ind{\pi_2}\bn) = E(\ind{\pi_2}\bn) = \set{(n_1, n_2, u) \in (N \cross N) \cross U}{n_2 = r(u)} \]
        and
        \[ E(\bt_M\restr{N}) = \set{(n, m_1, m_2) \in N \cross (M \cross M)}{n = m_1}. \]

        % homeomorphy
        We can easily define a homeomorphism $\psi: E(\bt_M \circ \ind{\pi_2}\bn) \isomto E(\bt_N\restr{M})$ given by
        \[ \psi(m_1, m_2, u) = (m_1, m_1, u) \tand \inv{\psi}(m, n_1, n_2) = (m, r(n_2), n_2). \]
        Note that $\psi$ is a homeomorphism since $\psi$ and $\inv{\psi}$ are component-wise continuous.

        % commutativity
        It remains to be shown that $\psi$
        commutes with the injection and projection maps of $\bt_M \circ \ind{\pi_2}\bn$ and $\bt_N\restr{M}$.
        This can be seen with the following equations:
        \begin{align}
            \psi(i_{\pi_2}(\Delta(m))) = \psi(m, m, \iota(m)) = (m, m, m) = (m, \Delta(\iota(m))) \\
            \pi_1(j_{\pi_2}(m_1, m_2, u)) = m_1 = j_\iota(m_1, m_1, u) = j_\iota(\psi(m_1, m_2, u)) 
        \end{align}
        
        % step 2
        \item $\bt_M \circ \ind{\pi_1}\bn \cong \whitney{\bt_M}{\bn}$
        
        % total space
        In this case, the two total spaces
        \[ E(\bt_M \circ \ind{\pi_1}\bn) = E(\ind{\pi_1}\bn) = \set{(m_1, m_2, u) \in (M \cross M) \cross U}{m_1 = r(u)} \]
        and
        \[ E(\whitney{\bt_M}{\bn}) = \set{(m_1, m_2, u) \in (M \cross M) \cross U}{m_1 = r(u)} \]
        are equal.
        % homeomorphy + commutativity
        Additionally,
        the injection and projection maps agree, as the following equations show:
        \begin{align}
            & i_{\pi_1}(\Delta(m)) = i_{\pi_1}(m, m) = (m, m, \iota(m)) = (\Delta(m), \iota(m)) \\
            & \pi_1(j_{\pi_1}(m_1, m_2, u)) = \pi_1(m_1, m_2) = m_1 = r(u) = j_{\oplus}(m_1, m_2, u)
        \end{align}

        % step 3
        \item $\bt_M \circ \ind{\pi_1}\bn \cong \bt_M \circ \ind{\pi_2}\bn$
        
        We show that there exists a neighborhood $D \sub N \cross N$ of $\Delta(M)$ such that
        $\pi_1\restr{D}$ is homotopic to $\pi_2\restr{D}$:

        Firstly, we assume that $M$ is embedded in Euclidean space (see\cite[p.60]{dimension}).
        Let $V \sub N$ be a neighborhood retract of $M$.
        We define $D$ as follows:
        \[ D = \set{(m, m') \in M \cross M}{ tm + (1 - t)m' \in V, \forall t \in \I} \]
        We are given a homotopy $H: \cyl{D} \to N$ between $\pi_1$ and $\pi_2$ by
        \[ H((m, m'), t) = tm + (1 - t)m'. \]

        Applying the Homotopy Theorem yields $\ind{\pi_1}\bn\restr{D} \cong \ind{\pi_2}\bn\restr{D}$, and
        by restricting the total spaces accordingly, we get
        $\bt_M \circ \ind{\pi_1}\bn \cong \bt_M \circ \ind{\pi_2}\bn$.
    \end{steps}
    The claim follows by Step 1, 2 and 3.
\end{myproof}

% afterword
\begin{myparagraph}
    This proposition also shows that the normal microbundle
    underlies the same intuition as the normal vector bundle. 
    The sum of the tangent- and the normal microbundle of the submanifold
    `span' the tangent microbundle of its surrounding space.
\end{myparagraph}
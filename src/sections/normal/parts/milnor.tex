\begin{scope} \newcommand{\rwhitney} { \whitney{\ind{r}\bt_N}{\ind{r}\eta} } \newcommand{\rtn} { \ind{r}\bt_N }

\begin{mystatement}{theorem}[Milnors Theorem]{normal::milnor}[62] Let $N \sub M$ be a submanifold. For a sufficiently large $q \in \N$, $N = \cyldown{N}$ has a normal microbundle in $M \cross \R^q$. \end{mystatement}

\begin{myproof} We assume that $M$ is embedded in Euclidean space $\R^{2m + 1}$ \cite[p.60]{dimension}. Additionally, let $V$ be an open neighborhood of $N$ in $M$ together with a retraction $r: V \to N$.

We show the theorem in multiple steps. \begin{steps} \item $N$ has a normal microbundle $\eta$ in $E(\ind{r}\bt_N)$ such that $\whitney{\bt_N}{\eta} \cong \be[q]{N}$

Consider the extension $\ind{r}\bt_N$ for $\bt_N$. Since $V$ is an open set, it's a simplicial complex. Hence, we can apply \myintref{whitney::theorem} to $\ind{r}\bt_N$ to obtain a microbundle $\eta'$ such that $\whitney{\ind{r}\bt_N}{\eta'} \cong \be[q]{V}$.

We conclude that $\whitney{\bt_N}{\eta'\restr{N}} = \whitney{\ind{r}\bt_N\restr{N}}{\eta'\restr{N}} \cong (\whitney{\ind{r}\bt_N}{\eta'})\restr{N} \cong \be[q]{N}$.

\item $E(\rtn)$ has a normal microbundle in $E(\rwhitney)$

Note that $\rwhitney \cong \ind{r}(\whitney{\bt_N}{\eta})$ is trivial. Hence, by restricting the total space $E(\rwhitney)$ if necessary, we may assume that \[ E(\rwhitney) = V \cross \R^q \] for some $q \in \N$. Particularly, we derive that $E(\rwhitney)$ is a manifold.

We denote the injection and projection of $\rtn$ by $i_{\bt}$ and $j_{\bt}$, and the injection and projection of $\ind{r}\eta$ by $i_{\eta}$ and $j_{\eta}$.

We consider $E(\rtn)$ to be a submanifold of $E(\rwhitney)$ embedded by \[ e \mapsto (e, j_{r}(i_{\eta}(e))). \] We have a normal microbundle of $E(\rtn)$ in $E(\rwhitney)$ given by $\ind{j_{\bt}}(\ind{r}\eta)$. That is because the total space \[ E(\ind{j_r}(\ind{r}\eta)) = \set{(e, e') \in E(\rtn) \cross E(\ind{r}\eta)}{j_{\bt}(e) = j_\eta(e')} \] equals $E(\rwhitney)$ and because its inclusion matches the above embedding.

\item $N$ has a normal microbundle in $M \times \R^q$

Since $N$ has a normal microbundle in $E(\rtn)$ (\myintref{microbundle::total}) and $E(\rtn)$ has a normal microbundle in $E(\rwhitney)$, we conclude with \myintref{normal::transitivity} that $N$ has a normal microbundle in $E(\rwhitney)$.

As elaborated in Step 2, we may assume that $E(\rwhitney) = V \cross \R^n$. Since $V$ is open in $M$, $V \cross \R^n$ is open in $M \cross \R^n$. \end{steps} We conclude that $N$ has a normal microbundle in $M \cross \R^n$. \end{myproof}

\begin{myparagraph} With the provided proofs for \myintref{whitney::theorem} and \myintref{normal::milnor}, we can calculate an upper bound $m(2^{2m + 2} - 2)$ for $q$, where $m$ is the dimension of $M$ \cite[p.63]{milnor}. However, when provided with sharper proofs, one can substantially reduce the upper bound to a quadratic $(m + 1)^2 - 1$ (see \cite[p.232]{hirsch}). \end{myparagraph} \end{scope}
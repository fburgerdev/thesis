\begin{scope}
    % defines
    \newcommand{\rwhitney} {
        \whitney{\ind{r}\bt_N}{\ind{r}\eta}
    }
    \newcommand{\rtn} {
        \ind{r}\bt_N
    }

    % foreword
    \begin{myparagraph}
        Finally, we gathered all the tools to prove Milnor's theorem.
    \end{myparagraph}

    % theorem
    \begin{mystatement}{theorem}[Milnors Theorem]{normal::milnor}
        For a sufficently large $q \in \N$, $N = \cyldown{N}$ has a normal microbundle in $M \cross \R^q$.
    \end{mystatement}

    % proof
    \begin{myproof}
        We show the theorem in multiple steps:
        \begin{enumerate}
            % first
            \item There exists a microbundle $\eta$ over $N$ such that $\whitney{\bt_N}{\eta} \cong \be[q]{N}$:
            
            From the \extref{Whitney Embedding Thereom} it follows that
            we can embed $M$ in euclidean space $\R^{2m + 1}$.

            Additionally, since there exists a retraction $r: V \to N$
            where $V$ is an open neighborhood of $N$ in $M$ we can extend $\bt_N$
            to a microbundle $\bt_N'$ over $V$.
            Since $V$ is an open subset of euclidean space, it's a simplicial complex.

            Hence, we can apply \myintref{whitney::theorem} to the extended microbundle $\bt_N'$
            to obtain a microbundle $\eta'$ such that $\whitney{\bt_N'}{\eta} \cong \be[q]{V}$.

            We conclude that
            \[
                \whitney{\bt_N}{\eta'\restr{N}}
                = \whitney{\bt_N'\restr{N}}{\eta'\restr{N}}
                = (\whitney{\bt_N'}{\eta'})\restr{N} = \be[q]{N}.
            \]

            % second
            \item $E(\rtn) \sub E(\rwhitney)$ has a normal microbundle:

            Since the total space
            \[ E(\rwhitney) = \set{(e, e') \in E(\rtn) \cross E(\ind{r}\eta)}{j(e) = j'(e)} \]
            we can consider $E(\rtn) \sub E(\rwhitney)$ embedded via
            \[ \iota: e \mapsto (e, i'(j(e))) \]
            with the inverse $\pi_1: (e, e') \mapsto e$.

            Because $\rwhitney \cong \ind{r}(\whitney{\bt_N}{\eta})$ is trivial,
            it follows that $E(\rwhitney) \sub M \cross \R^k$ open and hence being a manifold.

            We have a normal microbundle of $E(\rtn)$ in $E(\rwhitney)$ via
            \[ \bundledef{\bn}{E(\rtn)}{E(\rwhitney)}{\iota}{\pi_1}. \]

            To show local triviality,
            let $(U, V, \phi)$ be a local trivialization of $i'(j(e))$ in $\ind{r}\eta$
            for an arbitrary $e \in E(\rtn)$.

            By choosing
            \begin{itemize}
                \item $U' := \inv{j}(U)$
                \item $V' := (U' \cross V) \cap E(\rwhitney)$
                \item $\phi': V' \isomto U' \cross \R^{n_\eta}$ with $\phi'(e, e') = (e, \snd{\phi}(e'))$
            \end{itemize}
            we have a local trivialization of $e$ in $\bn$.

            That is because both $U' \sub E(\rtn)$ and $V' \sub E(\rwhitney)$ are open sets
            and $\phi'$ is a homeomorphism with its inverse
            $\inv{\phi'}(e, x) = (e, \inv{\phi}(j(e), x))$.
            
            Also, $\phi'$ commutes with injection
            \[ \phi'(\iota(e)) = \phi'(e, i'(j(e))) = (e, \snd{\phi}(i'(j(e)))) = (e, 0) = (id \cross 0)(e) \]
            and projection maps
            \[ \pi_1(e, e') = \pi_1(e, \snd{\phi'}(e, e')) = \pi_1(\phi'(e, e')). \]
        \end{enumerate}
        % transitivity
        Since $N \sub M \sub E(\rtn)$ has a normal microbundle (using \myintref{normal::total}),
        it follows from \myintref{normal::transitivity} that
        $N \sub E(\whitney{\ind{r}\bt_N}{\ind{r}\bt'})$ has a normal microbundle.

        By restricting $E(\rwhitney)$ to an open subset if necessary, we may assume that
         \[ E(\rwhitney) = M \cross \R^q \]
        for some $q \in \N$ using \myintref{induced::trivial}.

        This completes the proof.
    \end{myproof}

    % afterword
    \begin{myparagraph}
        
    \end{myparagraph}
\end{scope}
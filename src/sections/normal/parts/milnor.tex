\begin{scope}
    % defines
    \newcommand{\rwhitney} {
        \whitney{\ind{r}\bt_N}{\ind{r}\eta}
    }
    \newcommand{\rtn} {
        \ind{r}\bt_N
    }

    % foreword
    \begin{myparagraph}
        Finally, we gathered all the tools to prove Milnor's theorem.
    \end{myparagraph}

    % theorem
    \begin{mystatement}{theorem}[Milnors Theorem]{normal::milnor}[62]
        For a sufficently large $q \in \N$, $N = \cyldown{N}$ has a normal microbundle in $M \cross \R^q$.
    \end{mystatement}

    % proof
    \begin{myproof}
        We assume that $M$ is embedded in euclidean space $\R^{2m + 1}$ \cite[p.60]{dimension}.
        Additionally, let $V$ be an open neighborhood of $N$ in $M$ together with a retraction $r: V \to N$.

        We show the theorem in multiple steps.
        \begin{steps}
            % step 1
            \item $N$ has a normal microbundle $\eta$ in $M$ such that $\whitney{\bt_N}{\eta} \cong \be[q]{N}$
            
            Consider the extension $\ind{r}\bt_N$.
            Since $V$ is an open set, it's a simplicial complex.
            Hence, we can apply \myintref{whitney::theorem} to $\ind{r}\bt_N$
            to obtain a microbundle $\eta'$ such that $\whitney{\ind{r}\bt_N}{\eta'} \cong \be[q]{V}$.

            We conclude that $\whitney{\bt_N}{\eta'\restr{N}} = \whitney{\ind{r}\bt_N\restr{N}}{\eta'\restr{N}} = (\whitney{\ind{r}\bt_N}{\eta'})\restr{N} = \be[q]{N}$.

            % step 2
            \item $E(\rtn)$ has a normal microbundle in $E(\rwhitney)$

            We denote the injection and projection of $\rtn$ by $i_{\bt}$ and $j_{\bt}$,
            and the injection and projection of $\ind{r}\eta$ by $i_{\eta}$ and $j_{\eta}$.

            Firstly, note that $\rwhitney \cong \ind{r}(\whitney{\bt_N}{\eta})$ is trivial, so
            $E(\rwhitney)$ is an open subset of $\R^q$ (for some $q \in \N$) and hence a manifold.

            We consider $E(\rtn)$ to be a subset of $E(\rwhitney)$ embedded via
            \[ (v, e) \mapsto ((v, e), (v, i_{\eta}(v))). \]

            We are given a normal microbundle of $E(\rtn)$ in $E(\rwhitney)$ by $\ind{j_{\bt}}(\ind{r}\eta)$.
            That is because the total space
            \[ E(\ind{j_r}(\ind{r}\eta)) = \set{(e, e') \in E(\rtn) \cross E(\ind{r}\eta)}{j_{\bt}(e) = j_\eta(e')} \]
            equals $E(\rwhitney)$ and because its inclusion is the above embedding
            \[ ((v, e), i_\eta(j_{\bt}(v, e))) = ((v, e), i_\eta(v)). \]
            
            % step 3
            \item $N$ has a normal microbundle in $M \times \R^q$
            
            Since $N$ has a normal microbundle in $E(\rtn)$ using \myintref{microbundle::total},
            it follows from \myintref{normal::transitivity} that
            $N$ has a normal microbundle in $E(\rwhitney)$.
     
            By restricting $E(\rwhitney)$ to an open neighborhood of $i_{\oplus}(V)$ if necessary, we may assume that
             \[ E(\rwhitney) = M \cross \R^q \]
            using \myintref{induced::trivial}.
        \end{steps}
        Applying \myintref{normal::transitivity} and $E(\rwhitney) = M \cross \R^q$ completes the proof.
    \end{myproof}

    % dimension
    \begin{myparagraph}
        With the proof of \myintref{whitney::theorem} and of \myintref{normal::milnor},
        we can calculate an upper bound $m(2^{2m + 2} - 2)$ for $q$ where $m$ is the dimension of $M$ (\cite[p.63]{milnor}).
        However with slightly sharper proofs,
        one can substantially reduce the upper bound
        to a quadratic upper bound $(m + 1)^2 - 1$ (see \cite[p.232]{hirsch}). 
    \end{myparagraph}
\end{scope}
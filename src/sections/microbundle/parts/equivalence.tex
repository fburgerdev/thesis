% foreword
\begin{myparagraph}
    The following statement is fundamental for the theory of
    microbundles over topological manifolds.
    It justifies that the tangent microbundle can be regarded
    as a generalization of the tangent vector bundle.
\end{myparagraph}

% theorem
\begin{mystatement}{theorem}{microbundle::equivalence}[56]
    Let $M$ be a smooth $d$-manifold.
    Then the underlying microbundle of $\xi: TM \to M$ and the tangent microbundle $\bt_M$
    are isomorphic. 
\end{mystatement}

% proof
\begin{myproof}
    We equip $M$ with a Riemannian metric,
    which allows us to define the usual 
    exponential map $\exp: V \to M$ where $V \sub TM$ is a neighborhood
    of the zero-cross section of $M$.

    Consider $id \cross \exp$.
    % IFT
    Using the Inverse Function Theorem for smooth manifolds (see\cite[thm.4.5]{lee})
    for arbitrary $(p, \nu) \in V$, it follows that $id \cross \exp$ is a local diffeomorphism
    and hence a local homeomorphism.
    % local homeomorphism -> homeomorphism
    Furthermore,
    the zero-cross section is mapped homeomorphically to the diagonal.
    By applying Lemma 4.1 from \cite[lm.4.1]{whitehead} (manifolds are locally compact and separable),
    it follows that $id \cross \exp$ maps a neighborhood
    of the zero-cross section to a neighborhood of the diagonal.
    % commutativity
    Commutativity with the injection maps is given by
    \[ (id \cross \exp)(i_{\abs{\eta}}(p)) = (id \cross \exp)(p, 0) = (p, p) = \Delta(p) \]
    and with the projection maps by
    \[ j_{\abs{\eta}}(p, \nu) = p = \pi_1(p, \exp(\nu)) = \pi_1((id \cross \exp)(p, \nu)), \]
    which concludes the proof.
\end{myproof}
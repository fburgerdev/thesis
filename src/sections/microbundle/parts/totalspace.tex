% foreword
\begin{myparagraph}
    As the definition of isomorphy already indicates, when studying microbundles,
    we are not interested in the entire total space
    but only in an arbitrary small neighborhood of the base space (more precise, the image $i(B)$).
    The following proposition makes this even clearer.
    
    % Throughout the paper we will often take use of homeomorphisms
    % \[ \mu_\eps: \ball[\eps] \isomto \R^n \twith \mu_\eps(0) = 0 \]
    % for example given by
    % \[ x \mapsto \tan(\frac{\abs{x}\cdot\pi}{2\eps}) x \]
    % in order to show properties like local triviality.
\end{myparagraph}

% proposition
\begin{mystatement}{proposition}{microbundle::total}[54]
    Given a microbundle $\bundledef{\bb}{B}{E}{i}{j}$ over $B$,
    restricting the total space $E$ to an
    arbitrary neighborhood $E' \sub E$ of $i(B)$ leaves the microbundle unchanged.
    That is, the microbundle
    \[ \bundledef{\bb'}{B}{E'}{i}{j\restr{E'}} \]
    is isomorphic to $\bb$.
\end{mystatement}

% proof
\begin{myproof}
    We prove the proposition in two steps.
    \begin{steps}
        % microbundle
        \item $\bb'$ is a microbundle
        
        Continuity for $i$ and $j$ as well as $j\restr{E'} \circ i = id_B$ are already given since $\bb$ is a microbundle.

        So we only need to show that $\bb'$ is locally trivial.
        For an arbitrary $b \in B$, choose a local trivialization $(U, V, \phi)$ of $b$ in $\bb$.
        By restricting $\phi$ to $V \cap E'$,
        we obtain a homeomorphism on its image as required in \myintref{microbundle::local},
        hence showing local triviality.
        
        % isomorphy
        \item $\bb'$ is isomorphic to $\bb$

        Since $E'$ is a subset of $E$,
        we can simply use the identity $E' \to E' \sub E$
        as our homeomorphism between neighborhoods of $i(B)$.
        Furthermore, the injection and projection maps for $\bb$ and $\bb'$ are the same,
        so they clearly commute with the identity.
    \end{steps}
    This completes the proof.
\end{myproof}
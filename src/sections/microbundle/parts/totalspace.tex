% foreword
\begin{myparagraph}
    As the definition of isomorphy already implies, when studying microbundles,
    we are not interested in the entire total space
    but only in an arbitrary small neighborhood of the base space (more precise, the image $i(B)$).
    This is one of the biggest conceptual differences between
    microbundles and vector bundles.
    The following proposition makes this even clearer.
    
    Throughout the paper we will often take use of homeomorphisms
    \[ \mu_\eps: \ball[\eps] \isomto \R^n \twith \mu_\eps(0) = 0 \]
    for example given by
    \[ x \mapsto \tan(\frac{\abs{x}\cdot\pi}{2\eps}) x \]
    in order to show properties like local triviality.
\end{myparagraph}

% proposition
\begin{mystatement}{proposition}{microbundle::total}
    Given a microbundle $\bundledef{\bb}{B}{E}{i}{j}$ over $B$,
    restricting the total space $E$ to an
    arbitrary neighborhood $E' \sub E$ of $i(B)$ leaves the microbundle unchanged.
    That is, the microbundle
    \[ \bundledef{\bb'}{B}{E'}{i}{j\restr{E'}} \]
    is isomorphic to $\bb$.
\end{mystatement}

% proof
\begin{myproof}
    We prove the proposition in two steps.
    \begin{enumerate}
        % microbundle
        \item $\bb'$ is a microbundle:
        
        Continuity and $id_B = j\restr{E'} \circ i$ are already given since $\bb$ is a microbundle.

        So we only need to show that $\bb'$ is locally trivial.
        For an arbitrary $b \in B$, choose a local trivialization $(U, V, \phi)$ of $b$ in $\bb$.
        
        The image $\phi(V \cap E')$ is a neighborhood of $(b, 0)$.
        This follows from $\phi(i(b)) = (b, 0)$ and $V \cap E'$ being a neighborhood of $i(b)$.

        Hence, for a sufficiently small $\eps > 0$ there
        exists a $U' \cross \ball[\eps] \sub \phi(V \cap E')$ such that
        $U'$ is an open neighborhood of $b$.

        Utilizing a homeomorphism $\mu_\eps: \ball[\eps] \isomto \R^n$,
        we have a local trivialization $(U', V', \phi')$ with
        \[ \phi': V' \xto{\phi} U' \cross \ball[\eps] \xto{id \cross \mu_\eps} U' \cross \R^n \]
        and $V' := \inv{\phi}(U' \cross \ball[\eps])$.

        That is because $\phi'$ commutes with the injection
        \[ \phi'(i(b)) =  (id \cross \mu_\eps)(\phi(i(b))) = (id \cross \mu_\eps)(b, 0) = (b, 0) = (id \cross 0)(b)\]
        and projection maps
        \[ j(e) = \pi_1(\phi(e)) = \pi_1((id \cross \mu_\eps)(\phi(e))) = \pi_1(\phi'(e)). \]
        
        % isomorphy
        \item $\bb'$ is isomorphic to $\bb$:

        Since $E' \sub E$, we can simply take the identity $E' \to E' \sub E$
        as our homeomorphism between neighborhoods of $i(B)$.
        Furthermore, the injection and projection maps for $\bb$ and $\bb'$ are the same,
        so they clearly commute with the identity.
    \end{enumerate}
\end{myproof}
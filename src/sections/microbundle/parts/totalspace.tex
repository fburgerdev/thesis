% foreword
The following proposition shows that, when studying microbundles,
we are not interested in the entire total space
but only in an arbitrarily small neighborhood of the base space.
This is actually one of the biggest differences behind
the concept of microbundles and classical vector-bundles.

% prop: restricting the total space
\begin{myproposition}
    % statement
    For a microbundle $\bundledef{\bb}{B}{E}{i}{j}$ over $B$, we can be restrict the total space $E$ to an arbitray neighborhood $E' \sub E$ of $i(B)$
    where the resulting microbundle is isomorphic to $\bb$.
    % proof
    \begin{myproof}
        For an arbitray $b \in B$, choose a local trivialization $(U, V, \phi)$.

        The intersection $V \cap E'$ is a neighborhood of $i(b)$ because $V$ and $E'$ both are.
        It follows that $\phi(V \cap E')$ is a neighborhood of $(b, 0)$. Hence there exist $U' \sub B$ open and $\ball[\varepsilon] \sub \R^n$ such that $U' \times \ball[\varepsilon] \sub \phi(V \cap E')$. 
        Now we construct our local trivialization by choosing $V' := \phi^{-1}(U' \times \ball[\varepsilon])$ and the fact that $\ball[\varepsilon] \cong \R^n$:
        \[ U' \times \R^n \cong U' \times \ball[\varepsilon] \cong V' \]

        We easily see that the resulting microbundle is isomorphic to $\bb$ via the identity.
    \end{myproof}
\end{myproposition}
% foreword
\begin{myparagraph}
    Let $r: B \isomto B$ denote the `reflection' given by
    \[ r([x, t]) = [x, 1 - t] \]
    and let $c: B \vee B \to B$ denote
    the identity on the first summand and $r$ on the second summand, i.e,
    \[ c(b, i) = \case{b}{r(b)}{i = 1}. \]
\end{myparagraph}

% lemma
\begin{mystatement}{lemma}{suspension::reflection}[70]
    The induced microbundle $\ind{\phi}(\bb \vee \ind{r}\bb)$ is trivial.
\end{mystatement}

% proof
\begin{myproof}
    The composition $c \circ \phi$ is null-homotopic via
    the homotopy $H: B \cross \I \to B$ given by
    \[ H([x, t], s) = f(\phi(x, t \cdot s)). \]
    Applying \myintref{homotopy::theorem} yields
    $\ind{\phi}(\ind{c}\bb) \cong \ind{(c \circ \phi)}\bb \cong \ind{c_{B, b_0}}\bb \cong \be{B}$
    as rooted-isomorphy.
    Note that the induced microbundle preserves rootings
    (see \myintref{suspension::induced}).
    By applying the previous lemma,
    it follows that
    \[ \ind{\phi}(\bb \vee \ind{r}\bb) = \ind{\phi}(\ind{(id \cup r)}(\bb \vee \bb)) = \ind{\phi}(\ind{c}\bb) \]
    and hence $\ind{\phi}(\bb \vee \ind{r}\bb) \cong \be{B}$.
\end{myproof}
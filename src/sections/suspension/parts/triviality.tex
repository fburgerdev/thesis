\begin{mystatement}{lemma}{suspension::triviality}[70] The following (non-rooted) isomorphy holds: \[ \ind{\phi}(\bb \vee \be{B}) \cong \bb \cong \ind{\phi}(\be{B} \vee \bb) \] \end{mystatement}

\begin{myproof} We prove the lemma in two steps. \begin{steps} \item $\ind{c_1}\bb \cong \bb \vee \be{B}$

Let $E(\bb \vee \be{B})$ be constructed via the representative $f: V \to b_0 \cross \R^n$ for $R$.

Without loss of generality, we may assume that $V = E(\bbb)$ by removing the closed set $E(\bbb) - V$ from $E(\bb)$ if necessary.

Consider $\psi: E(\ind{c_1}\bb) \isomto E(\bb \vee \be{B})$ given by \[ \psi((b, i), e) = \case{e}{(b, \snd{f}(e))}{i = 1}. \] Note that $\psi$ is well-defined, because \[ \psi((b_0, 1), e) = e = f(e) = (b_0, \snd{f}(e)) = \psi((b_0, 2), e). \] Furthermore, $\psi$ is a homeomorphism as both of its summands are homeomorphisms.

It remains to be shown that $\psi$ commutes with the injection and projection maps of $\ind{c_1}\bb$ and $\bb \vee \be{B}$. This can be seen with the following equations: \begin{align} &\psi(i_{c_1}(b, i)) = \casenif{\psi((b, 1), i(b)) = i(b) = i_\vee(b, 1)}{\psi((b, 2), i(b_0)) = f(i(b)) = (b, 0) = (id \cross 0)(b) = i_\vee(b, 2)} \\ &j_{c_1}((b, i), e) = \casenif{j(e) = j_\vee(e) = j_\vee(\psi((b, 1), e))}{(b, 2) = \pi_1(\psi((b, 2), e)) = j_\vee(\psi((b, 2), e))} \end{align}

\item $\ind{\phi}(\bb \vee \be{B}) \cong \bb$

Using the fact that $c_1 \circ \phi = id_{B \vee B}$, we conclude that \[ \ind{\phi}(\bb \vee \be{B}) \cong \ind{\phi}(\ind{c_1}\bb) \cong \ind{(c_1 \circ \phi)}\bb \cong \bb. \] For reasons of symmetry, it follows that $\bb \cong \ind{\phi}(\be{B} \vee \bb)$. \end{steps} This completes the proof. \end{myproof}
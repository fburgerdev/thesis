% foreword
\begin{myparagraph}
    Now that we introduced rooted-microbundles,
    we are able to define the wedge sum.
    As we will see in the subsequent proof,
    we require rootings for its definition.
    Particulary, the wedge sum depends on the specific choices of these rootings,
    justifying the requirement for rooted-microbundles.

    Given a quotient space $A \sqcup B / \sim$ and maps $f: A \to C$ and $g: B \to C$, we define
    $f \cup g: (A \sqcup B / \sim) \to C$ by
    \[ x \mapsto \begin{cases} f(x) & \text{if } x \in A \\ g(x) &\text{if } x \in B \end{cases}.\]
    Clearly, this map is only well-defined if $a \sim b \implies f(a) = g(b)$.
\end{myparagraph}

% definition
\begin{mystatement}{definition}{suspension::wedge}[70] % ? remove citation
    Let $\ba$ and $\bb$ be two rooted microbundles over $A$ and $B$.
    % microbundle
    The \defterm{wedge sum} $\ba \vee \bb$ of $\ba$ and $\bb$ is a microbundle
    \[ A \vee B \xto{i_a \cup i_b} E(\ba \vee \bb) \xto{j_a \cup j_b} A \vee B \]
    with the total space defined as
    \[ (E(\ba) \sqcup E(\bb)) / f(e_a) \sim e_a \]
    where $f: W_a \isomto W_b$ is a representative for $\inv{R_b} \circ R_a$.
    
    % rooting
    We equip $\ba \vee \bb$ with a rooting
    \[ \germdef{R}{E((\ba \vee \bb)\restr{a_0})}{\be{a_0}} \]
    represented by any representative for $R_a$ (or $R_b$).
\end{mystatement}

% proof
\begin{myproof}[that $\ba \vee \bb$ is a (rooted) microbundle]
    Let $f: W_a \isomto W_b$ be a representative for $\inv{R_b} \circ R_a$.

    % rooted microbundle
    \begin{sectionize}
        \item $\ba \vee \bb$ is a rooted microbundle
        \begin{itemize}
            % injection
            \item The injection map $i_a \cup i_b$ is well-defined because 
            \[ [i(a_0)] = [i_a(a_0)] = [f(i_a(a_0))] = [i_b(b_0)] = [i(b_0)] \]
            and continuous since both $i_a$ and $i_b$ are continuous.
            % projection
            \item The projection map $j_a \cup j_b$ is well-defined because
            \[ \forall e \in W_a: [j(e)] = [j_a(e)] = [a_0] = [b_0] = [j_b(f(e))] = [j(f(e))] \]
            and continuous since both $j_a$ and $j_b$ are continuous.
            % compatibility
            \item The composition $j \circ i$ is the identity because
            \[ \forall a \in A: j(i(a)) = j(i_a(a)) = j_a(i_a(a)) = a \]
            since $j_a \circ i_a = id_A$ (symmetrical for $B$).
        \end{itemize}
        % local triviality
        It remains to be shown that $\ba \vee \bb$ is locally trivial.

        Let $x \in A \vee B$.
        For reasons of symmetry, we can assume that $x \in A$.
        \begin{caselist}
            % trivial case
            \item $x \neq a_0$
            
            % construct
            Choose a local trivialization $(U, V, \phi)$ for $x$ in $\ba$.
            Without loss of generality,
            we can assume that $U \cap B = \emptyset$ by subtracting
            $\{a_0\}$ from $U$ if necessary.
            Note that $\{a_0\}$ is closed since $A$ is hausdorff.
            
            % show
            Now we can simply use this trivialization for $\ba \vee \bb$, because
            $U \sub A$ is open in $A \vee B$ and
            $V \sub E(\ba)$ is open in $E(\ba \vee \bb)$.
            Furthermore, since $i$ and $j$ reduce to $i_a$ and $j_a$,
            it follows that $\phi$ commutes with $i$ and $id \cross 0$
            as well as with $j$ and $\pi_1$.

            % special case
            \item $x = a_0$
            
            Let $(U_a, V_a, \phi_a)$ and $(U_b, V_b, \phi_b)$ be local trivializations
            for $a_0 = b_0$ in $\ba$ and $\bb$.

            % open
            Since $W_a \sub E(\bab)$ is open,
            there exists an open subset
            $W_a' \sub E(\ba)$ such that $W_a = W_a' \cap E(\bab)$.

            Let $U_a' \sub A$ be an open neighborhood of $a_0$ and $\eps > 0$ such that
            \[ V_a' = U_a' \cross \ball[\eps] \sub \phi_a(W_a'). \]
            This allows us to define the map
            \[ \phi_a': V_a' \isomto \phi_a'(V_a') \sub A \cross \R^n \twith \]
            \[ \phi_a'(e) = (j_a(e), (\snd{\phi_b} \circ f \circ \inv{\phi_a})(a_0, \snd{\phi_a}(e))). \]

            Now we can show local triviality in $a_0$ using the homeomorphism
            \[ \phi_a' \cup \phi_b: V_a' \cup V_b \isomto \phi_a'(V_a' \cup V_b) \sub (A \vee B) \cross \R^n \]

            This map is well-defined, because
            \[ \phi_a'(e) = (a_0, (\snd{\phi_b} \circ f \circ \inv{\phi_a})(a_0, \snd{\phi_a}(e))) \]
            \[ = (b_0, \snd{\phi_b}(f(e))) = (j_b(f(e)), \snd{\phi_b}(f(e))) = \phi_b(f(e)). \]

            Homeomorphy follows from the fact
            that both $\phi_a'$ and $\phi_b$ are homeomorphisms,
            and that $\phi_a'(e_a) = \phi(e_b) \implies f(e_a) = e_b$. 

            Commutativity with $i_a \cup i_b$ and $id \cross 0$
            as well as between $j_a \cup j_b$ and $\pi_1$
            is inherited from $\phi_a$ and $\phi_b$.
            Note that $\phi_a(i_a(a)) = (a, 0) = \phi_a'(i_a(a))$.

            Applying \myintref{microbundle::local} yields that $\ba \vee \bb$ is locally trivial.
        \end{caselist}
        
        % well-defined
        \item $\ba \vee \bb$ is well-defined

        Let $f'$ be another representative for $\inv{R_b} \circ R_a$ and $(\ba \vee \bb)'$ the resulting wedge sum.
        We need to find an isomorphism-germ that extends $\inv{R'} \circ R$.
        
        In order to do this,
        choose an open neighborhood $V \sub E(\ba\restr{a_0})$
        of $i_a(a)$ where $f$ and $f'$ agree.
        
        By subtracting the closed set $\inv{j_a}(a_0) - V$
        from $E(\ba \vee \bb)$ and $E(\ba \vee \bb)'$,
        the microbundles remain unchanged due to \myintref{microbundle::total}.
        
        But now the total spaces $E(\ba \vee \bb)$ and $E((\ba \vee \bb)')$ are the same.
        That is because $E(\ba \vee \bb)$ and $E((\ba \vee \bb)')$
        could only differ in $\inv{j_a}(a_0) - V$.

        
        Furthermore, since injection and projection are defined exactly the same,
        it follows that the identity $\germ{(\ba \vee \bb)}{(\ba \vee \bb)'}$
        is an isomorphism-germ.
        Together with
        \[ \inv{R'} \circ R = \inv{R} \circ R = I, \]
        which completes the proof.
    \end{sectionize}
\end{myproof}
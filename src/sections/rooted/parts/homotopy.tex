% foreword theorem
\begin{myparagraph}
    Given a rooted microbundle $\bb$ and homotopic based maps $f, g: A \to B$,
    the Homotopy Theorem yields that $\ind{f}\bb$ and $\ind{g}\bb$
    are isomorphic (not rooted-isomorphic).
    
    With the preliminary work in \myintref{section::homotopy_prove}, we can derive
    a version of the Homotopy Theorem that also accounts for rooted isomorphy.
\end{myparagraph}

% theorem
\begin{mystatement}{theorem}[Rooted Homotopy Theorem]{suspension::homotopy}[69]
    Let $\bb$ be a rooted microbundle over $B$ and let $f, g: A \to B$ be two based maps
    where $A$ is paracompact hausdorff.
    If there exists a homotopy $H: \cyl{A} \to B$ between $f$ and $g$ that leaves the base point fixed,
    then the two rooted microbundles $\ind{f}\bb$ and $\ind{g}\bb$ are rooted isomorphic.
\end{mystatement}

% foreword lemma
\begin{myparagraph}
    In order to prove this,
    we need to show a `rooted version' of \myintref{homotopy::lemma2}.
    
    First, note that
    \[
        E(\ind{H}\bb[\cyl{a_0}]) = E(\ind{\iota}(\ind{H}(\bb)))
        \cong E(\ind{(H \circ \iota)}\bb) = E(\ind{c_{\cyl{a_0}, b_0}}\bb),
    \]
    whose total space is of the form $(a_0 \cross [0, 1]) \cross E(\bb)$.
    Based on this, we can define an isomorphism-germ
    $\germdef{\overline{R}}{\ind{H}\bb[\cyl{a_0}]}{\be{\cyl{a_0}}}$
    represented by
    \[ \overline{r}(a_0, t, v) = (a_0, t, \snd{r}(v)), \]
    where $r: V \to b_0 \cross \R^n$ is a representative for $R$.
    Note that $\overline{r}$ is a homeomorphism on its image,
    since its components are homeomorphisms on their image.
\end{myparagraph}

% lemma
\begin{mystatement}{lemma}{suspension::sharper}[69]
    Let $\bb$ be a rooted microbundle over $B$ and
    let $H: \cyl{A} \to B$ be a map that leaves the base point fixed.
    Then there exists a neighborhood $V$ of $a_0$ together with an isomorphism-germ
    \[ \germ{\ind{H}\bb[\cyl{V}]}{\be{\cyl{V}}} \]
    extending $\overline{R}$ (as defined above).
\end{mystatement}

% proof lemma
\begin{myproof}
    By applying \myintref{homotopy::lemma2},
    it follows that there exists an isomorphism-germ
    \[ \germdef{Q}{\ind{H}\bb[\cyl{V}]}{\be{\cyl{V}}} \]
    for a sufficiently small neighborhood $V$ of $a_0$.
    However, $Q$ does not extend $\overline{R}$ in general.

    In order to fix this, consider
    \[ \germdef{Q \circ \inv{\overline{R}}}{\be{\cyl{a_0}}}{\be{\cyl{a_0}}} \]
    together with a representative $f: U_f \to (\cyl{a_0}) \cross \R^n$.

    Similar to the construction of $\overline{R}$, we can construct an isomorphism-germ
    \[ \germdef{P}{\be{\cyl{V}}}{\be{\cyl{V}}} \]
    extending $Q \circ \inv{\overline{R}}$ represented by
    \[ (a, t, x) \mapsto (a, f(a_0, t, x)) \]
    considering $f(a_0, t, x) \in \I \cross \R^n$.

    Restricted to $\be{\cyl{a_0}}$, $P$ agrees with $Q \circ \inv{\overline{R}}$ and thus
    \[
        \inv{Q} \circ P\restr{\be{\cyl{a_0}}}
        = (\inv{Q} \circ (Q \circ \inv{\overline{R}}))
        = ((\inv{Q} \circ Q) \circ \inv{\overline{R}})
        = \inv{\overline{R}}.
    \]
    % \[ \implies (\inv{P} \circ Q)\restr{\ind{H}\bb[\cyl{a_0}]} = \overline{R}. \]
    Since $P$ and $Q$ are both isomorphism-germs,
    \[ \germdef{\inv{P} \circ Q}{\ind{H}\bb[\cyl{V}]}{\be{\cyl{V}}} \]
    is an isomorphism-germ extending $\overline{R}$.
\end{myproof}

% foreword proof
\begin{myparagraph}
    We are now able to show the Rooted Homotopy Theorem.

    To understand the proof,
    it is useful to have the constructions of \myintref{homotopy::lemma3} in mind,
    because we will modify them slightly in order to preserve the rootings.
\end{myparagraph}

% proof theorem
\begin{myproof}[of the Rooted Homotopy Theorem]
    We need to show that $\ind{f}\bb$ and $\ind{g}\bb$ are rooted isomorphic,
    that is, there exists an isomorphism-germ $\germ{\ind{f}\bb}{\ind{g}\bb}$
    extending $\inv{R_g} \circ R_f = I$
    where $I$ denotes the identity germ.

    For the initial Homotopy Theorem,
    we constructed a bundle-germ
    \[ \germdef{F}{\ind{H}\bb}{\ind{H}\bb[\cylup{A}]} \]
    covering $(a, t) \mapsto (a, 1)$
    and restricted it to $\ind{H}\bb[\cyldown{A}]$.
    The required isomorphism-germ was then obtained by
    identifying $\ind{f}\bb$ with $\ind{H}\bb[\cyldown{A}]$ and
    $\ind{g}\bb$ with $\ind{H}\bb[\cyldown{A}]$.

    We must make slight modifications
    to the construction of $F$ such that it extends
    $\germ{\ind{f}\bbb \cong \ind{H}\bb[\cyldown{a_0}]}{\ind{H}\bb[\cylup{a_0}] \cong \ind{g}\bbb}$
    represented by
    \[ (a_0, e) = ((a_0, 0), e) \mapsto ((a_0, 1), e) = (a_0, e). \]

    This can be achieved by choosing a locally finite open cover $\coll{V_\alpha}$
    of $A$ (as in \myintref{homotopy::lemma3}), removing the base point $a_0$ from every set
    and adding $V$ obtained from \myintref{suspension::sharper}.
    Since $a_0 \in V$, the resulting collection is still a
    locally finite open cover of $A$.
    
    In the following, we denote constructions over $V$
    with subscript $V$ and constructions over the other sets
    from the cover with subscript $\alpha$.

    We continue with the proof of \myintref{homotopy::lemma3}.
    Note that $\lambda_V(a_0) = 1$.
    That is because we removed $a_0$ from every other set, and hence $\lambda_\alpha(a_0) = 0$.

    Lastly, we construct the extension $R_V$ for $r_V$
    like in \myintref{section::homotopy_prove},
    but instead of choosing an arbitrary trivialization
    $E(\ind{H}\bb\restr{A_V}) \cong A_V \cross \R^n$
    for the construction we use a representative $r$
    for the bundle-germ constructed in \myintref{suspension::sharper}.
    
    This has the advantage that the representative
    \[
        E(\ind{H}\bb[A_V]) \xto{r}
        A_V \cross \R^n \xto{r_V \cross id} (A_V \cap A'_V) \cross \R^n
        \xto{\inv{r}} E(\ind{H}\bb[A_V \cap A'_V])
    \]
    for $R_V$ maps elements $((a_0, 0), e)$ to $((a_0, 1), e)$.
    Additionally,
    every other $R_\alpha$ leaves $\ind{H}\bb[\cyldown{a_0}]$ unaffected
    because $r_\alpha(a_0, t) = (a_0, \max(\underbrace{\lambda_\alpha(t)}_{= 0}, t)) = (a_0, t)$.

    It follows that, by piecing together the $R_\alpha$ and $R_V$ like in \myintref{homotopy::lemma3},
    we obtain a bundle germ $\germdef{F}{\ind{H}\bb}{\ind{H}\bb[\cylup{A}]}$
    extending $\inv{R_g} \circ R_f$.
    This completes the proof.
\end{myproof}
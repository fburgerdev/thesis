\begin{scope}
    % defines
    \newcommand{\bbleft} {
        \bb[B {\cross} [0, \half]]
    }
    \newcommand{\bbright} {
        \bb[B {\cross} [\half, 1]]
    }
    \newcommand{\bbhalf} {
        \bb[B {\cross} \coll{\half{}}]
    }

    % % foreword
    % \begin{myparagraph}
    %     Using the previous lemma,
    %     we can show a criteria on whether a microbundle
    %     over a cylinder base space is trivial.        
    % \end{myparagraph}

    % lemma
    \begin{mystatement}{lemma}{homotopy::lemma1}[67]
        Let $\bb$ be a microbundle over $\cyl{B}$.
        If $\bbleft$ and $\bbright$ are both trivial,
        then $\bb$ itself is trivial.
    \end{mystatement}

    % proof
    \begin{myproof}
        Consider the identity bundle-germ over $\bbhalf$,
        defined as the bundle-germ represented by the identity on $E(\bbhalf)$.
        
        Since $\bbright$ and $\bbhalf$ are both trivial,
        there exist isomorphism-germs
        \[
            \germdef{R}{\bbright}{\be{B {\cross} [\half, 1]}} \tand
            \germdef{L}{\bbhalf}{\be{B {\cross} \coll{\half}}}.
        \]

        We can define a bundle-germ $\germ{\bbright}{\bbhalf}$
        extending the identity on $\bbhalf$ using
        \[
            \germdef{M}{\be{B \cross [\half, 1]}}{\be{B \cross \coll{\half}}} \twith
            (b, t, x) \mapsto (b, \half, x)
        \]
        to form the composition $\inv{L} \circ M \circ R$.

        Using the previous lemma, we can piece this together with the
        identity bundle-germ on $\bbleft$
        (note that the bundle-germs agree on their intersection)
        resulting in a bundle-germ
        \[ \germ{\bb}{\bbleft}. \]

        \myintref{homotopy::corollary} infers that $\bb$
        is isomorphic to $\ind{r}\bbleft$
        where $r: B \cross [0, 1]$ is the
        retraction $(b, t) \mapsto (b, \min(t, \half))$.
        But $\bbleft$ is trivial, hence $\ind{r}\bbleft$ is
        trivial as well (see \myintref{induced::trivial}),
        which concludes the proof.
    \end{myproof}
\end{scope}
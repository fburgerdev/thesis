\begin{file}
% local
\newcommand{\A}[1][] {
    A_\alpha#1
}

% local
\newcommand{\bbleft} {
    \bb\restr{B \times{} [0, \half]}
}
\newcommand{\bbright} {
    \bb\restr{B \times{} [\half, 1]}
}
\newcommand{\bbhalf} {
    \bb\restr{B \times{} \{ \half \}}
}

% statement
\lemma{\parttitle{piecewise triviality}} \\
Let $\bb$ be a microbundle over $B \times [0, 1]$ such that both $\bbleft$ and $\bbright$ are trivial.
Then $\bb$ itself is already trivial.
% proof
\begin{proof}
Since $\bbright$ is trivial, we can extend the identity bundle map-germ on $\bbhalf$ to $\bbright \double \bbhalf$.
Using the previous lemma, we can piece this together with the identity bundle map-germ on $\bbleft$ to
\[ \bb \double \bbleft \]

From the corollary it follows that $\bb \cong \bbleft$.
\end{proof}

% statement
\lemma{\parttitle{}} \\
Let $\bb$ be a microbundle over $B \times [0, 1]$.
Every $b \in B$ has a neighborhood $V$ where $\bb\restr{V \times [0, 1]}$ is trivial.
%proof
\begin{proof}
Let $b \in B$.
For every $t \in [0, 1]$, choose a neighborhood $U_t := V_t \times (t - \varepsilon_t, t + \varepsilon_t)$ of $(b, t)$ such that $\bb\restr{U_t}$ is trivial.
Since $\{b\} \times [0, 1]$ is compact, we can choose a finite covering of the $U_t$ and define $V$ to be the intersection of the corresponding $V_t$.
Then there exists a subdivision $0 = t_0 < \cdots < t_k = 1$ where the $\bb\restr{V \times [t_{i}, t_{i + 1}]}$ are trivial.
Iteratively applying the previous lemma, it follows that $\bb\restr{V \times [0, 1]}$ is trivial.
\end{proof}

% lemma
\lemma{\parttitle{}} \\
Let $\bb$ be a microbundle over $B \times [0, 1]$ where $B$ is paracompact.
Then there exists a bundle map-germ $F: \bb \to \bb\restr{B \times \{1\}}$
covering the standard retraction $r: B \times [0, 1] \to B \times \{1\}$ .
% proof
\begin{proof}
First, we assume a locally finite covering $\{V_\alpha\}$ of closed sets where $\bb\restr{V_\alpha \times [0, 1]}$ is trivial.
The existence of such a covering is justified by the previous lemmas.
Since $B$ is paracompact, we can choose bump functions 
\[ \lambda_\alpha: B \to [0, 1] \]
with $\text{supp}(\lambda_\alpha) \sub V_\alpha$.
Additionally, we assume that 
\[ \max_{\alpha}(\lambda_\alpha(b)) = 1, \forall b \in B \]
Now we define a retraction $r_\alpha: B \times [0, 1] \to B \times [0, 1]$ with
\[ r_\alpha(b, t) := (b, \max(t, \lambda_\alpha(b))) \]

We construct bundle map-germs $R_\alpha: \bb \to \bb$ covering $r_\alpha$.
We can divide $B \times [0, 1]$ into $A_\alpha := \text{supp}(\lambda_\alpha) \times [0, 1]$ and $A'_\alpha := \{(b, t) \mid t \ge \lambda_\alpha(b)\}$.
Since $A_\alpha \sub V_\alpha \times [0, 1]$, $\bb\restr{A_\alpha}$ is trivial.
It follow that the identity bundle germ on $\bb\restr{\A \cap \A[']}$ can be extended to a bundle germ $\bb\restr{\A} \double \bb\restr{\A \cap \A[']}$.
Piecing this together with the identity bundle germ $\bb\restr{\A[']}$, we obtain the desired bundle germ $R_\alpha$.

Applying the well-ordering theorem, which is equivalent to the axiom of choice, we can assume an ordering of $\{ V_\alpha \}$.
Let $\{B_\beta\}$ be a locally finite covering of $B$ with closed sets where $B_\beta$ intersects only $V_{\alpha_1} < \cdots < V_{\alpha_k}$ a finite collection.
Now the composition $R_{\alpha_1} \circ \ldots \circ R_{\alpha_k}$ restricts to a bundle germ $R(\beta): \bb\restr{B_\beta} \times [0, 1] \double \bb\restr{B_\beta} \times [1]$.
Pieced together with the previous lemma, we obtain $R: \bb \times [0, 1]\to \bb \times [1]$ which concludes the proof.
\end{proof}

% theorem
\theorem{\parttitle{Homotopy Theorem}} \\
Let $\bb$ be a microbundle of $B$ and $f, g: A \to B$ be two maps.
\[ f \simeq g \implies f^*\bb \cong g^*\bb \]
% proof
\begin{proof}
Let $H: A \times [0, 1] \to B$ be a homtopy between $f$ and $g$.
By the previous lemma, there exists a bundle germ $R: H^*\bb \double H^*\bb\restr{B \times [1]}$ covering the standard retraction $B \times [0, 1] \to B \times [1]$.
From the composition
\[ f^*\bb \sub H^*\bb \double_R H^*\bb\restr{B \times [1]} = g^*\bb \]
we obtain an isomorphism germ $f^*\bb \double g^*\bb$.
It follow that $f^*\bb \cong g^*\bb$.
\end{proof}

\end{file}
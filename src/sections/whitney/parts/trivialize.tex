% foreword
\begin{myparagraph}
    In the remainder of this section, we show the above-mentioned theorem about the
    existence of an `inverse' in respect to the Whitney sum.
    This statement is essential for the proof of Milnors \Cref{normal::milnor}.

    In order to show this, we require the following proposition,
    whose proof will be deferred until \myintref{chapter::suspension}.

    % ? rephrase
\end{myparagraph}

% bouquet proposition
\begin{mystatement}{proposition}[Bouquet Lemma]{whitney::bouquet}[59]
    Let $\bb$ be a microbundle over a `bouquet' of spheres $B$, meeting in a single point.
    Then there exists a map $r: B \to B$ such that $\whitney{\bb}{\ind{r}\bb}$ is trivial.
\end{mystatement}

% theorem trivialize
\begin{mystatement}{theorem}{whitney::theorem}[59]
    Let $\bb$ be a microbundle over a $d$-simplicial complex $B$.
    Then there exists a microbundle $\bn$ over $B$ such that
    the Whitney sum $\whitney{\bb}{\bn}$ is trivial.
\end{mystatement}

% proof trivialize
\begin{myproof}
    We prove the theorem by induction over $d$.

    % start
    (Start of induction)

    A $1$-simplicial complex is just a bouquet of circles.
    Hence, the start of induction follows directly from \myintref{whitney::bouquet}.   

    % step
    (Inductive Step)

    Let $B'$ be the $(d - 1)$-skeleton of $B$ and let $\bn'$ be its corresponding microbundle
    such that $\whitney{\bb[B']}{\bn'}$ is trivial.

    \begin{steps}
        % simplex
        \item $\whitney{\bn'}{\be{B'}}$ can be extended over any $d$-simplex $\sigma$

        Consider the equation
        \[
            (\whitney{\bn'}{\be{B'}})\restr{\partial\sigma}
            = \whitney{\bn'\restr{\partial\sigma}}{\be{B'}\restr{\partial\sigma}}
            = \whitney{\bn'\restr{\partial\sigma}}{\bb[\partial\sigma]}
            = (\whitney{\bn'}{\bb[B']})\restr{\partial\sigma}
        \]
        in which we used \myintref{induced::simplex}
        for $\be{B'}\restr{\partial\sigma} = \bb[\partial\sigma]$.
        Since $(\whitney{\bn'}{\bb[B']})\restr{\partial\sigma}$ is trivial, the claim follows from \myintref{induced::simplex}.

        % extend
        \item $\whitney{\bn'}{\be{B'}}$ can be extended over $B$

        One difficulty is that the individual $d$-simplices are not well-separated.
        To deal with this, we consider $B''$ which is defined to be $B$
        with small open $d$-cells removed from every $d$-simplex.
        Since $B'$ is a retract of $B''$, we can extend $\whitney{\bn'}{\be{B'}}$ to a microbundle $\nu$ over $B''$.

        Now we extend $\nu$ over $B$ by taking all extensions of $\nu$
        over every simplex
        using (Step 1), and identifying its total spaces together along $E(\nu)$.
        Similarly, injection and projection are obtained
        by piecing the injection and projection maps of the individual extensions together.

        We denote the resulting microbundle by $\eta$.

        % trivial
        \item \blankbreak{}
        Consider the mapping cone $B \sqcup_\iota CB'$ over the inclusion $B' \incl B$.
        The following equation shows that $(\whitney{\bb}{\eta})\restr{B'}$ is trivial:
        \[
            (\whitney{\bb}{\eta})\restr{B'}
            \cong \whitney{\bb[B']}{\eta\restr{B'}}
            \cong \whitney{\bb[B']}{(\whitney{\bn'}{\be{B'}})}
            \cong \whitney{(\whitney{\bb[B']}{\bn'})}{\be{B'}}
            \cong \whitney{\be{B'}}{\be{B'}}
        \]
        \myintref{induced::cone} then yields a microbundle $\xi$ over $B \sqcup_\iota CB'$
        extending $\whitney{\bb}{\eta}$.

        The mapping cone $B \sqcup_\iota CB'$ has the homotopy type of a bouquet of spheres,
        which can be seen as follows:
        \begin{itemize}
            \item A $d$-simplex is homotopic to a $d$-disc.
            \item A $d$-disc whose boundary is collapsed to a single point is a $d$-sphere.
            \item One can define a homotopy between $\iota(B')$ and the tip of the cone by
            traveling along $CB'$.
        \end{itemize}
        
        Using \myintref{homotopy::theorem} and \myintref{whitney::bouquet},
        we conclude that there exists a microbundle $\bn$ such that $(\whitney{\xi}{\bn})\restr{B}$ is trivial.
        The equation
        \[
            \be{B}
            = (\whitney{\xi}{\bn})\restr{B}
            = \whitney{\xi\restr{B}}{\bn\restr{B}}
            = \whitney{(\whitney{\bb}{\eta})}{\bn\restr{B}}
            = \whitney{\bb}{(\whitney{\eta}{\bn\restr{B}})}
        \]
        completes the proof.
    \end{steps}
\end{myproof}
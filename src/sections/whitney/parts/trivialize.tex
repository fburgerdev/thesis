% foreword
\begin{myparagraph}
    Lastly, we show the above mentioned theorem about Whitney sums
    which will be essential in the proof of Milnors Theorem.

    For its prove, we need the following proposition
    which will be shown in \myintref{chapter::suspension}.
\end{myparagraph}

% bouquet proposition
\begin{mystatement}{proposition}{whitney::bouquet}[59]
    Let $\bb$ be a microbundle over a `bouquet' of spheres $B$, meeting in a single point.
    Then there exists a map $r: B \to B$ such that $\whitney{\bb}{\ind{r}\bb}$ is trivial.
\end{mystatement}

% theorem trivialize
\begin{mystatement}{theorem}{whitney::theorem}[59]
    Let $\bb$ be a microbundle over a $d$-simplicial complex $B$.
    Then there exists a microbundle $\bn$ over $B$ such that
    the Whitney sum $\whitney{\bb}{\bn}$ is trivial.
\end{mystatement}

% proof trivialize
\begin{myproof}
    We prove the theorem by induction over $d$.

    % start
    (Start of induction)

    A $1$-simplicial complex is just a bouquet of circles.
    Hence, the start of induction follows directly from \myintref{whitney::bouquet}.   

    % step
    (Inductive Step)

    Let $B'$ be the $(d - 1)$-skeleton of $B$ and let $\bn'$ be its corresponding microbundle
    such that $\whitney{\bb[B']}{\bn'}$ is trivial.

    \begin{enumerate}
        % simplex
        \item $\whitney{\bn'}{\be{B'}}$ can be extended over any $d$-simplex $\sigma$:

        Consider the equation
        \[
            (\whitney{\bn'}{\be{B'}})\restr{\partial\sigma}
            = \whitney{\bn'\restr{\partial\sigma}}{\be{B'}\restr{\partial\sigma}}
            = \whitney{\bn'\restr{\partial\sigma}}{\bb[\partial\sigma]}
            = (\whitney{\bn'}{\bb[B']})\restr{\partial\sigma}
        \]
        in which we used the previous lemma and \myintref{induced::simplex}
        for $\be{B'}\restr{\partial\sigma} = \bb[\partial\sigma]$.
        Since $(\whitney{\bn'}{\bb[B']})\restr{\partial\sigma}$ is trivial, the claim follows from \myintref{induced::simplex}.

        % extend
        \item $\whitney{\bn'}{\be{B'}}$ can be extended over $B$:

        The difficulty is that the individual $d$-simplices are not well-seperated.
        Let $B''$ denote $B$ with small open $d$-cells removed from every $d$-simplex.
        Since $B'$ is a retract of $B''$ we can extend $\whitney{\bn'}{\be{B'}}$ to a microbundle $\nu$ over $B''$.

        Now we can extend $\nu$ over $B$ by taking all extensions of $\nu$
        over every simplex
        using (1.), and glueing its total spaces together along $E(\nu)$.
        Similarly, the injection and projection can be obtained
        by glueing the injection and projection maps over every simplex together.

        We denote the resulting microbundle by $\eta$.

        % trivial
        \item
        Consider the mapping cone $B \sqcup_\iota CB'$ over the inclusion $B' \incl B$.
        Since
        \[
            (\whitney{\bb}{\eta})\restr{B'}
            = \whitney{\bb[B']}{\eta\restr{B'}}
            = \whitney{\bb[B']}{(\whitney{\bn'}{\be{B'}})}
            = \whitney{(\whitney{\bb[B']}{\bn'})}{\be{B'}}
            = \whitney{\be{B'}}{\be{B'}}
        \]
        % ! associativity not shown
        is trivial, it follows from \myintref{induced::cone} that
        we can extend $\whitney{\bb}{\eta}$ over $B \sqcup_\iota CB'$,
        which will be denoted by $\xi$.

        The mapping cone $B \sqcup_\iota CB'$ has the homotopy type of a bouquet of spheres
        by carrying $B'$ along $CB'$ collapsing to a single point.
        Since any $d$-simplex is homotopic to a $d$-disc and its boundary is collapsed,
        we obtain the homotopy of a $(d - 1)$-sphere.
        
        Using \myintref{homotopy::theorem} and \myintref{whitney::bouquet},
        we conclude that there exists a microbundle $\bn$ such that $(\whitney{\xi}{\bn})\restr{B}$ is trivial.
        The equation
        \[
            \be{B}
            = (\whitney{\xi}{\bn})\restr{B}
            = \whitney{\xi\restr{B}}{\bn\restr{B}}
            = \whitney{(\whitney{\bb}{\eta})}{\bn\restr{B}}
            = \whitney{\bb}{(\whitney{\eta}{\bn\restr{B}})}
        \]
        completes the proof.
    \end{enumerate}
\end{myproof}
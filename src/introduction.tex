\begin{myparagraph}
    In differential geometry,
    the tangent- and normal bundle of smooth manifolds
    play an essential role in understanding their underlying geometry.

    One can define the tangent space
    in a point $p \in M$ using derivations
    \[ T_p M = \set{\nu: C^\infty(M) \to \R^n \text{ linear}}{\nu(fg) = f(p)\nu(g) + \nu(f)g(p)} \]    
    or using tangent curves
    \[ T_p M = \set{\gamma \in C^\infty((-1, 1), M)}{\gamma(0) = p } / \sim \]
    where $\gamma \sim \gamma' \iff \frac{d}{dx}(\psi \circ \gamma)(0) = \frac{d}{dx}(\psi \circ \gamma')(0)$
    and $\psi$ is a chart for $p$.
    
    The tangent space in a point allows for the definition of the tangent bundle
    \[ TM := \bigsqcup_{p \in M} T_p M\]
    together with the section
    \[ TM \xto{\pi} M \twith \pi(p, \nu) = p. \]

    Lets say one wants to define a tangent bundle over a topological manifold
    without given any smooth structure.
    We cannot use the same constructions as in the smooth case, because
    as we have just seen the definition of
    the tangent space always requires the notion of differentiability.

    What we can do is to generalize the concept of tangent bundles
    so that it can be applied to topological manifolds,
    in the hope that many results transfer to this generalization.

    In the paper `Microbundles, Part I', J. Milnor introduces a concept
    of tangent bundles over topological manifolds.
    This tangent bundle is not a vector bundle as in the smooth case,
    instead the tangent bundle is a `microbundle'
    which is a weakening of the definition of a vector bundle.

    One can transfer many constructions and results for
    vector bundles over to microbundles,
    for example the `whitney sum' or the `induced bundle'. 

    Furthermore, one can also define a microbundle analogue
    to the normal bundle for smooth manifolds.

    In the smooth case, given an embedded submanifold $P \sub M$,
    the normal space in a point $p \in P$
    is defined to be the quotient
    \[ N_p P = T_p M / T_p N. \]
    Similar to the tangent bundle, the normal bundle is defined as
    \[ NP  = \bigsqcup_{p \in P} N_p P \]
    together with the section
    \[ NP \xto{\pi} N \twith \pi(p, \nu) = p. \]
    
    So for smooth embedded manifolds there always exists a normal bundle
    in its surrounding space.

    This result doesn't transfer over to microbundles.
    One can find examples for embedded topological manifolds
    such that there doesn't exist a normal microbundle.

    However, Milnor shows that one can always find a normal microbundle
    of the submanifold $P$ in a tubular space $M \cross \R^q$
    for a sufficiently large $q \in \N$.

    Proving this statement while presenting the concept of
    microbundles along the way is the content of this thesis.

    It is based on Milnors paper `Microbundles, Part I',
    adopting much of its structure and proofs.
    Mostly there will be provided more details, proving every statement explicitly, than in Milnors original work.
\end{myparagraph}
\begin{myparagraph}
    % smooth
    In differential geometry,
    the tangent bundle of a smooth manifold
    is a fundamental concept in understanding the manifolds underlying geometry.
    
    % smooth :: tangent space
    Given a smooth manifold $M$,
    one can define the tangent space
    in a point $p \in M$ using derivations
    \[
        T_p M = \set{\nu: C^\infty(M) \to \R^n \text{ linear}}{\nu(fg)
        = f(p)\nu(g) + \nu(f)g(p)}
    \]    
    or using tangent curves
    \[ T_p M = \set{\gamma \in C^\infty((-1, 1), M)}{\gamma(0) = p } / \sim \]
    where $\gamma \sim \gamma' \iff \frac{d}{dx}(\psi \circ \gamma)(0) = \frac{d}{dx}(\psi \circ \gamma')(0)$
    and $\psi$ is a chart for $p$.
    
    % smooth :: tangent bundle
    The tangent space in a point allows for the definition of the tangent bundle
    \[ TM := \bigsqcup_{p \in M} T_p M\]
    together with the section
    \[ TM \xto{\pi} M \twith \pi(p, \nu) = p. \]

    % topological
    Lets say one wants to define the tangent bundle over a topological manifold.
    % topological :: problem
    We cannot use the same construction as in the smooth case, because
    as we have just seen the definition of
    the tangent space always requires the notion of differentiability.
    However, a topological manifold in general
    does not admit a differentiable structure (see\cite{kervaire})
    and even if there exists one,
    it is generally not unique (e.g. the $7$-sphere\cite{milnor7sphere}).

    % topological :: naive
    So a different construction to the smooth tangent bundle is necessary.
    One plausible approach would be to define the tangent space in a point $p \in M$
    to be $\coll{x} \cross U_x$, where $U_x \isomto \R^n$ using a chart of $p$.
    However, with this construction we would have choose the neighborhoods $U_x$
    accordingly such that they vary continuously over $M$.
    Furthermore, it is unknown whether this construction would be topologically
    invariant to the choices of $U_x$ (compare\cite[p.53]{milnor}).

    % topological :: solution
    In 1964,
    the well-known mathematician John Milnor
    published a paper called ``Microbundles, Part I''
    introducing a unique way to solve this problem:

    The core idea is, that we drop the assumption
    that the tangent bundle is a vector bundle and hence
    every fiber is homeomorphic to euclidean spaces.
    Instead, we only require the fibers to be a `germ' of euclidean space,
    i.e. a topological space with an open subset homeomorphic to euclidean space.
    So in contrast to the above approach,
    we could now take any neighborhood of $p$ as the tangent space in $p$ since
    one can always find a chart within.

    If the respective neighborhoods can be chosen freely,
    we could always choose the entire space $M$ for the sake of simplicity.

    It follows that our resulting total space is of the form $M \cross M$.
    
    % topological :: microbundle
    Milnor establishes this approach by generalizing this type of bundle.
    He calls them `microbundles'.
    In the above mentioned paper, Milnor shows that many fundamental properties
    and constructions immediatly carry over other to microbundles,
    e.g. induced microbundles or the Whitney sum (see \myintref{chapter::constructions}).

    % topological :: normal bundle
    Furthermore, Milnor defines what the topological analogue
    of the normal bundle of a smooth manifold should look like.
    In the smooth case, given an embedded submanifold $P \sub M$,
    the normal space in a point $p \in P$
    is defined to be the quotient
    \[ N_p P = T_p M / T_p N. \]
    Similar to the tangent bundle, the normal bundle is defined as
    \[ NP  = \bigsqcup_{p \in P} N_p P \]
    together with the section
    \[ NP \xto{\pi} N \twith \pi(p, \nu) = p. \]
    So for smooth embedded manifolds there always exists a normal bundle
    in its surrounding space.
    However, this is generally not true for normal microbundles (see\cite{rourke}).
    Instead, Milnor could show that there exists a normal microbundle of an embedded manifold
    in a stabalized surrounding manifold $M \times \R^q$ for some $q \in \N$. 

    % applications
    This is a very important result, since it allows for example to transfer many results
    of cobordism theory by Thom over to topological manifolds
    with the use of normal microbundles.

    % thesis structure
    Proving the statement of normal microbundles while establishing the concept of
    microbundles along the way is the content of this thesis.

    It is based on Milnors paper `Microbundles, Part I',
    adopting much of its structure and proofs.
    Mostly there will be provided more details,
    proving every statement explicitly,
    than in Milnors original work.
\end{myparagraph}
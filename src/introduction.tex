\begin{myparagraph}
    % smootg
    % smooth :: tangent space
    Given a smooth manifold $M$,
    one can define the tangent space
    in a point $p \in M$ using `derivations'
    \[
        T_p M = \set{\nu: C^\infty(M) \to \R^n \text{ linear}}{\nu(fg)
        = f(p)\nu(g) + \nu(f)g(p)}
    \]
    or using `tangent curves'
    \[ T_p M = \set{\gamma \in C^\infty((-1, 1), M)}{\gamma(0) = p } / \sim \]
    where $\gamma \sim \gamma' \iff \frac{d}{dx}(\psi \circ \gamma)(0) = \frac{d}{dx}(\psi \circ \gamma')(0)$
    and $\psi$ is a chart for $p$.
    % ? omit

    % smooth :: tangent bundle
    The tangent space allows for the definition of the tangent bundle
    \[ TM := \bigsqcup_{p \in M} T_p M\]
    together with the projection
    \[ TM \xto{\pi} M, \pi(p, \nu) = p. \]

    % topological
    Suppose we want to define the tangent bundle over a topological manifold $M$.
    % topological :: problem
    As we see from the two definitions of the tangent space,
    the tangent bundle requires the notion of differentiability.
    However, M. Kervaire showed in 1960 that there exists a $10$-dimensional manifold
    that does not admit a differentiable structure (see\cite{kervaire}).
    Furthermore, even if there exists such a structure for a topological manifold,
    it is generally not unique (e.g. the $7$-sphere\cite{milnor7sphere}).
    % TODO: elaborate

    Therefore,
    we need a different construction than the one for the smooth case.
    
    % topological :: naive
    One plausible approach would be to define the fibers of the tangent bundle
    to be of the form $\coll{p} \cross U_p$,
    where $U_p$ is a neighborhood of $p$ which is
    homeomorphic to $\R^d$ via a chart.
    However, this construction raises the problem
    on choosing neighborhoods $U_p$
    such that they vary continuously over $M$.
    Furthermore,
    it is questionable whether this construction is topologically
    invariant if it depends on specific choices of neighborhoods $U_p$.

    % topological :: solution
    In 1964, John Milnor published `Microbundles, Part I',
    introducing a unique way to think
    about tangent bundles over topological manifolds:

    The core idea behind this approach is
    to drop the assumption
    that the tangent bundle is a vector bundle and hence
    every fiber must be homeomorphic to euclidean spaces.
    Instead, we require that the fibers are a `germ' of euclidean space,
    i.e. a topological space with an open subset homeomorphic to euclidean space.
    % TODO: rephrase to make clearer that fibers don't have to be static
    In contrast to the above approach,
    we can now choose the neighborhoods $U_p$ of $p$ regardless of any corresponding charts.
    That is because we can always find
    the domain of a chart (which is homeomorphic to $\R^d$) within arbitrary neighborhoods of $p$.

    If the respective neighborhoods can be chosen freely,
    we may as well always choose the entire space $M$ for the sake of simplicity.

    So it follows that our resulting total space is of the form $M \cross M$, which,
    analoguous to the smooth case, comes equipped with the projection
    \[ M \cross M \xto{\pi} M, \pi(m, m') = m. \]
    
    % topological :: microbundle
    In order to develop this approach, Milnor introduces a new type of bundle
    that arise from this.
    He calls them `microbundles'.
    Many fundamental properties
    and constructions of vector bundles immediatly carry over other to microbundles,
    e.g. induced microbundles or the Whitney sum (see \myintref{chapter::constructions}).
    Moreover, Milnor shows that if a manifold can be equipped with a smooth structure,
    then the tangent vector bundle regarded as microbundle is isomorphic to the
    tangent microbundle.

    % topological :: normal bundle
    Furthermore, Milnor defines what the microbundle analogue
    of the normal bundle is.
    Given a smooth embedded submanifold $P \sub M$,
    there always exists a normal bundle $NP$ defined fiber-wise with
    \[ N_p P = T_p M / T_p N. \]
    However, for the topological case it is not as easy.
    The two mathematicians C. Rourke and B. Sanderson could construct a $19$-dimensional manifold
    embedded in $S^{29}$ that does not admit a normal microbundles (see\cite{rourke}).
    Instead, Milnor could show that there always exists a normal microbundle
    of an embedded manifold
    in a stabilized surrounding manifold $M \times \R^q$ for some $q \in \N$.

    % applications
    Milnors studies on microbundles allowed R. Williamson
    to transfer fundamental results of cobordism theory from the smooth case
    over to the topological case, particulary by utilising the existence
    of a normal microbundle in a stabilized surrounding manifold.
    % TODO: cite williamson

    % thesis structure
    This thesis introduces the concept of microbundles and presents
    a proof for the above theorem.
    It is based on Milnors paper `Microbundles, Part I',
    adopting much of its structure and proofs.
\end{myparagraph}